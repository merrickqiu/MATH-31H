\documentclass{article}

\usepackage{amsmath}
\usepackage{amssymb}
\usepackage{hyperref}
\usepackage{indentfirst}
\usepackage{matlab-prettifier}

%User defined commands
\newcommand{\sgn}{\operatorname{sgn}}

\begin{document}
\begin{center}
	\huge{\bf Math 170A: Homework 1} \\
	Merrick Qiu
\end{center}

\section*{Question 1}
\begin{enumerate}
    \item \begin{align*}
        X &= L_{ji}^aA \\
        &= (I_n + B_{ji}^a)A \\
        &= A + B_{ji}^aA \\
    \end{align*}
    For all rows $k \neq j$, the $k$th row of $B_{ji}^aA$ is all zeros
    since the $k$th row of $B_{ji}^a$ is all zeros.
    The $j$th row of $B_{ji}^aA$ is $aR_i$ since the $ji$th entry 
    of $B_{ji}^a$ is $a$.
    Adding $A$ to $B_{ji}^aA$ yields in all the rows being equal to 
    the row in $A$ except for the $j$th row, which is equal to $R_j + aR_i$.
    Thus, $X = L_{ji}^aA = A + B_{ji}^aA$ 
    is the result of the row operation $R_j + aR_i \rightarrow R_j$.
    \item A matrix that is $(L_{ji}^a)^{-1}$ will perform the row operation $R_j - aR_i$.
    From part (a), we know that this matrix is 
    \begin{align*}
        (L_{ji}^a)^{-1} &= L_{ji}^{-a} \\
        &= I_n + B_{ji}^{-a} \\
        &= I_n - B_{ji}^{a}
    \end{align*}
\end{enumerate}
\newpage 

\section*{Question 2}
Since $1 \leq i < j < n$ and $1 \leq i < l < k \leq n$,
$B_{ji}^aB_{kl}^b = 0$.
\begin{align*}
    L_{ji}^aL_{kl}^b &= (I + B_{ji}^a)(I + B_{kl}^b) \\
    &= I + B_{ji}^a + B_{kl}^b + B_{ji}^aB_{kl}^b \\
    &= I + B_{ji}^a + B_{kl}^b \\
    &= I + D
\end{align*}
\newpage

\section*{Question 3}
\begin{enumerate}
    \item For the base case row $n$, $a_{nn} x_n = 0$ so $x_n = 0$.
    For row $j$ where $i< j < n$, assume that $x_k = 0$ for all $k > j$.
    Then we have that $a_{jj}x_j + \sum_{k=j+1}^n a_{jk}x_k = 0$.
    Substituting in $x_k = 0$ for all $k > j$, implies that $x_j = 0$.
    By induction we have that $x_j = 0$ for all  $i< j \leq n$,
    so $x$ has the same pattern of zeros as $b$.
    \item We have that $AA^{-1} = I$, 
    which implies that for $j$th column of $A^{-1}$
    \[  
        Ax_j = e_j \quad j=1,\cdots, n
    \]
    where $n$ is the size of the matrix and $e_j$ is the j-th unit vector.
    Since $e_j$ has all zeros after $j$, invoking part (a)
    implies that $x_j$ also has all zeros after $j$.
    Applying this to every column in $A^{-1}$ shows that 
    $A^{-1}$ is also upper triangular.
\end{enumerate}
\newpage

\section*{Question 4}
\lstinputlisting[
frame=single,
numbers=left,
style=Matlab-editor
]{multiplyA_BX.m}
Line 12 has $2$ flops and it is run $n^2$ times.
This is also true of line 19 so there are a total of 
$4n^2$ flops to calculate $A(Bx)$.
\newpage

\lstinputlisting[
frame=single,
numbers=left,
style=Matlab-editor
]{multiplyAB_X.m}
Line 13 has 2 flops and it is run for $n^3$ times.
Line 21 has 2 flops and it is run for $n^2$ times.
There are a total of $2n^3 + 2n^2$ flops
to calculate $(AB)x$. \\

Comparing the two algorithms, we see that 
$4n^2 < 2n^3 + 2n^2$ so computing $A(Bx)$ is
more efficient than $(AB)x$.








\end{document}