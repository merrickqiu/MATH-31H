\documentclass{article}

\usepackage{amsmath}
\usepackage{amssymb}
\usepackage{hyperref}
\usepackage{indentfirst}
\usepackage{matlab-prettifier}

%User defined commands
\newcommand{\sgn}{\operatorname{sgn}}

\begin{document}
\begin{center}
	\huge{\bf Math 170A: Homework 2} \\
	Merrick Qiu
\end{center}

\section*{Question 1}
The matrix does not have an LU decomposition because $A_{1,1} = 0$.
The operations on $A$ are as follows.
\begin{align*}
    &\begin{bmatrix}
        0 & 1 & 1 & 2 \\
        2 & 0 & 0 & 0 \\
        1 & 1 & 1 & 3 \\
        1 & 0 & 2 & 4
    \end{bmatrix} \implies 
    \begin{bmatrix}
        2 & 0 & 0 & 0 \\
        0 & 1 & 1 & 2 \\
        1 & 1 & 1 & 3 \\
        1 & 0 & 2 & 4
    \end{bmatrix} \implies
    \begin{bmatrix}
        2 & 0 & 0 & 0 \\
        0 & 1 & 1 & 2 \\
        0 & 1 & 1 & 3 \\
        0 & 0 & 2 & 4
    \end{bmatrix} \implies \\
    &\begin{bmatrix}
        2 & 0 & 0 & 0 \\
        0 & 1 & 1 & 2 \\
        0 & 1 & 1 & 3 \\
        0 & 0 & 2 & 4
    \end{bmatrix} \implies
    \begin{bmatrix}
        2 & 0 & 0 & 0 \\
        0 & 1 & 1 & 2 \\
        0 & 0 & 0 & 1 \\
        0 & 0 & 2 & 4
    \end{bmatrix} \implies 
    \begin{bmatrix}
        2 & 0 & 0 & 0 \\
        0 & 1 & 1 & 2 \\
        0 & 0 & 2 & 4 \\
        0 & 0 & 0 & 1
    \end{bmatrix}
\end{align*}

The operations on $L$ are as follows.
\begin{align*}
    &\begin{bmatrix}
        0 & 0 & 0 & 0 \\
        0 & 0 & 0 & 0 \\
        0 & 0 & 0 & 0 \\
        0 & 0 & 0 & 0
    \end{bmatrix} \implies 
    \begin{bmatrix}
        0 & 0 & 0 & 0 \\
        0 & 0 & 0 & 0 \\
        0 & 0 & 0 & 0 \\
        0 & 0 & 0 & 0
    \end{bmatrix} \implies 
    \begin{bmatrix}
        0 & 0 & 0 & 0 \\
        0 & 0 & 0 & 0 \\
        0.5 & 0 & 0 & 0 \\
        0.5 & 0 & 0 & 0
    \end{bmatrix} \implies \\
    &\begin{bmatrix}
        0 & 0 & 0 & 0 \\
        0 & 0 & 0 & 0 \\
        0.5 & 0 & 0 & 0 \\
        0.5 & 0 & 0 & 0
    \end{bmatrix} \implies
    \begin{bmatrix}
        0 & 0 & 0 & 0 \\
        0 & 0 & 0 & 0 \\
        0.5 & 1 & 0 & 0 \\
        0.5 & 0 & 0 & 0
    \end{bmatrix} \implies 
    \begin{bmatrix}
        0 & 0 & 0 & 0 \\
        0 & 0 & 0 & 0 \\
        0.5 & 0 & 0 & 0 \\
        0.5 & 1 & 0 & 0
    \end{bmatrix} 
\end{align*}
We swapped $R_1$ with $R_2$ and $R_3$ with $R_4$ so the permutation matrix is 
\[
    P = \begin{bmatrix}
        0 & 1 & 0 & 0 \\
        1 & 0 & 0 & 0 \\
        0 & 0 & 0 & 1 \\
        0 & 0 & 1 & 0
    \end{bmatrix}
\]
We add the identity to $L$ to get
\[
    L = \begin{bmatrix}
        1 & 0 & 0 & 0 \\
        0 & 1 & 0 & 0 \\
        0.5 & 0 & 1 & 0 \\
        0.5 & 1 & 0 & 1
    \end{bmatrix}
\]
U is the resulting matrix after the the row operations 
\[
    U = \begin{bmatrix}
        2 & 0 & 0 & 0 \\
        0 & 1 & 1 & 2 \\
        0 & 0 & 2 & 4 \\
        0 & 0 & 0 & 1
    \end{bmatrix}
\]
\newpage 
\section*{Question 2}
\begin{align*}
    r_{11} &= \sqrt{a_{11}} = 1 \\
    r_{12} &= \frac{a_{12}}{r_{11}} = -2 \\
    r_{13} &= \frac{a_{13}}{r_{11}} = 0 \\
    r_{22} &= \sqrt{a_{22}-r_{12}^2} = 3 \\
    r_{23} &= \frac{a_{23} - r_{12}r_{13}}{r_{22}} = 2\\
    r_{33} &= \sqrt{a_{33} - r_{13}^2 - r_{23}^2} = 1
\end{align*}
\[
    R = \begin{bmatrix}
        1 & -2 & 0 \\
        0 & 3 & 2 \\
        0 & 0 & 1
    \end{bmatrix}
\]
Since we were able to perform all the math for the Cholesky-factorization 
and since the matrix is symmetric, we know that $A$ is positive-definite.

The first and second row of $B$ are not independent, so 
$B$ has an eigenvalue of 0 and it does not 
have a Cholesky factorization.
\newpage 

\section*{Question 3}
$B$ is symmetric since 
\begin{align*}
    B^T &= (X^TAX)^T \\
    &= X^TA^T(X^T)^T \\
    &= X^TAX \\
    &= B
\end{align*}

Let $y \in \mathbb{R}^n$ and define $x = Xy$.
$B$ is positive definite since
\begin{align*}
    y^TBy &= y^T(X^TAX)y \\
    &= (Xy)^T A (Xy) \\
    &= x^T A x \\
    &> 0
\end{align*}
\newpage 

\section*{Question 4}
We can see that $AP^T$ is equivalent to 
tranposing A so that the columns are now rows,
permuting the rows, and then transposing the resulting matrix 
back such that the columns are permuted.
\[
    AP^T = ((AP^T)^T)^T = (PA^T)^T
\]
In row $i$ of $P$, the 1 is in column $p(i)$.
Since $P^T$ permutes the columns in $P$, it will send the $p(i)$th 
column to the $i$th column.
This will result in the 1 being in position $A_{ii}$ for all $i$,
which is the identity matrix.
\newpage 

\section*{Question 5}
The Backsub function used is the same as listed in the problem.
\lstinputlisting[
frame=single,
numbers=left,
style=Matlab-editor
]{ge_pp_solve.m}




\end{document}