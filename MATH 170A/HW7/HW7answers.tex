\documentclass{article}

\usepackage{amsmath}
\usepackage{amssymb}
\usepackage{hyperref}
\usepackage{indentfirst}
\usepackage{matlab-prettifier}
\usepackage[shortlabels]{enumitem}
\usepackage{graphicx}
\usepackage{physics}

%User defined commands
\newcommand{\inv}[1]{#1^{-1}}
\newcommand{\cond}{\kappa_{||\cdot||}}

\begin{document}
\begin{center}
	\huge{\bf Math 170A: Homework 7} \\
	Merrick Qiu
\end{center}

\section*{Q1}
We have that 
\[
    I - \delta A = 
    \begin{bmatrix}
        1 & 0 \\
        0 & 1 \\
    \end{bmatrix} -
    \begin{bmatrix}
        \frac{1}{3} & 0 \\
        0 & 1 \\
    \end{bmatrix}
    \begin{bmatrix}
        1 & 1 \\
        -2 & 3 \\
    \end{bmatrix} = 
    \begin{bmatrix}
        \frac{2}{3} & -\frac{1}{3} \\
        2 & -2 \\ 
    \end{bmatrix}
\]
The characteristic polynomial is 
$(\frac{2}{3} - \lambda)(-2-\lambda) + \frac{2}{3} = \lambda^2 + \frac{4}{3}\lambda - \frac{2}{3}$
The eigenvalues are $\frac{-\frac{4}{3}\pm \sqrt{\frac{16}{9}+ \frac{8}{3}}}{2}$
which is $\lambda_1 = -\frac{2}{3} - \frac{\sqrt{10}}{3}$ and 
$\lambda_1 = -\frac{2}{3} + \frac{\sqrt{10}}{3}$.
We can see that $|\lambda_1| > 1$ so the Jacobi iterative method 
will not always converge.
\newpage

\section*{Q2}
We have that 
\[
    x_1^{(1)} = \frac{1 - 0}{1} = 1 
\]
\[
    x_2^{(1)} = \frac{1 - 1}{3} = 0 
\]
\[
    x_1^{(2)} = \frac{1 - 0}{1} = 1 
\]
\[
    x_2^{(2)} = \frac{1 - 1}{3} = 0 
\]
\[
    \vdots
\]
\[
    x_1^{(5)} = \frac{1 - 0}{1} = 1 
\]
\[
    x_2^{(5)} = \frac{1 - 1}{3} = 0 
\]
We see that Gauss-seidel converges to the answer after one iteration
and remains at that value for each additional iteration.
\newpage

\section*{Q3}
Since $A^* = -A$, that means $A_{i,j} = -\overline{A_{j,i}}$,
which is only possible if every entry has 
$\Re(A_{i,j}) = -\Re(A_{j,i})$ and $\Im(A_{i,j}) = \Im(A_{j,i})$.
Thus $B=iA$ is a matrix with 
$\Im(A_{i,j}) = -\Im(A_{j,i})$ and $\Re(A_{i,j}) = \Re(A_{j,i})$,
meaning $B$ is a hermetian matrix.
Thus $B$ has all real eigenvalues, which are just the eigenvalues of $A$
times $i$. Thus $A$ must have all imaginary eigenvalues.
\newpage 

\section*{Q4}
\[
    AA^T = 
    \begin{bmatrix}
        2 & -1 & 1 \\
        -1 & 1 & 0 \\
        1 & 0 & 1
    \end{bmatrix}
\]
The characteristic polynomial is 
\[
    (2-\lambda)(1-\lambda)(1-\lambda) - (1-\lambda) - (1-\lambda)
    = -\lambda^3 + 4 \lambda^2 - 3\lambda
    = -\lambda(\lambda-3)(\lambda-1)
\]
The eigenvalues are $\lambda = 0, 1, 3$.
By inspection, we can see that the vectors that satisfy $(A-\lambda I)v = 0$ are
\[
    v_0 = 
    \begin{bmatrix}
        1 \\ 
        1 \\
        -1
    \end{bmatrix}
    v_1 = 
    \begin{bmatrix}
        0 \\
        1 \\
        1
    \end{bmatrix}
    v_3 = 
    \begin{bmatrix}
        2 \\
        -1 \\
        1
    \end{bmatrix}
\]
\[
    A^TA = \begin{bmatrix}
        2 & -1 \\
        -1 & 2
    \end{bmatrix}
\]
The characteristic polynomial is 
\[
    (2-\lambda)(2-\lambda) - 1 = \lambda^2 - 4\lambda + 3 = (\lambda -1)(\lambda-3)
\]
The eigenvalues are $\lambda = 1, 3$.
The eigenvectors are \
\[
    v_1 = 
    \begin{bmatrix}
        1 \\
        1 \\
    \end{bmatrix}
    v_3 = 
    \begin{bmatrix}
        1 \\
        -1 \\

    \end{bmatrix}
\]
We can put together the normalized eigenvectors of $AA^T$ to form $U$,
the normazliedeigenvectors of $A^TA$ to form $V$, and the square root of 
the eigenvalues from $V$ to form the diagonals of $\Sigma$.
\[
  A = \begin{bmatrix}
    \frac{\sqrt{6}}{3} & 0 & \frac{\sqrt{3}}{3} \\
    -\frac{\sqrt{6}}{6} & \frac{\sqrt{2}}{2} & \frac{\sqrt{3}}{3} \\
    \frac{\sqrt{6}}{6} & \frac{\sqrt{2}}{2} & -\frac{\sqrt{3}}{3}
  \end{bmatrix}  
  \begin{bmatrix}
    \sqrt{3} & 0 \\
    0 & 1 \\
    0 & 0
  \end{bmatrix}
  \begin{bmatrix}
    \frac{\sqrt{2}}{2} & -\frac{\sqrt{2}}{2} \\
    \frac{\sqrt{2}}{2} & \frac{\sqrt{2}}{2}
  \end{bmatrix}
\]
\newpage 

\section*{Q5}
\begin{enumerate}
    \item The eigenvalues of $B$ are the eigenvalues of $A$ minus 0.25.
    The eigenvalues of $C$ are the reciprocal of the eigenvalues of $B$.
    \item The eigenvalue is $0.5$, which corresponds to an eigenvalue of $0.25$
    in $B$, which corresponds to $4$, which is the largest eigenvalue of $C$.
    Essentially we want to find the smallest eigenvalue in $B$ because this is the 
    largest eigenvalue in $C$, which will get amplified by the power method.
    \item The eigenvalue is $-0.25$, which corresponds to $0.25$ in $B$,
    which corresponds to $4$, which is the largest eigenvalue of $C$.
    \item It results in $3.5804e-04$, which is very small so $q$ is a very
    good approximate.
\end{enumerate}
\end{document}