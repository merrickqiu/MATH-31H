\documentclass{article}

\usepackage{amsmath}
\usepackage{amssymb}
\usepackage{hyperref}
\usepackage{indentfirst}
\usepackage{matlab-prettifier}
\usepackage[shortlabels]{enumitem}
\usepackage{graphicx}

%User defined commands
\newcommand{\inv}[1]{#1^{-1}}
\newcommand{\abs}[1]{|#1|}
\newcommand{\norm}[1]{||#1||}
\newcommand{\cond}{\kappa_{||\cdot||}}

\begin{document}
\begin{center}
	\huge{\bf Math 170A: Homework 6} \\
	Merrick Qiu
\end{center}

\section*{Q1}
\begin{enumerate}
    \item 
    \begin{align*}
        ||UA||_F^2 &= trace((UA)^TUA) \\
        &= trace(A^TU^TUA) \\
        &= trace(A^TA)\\
        &= ||A||_F^2
    \end{align*}
    \item 
    \begin{align*}
        ||AV||_F^2 &= ||(AV)^T||_F^2  \\
        &= trace(AV(AV)^T) \\
        &= trace(AVV^TA) \\
        &= trace(AA^T) \\
        &= ||A^T||_F^2 \\
        &= ||A||_F^2 \\
    \end{align*}
    \item Since we can write any $n \times n$ matrix as 
    $A = U\Sigma V$ using singular value decomposition,
    and since $U$ and $V$ are orthonormal, $A$ has the same 
    frobenius norm as $\Sigma$, which has frobenius norm 
    $\sqrt{\sum_{i=1}^n \sigma_i^2}$.
\end{enumerate}
\newpage 

\section*{Q2}
Normalizing $q_0$ yields $[1, \frac{b}{a}]^T$, but repeatedly
applying $A$ simply switches the coordinates back and forth
\[
    \tilde{q_0} = [1, \frac{b}{a}]^T
\]
\[
    A\tilde{q_0} = [\frac{b}{a},1]^T
\]
\[
    A^2\tilde{q_0} = [1, \frac{b}{a}]^T
\]
\[
    A^3\tilde{q_0} = [\frac{b}{a},1]^T
\]
\[
    \vdots
\]

The power method relies on there existing a largest eigenvalue,
but the two eigenvalues of this matrix, $1$ and $-1$ have the same magnitude.
\newpage 

\section*{Q3}
The power method will converge to $v_2$.
Since $q$ does not depend on $v_1$, the power method will 
not be able to amplify $v_1$ so it will instead amplify $v_2$,
the eigenvector with the next largest eigenvalue.
This is why a random initial $q$ needs to be chosen for the power method.
\newpage 
\section*{Q4}
\begin{enumerate}
    \item The characteristic polynomial is $(\lambda - 1)^2 = \lambda^2 - 2\lambda + 1$,
    eigenvalue $\lambda = 1$ which has algebraic multiplicity 2
    from its factorization.
    \item Using the quadratic equation,
    \[
        \hat{\lambda} = \frac{2 \pm \sqrt{4 - 4(1-\epsilon)}}{2} = 1 \pm \sqrt{\epsilon}
    \]
    \item 
    \[
        |\hat{\lambda} -\lambda|=  |1 \pm\sqrt{10^{-12}}{2} - 1| = \times 10^{-6}
    \]
    \[
        \frac{10^{-6}}{10^{-12}} = 10^6 \text{ times bigger}
    \]
    \item Since a very small change in the coefficients can lead to an arbitrarily
    large relative change in the eigenvalues(by choosing a sufficiently small epsilon),
    the computation is numerically unstable.
\end{enumerate}
\newpage 
\section*{Q5}
\begin{enumerate}
    \item MATLAB says the rank of $A$ is 3
    \item The pseudoinverse is 
    \[  
        \begin{bmatrix}
            0.5 & 0 & 0 & -1 \\
            1 & -0.4 & -0.2 & -1 \\
            -1 & 0.4 & 0.2 & 2
        \end{bmatrix}
    \]
    \item Yes they agree, $A^\dagger A = I$ and 
    \[
        AA^\dagger = U_rU_r^T = \begin{bmatrix}
            1 & 0 & 0 & 0 \\
            0 & 0.8 & 0.4 & 0 \\
            0 & 0.4 & 0.2 & 0 \\
            0 & 0 & 0 & 1
        \end{bmatrix}
    \]
\end{enumerate}
\end{document}