
\documentclass{article}
\usepackage{amsmath}
\usepackage{amssymb}
\usepackage{hyperref}
\usepackage{mathrsfs}
\usepackage{enumerate}
\usepackage{bm}
\usepackage{physics}
\usepackage{polynom}
\setlength{\parindent}{0pt}
\usepackage[parfill]{parskip}
\usepackage[margin=1in]{geometry}
\usepackage{tikz}


\begin{document}
\begin{center}
	\huge{\bf Math 100B: Homework 6} \\
	Merrick Qiu
\end{center}

\section*{Problem 1}
Polynomials of degree $0$ are units so they aren't irreducible. 
Polynomials of degree $1$ must be the product of a degree $0$ polynomial and a degree $1$ polynomial so they are irreducible.
Polynomials of degree $2$ or $3$ that are reducible must have a root since their product contains at least one degree $1$ 
polynomial.
Checking the roots of all monic polynomials gives us the following table.
Therefore all the irreducibles up to associates are,
$x,x+1,x+2,x^2+1,x^2+x+2,x^2+2x+2,x^3+2x+1,x^3+2x+2,x^3+x^2+2,x^3+x^2+x+2,x^3+x^2+2x+1,x^3+2x^2+1, x^3+2x^2+x+1, x^3+2x^2+2x+2$.
\begin{table}[h]
    \centering
    \begin{tabular}{|c|c|c|c|}
        \hline 
         & $f(0)$ & $f(1)$ & $f(2)$ \\ 
        \hline 
        $x^2$ & $0$ & $1$ & $1$ \\
        \hline 
        $x^2+1$ & $1$ & $2$ & $2$ \\
        \hline 
        $x^2+2$ & $2$ & $0$ & $0$ \\
        \hline       
        $x^2+x$ & $0$ & $2$ & $0$ \\
        \hline 
        $x^2+x+1$ & $1$ & $0$ & $1$ \\
        \hline
        $x^2+x+2$ & $2$ & $1$ & $2$ \\
        \hline 
        $x^2+2x$ & $0$ & $0$ & $2$ \\ 
        \hline
        $x^2+2x+1$ & $1$ & $1$ & $0$ \\
        \hline 
        $x^2+2x+2$ & $2$ & $2$ & $1$ \\
        \hline 
        $x^3$ & $0$ & $1$ & $2$ \\
        \hline
        $x^3+1$ & $1$ & $2$ & $0$ \\
        \hline
        $x^3+2$ & $2$ & $0$ & $1$ \\
        \hline 
        $x^3+x$ & $0$ & $2$ & $1$ \\
        \hline 
        $x^3+x+1$ & $1$ & $0$ & $2$ \\
        \hline 
        $x^3+x+2$ & $2$ & $1$ & $0$ \\
        \hline 
        $x^3+2x$ & $0$ & $0$ & $0$ \\
        \hline 
        $x^3+2x+1$ & $1$ & $1$ & $1$ \\
        \hline
        $x^3+2x+2$ & $2$ & $2$ & $2$ \\
        \hline 
        $x^3+x^2$ & $0$ & $2$ & $0$ \\
        \hline 
        $x^3+x^2+1$ & $1$ & $0$ & $1$ \\
        \hline 
        $x^3+x^2+2$ & $2$ & $1$ & $2$ \\
        \hline 
        $x^3+x^2+x$ & $0$ & $0$ & $2$ \\
        \hline
        $x^3+x^2+x+1$ & $1$ & $1$ & $0$ \\
        \hline 
        $x^3+x^2+x+2$ & $2$ & $2$ & $1$ \\
        \hline 
        $x^3+x^2+2x$ & $0$ & $1$ & $1$ \\
        \hline 
        $x^3+x^2+2x+1$ & $1$ & $2$ & $2$ \\
        \hline 
        $x^3+x^2+2x+2$ & $2$ & $0$ & $0$ \\
        \hline 
        $x^3+2x^2$ & $0$ & $0$ & $1$ \\
        \hline 
        $x^3+2x^2+1$ & $1$ & $1$ & $2$ \\
        \hline 
        $x^3+2x^2+2$ & $2$ & $2$ & $0$ \\
        \hline 
        $x^3+2x^2+x$ & $0$ & $1$ & $0$ \\
        \hline 
        $x^3+2x^2+x+1$ & $1$ &$2$ & $1$ \\
        \hline 
        $x^3+2x^2+x+2$ & $2$ & $0$ & $2$ \\
        \hline 
        $x^3+2x^2+2x$ & $0$ & $2$ & $2$ \\
        \hline 
        $x^3+2x^2+2x+1$ & $1$ & $0$ & $0$ \\
        \hline 
        $x^3+2x^2+2x+2$ & $2$ & $1$ & $1$ \\
        \hline 
    \end{tabular}
\end{table}
\newpage 

\section*{Problem 2}
\begin{enumerate}[(a)]
    \item We can use De Moivre's theorem to find the four roots of $-1$,
    where $m=0,1,2,3$.
    \[
        \left(\cos(\frac{\pi+2\pi m}{4}) + i\sin(\frac{\pi+2\pi m}{4})\right)^4 
        = \cos (\pi + 2\pi m) + i\sin (\pi + 2\pi m)  
        = -1
    \]
    Any degree $1$ polynomial is irreducible since it can only be written as the product 
    of a degree $1$ polynomial with some constant.
    \[
        x^4+1 = \left(x - \frac{1}{\sqrt{2}} - \frac{i}{\sqrt{2}}\right)
        \left(x + \frac{1}{\sqrt{2}} - \frac{i}{\sqrt{2}}\right)
        \left(x + \frac{1}{\sqrt{2}} + \frac{i}{\sqrt{2}}\right)
        \left(x - \frac{1}{\sqrt{2}} + \frac{i}{\sqrt{2}}\right)
    \]
    \item Although none of the factors in $\mathbb{C}[x]$ are in $\mathbb{R}[x]$,
    we can combine the terms to get factors that are in $\mathbb{R}[x]$.
    These terms are irreducible because if they weren't, that would imply real roots.
    \begin{align*}
         x^4+1 &= \left(x - \frac{1}{\sqrt{2}} - \frac{i}{\sqrt{2}}\right)
        \left(x - \frac{1}{\sqrt{2}} + \frac{i}{\sqrt{2}}\right) 
        \left(x + \frac{1}{\sqrt{2}} - \frac{i}{\sqrt{2}}\right)
        \left(x + \frac{1}{\sqrt{2}} + \frac{i}{\sqrt{2}}\right) \\
        &= (x^2-\sqrt{2}x + 1)(x^2+\sqrt{2}x + 1)
    \end{align*} 
    \item Over $\mathbb{Q}$, $x^4+1$ is already irreducible since 
    $\mathbb{Q}[x] \subseteq \mathbb{R}[x]$ which are both UFDs,
    but the factors in $\mathbb{R}[x]$ are irrational and combining them just gives you $x^4+1$. 
    \item Since $x^4+1$ is irreducible in $\mathbb{Q}$, it is also irreducible in $\mathbb{Z}$.
    Since the reduction mod $p$ mapping $\mathbb{Z}[x] \to (\mathbb{Z}/3\mathbb{Z})[x]$ is an isomorphism,
    this means that it is also irreducible in $(\mathbb{Z}/3\mathbb{Z})[x]$.
    See problem 3(a) for more details.
\end{enumerate}
\newpage
\section*{Problem 3}
\begin{enumerate}
    \item Suppose that $f(x) = g(x)h(x)$ was reducible in $\mathbb{Z}[x]$, where degree of $g$ and $h$ are greater than $0$. 
    Then the homomorphism implies 
    \begin{align*}
        \overline{f}(x) &= (\phi(f))(x) \\
        &=(\phi(g)\phi(h))(x) \\
        &= \overline{g}(x)\overline{h}(x)
    \end{align*}
    which is a contradiction since $\overline{f}(x)$ is irreducible but $\overline{g}(x)$ and $\overline{h}(x)$
    are not units and nonzero since $\overline{a_n} \neq 0$.
    Therefore $f(x)$ must be irreducible.
    \item $80x^3-8x+100$ is irreducible in $\mathbb{Q}$ iff irreducible in $\mathbb{Z}$ if irreducible in $\mathbb{Z}/p\mathbb{Z}$ for some prime $p$.
    Choosing $p=5$ works since $-3x$ is irreducible in $\mathbb{Z}/5\mathbb{Z}$.
\end{enumerate}
\newpage
\section*{Problem 4}
    We can write $x^9-1 = (x-1)(x^2+x+1)(x^6+x^3+1)$.
    $(x-1)$ is irreducible since it is degree $1$,
    $(x^2+x+1)$ is irreducible since it is a degree $2$ polynomial with no roots.
    $(x^6+x^3+1)$ is irreducible by Eisenstein's criterion. 
    
    Let $\Phi(x) = x^6+x^3+1$.
    We can substitute $x \to x+1$ to get 
    \[
        \Phi(x+1) = (x+1)^6 + (x+1)^3 + 1 = 
        x^6 + 6x^5 + 15x^4 + 21x^3 + 18x^2 + 9x + 3.
    \]
    Notice that all ther coefficients except for the leading coefficient 
    are divisible by $3$ and the constant term is not divisible by $9$,
    so it is irreducible.
    \newpage
\section*{Problem 5}
\begin{enumerate}[(a)]
    \item $2x^3 + x -4$ is irreducible since it only has irrational roots 
    that can be calculated through the cubic formula,
    but any factorization of $2x^3+x-4$ must have a degree $1$ polynomial.
    \item By the Eisenstein criterion, all the coefficients 
    except for the leading coefficient are divisble by $2$ but the constant
    is not divisible by $4$ so it is irreducible.
    \item First notice that $x^4+10x^2+1$ does not have rational roots 
    so it must factor into two degree two polynomials.
    \[
        (x^2+ax+b)(x^2+cx+d) = x^4 +(a+c)x^3 + (ac+b+d)x^2+(ad+bc)x + bd \\
    \]
    This gives up the following system of equations 
    \begin{align*}
        a+c &= 0 \\
        ac+b+d &= 10 \\
        ad+bc &= 0 \\
        bd &= 1 
    \end{align*}
    From the last equation, either $b=d=1$ or $b=d=-1$.
    In the first case, the second equation gives $ac=8$
    but the third equation gives $a+c=0$ which is a contradiction.
    Similarly, $ac=12$ but $-a-c=0$ is a contradiction.
    Therefore $x^4+10x^2+1$ is irreducible since it cannot be factorized
    into degree $2$ polynomials. 
\end{enumerate}
\newpage
\section*{Problem 6}
As proved in class, there exist a rational number $r \in \mathbb{Q}$
such that $h(x) = f'(x)g'(x)$ where $f'(x) = rf(x)$ and $g'(x) = r^{-1}g(x)$
have integer coefficients. 
The product of two coefficients of $f'(x)$ and $g'(x)$ is clearly an integer
and the product of two coefficients of $f(x)$ and $g(x)$ is equivalent to the 
product of the associated product of coefficients from $f'(x)$ and $g'(x)$
since $rr^{-1} = 1$.
Therefore the product of any coefficient of $f$ with any coefficient of $g$ is an integer.


\end{document}