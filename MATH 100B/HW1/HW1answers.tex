\documentclass{article}

\usepackage{amsmath}
\usepackage{amssymb}
\usepackage{hyperref}
\usepackage{mathrsfs}
\usepackage{enumerate}
\usepackage{bm}
\usepackage{physics}
\usepackage{polynom}
\setlength{\parindent}{0pt}
\usepackage[parfill]{parskip}
\usepackage[margin=1in]{geometry}

\begin{document}
\begin{center}
	\huge{\bf Math 100B: Homework 1} \\
	Merrick Qiu
\end{center}

\section*{1. (Artin 11.1.7(a))}
Let $A^C = U - A$ be the complement of $A$.
Addition is defined as the symmetric difference, which can be expressed as
\[
	A+B = A\cup B - A\cap B = (A\cap B^C) \cup (A^C\cap B).
\]
Addition is commutative since
\[
	A+B = (A\cap B^C) \cup (A^C\cap B) = (B\cap A^C) \cup(B^C\cap A) = B+A.
\]
Addition is associative since
\begin{align*}
	(A + B) + C &= ((A\cap B^C) \cup (A^C\cap B)) + C \\
	&= ((A\cap B^C) \cup (A^C\cap B)) \cap C^C \cup 
	((A\cap B^C) \cup (A^C\cap B))^C \cap C \\
	&= (A\cap B\cap C) \cup (A\cap B^C\cap C^C) \cup (A^C\cap B\cap C^C) \cup 
        (A^C\cap B^C \cap C) \\
	&= A\cap ((B\cap C^C) \cup (B^C\cap C))^C \cup A^C\cap ((B\cap C^C) \cup (B^C\cap C))\\
	&= A + ((B\cap C^C) \cup (B^C\cap C)) \\ 
	&= A+(B+C).
\end{align*}
The empty set is the additive identity since 
\[
	A + \emptyset = A\cup \emptyset - A\cap \emptyset = A.
\]
Each element has itself as its additive inverse since 
\[
	A + A = A\cup A - A\cap A = \emptyset.
\]
Multiplication is commutative and associative since 
intersection is commutative and associative.
$U$ is the multiplicative identity since any set intersection
with $U$ is itself.

The distributive law holds since 
\begin{align*}
	(A+B)C &= (A\cup B - A\cap B) \cap C \\
	&= (A\cup B)\cap C - A\cap B\cap C \\
	&= (A\cap C)\cup (B\cap C) - A\cap C\cap B\cap C \\
	&= (A\cap C) + (B\cap C) \\
	&= AC + BC 
\end{align*}

$R$ is a ring because it satisfies all of the axioms.
\newpage

\section*{2. (Artin 11.1.6(a))}
Since $\mathbb{Q}$ is a ring we
just need to check that $S$ is closed
under subtraction, multiplication, and contains $1$.
Let $\frac{a}{b}$ and $\frac{c}{d}$ be two elements in $S$
where $b$ and $d$ are not divisible by $3$.

$S$ is closed under subtraction and multiplication
since the result can be written with denominator $bd$ 
which is also not divisible by $3$.
\[
	\frac{a}{b} - \frac{c}{d} = \frac{ad-cb}{bd}.
\]
\[
	\frac{a}{b} \cdot \frac{c}{d} = \frac{ac}{bd}.
\]
$S$ also contains $1$ since $1 = \frac{1}{1}$.
Therefore $S$ is a subring.
\newpage 

\section*{3. (Artin 11.1.9)}
Addition is commutative if multiplication is commutative
and distributivity holds since 
\begin{align*}
	ab &= ab + 0b \\
	&= (a+0)b \\
	&= b(a+0) \\
	&= ba + b0 \\
	&= ba.
\end{align*}
\newpage 

\section*{4.}
If $a^n = 0$, then $(1-a)^{-1} = 1+a+\ldots+a^{n-2}+a^{n-1}$ since 
\begin{align*}
	(1-a)(1+a+\ldots+a^{n-1}) &= (1+a+\ldots+a^{n-1}) - (a+a^2+\ldots+a^{n-1}+a^n) \\
	&= (1+a+\ldots+a^{n-1}) - (a+a^2+\ldots+a^{n-1}) \\
	&= 1.
\end{align*}
\newpage

\section*{5.}
The identities are $f(x) = 0$ and $f(x) = 1$.
The units are functions that are not equal to $0$
at any point since their inverse 
is $[f^{-1}](x) = (f(x))^{-1}$.
The only nilpotent function is $f(x) = 0$.
The zero-divisors are the nonzero functions that 
are equal to $0$ at some point.
If $f$ is a function with this property we can define 
\[
	g(x) = \begin{cases}
		1 & f(x) = 0 \\
		0 & f(x) \neq 0
	\end{cases}
\]
so that $g \neq 0$ but $fg = 0$.

Since all functions are either $0$ at some point
or at no points, all functions are either 
a zero-divisor or a unit.
\newpage 

\section*{6.}
The units are the numbers with a multiplicative inverse,
which holds for $a$ when $\gcd(a,n) = 1$.
The nilpotent elements are the numbers whose prime
factors contain the prime factors of $n$.
For example if $n=242 = 2\cdot 11^2$ then 
$a=132=2^2\cdot 3 \cdot 11$ is a nilpotent element
since it contains the prime factors $2$ and $11$.
We can see that $132^2 \equiv 17424 \equiv 0 \mod 242$.
The zero-dvisors are the numbers that divide $n$, 
which holds for $a$ when $\gcd(a,n) \neq 1$.

Since all numbers either have a gcd of $1$ or not $1$,
all elements are either a unit or a zero-divisor.

\newpage 

\section*{7.}
\begin{enumerate}[(a)]
	\item Since $\sqrt{r} \notin \mathbb{Q}$,
	there does not exist $\sqrt{r} = \frac{a}{b}$
	for $a,b \in \mathbb{Q}$.
	Therefore $a_2\sqrt{r}$ and $b_2\sqrt{r}$
	are irrational too while $a_1$ and $b_1$ are rational.
	Therefore if $a_1 + a_2\sqrt{r} = b_1 + b_2\sqrt{r}$,
	then it must be that $a_1 = b_1$ and $a_2 = b_2$.
	\item Subtraction is closed since 
	\[	
		(a_1 + a_2\sqrt{r}) - (b_1 + b_2\sqrt{r}) 
		= (a_1-b_1) + (a_2-b_2)\sqrt{r} 
		\in \mathbb{Q}[\sqrt{r}]
	\]
	Multiplication is closed since 
	\[
		(a_1 + a_2\sqrt{r})(b_1 + b_2\sqrt{r})
		= (a_1b_1 + a_2b_2r) + (a_1b_2+a_2b_1)\sqrt{r}
		\in \mathbb{Q}[\sqrt{r}]
	\]
	The element $1 = 1+0\sqrt{r}$ is the multiplicative identity.

	The inverse of an element $a_1 + a_2\sqrt{r}$ is 
	$\frac{1}{a_1^2 + a_2^2r}(a_1 - a_2\sqrt{r})$ since 
	\[
		\frac{1}{a_1^2 + a_2^2r}(a_1 - a_2\sqrt{r})(a_1 + a_2\sqrt{r})
		= \frac{1}{a_1^2 + a_2^2r}(a_1^2 + a_2^2r) 
		= 1
	\]
	Therefore $\mathbb{Q}[\sqrt{r}]$ is a subfield.
\end{enumerate}
\newpage 

\section*{8.}
We can perform polynomial long division.
\[
	\polylongdiv{x^5 + 4x^4 + 2x^3 + 3x^2}{x^2 + 3}
\]
Therefore $q(x) = x^3+4x^2-x-9$ and $r(x) = 3x+27$.
$r(x) = 0$ when $n=1,3$.









\end{document}