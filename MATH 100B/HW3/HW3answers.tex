
\documentclass{article}
\usepackage{amsmath}
\usepackage{amssymb}
\usepackage{hyperref}
\usepackage{mathrsfs}
\usepackage{enumerate}
\usepackage{bm}
\usepackage{physics}
\usepackage{polynom}
\setlength{\parindent}{0pt}
\usepackage[parfill]{parskip}
\usepackage[margin=1in]{geometry}
\usepackage{tikz}


\begin{document}
\begin{center}
	\huge{\bf Math 100B: Homework 3} \\
	Merrick Qiu
\end{center}

\section*{Problem 1}
\begin{enumerate}[(a)]
	\item If $r,s \in R$ and $r^n = 0$ for some $n$
		then $(rs)^n = r^ns^n = 0(s^n) = 0$ so $rs \in N$.
		If we have $s^m = 0$ for some $m$ then 
		$(r+s)^{nm} = r^{nm} + Prs + s^{nm} = Prs \in N$
		since $rs \in N$
		($P$ is some polynomial from the middle terms of the binomial expansion).
		Thus $(Prs)^p = 0$ for some $p$ so $(r+s)^{nmp} = 0$ and $r+s \in N$.
		Therefore $N$ is an ideal since it is closed under addition and closed under multiplication
		with an arbitrary element from the ring.
	\item If $r \in R$ was a nonzero nilpotent element then $r+N = 0+N$ since $r \in N$ by definition.
	\item Let $r \in N$ arbitrary with $r^n = 0$. 
	Notice that $0 \in P$ so either $r \in P$ or $r^{n-1} \in P$.
	If $r^{n-1} \in P$ then either $r \in P$ or $r^{n-2} \in P$.
	Inducting over the exponent, we have that $r\in P$, but since $r$ was arbitrary,
	$N \subset R$.
\end{enumerate}

\newpage 

\section*{Problem 2}
\begin{enumerate}[(a)]
	\item If $f \in I_X$ and $g \in R$ then $f(a) = 0$ for all $a \in X$.
	Thus $(fg)(a) = 0$ for all $a$ so $fg \in I_X$.
	If $f,g \in I_X$ then $f(a) = 0$ and $g(a) = 0$ for all $a \in X$.
	Thus $(f+g)(a) = 0$ for all $a$ so $f+g \in I_X$.
	Therefore $R$ is a ring.
	The function 
	\[
		f(a) = \begin{cases}
			0 & a \in X \\
			1 & a \notin X
		\end{cases}
	\]
	is the generator of the principal ideal $R$ since any function $h \in R$
	can be written as the product of $f$ and some function $g \in R$ that matches $h$
	for all values not in $X$.
	\item $I_X$ is a maximal ideal when $\mathbb{R} - X$ only contains a single point
	since the only other ideal that contains it is the entire ring $R$.
	If $\mathbb{R} - X$ contains more than one point then $I_X \subset I_Y$ where $Y$
	is $X$ with an additional missing point added so $I_X$ is not maximal.

	$I_X$ is a prime ideal when $X$ contains less than two points. 
	If $X$ is the empty set then $I_X = R$ which is trivally prime.
	If $X$ contains a single point $a$ and
	if $fg \in I_X$ then $fg(a) = 0$ implies either $f(a) = 0$ or $g(a) = 0$ so $I_X$ is prime.
	If $X$ contains two or more points $a,b \in X$ then it is possible for $fg \in I_x$
	If $f(a) = 0$ and $g(b) = 0$ but $f(b) \neq 0$ and $g(a) \neq 0$.
\end{enumerate}
\newpage 

\section*{Problem 3}
\begin{enumerate}[(a)]
	\item If $r \in I \cap R$ and $s \in R$ then $rs \in I$ since $I$ is an ideal 
	and $rs \in R$ because $r \in R$ and$s \in R$.
	Thus $rs \in I \cap R$.
	If $r,s \in I \cap R$ then $r+s \in I$ and $r+s \in R$ since $I$ and $R$ are both subgroups under addition.
	Thus $r+s \in I \cap R$ and $I \cap R$ is an ideal of $R$.
	\item If $a,b \in R$ and $ab \in I \cap R$ then either $a \in I$ or $b \in I$ because $I$ is a prime ideal of $S$.
	However $a,b \in R$ so either $a \in I \cap R$ or $b \in I \cap R$, which means $I \cap R$ is a prime in $R$.
	\item No. If $I$ is a subring and $R=I$, then $I \cap R = R$ is definitionally is not maximal.
	For example if $I = R = \mathbb{R}$ and $S = \mathbb{C}$ then $\mathbb{R}$ is not a maximal ideal of $\mathbb{R}$.
\end{enumerate}

\newpage 

\section*{Problem 4}
Suppose that $I = (p)$ was a principal ideal generated by $p \in R$.
This means that $2 = pq$ for some $q \in R$, 
and so $p$ must be a constant.
It must also be that $x = pr$ for some $r \in R$ so $q$ must be a linear polynomial 
and $p = -1,1$ so that $p$ divides $1$(which is the coefficient of $x$).
However neither $-1$ or $1$ generate $I$ so it cannot be that $I$ is a principal ideal
and $R$ is not a principal ideal domain.
\newpage 

\section*{Problem 5}
\begin{enumerate}[(a)]
	\item $(\implies)$ Let $f,g \in F[x]$ and $(f) \subseteq (g)$.
	Since $f \in (f)$ it is also $f \in (g)$.
	Therefore $f = gh$ for some $h \in F[x]$.
	
	$(\impliedby)$ Let $f,g \in F[x]$ and $f = gh$ for some $h \in F[x]$.
	From the definition of the principal ideal,
	\begin{align*}
		(f) &= \{fr|\forall r \in F[x]\} \\
			&= \{ghr|\forall r \in F[x]\} \\
			&\subseteq \{gr|\forall r \in F[x]\} \\
			&= (g).
	\end{align*}
	\item The kernel of $\phi$ is 
	$\ker \phi = \{f(x) \in F[x] \colon f(a) = f(b) = 0\}$.
	Therefore $\ker \phi = (x-a)(x-b)$ since it 
	has roots $a,b$ and no polynomial of degree $\leq 1$
	has $a$ and $b$ as roots.

	The homomorphism $\phi$ is surjective since
	for arbitrary $(p,q) \in F \times F$,
	$\phi\left(\frac{(a-x)q + (x-b)p}{a-b}\right) = (p,q)$.

	By the first isomorphism theorem, $F[x]/(f) \cong F \times F$.

	\item Since $F$ is a field it only has the zero ideal and the unit ideal.
	So $F \times F$ only has the four ideals 
	$\{0\} \times \{0\}$, $\{0\} \times F$,  $F \times \{0\}$, $F \times F$.
	According to the correspondence theorem for quotient rings,
	the ideals of $F[x]/(g(x))$ are in correspondence with 
	 the ideals of $F[x]$ that contain $g(x)$,
	and since $F[x]$ is a principal ideal domain its only ideals 
	that contain $g(x)$ are $F[x]$, $((x-a)^2)$, and $(x-a)$.
	Therefore $F \times F$ is not isomorphic to $F[x]/(g(x))$
	since they have a different number of ideals.
\end{enumerate}
\newpage 

\section*{Problem 6}

If $R$ is a ring with finitely many elements such that 
every element of $R$ is idempotent, then 
$R$ is ismomorphic to $n$ copies of $\mathbb{Z}/2\mathbb{Z}$.
However since $A \cap A = A$ for all $A \in R$,
all $n$ elements of $R$ are idempotent.













\end{document}