
\documentclass{article}
\usepackage{amsmath}
\usepackage{amssymb}
\usepackage{hyperref}
\usepackage{mathrsfs}
\usepackage{enumerate}
\usepackage{bm}
\usepackage{physics}
\usepackage{polynom}
\setlength{\parindent}{0pt}
\usepackage[parfill]{parskip}
\usepackage[margin=1in]{geometry}
\usepackage{tikz}


\begin{document}
\begin{center}
	\huge{\bf Math 100B: Homework 5} \\
	Merrick Qiu
\end{center}

\section*{Problem 1}
Let $f, g \in R$ and we want to show $f = g q + r$ with $q, r \in R$
and $N(r) < N(g)$.
Since $R \subset \mathbb{C}$, we can write 
$\frac{f}{g} = q + \frac{r}{g}$ in $\mathbb{C}$.
If $N(\frac{r}{g}) < 1$, then $N(r) < N(g)$ due to the
multiplicativity of the norm.

The elements of $R$ form a rectangular grid of side length $1$ and 
$\sqrt{2}$ inside of $\mathbb{C}$.
Therefore for any $\frac{f}{g} \in \mathbb{C}$, we can find a $q \in R$
that is at most $\frac{\sqrt{3}}{2} < 1$ away from $\frac{f}{g}$.
Therefore $N(\frac{f}{g} - q) = N(\frac{r}{g}) < 1$ in the equation
$\frac{f}{g} = q + \frac{r}{g}$.
Multiplying by $g$ on both sides, we get $f = gq + r$ with $N(r) < N(g)$.

Since $R$ is a Euclidean domain, it is also a PID and a UFD.
\newpage 

\section*{Problem 2}
\begin{enumerate}[(a)]
	\item If $2 = xy$ then $N(2) = N(x)N(y)$.
		$N(2) = 4$ so  $N(x) = N(y) = 2$ if $2$ was reducible.
		However there does not exist any $x = a+b\sqrt{2}$ such that $a^2 - db^2 = 2$ when $d \leq 3$
		so $2$ is irreducible.
	\item If  $d =2n$ is even then $2n = d = \sqrt{d}\sqrt{d}$ but $2$ does not divide $\sqrt{d}$ so $2$ is not prime.
		If $d = 2n+1$ is odd then $-2n = 1-d =(1+\sqrt{d})(1-\sqrt{d})$ but $2$ does not divide $1+\sqrt{d}$ or $1-\sqrt{d}$ so $2$ is not prime.
		Therefore $R$ is not a UFD.
\end{enumerate}
\newpage 

\section*{Problem 3}
\begin{enumerate}[(a)]
	\item If $p = a^2+2b^2$, then it can be written as the product $p = (a+b\sqrt{-2})(a-b\sqrt{2})$.
		Since $N(a+b\sqrt{-2}) = N(a-b\sqrt{-2}) = p$, both these elements are irreducible.

		If $p$ cannot be written as $p = a^2+2b^2$ then $p$ is irreducible in $R$.
		If it was reducible, then $p = xy$ and $N(x) = N(y) = p$.
		However if $x = a+b\sqrt{-2}$ then $N(x) = a^2+2b^2$ which is a contradiction.

	\item We can write $2 = 0^2 + 2(1)^2$ so $2$ falls into case (ii).
	For the case when $p \equiv 5 \mod 8$ or $p \equiv 7 \mod 8$,
	notice that $a^2 = 0,1,4\mod 8$ and $2b^2 = 0,2\mod 8$
	so $a^2 + 2b^2 = 0,1,3,4,6 \mod 8$. so it is not possible to write 
	$p = a^2 + 2b^2$.
		
\end{enumerate}
\newpage 

\section*{Problem 4}
We can write 
\begin{align*}
	1122 &= (2)(3)(11)(17) \\
	&=	\left[(0 + \sqrt{-2})(0 - \sqrt{-2})\right]\left[(1 + \sqrt{-2})(1 - \sqrt{-2})\right]\left[(3 + \sqrt{-2})(3 - \sqrt{-2})\right]\left[(3 + 2\sqrt{-2})(3 - 2\sqrt{-2})\right]
\end{align*}

Therefore we can write $1122 = \gamma\overline{\gamma}$ where gamma is the product with four of the factors selected above in the following ways.
Due to symmetry we can choose $(0 + \sqrt{-2})$ as our first factor.
\begin{align*}
	\gamma &= (0 + \sqrt{-2})(1 + \sqrt{-2})(3 + \sqrt{-2})(3 + 2\sqrt{-2}) =  -28 - 13\sqrt{-2}\\
	\gamma &= (0 + \sqrt{-2})(1 + \sqrt{-2})(3 + \sqrt{-2})(3 - 2\sqrt{-2}) =  -20 + 19\sqrt{-2}\\
	\gamma &= (0 + \sqrt{-2})(1 + \sqrt{-2})(3 - \sqrt{-2})(3 + 2\sqrt{-2}) =  -32 + 7\sqrt{-2}\\
	\gamma &= (0 + \sqrt{-2})(1 + \sqrt{-2})(3 - \sqrt{-2})(3 - 2\sqrt{-2}) =  8 + 23\sqrt{-2}\\
	\gamma &= (0 + \sqrt{-2})(1 - \sqrt{-2})(3 + \sqrt{-2})(3 + 2\sqrt{-2}) =  -8 + 23\sqrt{-2}\\
	\gamma &= (0 + \sqrt{-2})(1 - \sqrt{-2})(3 + \sqrt{-2})(3 - 2\sqrt{-2}) =  32 + 7\sqrt{-2}\\
	\gamma &= (0 + \sqrt{-2})(1 - \sqrt{-2})(3 - \sqrt{-2})(3 + 2\sqrt{-2}) =  20 + 19\sqrt{-2}\\
	\gamma &= (0 + \sqrt{-2})(1 - \sqrt{-2})(3 - \sqrt{-2})(3 - 2\sqrt{-2}) =  28 - 13\sqrt{-2}\\
\end{align*}

\begin{align*}
	1122 &= (-28 - 13\sqrt{-2})(-28 + 13\sqrt{-2}) = 28^2 + 2\cdot 13^2 \\
	1122 &= (-20 + 19\sqrt{-2})(-20 - 19\sqrt{-2}) = 20^2 + 2\cdot 19^2 \\
	1122 &= (-32 + 7\sqrt{-2})(-32 - 7\sqrt{-2}) = 32^2 + 2\cdot 7^2 \\
	1122 &= (8 + 23\sqrt{-2})(8 - 23\sqrt{-2}) = 8^2 + 2\cdot 23^2 \\
\end{align*}

These are the only ways to write $1122$ as $a^2 + 2b^2$ since $R$ is a UFD and 
the existence of a another way would imply a different factorization of $1122$.
\newpage 

\section*{Problem 5}
\begin{enumerate}[(a)]
	\item First we show that $\mathbb{Z}[\sqrt{-2}]/(p)$ is a field.
	The evaluation homomorphism $\phi: \mathbb{Z}[x] \to \mathbb{Z}[\sqrt{-2}]$ that sends $x \to \sqrt{-2}$
	has $(x^2+2) \subseteq \ker \phi$. 
	To show the converse containment, notice that when $f \in \ker \phi$, $f = (x^2+2)q + r$ with $\deg r < x^2+2$.
	But there exists no $r \in \ker \phi$ with $\deg r < x^2+2$ so $r = 0$ and $(x^2+2) = \ker \phi$. 
	By the first isomorphism theorem, $\mathbb{Z}[\sqrt{-2}] \cong \mathbb{Z}[x]/(x^2+2)$.
	Since $(p)$ is irreducible in the Euclidean domain $\mathbb{Z}[\sqrt{-2}]/(p)$, it is maximal.
	By the correspondence theorem, $\mathbb{Z}[\sqrt{-2}]/(p)$ can only have two ideals so it is a field.

	We can write 
	\begin{align*}
		\mathbb{Z}[\sqrt{-2}]/(p) &\cong \frac{\mathbb{Z}[x]/(x^2+2)}{(p, x^2+2)/(x^2+2)} \\
		&\cong \mathbb{Z}[x]/(p, x^2+2) \\
		&\cong \frac{(\mathbb{Z}/p\mathbb{Z})[x]}{(x^2+2)}
	\end{align*}
	Elements in $\frac{(\mathbb{Z}/p\mathbb{Z})[x]}{(x^2+2)}$ are degree $1$ polynomials in $\mathbb{Z}/p\mathbb{Z}$, so there are $p^2$ elements in the field.
	\item Since $p = a^2 + 2b^2 = (a+b\sqrt{-2})(a-b\sqrt{-2})$, 
	$b$ is invertible modulo $p$ since $b$ is nonzero(if $b$ is zero, then $p=a^2$ which is a contradiction since $p$ is prime).
	Solving for $-2$ in the equation $a^2  \equiv - 2b^2 \mod p$ gives us that
	that $\left(\frac{a}{b}\right)^{2} = -2$.
	Therefore $x^2 + 2 = (x - \frac{a}{b})(x + \frac{a}{b})$ and so by the chinese remainder theorem and then evaluating at $x = \pm \frac{a}{b}$, we get
	\begin{align*}
		\mathbb{Z}[\sqrt{-2}]/(p) &\cong  \frac{(\mathbb{Z}/p\mathbb{Z})[x]}{(x^2+2)} \\
		&\cong \frac{(\mathbb{Z}/p\mathbb{Z})[x]}{(x - \frac{a}{b})} \times \frac{(\mathbb{Z}/p\mathbb{Z})[x]}{(x + \frac{a}{b})} \\
		&\cong \mathbb{Z}/p\mathbb{Z} \times \mathbb{Z}/p\mathbb{Z} \\
	\end{align*}
	At the same time,
	\[
		\mathbb{Z}[\sqrt{-2}]/(p) \cong \mathbb{Z}[\sqrt{-2}]/(a-b\sqrt{-2}) \times \mathbb{Z}[\sqrt{-2}]/(a+b\sqrt{-2}) \\
	\]
	Therefore by matching the rings in the ring product we get that
	\[
		\mathbb{Z}[\sqrt{-2}]/(a+b\sqrt{-2}) \cong \mathbb{Z}/p\mathbb{Z}.
	\]

\end{enumerate}
\newpage






\end{document}