
\documentclass{article}
\usepackage{amsmath}
\usepackage{amssymb}
\usepackage{hyperref}
\usepackage{mathrsfs}
\usepackage{enumerate}
\usepackage{bm}
\usepackage{physics}
\usepackage{polynom}
\setlength{\parindent}{0pt}
\usepackage[parfill]{parskip}
\usepackage[margin=1in]{geometry}
\usepackage{tikz}


\begin{document}
\begin{center}
	\huge{\bf Math 100B: Homework 4} \\
	Merrick Qiu
\end{center}

\section*{Problem 1}
We can reproduce Euclids proof that $\mathbb{Z}$ has infinitely many primes to show that 
$F[x]$ has infinitely many monic irreducible polynomials.
Suppose that there were only a finite number of monic irreducible polynomials in $F[x]$, say $p_1, p_2, \ldots, p_n$. 
Then consider the polynomial $p_{n+1} = p_1p_2\cdots p_n + 1$ which is also a monic polynomial.
Since $p_{n+1}$ is not divisible by any of the $p_i$, it must be irreducible.

Therefore the monic irreducible polynomials are not given by $p_1, p_2, \ldots, p_n$ which is a contradiction.
Therefore there must be a infinite number of monic irreducible polynomials.
Since the maximal ideals of $F[x]$ are the principal ideals generated by the monic irreducible polynomials,
there are also an infinite number of maximal ideals.
\newpage 

\section*{Problem 2}
\begin{enumerate}[(a)]
	\item This is simply proving Fermats Little theorem, which we can do by looking at the multiplicative group $\mathbb{F}_p^\times$.
	For any element $a \in \mathbb{F}_p^\times$, the order give by $k$ must divide $p-1$ by Lagrange's theorem.
	If $p-1 = kn$ for some $n$ then $a^{p-1}-1 = a^{kn}-1 = (a^k)^n -1 = 1-1 = 0$.
	Therefore all elements in $\mathbb{F}_p^\times$ are roots of $f(x)$.
	\item Since every nonzero element of $\mathbb{F}_p$ is a root of $f(x)$,
	we can write $f(x) = (x-1)(x-2)\ldots(x-(p-1))g(x)$ for some polynomial $g(x)$.
	However in order for the leading coefficients and degrees of the left and right hand side to match,
	$g(x) = 1$ so we can simply write $f(x) = (x-1)(x-2)\ldots(x-(p-1))$.
	\item The constant term of $f(x)$ is $-1$ and the constant term of the right hand side is 
	the product $(p-1)!$ modulo p so it must be that $(p-1)! \equiv -1 \mod p$. 
\end{enumerate}
\newpage 

\section*{Problem 3}
\begin{enumerate}
	\item Let $f(x) \in R$ with leading coefficient $a$ and $g(x)$ has leading coefficient $b$.
	If we set $f(x) - \frac{a}{b}g(x) = r(x)$ then $\deg r < \deg g$ and $f(x) = r(x) + \frac{a}{b}g(x)$.
	This representation is unique since if $f(x) = r(x) + \frac{a}{b}g(x) = s(x) + cg(x)$ with $\deg s < \deg g$ and $c \neq \frac{a}{b}$,
	then that would imply that $r(x) - s(x) = (c-\frac{a}{b})g(x)$, 
	but this is a contradiction since the degrees of the left and right hand side do not match.
	\item Each element of $R$ corresponds to a coset $r(x) + (g(x))$ where $r(x)$ has degree $n-1$.
	Since $r(x)$ has $n$ different coefficients and each of these coefficients can take on $p$ different values,
	there are in total $p^n$ different cosets in $\mathbb{F}_p[x]/(g(x))$.
\end{enumerate}
\newpage 

\section*{Problem 4}
\begin{enumerate}
	\item By the previous problem we have that $E$ has a total of $3^2 = 9$ elements. 
	It is a field because it is the quotient of a polynomial ring by an irreducible polynomial.
	\item $E^\times$ is cyclic since $x+1$ generates it.$(x+1)^2 = 2x$, $(x+1)^4 = 2$, and $(x+1)^8 = 1$.
	\item $\mathbb{F}_3[x]/(x^3+1)$ is a field with 27 elements.
\end{enumerate}
\newpage 

\section*{Problem 5}
\begin{enumerate}
	\item The units of $R$ are the constants $a_0 \in F$ with degree $0$.
	If $x^2$ are reducible then it can be written as the product of two degree $1$ polynomials,
	but since $R$ contains no degree $1$ polynomials $x^2$ must be irreducible.
	Likewise $x^3$ must be written as the product of a degree $1$ polynomial and 
	a degree $2$ polynomial, but $R$ contains no degree $1$ polynomials.
	\item We can factor $x^6 = x^2\cdot x^2\cdot x^2 = x^3\cdot x^3$ 
	in two ways into irreducible elements, so $R$ is not a unique factorization domain.
\end{enumerate}
\newpage 

\section*{Problem 6}
\begin{enumerate}
	\item $RS^{-1}$ is closed under subtraction since for $\frac{r_1}{s_1}, \frac{r_2}{s_2} \in RS^{-1}$
	\[
		\frac{r_1}{s_1} - \frac{r_2}{s_2} = \frac{r_1s_2 - r_2s_1}{s_1s_2} \in RS^{-1}
	\]
	since $r_1s_2 - r_2s_1 \in R$ and $s_1s_2 = S^{-1}$.

	$RS^{-1}$ is closed under subtraction since for $\frac{r_1}{s_1}, \frac{r_2}{s_2} \in RS^{-1}$
	\[
		\frac{r_1}{s_1}\cdot\frac{r_2}{s_2} = \frac{r_1r_2}{s_1s_2} \in RS^{-1}
	\]
	since $r_1r_2 \in R$ and $s_1s_2 = S^{-1}$.
	Since $\frac{1}{1} \in RS^{-1}$ as well, $RS^{-1}$ is a subring of $F$.
	\item Define $\hat{\phi}\left(\frac{r}{s}\right) = \phi(r)\phi(s)^{-1}$.
	This is well defined since $\phi(s)$ is a unit and 
	it sends two equivalent fractions to the same element.
	If $\frac{r}{s} = \frac{a}{b}$ then $rb = as$ which can be written as $rs^{-1} = ab^{-1}$ and 
	\[
		\hat{\phi}\left(\frac{r}{s}\right) 
		= \phi(r)\phi(s)^{-1} 
		= \phi(rs^{-1})
		= \phi(ab^{-1})
		= \phi(a)\phi(b)^{-1} 
		= \hat{\phi}\left(\frac{a}{b}\right).
	\]
	It is a homomorphism since 
	\[
		\hat{\phi}\left(\frac{1}{1}\right) = 1 
	\]
	\begin{align*}
		\hat{\phi}\left(\frac{r}{s} + \frac{a}{b}\right) &= \hat{\phi}\left(\frac{rb + as}{sb}\right) \\
		&= \phi(rb+as)\phi(sb)^{-1} \\
		&= \phi((rb+as)(sb)^{-1}) \\
		&= \phi(rs^{-1}+ab^{-1}) \\
		&= \phi(r)\phi(s)^{-1} + \phi(a)\phi(b)^{-1} \\
		&=\hat{\phi}\left(\frac{r}{s}\right) + \hat{\phi}\left(\frac{a}{b}\right) \\
	\end{align*}
	\begin{align*}
		\hat{\phi}\left(\frac{r}{s}\cdot\frac{a}{b}\right) &= \hat{\phi}\left(\frac{ra}{sb}\right) \\
		&= \phi(ra)\phi(sb)^{-1} \\
		&= \phi(r)\phi(s)^{-1}\phi(a)\phi(b)^{-1} \\
		&=\hat{\phi}\left(\frac{r}{s}\right)\hat{\phi}\left(\frac{a}{b}\right) \\
	\end{align*}
	It is unique since another homomorphism with these properties $\varphi$ would have 
	\begin{align*}
		\varphi(\frac{r}{s})\varphi(s) &= \varphi(\frac{r}{s}\frac{s}{1}) \\
		&= \varphi(\frac{r}{1}) \\
		&= \phi(r) \\
	\end{align*}
	Also $\varphi(s) = \phi(s)$ so 
	\[
		\varphi(\frac{r}{s}) = \phi(r) \phi(s)^{-1} = \hat{\phi}(\frac{r}{s})
	\]
\end{enumerate}
\newpage

\section*{Problem 7}
Since $\phi(\frac{a}{p^k}) = \phi(a)\phi(p)^{-k}$ for $\frac{a}{p^k} \in RS^{-1}$, 
each homomorphism is uniquely determined by where it sends $a$ and $p$.
Since $a$ is an integer and $\phi(1) = 1$, $\phi(a) = \phi(1+\ldots+1) = \phi(1)+\ldots+\phi(1) = a$.
Since $p$ must be sent to a unit of $\mathbb{Z}/n\mathbb{Z}$,
and there are $\varphi(n)$ units in $\mathbb{Z}/n\mathbb{Z}$,
there are $\varphi(n)$ different homomorphisms,
where $\varphi(n)$ is eulers totient function.














\end{document}