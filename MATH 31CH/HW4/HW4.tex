\documentclass[12pt]{article}
\usepackage[utf8]{inputenc}
\usepackage[english]{babel}
\usepackage[letterpaper,top=2cm,bottom=2cm,left=3cm,right=3cm,marginparwidth=1.75cm]{geometry}
\usepackage{amsmath}
\usepackage{tikz}
\usepackage{graphicx}
\usepackage{bigints}
\usepackage{amssymb}
\usepackage[colorlinks=true, allcolors=blue]{hyperref}

\newcommand{\vt}[1]{\textbf{#1}}
\newcommand{\pder}[3][]{\frac{\partial^{#1} #2}{\partial #3^{#1}}}

\title{Math 31CH HW4}
\author{Merrick Qiu}

\begin{document}
\maketitle
\newpage



\subsection*{Exercise 5.3.1}
\textbf{Part a.}
Let $(r(t),\theta(t))$ be a parametrization of a curve in polar coordinates. Show that the length of the piece of curve between $t = a$ and $t = b$ is given by the integral $\int_{a}^b\sqrt{(r'(t))^2+(r(t))^2(\theta'(t))^2}dt$.
\medskip

\textbf{Solution.}
The derivative of the coordinates are
\[
\gamma'(t) = 
  \begin{bmatrix}
    r'(t)\cos(\theta(t)) - r(t)\theta'(t)\sin(\theta(t)) \\
    r'(t)\sin(\theta(t)) + r(t)\theta'(t)\cos(\theta(t))
  \end{bmatrix}
\]
The squared sum of the x and y coordinates are
\begin{align*}
  \gamma'(t)_x^2 + \gamma'(t)_y^2 
  &= r'(t)^2\cos^2(\theta(t)) 
    + r(t)^2\theta'(t)^2\sin^2(\theta(t))
    + r'(t)^2\sin^2(\theta(t)) 
    + r'(t)^2\theta'(t)^2\cos^2(\theta(t)) \\
  &= r'(t)^2(\cos^2(\theta(t)) + \sin^2(\theta(t)))
    + r(t)^2\theta'(t)^2(\cos^2(\theta(t)) + \sin^2(\theta(t))) \\
  &= r'(t)^2 + r(t)^2\theta'(t)^2
\end{align*}
Therefore the path length is 
\[
  \int_{a}^b |\gamma'(t)| \,dt =
  \int_{a}^b \sqrt{\gamma'(t)_x^2 + \gamma'(t)_y^2} \,dt =
  \int_{a}^b \sqrt{(r'(t))^2+(r(t))^2(\theta'(t))^2} \,dt
\]


\bigskip

\noindent \textbf{Part b.}
Consider the spiral in polar coordinates when $r(t) = e^{-\alpha t}$ and $\theta(t) = t$, for $\alpha >0$. What is its length between $t = 0$ and $t = b$? What is the limit of this length as $\alpha\rightarrow 0$.
\medskip

\textbf{Solution.}
  The path length is
  \begin{align*}
    \int_{0}^b \sqrt{(r'(t))^2+(r(t))^2(\theta'(t))^2} \,dt
    &= \int_{0}^b \sqrt{\alpha^2e^{-2\alpha t}+e^{-2\alpha t}} \,dt \\
    &= \int_{0}^b \sqrt{\alpha^2+1} e^{-\alpha t} \,dt \\
    % &= \left[-\frac{\sqrt{\alpha^2+1}}{\alpha} e^{-\alpha t}\right]_0^b \\
    &= -\frac{\sqrt{\alpha^2+1}}{\alpha} (e^{-\alpha b} - 1)
  \end{align*}

  The limit of of the path length as $\alpha\rightarrow 0$ is 
  \begin{align*}
    \lim_{\alpha \to 0} -\frac{\sqrt{\alpha^2+1}}{\alpha} (e^{-\alpha b} - 1)
    &= \lim_{\alpha \to 0} -\frac{\sqrt{\alpha^2+1}}{\alpha} ((1-b\alpha+\frac{b^2\alpha^2}{2} \hdots) - 1) \\
    &= \lim_{\alpha \to 0} -\sqrt{\alpha^2+1} (b(-1 + \frac{b\alpha}{2} \hdots)) \\
    &= b
  \end{align*}
\bigskip

\noindent \textbf{Part c.}
Show that the spiral turns infinitely many times around the origin as $t\rightarrow\infty$. Does the length tend to $\infty$ as $b\rightarrow \infty$?
\medskip

\textbf{Solution.}
$\theta$ grows without bound as $t$ increases, so it turns infinitely many times.
\[
  \lim_{t \to \infty} \theta(t) 
  = \lim_{t \to \infty} t
  = \infty
\]
The length does not tend towards $\infty$ since 
\[
  \lim_{b \to \infty} -\frac{\sqrt{\alpha^2+1}}{\alpha} (e^{-\alpha b} - 1)
  = \frac{\sqrt{\alpha^2+1}}{\alpha}
\]
\newpage









\subsection*{Exercise 5.3.3}
\textbf{Part a.}
Suppose that $t\mapsto (r(t),\theta(t),\phi(t))$ is a parametrization of a curve in  $\mathbb R^3$, written in spherical coordinates. Find the formula analogous to the integral in Exercise 5.3.1, part a, for the length of the arc between $t=a$ and $t = b$.
\medskip

\textbf{Solution.}
The coordinates are given by 
\[
  \gamma(t) =
  \begin{bmatrix}
    r(t) \cos(\theta(t)) \cos(\varphi(t)) \\
    r(t) \sin(\theta(t)) \cos(\varphi(t)) \\
    r(t) \sin(\varphi(t))
  \end{bmatrix}
\]
The derivative is 
\[
  \gamma(t) =
  \begin{bmatrix}
    r'(t) \cos(\theta(t)) \cos(\varphi(t)) 
      - r(t) \theta'(t) \sin(\theta(t)) \cos(\varphi(t)) 
      - r(t) \varphi'(t) \cos(\theta(t)) \sin(\varphi(t)) \\
    r'(t) \sin(\theta(t)) \cos(\varphi(t))  
      + r(t) \theta'(t) \cos(\theta(t)) \cos(\varphi(t)) 
      - r(t) \varphi'(t) \sin(\theta(t)) \sin(\varphi(t)) \\
    r'(t) \sin(\varphi(t)) + r(t)\varphi'(t) \cos(\varphi(t))
  \end{bmatrix}
\]
After a lot of ugly algebra to calculate the norm,
\[
  \int_{a}^b |\gamma'(t)| \,dt =
  \int_{a}^b \sqrt{\gamma'(t)_x^2 + \gamma'(t)_y^2 + \gamma'(t)_z^2} \,dt =
  \int_{a}^b \sqrt{r'(t)^2 + r^2\phi'(t)^2 + r^2\cos^2(\varphi(t))\theta'(t)^2} \,dt
\]
\bigskip

\noindent \textbf{Part b.}
What is the length of the curve parametrized by $r(t) = cos\ t,\ \theta(t) = tan\ t,\ \phi(t) =  t$, between $t=0$ and $t = a$, where $0<a<\pi/2$?
\medskip

\textbf{Solution.}
\begin{align*}
  \int_{0}^a \sqrt{r'(t)^2 + r^2\phi'(t)^2 + r^2\cos^2(\varphi(t))\theta'(t)^2} \,dt
  &= \int_{0}^a \sqrt{\sin^2(t) + \cos^2(t) + \cos^2(t) \cos^2(t)\sec^4(t)} \,dt \\
  &= \int_{0}^a \sqrt{2} \,dt \\
  &= \sqrt{2} a
\end{align*}
\newpage








\subsection*{Exercise 5.3.5}
\textbf{Part a.}
Set up (but do not compute) the integral giving the surface area of the part of the surface of equation $z= \frac{x^2}{4}+\frac{y^2}{9}$, where $z\leq a^2$.
\medskip

\textbf{Solution.}
The manifold can be parameterized by 
\[
  \gamma(r, \theta) = 
  \begin{bmatrix}
    2r\cos \theta \\
    3r\sin \theta \\
    r^2
  \end{bmatrix}
\]
The Jacobian is 
\[
  D\gamma = 
  \begin{bmatrix}
    \pder{x}{r} & \pder{x}{\theta} \\
    \pder{y}{r} & \pder{y}{\theta} \\
    \pder{z}{r} & \pder{z}{\theta}
  \end{bmatrix} =
  \begin{bmatrix}
    2\cos \theta & -2r\sin \theta \\
    3\sin \theta & 3r\cos \theta \\
    2r & 0
  \end{bmatrix} 
\]
Multiplying by the transpose yields 
\[
  D\gamma^TD\gamma = 
  \begin{bmatrix}
    2\cos \theta & 3\sin\theta & 2r \\
    -2r\sin\theta & 3r\cos\theta & 0
  \end{bmatrix} 
  \begin{bmatrix}
    2\cos \theta & -2r\sin \theta \\
    3\sin \theta & 3r\cos \theta \\
    2r & 0
  \end{bmatrix} =
  \begin{bmatrix}
    4 + 5\sin^2 \theta + 4r^2 & 5 r\sin\theta\cos\theta \\
    5 r\sin\theta\cos\theta & 4r^2 + 5r^2\cos^2 \theta 
  \end{bmatrix}
\]
The volume of the transformation is
\[
  \sqrt{\det[D\gamma^TD\gamma]}
  = \sqrt{(4 + 5\sin^2 \theta + 4r^2)(4r^2 + 5r^2\cos^2 \theta) - (5 r\sin\theta\cos\theta)^2}
  = 2r\sqrt{9 + r^2(4 + 5\cos^2 \theta)}
\]
The integral is therefore, 
\[
  \int_0^{2\pi} \int_0^a 2r\sqrt{9 + r^2(4 + 5\cos^2 \theta)} \,dr \,d\theta
\]
\bigskip

\noindent \textbf{Part b.}
What is the volume of the region $\frac{x^2}{4}+\frac{y^2}{9}\leq z\leq a^2$?
\medskip

\textbf{Solution.}
We can use cylindrical coordinates to calculate the volume.
\[
  6\int_0^{2\pi} \int_0^a \int_{r^2}^{a^2} r \,dz \,dr \,d\theta
  = 6\int_0^{2\pi} \int_0^a  a^2r-r^3 \,dr \,d\theta
  = 6\int_0^{2\pi} \frac{a^4}{2} - \frac{a^4}{4} \,d\theta
  = 3\pi a^4
\]
I multply by 6 since the
\[
  \det \begin{bmatrix}
    2\cos \theta & -2r\sin \theta \\
    2\sin \theta & 3r\cos \theta
  \end{bmatrix} = 6
\]
\newpage








\subsection*{Exercise 5.3.6}
Let $S$ be the part of the paraboloid of revolution $z= x^2+y^2$ where $z\leq 9$. Compute the integral $\int_S(x^2+y^2+3z^2)|d^2x|$.

\medskip

\textbf{Solution.}
Lets use the transformation 
\[
  \gamma(r,\theta) = 
  \begin{bmatrix}
    r\cos \theta \\
    r\sin \theta \\ 
    r^2
  \end{bmatrix}
\]
The Jacobian of the transformation is 
\[
  D\gamma = 
  \begin{bmatrix}
    \cos \theta &  -r\sin \theta\\
    \sin\theta & r\cos \theta \\
    2r & 0
  \end{bmatrix}
\]
The determinant of $D\gamma^T D\gamma$ is 
\begin{align*}
   \det D\gamma^T D\gamma 
   &= \det \begin{bmatrix}
    \cos \theta & \sin \theta & 2r \\
    -r\sin \theta & r\cos \theta & 0
  \end{bmatrix}
  \begin{bmatrix}
    \cos \theta &  -r\sin \theta\\
    \sin\theta & r\cos \theta \\
    2r & 0
  \end{bmatrix} \\
  &= \det \begin{bmatrix}
    1+4r^2 & 0 \\
    0 & r^2
  \end{bmatrix} \\ 
  &= r^2 + 4r^4
\end{align*}
Using the substitution $t = \sqrt{1+4r^2}$
\begin{align*}
  \int_S(x^2+y^2+3z^2)|d^2x|
  &= \int_0^{2\pi} \int_0^{3} (r^3+3r^5)\sqrt{1 + 4r^2} \,dr \,d\theta \\
  &= 2\pi \int_0^{3} r^3\sqrt{1 + 4r^2} \,dr
    + 2\pi \int_0^{3} 3r^5\sqrt{1 + 4r^2} \,dr \\
  &= \frac{\pi}{8} \int_1^{\sqrt{37}} t^4-t^2 \,dt
    + \frac{3\pi}{32} \int_1^{\sqrt{37}} (t^2-1)(t^4-t^2) \,dt \\
  &= \frac{7+277574\sqrt{37}}{420}\pi \\
  &\approx 4020 \pi
\end{align*}

\newpage




\subsection*{Exercise 5.3.8}
What is the surface area of the part of the paraboloid of revolution  $z = x^2 + y^2$ where $z\leq 1$?

\medskip

\textbf{Solution.}
We can repurpose the integral from the previous problem 
\begin{align*}
  \int_S|d^2x|
  &= \int_0^{2\pi} \int_0^{1} r\sqrt{1 + 4r^2} \,dr \,d\theta \\
  &= 2\pi \left[ \frac{1}{12} (4r^2+1)^{\frac{3}{2}}\right]_0^1 \\
  &= 2\pi \frac{5\sqrt{5}-1}{12} \\
  &= \frac{5\sqrt{5}-1}{6} \pi
\end{align*}
\newpage






\subsection*{Exercise 5.3.13(a)}
Let $S^2$ be the unit sphere and let $S_1$ be the part of the cylinder of equation $x^2+y^2 = 1$ with $-1\leq z\leq 1$. Show that the horizontal radial projection $S_1\rightarrow S^2$ preserves area.

\medskip

\textbf{Solution.}
The unit sphere is given by the equation $x^2 + y^2 + z^2 = 1$.
The radial projection using cylindrical coordinates is 
\[
  \gamma(z, \theta) = 
  \begin{bmatrix}
    \sqrt{1-z^2} \cos \theta \\
    \sqrt{1-z^2} \sin \theta\\
    z
  \end{bmatrix}
\]
The Jacobian is 
\[
  D\gamma = 
  \begin{bmatrix}
    -\frac{z}{\sqrt{1-z^2}} \cos \theta & -\sqrt{1-z^2} \sin \theta \\
    -\frac{z}{\sqrt{1-z^2}} \sin \theta & \sqrt{1-z^2} \cos \theta \\
    1 & 0
  \end{bmatrix}
\]
The determinant of the Jacobian times its transpose is 
\begin{align*}
  \det D\gamma^T D\gamma 
  &= \det \begin{bmatrix}
    -\frac{z}{\sqrt{1-z^2}} \cos \theta & -\frac{z}{\sqrt{1-z^2}} \sin \theta & 1 \\
    -\sqrt{1-z^2} \sin \theta & \sqrt{1-z^2} \cos \theta & 0
  \end{bmatrix}
  \begin{bmatrix}
    -\frac{z}{\sqrt{1-z^2}} \cos \theta & -\sqrt{1-z^2} \sin \theta \\
    -\frac{z}{\sqrt{1-z^2}} \sin \theta & \sqrt{1-z^2} \cos \theta \\
    1 & 0
  \end{bmatrix} \\
  &= \det \begin{bmatrix}
    \frac{z^2}{1-z^2} + 1 & 0 \\
    0 & 1-z^2
  \end{bmatrix} \\
  &= 1
\end{align*}
Since $\sqrt{\det D\gamma^T D\gamma} = 1$,
the horizontal radial projection preserves area.
\newpage





\subsection*{Exercise 5.3.15}
\textbf{Part a.}
Show that when $\phi,\psi,\theta$ satisfy
$$
-\pi/2\leq \phi\leq \pi/2,\quad -\pi/2\leq \psi\leq \pi/2,\quad 0\leq \theta<2\pi
$$
the map $\gamma(\theta,\phi,\psi) = (\cos \psi \cos \phi \cos \theta, \cos \psi \cos \phi \sin \theta, \cos \psi \sin\phi,\sin \psi)$ parametrizes the unit sphere $S^3$ in $\mathbb R^4$.
\medskip

\textbf{Solution.}
  For the unit sphere, $x^2 + y^2 + z^2 + w^2 = 1$.
  Plugging the values in to show that the image of $\gamma$ is on the sphere, 
  \begin{align*}
    x^2 + y^2 + z^2 + w^2
    &= (\cos^2 \psi \cos^2 \phi \cos^2 \theta + \cos^2 \psi \cos^2 \phi \sin^2 \theta) + \cos^2 \psi \sin^2 \phi + \sin^2 \psi \\
    &= (\cos^2 \psi \cos^2 \phi  + \cos^2 \psi \sin^2 \phi) + \sin^2 \psi \\
    &= (\cos^2 \psi  + \sin^2 \psi) \\
    &= 1
  \end{align*}
   $w \in [-1, 1]$ since $-\pi/2\leq \psi\leq \pi/2$,
  so all $w$ coordinates of the sphere are covered.
  For a given $w$, the equation traces out 
  the entire $S^2$ sphere of radius $\cos \psi$.
  \[
    x^2 + y^2 + z^2 + \sin^2 \psi = 1 
    \implies x^2 + y^2 + z^2 = \cos^2 \psi
  \]
  Thus all points on $S^3$ are mapped onto.
  
\bigskip

\noindent \textbf{Part b.}
Use this parametrization to compute $vol_3(S^3)$.
\medskip

\textbf{Solution.}
The Jacobian of the transformation is 
\[
  D\gamma =
  \begin{bmatrix}
    -\cos \psi \cos \phi \sin \theta & -\cos \psi \sin \phi \cos \theta & -\sin \psi \cos \phi \cos \theta\\
    \cos \psi \cos \phi \cos \theta & -\cos \psi \sin \phi \sin \theta & -\sin \psi \cos \phi \sin \theta\\
    0 & \cos \psi \cos \phi & -\sin \psi \sin \phi\\
    0 & 0 & \cos \psi
  \end{bmatrix}
\]
The determinant of the Jacobian times its transform is 
\[
  \det D\gamma^T D\gamma =
  \det \begin{bmatrix}
    \cos^2 \psi \cos^2 \phi & 0 & 0 \\
    0 & \cos^2 \psi & 0 \\
    0 & 0 & 1
  \end{bmatrix} = 
  \cos^4 \psi \cos^2 \phi
\]
The volume is thus 
\begin{align*}
  \int_{0}^{2\pi}
  \int_{-\frac{\pi}{2}}^{\frac{\pi}{2}} 
  \int_{-\frac{\pi}{2}}^{\frac{\pi}{2}}
  \cos^2 \psi \cos \phi
  \,d\phi \,d\psi \,d\theta
  &= 2\pi
  \int_{-\frac{\pi}{2}}^{\frac{\pi}{2}} 
  2\cos^2 \psi \,d\psi \\
  &= \pi\left[(\sin(2x) + 2x)\right]_{-\frac{\pi}{2}}^{\frac{\pi}{2}} \\
  &= 2\pi^2
\end{align*}
  
\newpage





\subsection*{Exercise 6.1.3}
Compute the following numbers:
$$a.\ dx_1\wedge dx_4
        \begin{pmatrix}
          \begin{bmatrix}
           1 \\
           0 \\
           1 \\
           2
          \end{bmatrix},
          \begin{bmatrix}
           1\\-3\\-1\\2
         \end{bmatrix}
    \end{pmatrix}
    \quad
    b.\ (dx_1\wedge dx_2+2dx_2\wedge dx_3)
     \begin{pmatrix}
          \begin{bmatrix}
            1\\0\\1
          \end{bmatrix},
          \begin{bmatrix}
          -2\\1\\0
         \end{bmatrix}
    \end{pmatrix}
$$

$$
c.\ dx_4\wedge dx_2
\begin{pmatrix}
          \begin{bmatrix}
            1\\0\\1\\2
          \end{bmatrix},
          \begin{bmatrix}
           1\\-3\\-1\\2
         \end{bmatrix}
    \end{pmatrix}
    \quad
    d.\ dx_1\wedge dx_2\wedge dx_2
    \begin{pmatrix}
          \begin{bmatrix}
            1\\3\\1
          \end{bmatrix},
          \begin{bmatrix}
           -2\\1\\4
         \end{bmatrix}
          \begin{bmatrix}
           2\\2\\2
         \end{bmatrix}
    \end{pmatrix}
$$
\medskip

\textbf{Solution.}
\begin{enumerate}
  \item \[
    \det \begin{bmatrix}
      1 & 1 \\
      2 & 2 
    \end{bmatrix}
    = 0
  \]
  \item \[
    \det \begin{bmatrix}
      1 & -2 \\
      0 & 1
    \end{bmatrix}
    + 2 \det \begin{bmatrix}
      0 & 1 \\
      1 & 0
    \end{bmatrix}
    = 3 + 2(-1)
    = 1
  \]
  \item \[
    \det \begin{bmatrix}
      2 & 2 \\
      0 & -3 \\
    \end{bmatrix}
    = -6
  \]
  \item \[
    \det \begin{bmatrix}
      1 & -2 & 2 \\
      3 & 1 & 2 \\
      3 & 1 & 2 
    \end{bmatrix}
    = 1(0) + 2(0) + 2(0)
    = 0
  \]
\end{enumerate}
\newpage


\subsection*{Exercise 6.1.6}
Which of the following expressions make sense? Evaluate those that do.
$$a.\ dx_1\wedge dx_2
        \begin{pmatrix}
          \begin{bmatrix}
            1\\0\\1
          \end{bmatrix},
          \begin{bmatrix}
           2\\3\\1
         \end{bmatrix}
    \end{pmatrix}
    \quad
    b.\ dx_1\wedge dx_3
     \begin{pmatrix}
          \begin{bmatrix}
            1\\1
          \end{bmatrix},
          \begin{bmatrix}
          2\\3
         \end{bmatrix}
    \end{pmatrix}
$$

$$
c.\ dx_1\wedge dx_2
\begin{pmatrix}
          \begin{bmatrix}
            1\\1
          \end{bmatrix},
          \begin{bmatrix}
           2\\3
         \end{bmatrix}
         \begin{bmatrix}
           -2\\1
         \end{bmatrix}
    \end{pmatrix}
    \quad
    d.\ dx_1\wedge dx_2\wedge dx_4
    \begin{pmatrix}
          \begin{bmatrix}
            1\\0\\3
          \end{bmatrix},
          \begin{bmatrix}
           3\\7\\2
         \end{bmatrix}
          \begin{bmatrix}
           2\\0\\1
         \end{bmatrix}
    \end{pmatrix}
$$

$$
e.\ dx_1\wedge dx_2\wedge dx_3
\begin{pmatrix}
          \begin{bmatrix}
            1\\1
          \end{bmatrix},
          \begin{bmatrix}
           2\\3
         \end{bmatrix}
    \end{pmatrix}
    \quad
    f.\ dx_1\wedge dx_2\wedge dx_3
    \begin{pmatrix}
        \begin{bmatrix}
            1\\0\\3
          \end{bmatrix},
          \begin{bmatrix}
           3\\7\\2
         \end{bmatrix}
          \begin{bmatrix}
           2\\0\\1
         \end{bmatrix}
    \end{pmatrix}
$$
\medskip

\textbf{Solution.}
\begin{enumerate}
  \item \[
    \det \begin{bmatrix}
      1 & 2 \\
      0 & 3
    \end{bmatrix}
    = 3
  \]
  \item Does not make sense 
  \item Does not make sense 
  \item Does not make sense 
  \item Does not make sense 
  \item \[
    \det \begin{bmatrix}
      1 & 3 & 2 \\
      0 & 7 & 0 \\
      3 & 2 & 1
    \end{bmatrix}
    = 1(7) - 0 + 3(-14)
    = -35
  \]
\end{enumerate}
\newpage
















\subsection*{Exercise 6.1.8}
Verify that the wedge product of two 1-forms does not commute, and that the wedge product of a 2-form and a 1-form does commute.

\medskip

\textbf{Solution.}
Two 1-forms do not commute
\begin{align*}
  (\phi \wedge \omega) (v_1, v_2)
  &= \phi(v_1)\omega(v_2) - \phi(v_2)\omega(v_1) \\
  &= - (\omega(v_1)\phi(v_2) - \omega(v_2)\phi(v_1)) \\
  &= - (\omega \wedge \phi) (v_1, v_2)
\end{align*}
However, the wedge product of a 2-from and a 1-form does commute.
\begin{align*}
  (\phi \wedge \omega) (v_1, v_2, v_3)
  &= \phi(v_1, v_2)\omega(v_3)
    - \phi(v_1, v_3)\omega(v_2)
    + \phi(v_2, v_3)\omega(v_1) \\
  &= \omega(v_1)\phi(v_2, v_3)
    - \omega(v_2)\phi(v_1, v_3)
    +  \omega(v_3)\phi(v_1, v_2) \\
  &= (\omega \wedge \phi) (v_1, v_2, v_3) \\
\end{align*}

\newpage








\subsection*{Exercise 6.1.10}
Let $\vt a, \vt v, \vt w$ be vectors in $\mathbb R^3$ and let $\phi$ be the 2-form on $\mathbb R^3$ given by $\phi(\vt v,\vt w) = det(\vt a,\vt v,\vt w)$. Write $\phi$ as a linear combination of elementary 2-forms on $\mathbb R^3$, in terms of the coordinates of $\vt a$.

\medskip

\textbf{Solution.}
Using the cofactor formula on the first column yields 
\begin{align*}
  \phi(v, w)
  &= \det(a,v,w) \\
  &= a_1(dx_2\wedge dx_3)- a_2 (dx_1\wedge dx_3)+ a_3(dx_1\wedge dx_2)
\end{align*}
\newpage








\subsection*{Exercise 6.1.11}
Let $\phi$ and $\psi$ be 2-forms. Use definition 6.1.12 to write the wedge product $\phi\wedge\psi(\vt v_1,\vt v_2,\vt v_3,\vt v_4)$ as a combination of values of $\phi$ and $\psi$ evaluated on appropriate vectors (as in equations 6.1.28 and 6.1.32).

\medskip

\textbf{Solution.}
Simply sum up with the permutations, ensure the correct sign, 
and make sure that it follows the "shuffle" rule.
\begin{align*}
  (\phi \wedge \psi)(v_1, v_2, v_3, v_4) 
  &= \phi(v_1, v_2)\psi(v_3, v_4) \\
  &- \phi(v_1, v_3)\psi(v_2, v_4) \\
  &+ \phi(v_1, v_4)\psi(v_2, v_3) \\
  &+ \phi(v_2, v_3)\psi(v_1, v_4) \\
  &- \phi(v_2, v_4)\psi(v_1, v_3) \\
  &+ \phi(v_3, v_4)\psi(v_1, v_2) 
\end{align*}
\newpage





\subsection*{Exercise 6.1.6}
Which of the following expressions make sense? Evaluate those that do.
$$a.\ (x_1-x_4)dx_3\wedge dx_2
        \begin{pmatrix}
        P_0
        \begin{pmatrix}
          \begin{bmatrix}
           1\\2\\3\\4
          \end{bmatrix},
          \begin{bmatrix}
           0\\1\\-1\\1
         \end{bmatrix}
    \end{pmatrix}
    \end{pmatrix}
    \quad
    b.\ e^xdy
        \begin{pmatrix}
        P_{\begin{bmatrix}
         2\\1
        \end{bmatrix}}
        \begin{pmatrix}
          \begin{bmatrix}
           3\\2
          \end{bmatrix},
    \end{pmatrix}
    \end{pmatrix}
$$

$$
    c.\ x_1^2dx_3\wedge dx_2\wedge dx_1
        \begin{pmatrix}
        P_{\begin{bmatrix}
         2\\0\\0\\0
        \end{bmatrix}}
        \begin{pmatrix}
          \begin{bmatrix}
           1\\2\\3\\4
          \end{bmatrix},
          \begin{bmatrix}
           0\\1\\-1\\1
          \end{bmatrix},
          \begin{bmatrix}
          1\\-1\\-1\\0
          \end{bmatrix}
    \end{pmatrix}
    \end{pmatrix}
$$
\medskip

\textbf{Solution.}
\begin{enumerate}
  \item Does not make sense 
  \item $e^2 \det [2] = 2e^2$
  \item \[
    4 \det \begin{bmatrix}
      3 & -1 & -1 \\
      2 & 1 & -1 \\
      1 & 0 & 1
    \end{bmatrix}
    = 4(7)
    = 28
  \]
\end{enumerate}
\newpage
























\end{document}
