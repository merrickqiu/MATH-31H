\documentclass[12pt]{article}
\usepackage[utf8]{inputenc}
\usepackage[english]{babel}
\usepackage[letterpaper,top=2cm,bottom=2cm,left=3cm,right=3cm,marginparwidth=1.75cm]{geometry}
\usepackage{amsmath}
\usepackage{comment}
\usepackage{tikz}
\usepackage{graphicx}
\usepackage{bigints}
\usepackage{amssymb}
\usepackage[colorlinks=true, allcolors=blue]{hyperref}

\newcommand{\RR}{\mathbb{R}}
\newcommand{\pd}{\partial}

\title{Homework 6}
\author{Merrick Qiu}
\date{}

\begin{document}
\maketitle
\newpage

\textcolor{red}{Exercise 6.4.1:}
If the cone $M$ of equation $f(x,y,z) = x^2+y^2-z^2=0$ (Example 5.2.4) is oriented by $\nabla f$, does the parametrization $\gamma:(r,\theta)\mapsto (rcos\theta,rsin\theta,r)$ preserve orientation?

\bigskip

\textcolor{blue}{Solution:}
    The gradient of the locus is 
    \[
        \nabla f =
        \begin{bmatrix}
            2x \\
            2y \\
            -2z
        \end{bmatrix} = 
        \begin{bmatrix}
            2r\cos \theta \\
            2r\sin \theta \\
            -2r
        \end{bmatrix}
    \]
    The derivative of the parameterization is 
    \[
        D\gamma(r, \theta) =
        \begin{bmatrix}
            \cos \theta & -r\sin \theta\\
            \sin \theta & r\cos \theta \\
            1 & 0 
        \end{bmatrix}
    \]
    The orientation of the transformation is orientation reversing since
    \begin{align*}
        \det 
        \begin{bmatrix}
            2r\cos \theta & \cos \theta & -r\sin \theta\\
            2r\sin \theta & \sin \theta & r\cos \theta \\
            -2r & 1 & 0 
        \end{bmatrix} 
        &= -2r(r\cos^2 \theta + r\sin^2 \theta) -(2r^2\cos^2 \theta + 2r^2\sin^2 \theta) \\
        &= -2r^2 - 2r^2 \\
        &= -4r^2 \\
        &\leq 0
    \end{align*}
        
\newpage








\textcolor{red}{Exercise 6.4.4:}
What is the integral $\int_Sx_3dx_1\wedge dx_2\wedge dx_4$, where $S$ is the part of the $3$-dimensional manifold of equation
$$
x_4 = x_1x_2x_3\quad where\ 0\leq x_1,x_2,x_3\leq 1
$$
oriented by $\Omega = sgn\ dx_1\wedge dx_2\wedge dx_3$? Hint: This surface is a graph, so it is easy to parametrize.


\bigskip

\textcolor{blue}{Solution:}
    The manifold can be parameterized as 
    \[
        \gamma(x_1, x_2, x_3) =
        \begin{bmatrix}
            x_1 \\
            x_2 \\
            x_3 \\
            x_1x_2x_3
        \end{bmatrix}
    \]
    The derivative of the parameterization is 
    \[
        D\gamma = 
        \begin{bmatrix}
            1 & 0 & 0 \\
            0 & 1 & 0 \\
            0 & 0 & 1 \\
            x_2x_3 & x_1x_3 & x_1x_2
        \end{bmatrix}
    \]
    This parameterization is oriented since 
    \[
        dx_1\wedge dx_2\wedge dx_3 
        \begin{bmatrix}
            1 & 0 & 0 \\
            0 & 1 & 0 \\
            0 & 0 & 1 \\
            x_2x_3 & x_1x_3 & x_1x_2
        \end{bmatrix} = 1
    \]
    The integral is
    \begin{align*}
        \int_S x_3 dx_1\wedge dx_2\wedge dx_4
        &= \int_0^1 \int_0^1 \int_0^1 (x_3 dx_1\wedge dx_2\wedge dx_4)
            \left(
                P_{\begin{pmatrix} x_1 \\ x_2 \\ x_3 \\ x_1x_2x_3 \end{pmatrix}}
                \begin{bmatrix}
                    1 & 0 & 0 \\
                    0 & 1 & 0 \\
                    0 & 0 & 1 \\
                    x_2x_3 & x_1x_3 & x_1x_2
                \end{bmatrix}
            \right)
            \,dx_1 \,dx_2 \,dx_3 \\
        &= \int_0^1 \int_0^1 \int_0^1 (x_1x_2x_3) \,dx_1 \,dx_2 \,dx_3 \\
        &= \frac{1}{8}
    \end{align*} 
    
\newpage







\textcolor{red}{Exercise 6.5.4:}
Show that $\Phi_{\vec F\times \vec G} = W_{\vec F}\wedge W_{\vec G}$.


\bigskip

\textcolor{blue}{Solution:} \\
    Using the fact that the wedge product is distributive,
    \begin{align*}
        W_F\wedge W_G
        &= (F_1 dx_1 + F_2 dx_2 + F_3 dx_3) \wedge (G_1 dx_1 + G_2 dx_2 + G_3 dx_3) \\
        &= (F_2G_3 dx_2 \wedge  dx_3 + F_3G_2 dx_3 \wedge  dx_2) 
            + (F_3G_1 dx_3 \wedge dx_1 +  F_1G_3 dx_1 \wedge dx_3) \\
            &\hspace{189pt}+ (F_1G_2 dx_1 \wedge dx_2 +  F_2G_1 dx_2 \wedge dx_1) \\
        &= (F_2G_3 - F_3G_2) dx_2 \wedge dx_3
            + (F_3G_1 - F_1G_3) dx_3 \wedge dx_1 \\
            &\hspace{12em}+ (F_1G_2 - F_2G_1) dx_1 \wedge dx_2 \\
        &= \Phi_{F\times G} 
    \end{align*}
\newpage







\textcolor{red}{Exercise 6.5.5:}
Show that $M_{\vec F\cdot\vec G} = W_{\vec F}\wedge \Phi_{\vec G} = W_{\vec G}\wedge\Phi_{\vec F}$.


\bigskip

\textcolor{blue}{Solution:} \\
    Distributing and factoring shows $M_{\vec F\cdot\vec G} = W_{\vec F}\wedge \Phi_{\vec G}$
    \begin{align*}
        W_F \wedge \Phi_G
        &= (F_1 dx_1 + F_2 dx_2 + F_3 dx_3) \wedge 
            (G_1 dx_2 \wedge dx_3 + G_2 dx_3 \wedge dx_1 + G_3 dx_1 \wedge dx_2) \\
        &= F_1G_1 dx_1 \wedge dx_2 \wedge dx_3 
           + F_2G_2 dx_2 \wedge dx_3 \wedge dx_1 
           + F_3G_3 dx_3 \wedge dx_1 \wedge dx_2 \\
        &= (F_1G_1 + F_2G_2 + F_3G_3) dx_1 \wedge dx_2 \wedge dx_3  \\
        &= M_{F\cdot G}
    \end{align*}
    Since the dot product is commutative, 
    $M_{\vec F\cdot\vec G} = M_{\vec G\cdot\vec F} = W_{\vec G}\wedge \Phi_{\vec F}$.
\newpage







\textcolor{red}{Exercise 6.5.6:}
What is the work form field $W_{\vec F}(P_{a}(\vec u))$ of the vector field
$$
\vec F(x,y,z) = (x^2y,x-y,-z)
$$
at $\vec a = (0,1,2)$, evaluated on the vector $\vec u = (1,-1,1)$.


\bigskip

\textcolor{blue}{Solution:}
    $F(a)$ is
    \[
        \vec F(a) = (0, -1, -2)
    \]
    The work form evaluated at $\vec u$ is
    \[
        W_{\vec F}(P_{a}(\vec u)) = 
        (0, -1, -2) \cdot (1, -1, 1) =
        -1
    \]
\newpage














\textcolor{red}{Exercise 6.5.9:}


\textbf{a.} Construct an oriented parallelogram anchored at $(1,1,0)$ to which  the $2$-form $\Phi = ydy\wedge dz+ xdx\wedge dz-zdx\wedge dy$ of Example 6.5.3 will assign a positive number.


\textbf{b.} At what point $x$ might you anchor $P_x(\vec e_1,\vec e_2)$ if you wanted $\Phi$ evaluated on the parallelogram to return  a positive number? A negative number?


\bigskip

\textcolor{blue}{Solution:} \\
    The parallelogram $[\vec e_2, \vec e_3]$ yields a positive number
    \[
        ydy\wedge dz + xdx\wedge dz-zdx\wedge dy 
        \left(
            P_{\begin{pmatrix} 1 \\ 1 \\ 0 \end{pmatrix}}
            \begin{bmatrix}
                0 & 0\\
                1 & 0\\
                0 & 1
            \end{bmatrix}
        \right)
        = 1 + 0 - 0
        = 1
    \]
    Evaluating at point $(0, 0, -1)$ would yield a positive number
    \[
        ydy\wedge dz + xdx\wedge dz-zdx\wedge dy 
        \left(
            P_{\begin{pmatrix} 0 \\ 0 \\ -1 \end{pmatrix}}
            \begin{bmatrix}
                1 & 0\\
                0 & 1\\
                0 & 0
            \end{bmatrix}
        \right)
        = 0 + 0 -(-1)
        = 1
    \]
    Evaluating at point $(0, 0, 1)$ would yield a negative number
    \[
        ydy\wedge dz + xdx\wedge dz-zdx\wedge dy
        \left(
            P_{\begin{pmatrix} 0 \\ 0 \\ 1 \end{pmatrix}}
            \begin{bmatrix}
                1 & 0\\
                0 & 1\\
                0 & 0
            \end{bmatrix}
        \right)   
        = 0 + 0 - 1
        = -1
    \]
\newpage











\textcolor{red}{Exercise 6.5.13:}
Verify that $\det(\vec F(x),\vec v_1,\dots,\vec v_{n-1})$ is an $(n-1)$-form field, so that Definition 6.5.10 of the flux form on $\mathbb R^n$ makes sense.


\bigskip

\textcolor{blue}{Solution:}
    Through development of the first column,
    \[
        \det(\vec F,\vec v_1,\dots,\vec v_{n-1})
        = \sum_{i=1}^n (-1)^{i-1} F_i 
            \,dx_1 \wedge \hdots  \wedge dx_{i-1} \wedge 
            dx_{i+1} \wedge \hdots \wedge dx_n
    \]
    Since the flux form is a linear combination of elementary $(n-1)$-forms,
    $\det(\vec F(x),\vec v_1,\dots,\vec v_{n-1})$ is a $(n-1)$-form field.
\newpage







\textcolor{red}{Exercise 6.5.15:}
Given $\vec F(x,y,z) = (y^2,x+z,xz)$ and $f(x,y,z) = xz+zy$, the point $x= (1,1,-1)$, and the vectors $\vec v_1=(0,1,1),\vec v_2=(1,1,0),\vec v_3=(-1,1,1)$, what is

\textbf{a.} the work form $W_{\vec F}(P_x(\vec v_1))$?


\textbf{b.} the flux form $\Phi_{\vec F}(P_x(\vec v_1,\vec v_2))$?

\textbf{c.} the mass form $M_f(P_x(\vec v_1,\vec v_2,\vec v_3))$?


\bigskip

\textcolor{blue}{Solution:} \\
    $F(x)$ is
    \[
        \vec F(x) = (1, 0, -1)
    \]
    The work form is
    \[
        W_{\vec F}(P_x(\vec v_1)) 
        = (1, 0, -1) \cdot (0, 1, 1)
        = -1
    \]
    The flux form is
    \[
        \Phi_{\vec F}(P_x(\vec v_1,\vec v_2)) 
        = \det 
        \begin{bmatrix}
            1 & 0 & 1\\
            0 & 1 & 1\\
            -1 & 1 & 0
        \end{bmatrix}
        = -1 + 1
        = 0
    \]
    Since $f(x) = -2$, the mass form is
    \[
        M_f(P_x(\vec v_1,\vec v_2,\vec v_3))
        = -2 \det 
        \begin{bmatrix}
            0 & 1 & -1\\
            1 & 1 & 1\\
            1 & 0 & 1
        \end{bmatrix}
        = -2(1)
        = -2
    \]
\newpage





\textcolor{red}{Exercise 6.5.17:}
Let $R$ be the rectangle with vertices $(0,0),(0,a),(b,a),(b,0)$, with $a,b>0$, and oriented so that these vertices appear in that order. Find the work of the  vector field $\vec F(x,y) = (xy,ye^x)$ around the boundary of $R$.

\bigskip

\textcolor{blue}{Solution:} \\
    The work from $(0, 0)$ to $(0, a)$ can be parameterized with 
    $\gamma(t) = (0, t)$
    \[
        \int_0^a F(0, t) \cdot (0, 1) \,dt
        = \int_0^a (0, t) \cdot (0, 1) \,dt
        = \int_0^a t \,dt
        = \frac{a^2}{2} 
    \]
    The work from $(0, a)$ to $(b, a)$ can be parameterized with 
    $\gamma(t) = (t, a)$
    \[
        \int_0^b F(t, a) \cdot (1, 0) \,dt
        = \int_0^b (at, ae^t) \cdot (1, 0) \,dt
        = \int_0^b at \,dt
        = \frac{ab^2}{2} 
    \]
    The work from $(b, a)$ to $(b, 0)$ can be parameterized with 
    $\gamma(t) = (b, t)$
    \[
        \int_a^0 F(b, t) \cdot (0, 1) \,dt
        = -\int_0^a (bt, te^b) \cdot (0, 1) \,dt
        = -\int_0^a te^b \,dt
        = - \frac{a^2e^b}{2} 
    \]
    The work from $(b, 0)$ to $(0, 0)$ can be parameterized with 
    $\gamma(t) = (t, 0)$
    \[
        \int_b^0 F(t, 0) \cdot (1, 0) \,dt
        = -\int_0^b (0, e^t) \cdot (1, 0) \,dt
        = 0
    \]
    The total work is 
    \[
        \frac{a^2}{2} + \frac{ab^2}{2} - \frac{a^2e^b}{2} 
    \]
\newpage



\textcolor{red}{Exercise 6.5.18:}
Find the work of $\vec F(x,y,z) = (x^2,y^2,z^2)$ over the arc of helix parametrized by $\gamma(t) = (cost,sint,at)$ for $0\leq t\leq \alpha$, and oriented so that $\gamma$ is orientation preserving.


\bigskip

\textcolor{blue}{Solution:}
    Evaluating the integral gives 
    \begin{align*}
        \int_0^\alpha F(\cos t, \sin t, at) \cdot (-\sin t, \cos t, a) \,dt
        &= \int_0^\alpha (\cos^2 t, \sin^2 t, a^2t^2) \cdot (-\sin t, \cos t, a) \,dt \\
        &= \int_0^\alpha -\sin t\cos^2 t + \sin^2 t \cos t + a^3t^2 \,dt \\
        &= \frac{1}{3}\left[\cos^3 t + \sin^3 t + a^3t^3\right]_0^\alpha \\
        &= \frac{\cos^3 \alpha + \sin^3 \alpha + a^3\alpha^3}{3} - \frac{1}{3}
    \end{align*}
\newpage





\textcolor{red}{Exercise 6.5.20:}
What is the flux of the vector field $\vec F(x,y,z) = (x,-y,xy)$ through the surface $z = \sqrt{x^2+y^2}$, $x^2+y^2\leq 1$, oriented by the outward normal?

\bigskip

\textcolor{blue}{Solution:}
    % The derivative of the locus of the surface is 
    % \[
    %     Df =
    %     \begin{bmatrix}
    %         \frac{x}{x^2 + y^2} &
    %         \frac{y}{x^2 + y^2} &
    %         1
    %     \end{bmatrix}
    % \]
    % A basis would be 
    % \[
    %     \begin{bmatrix}
    %         x^2+y^2 & 0 \\
    %         0 & x^2+y^2 \\
    %         -x &  -y
    %     \end{bmatrix}
    % \]
    % This basis is oriented since 
    % \[
    %     \det \begin{bmatrix}
    %         x & x^2+y^2 & 0 \\
    %         y & 0 & x^2+y^2 \\
    %         z & -x &  -y
    %     \end{bmatrix}
    %     = x^2(x^2+y^2) - (x^2+y^2)(-y^2-z(x^2+y^2))
    %     = (z+1)(x^2+y^2)^2
    %     > 0
    % \]
    % Evaluating the integral gives 
    % \begin{align*}
    %     \int_{-1}^1 \int_{-\sqrt{1-x^2}}^{\sqrt{1-x^2}} 
    %     \det \begin{bmatrix}
    %         x & x^2+y^2 & 0 \\
    %         -y & 0 & x^2+y^2 \\
    %         xy & -x &  -y
    %     \end{bmatrix}
    %     \,dy \,dx
    %     &= \int_{-1}^1 \int_{-\sqrt{1-x^2}}^{\sqrt{1-x^2}} 
    %         x^5y + x^4 + 2x^3y^3 + xy^5 - y^4 \,dy \,dx
    % \end{align*}
    The surface can be parameterized as 
    \[
        \gamma(r, \theta) = 
        \begin{bmatrix}
            r\cos \theta \\
            r\sin \theta \\
            r
        \end{bmatrix}
    \]
    The derivative of the parameterization is
    \[
        D\gamma = 
        \begin{bmatrix}
            \cos \theta & -r\sin \theta \\
            \sin \theta & r\cos \theta\\
            1 & 0
        \end{bmatrix}
    \]
    The outward normal vector is 
    \[
        \vec n 
        = \nabla F
        = \begin{bmatrix}
            2x \\ 2y \\ -2z
        \end{bmatrix}
        = 2 \begin{bmatrix}
            r\cos \theta \\
            r\sin \theta \\
            -r
        \end{bmatrix}
    \]
    $\gamma$ is orientation reversing since 
    \begin{align*}
        \det \begin{bmatrix}
            r\cos \theta & \cos \theta & -r\sin \theta \\
            r\sin \theta & \sin \theta & r\cos \theta\\
            -r & 1 & 0
        \end{bmatrix}
        &= r(r\cos^2 \theta + r\sin^2 \theta) - 1(r^2\cos^2 \theta + r^2\sin^2 \theta) \\
        &= -2r^2 \\
        &\leq 0
    \end{align*}
    Integrating the flux through the surface yields 
    \begin{align*}
        -\int_0^{2\pi} \int_0^1 
        \det \begin{bmatrix}
            r\cos \theta & \cos \theta & -r\sin \theta \\
            -r\sin \theta & \sin \theta & r\cos \theta\\
            r^2\sin \theta \cos \theta & 1 & 0
        \end{bmatrix}\,dr \,d\theta
        &= -\int_0^{2\pi} \int_0^1 
            \frac{1}{2}r^3\sin (2\theta)-r^2\cos (2\theta) \,dr \,d\theta \\
        &= -\int_0^{2\pi} 
            \frac{1}{6}\sin (2\theta)-\frac{1}{2}\cos (2\theta) \,d\theta \\
        &= 0
    \end{align*}
\end{document}
