\documentclass[12pt]{article}
\usepackage[utf8]{inputenc}
\usepackage[english]{babel}
\usepackage[letterpaper,top=2cm,bottom=2cm,left=3cm,right=3cm,marginparwidth=1.75cm]{geometry}
\usepackage{amsmath}
\usepackage{comment}
\usepackage{tikz}
\usepackage{graphicx}
\usepackage{bigints}
\usepackage{amssymb}
\usepackage[colorlinks=true, allcolors=blue]{hyperref}

\newcommand{\RR}{\mathbb{R}}
\newcommand{\pd}{\partial}

\title{Math 31CH HW 7\\ Due May 31 at 11:59 pm by Gradescope Submission\\
6.6.5 (skip part b), 6.7.3, 6.7.4, 6.7.6, 6.7.9,  6.8.6, 6.8.7, 6.8.12}
\author{Merrick Qiu}
\date{}

\begin{document}
\maketitle


\begin{center}
   \textbf{EXERCISES FOR SECTION 6.6}
\end{center}



\textcolor{red}{Exercise 6.6.5 (skip part b):}
Consider the region $X=P \cap B \subset \RR^3$,
where $P$ is the plane of equation $x+y+z=0$ and $B$ is the ball $x^2 +y^2 +z^2 \leq 1$.
Orient $P$ by the normal
$\vec N = \begin{bmatrix}1\\1\\1\end{bmatrix}$
and orient the sphere $x^2 +y^2 +z^2 = 1$ by the
outward-pointing normal.

\textbf{a.} Which of $\operatorname{sgn} dx\wedge dy$,
$\operatorname{sgn} dx\wedge dz$,
$\operatorname{sgn} dy\wedge dz$
give the same orientation of $P$ as $\vec N $?

\textbf{c.} Is the parametrization
\begin{equation*}
    t \mapsto \begin{pmatrix}
    \dfrac{\cos (t)}{\sqrt{2}} - \dfrac{\sin (t)}{\sqrt{6}} \\
    -  \dfrac{\cos (t)}{\sqrt{2}} - \dfrac{\sin (t)}{\sqrt{6}} \\
    2\, \dfrac{\sin (t)}{\sqrt{6}}
    \end{pmatrix}
\end{equation*}
compatible with the boundary orientation of $\partial X$?

\textbf{d.} Do any of $\operatorname{sgn} dx$,
$\operatorname{sgn} dy$,
$\operatorname{sgn} dz$
define the orientation of $\partial X$ at every point?

\textbf{e.} Do any of $\operatorname{sgn} x \,dy - y\,dx$,
$\operatorname{sgn} x\,dz - z\, dx$,
$\operatorname{sgn} y \,dz - z \, dy$
define the orientation of $\partial X$ at every point?
\medskip

\textbf{Part A:} 
$\operatorname{sgn} dx \wedge dy$ and $\operatorname{sgn} dy \wedge dz$ 
give the same orientation of P as $\vec N$. 
\begin{align*}
    \det \begin{bmatrix}
        1 & v_x & w_x \\
        1 & v_y & w_y \\
        1 & v_z & w_z 
    \end{bmatrix}
    &= (v_yw_z - v_zw_y) - (v_xw_z - v_zw_x) + (v_xw_y - v_yw_x) \\
    &= (-v_y(w_x+w_y) + (v_x+v_y)w_y) - (-v_x(w_x+w_y) + (v_x+v_y)w_x) + (v_xw_y - v_yw_x) \\
    &= -v_yw_x-v_yw_y + v_xw_y+v_yw_y + v_xw_x+v_xw_y - v_xw_x-v_yw_x + v_xw_y - v_yw_x \\
    &= 3v_xw_y - 3v_yw_x = 3dx \wedge dy (\vec v, \vec w) \\
    &= -3(v_x(-w_x-w_y) - (-v_x-v_y)w_x) = -3dx \wedge dy (\vec v, \vec w) \\
    &= 3(v_y(-w_x-w_y) - (-v_x-v_y)w_y) = 3dy \wedge dz (\vec v, \vec w) 
\end{align*}

\textbf{Part C:}
$\gamma(t)$ is an outward pointing normal of $X$ and
the tangent vector of the parameterization is 
\[
    \gamma'(t) =
    \begin{bmatrix}
        -\frac{\sin(t)}{\sqrt{2}} - \frac{\cos(t)}{\sqrt{6}} \\
        \frac{\sin(t)}{\sqrt{2}} - \frac{\cos(t)}{\sqrt{6}} \\
        2\frac{\cos(t)}{\sqrt{6}}
    \end{bmatrix}
\]
The parameterization is not consistent since 
\begin{align*}
    \det \left[\vec N, \gamma(t), \gamma'(t)\right]
    &= 3dy \wedge dz \left[\gamma(t), \gamma'(t)\right] \\
    &= 3 \det 
    \begin{bmatrix}
        -\frac{\cos(t)}{\sqrt{2}} - \frac{\sin(t)}{\sqrt{6}} & \frac{\sin(t)}{\sqrt{2}} - \frac{\cos(t)}{\sqrt{6}} \\
        2\frac{\sin(t)}{\sqrt{6}} & 2\frac{\cos(t)}{\sqrt{6}}
    \end{bmatrix} \\
    &= 6 \left(\left(-\frac{\cos^2(t)}{2\sqrt{3}} - \frac{\sin(t)\cos(t)}{6}\right)
        - \left(\frac{\sin^2(t)}{2\sqrt{3}} - \frac{\sin(t)\cos(t)}{6}\right)\right)
    &= -\sqrt{3}
\end{align*}

\textbf{Part D}
Since $\gamma$ is orientation reversing,
any form directly orienting $\partial X$ must yield a negative yalue on $\gamma'(t)$
Since none of the components of $\gamma'(t)$ are always negative,
$\operatorname{sgn} dx$,
$\operatorname{sgn} dy$, and
$\operatorname{sgn} dz$ do not define the orientation for $\partial X$.

\textbf{Part E}
Calculating the forms on $\gamma'(t)$ yields that only $x\,dy - y\,dx$ and $y\,dz - z\,dy$ 
define the orientation on $\partial X$.
\begin{align*}
    x\,dy - y\,dx
    &\implies \left(\frac{\cos(t)}{\sqrt{2}} - \frac{\sin(t)}{\sqrt{6}}\right)\left(\frac{\sin(t)}{\sqrt{2}} - \frac{\cos(t)}{\sqrt{6}}\right)
        - \left(-\frac{\cos(t)}{\sqrt{2}} - \frac{\sin(t)}{\sqrt{6}}\right)\left(-\frac{\sin(t)}{\sqrt{2}} - \frac{\cos(t)}{\sqrt{6}}\right) \\
    &= -\frac{\cos^2(t)}{\sqrt{3}} - \frac{\sin^2(t)}{\sqrt{3}} \\
    &= -\frac{1}{\sqrt{3}}
\end{align*}
\begin{align*}
    x\,dz - z\,dx
    &\implies \left(\frac{\cos(t)}{\sqrt{2}} - \frac{\sin(t)}{\sqrt{6}}\right)\left(2\frac{\cos(t)}{\sqrt{6}}\right)
        - \left(2\frac{\sin(t)}{\sqrt{6}}\right)\left(-\frac{\sin(t)}{\sqrt{2}} - \frac{\cos(t)}{\sqrt{6}}\right) \\
    &= \frac{\cos^2(t)}{\sqrt{3}} + \frac{\sin^2(t)}{\sqrt{3}} \\
    &= \frac{1}{\sqrt{3}}
\end{align*}
\begin{align*}
    y\,dz - z\,dy
    &\implies \left(-\frac{\cos(t)}{\sqrt{2}} - \frac{\sin(t)}{\sqrt{6}}\right)\left(2\frac{\cos(t)}{\sqrt{6}}\right)
        - \left(2\frac{\sin(t)}{\sqrt{6}}\right)\left(\frac{\sin(t)}{\sqrt{2}} - \frac{\cos(t)}{\sqrt{6}}\right) \\
    &= \frac{-\cos^2(t)}{\sqrt{3}} - \frac{\sin^2(t)}{\sqrt{3}} \\
    &= -\frac{1}{\sqrt{3}}
\end{align*}
\newpage
\begin{center}
   \textbf{EXERCISES FOR SECTION 6.7}
\end{center}


\textcolor{red}{Exercise 6.7.3:}
In Example 6.7.7, confirm that:

\textbf{a.} $d\Phi_{\vec F_2 }= 0$.

\textbf{b.} $d\Phi_{\vec F_3} = 0$.
\smallskip

\textbf{Part A:}
\begin{align*}
    d\Phi_{\vec F_2 }
    &= d \left(\frac{-y}{x^2+y^2}\,dx + \frac{x}{x^2+y^2}\,dy\right) \\
    &= \left(D_1 \frac{-y}{x^2+y^2}\,dx + D_2 \frac{-y}{x^2+y^2}\,dy \right)\wedge \,dx + 
        \left(D_1 \frac{x}{x^2+y^2}\,dx + D_2 \frac{x}{x^2+y^2}\,dy\right)\wedge \,dy \\
    &= \left(\frac{2xy}{(x^2+y^2)^2}\,dx + \frac{y^2-x^2}{(x^2+y^2)^2}\,dy \right)\wedge \,dx +
        \left(\frac{y^2-x^2}{(x^2+y^2)^2}\,dx + \frac{-2xy}{(x^2+y^2)^2}\,dy\right)\wedge \,dy \\
    &= 0
\end{align*}
\textbf{Part B:}
\begin{align*}
    d\Phi_{\vec F_3}
    &= d \left(\frac{x}{(x^2+y^2+z^2)^\frac{3}{2}}\,dy\wedge dz + \frac{y}{(x^2+y^2+z^2)^\frac{3}{2}}\,dz \wedge dx + \frac{z}{(x^2+y^2+z^2)^\frac{3}{2}}\,dx\wedge dy\right) \\
    &= \frac{-2x^2+y^2+z^2}{(x^2 + y^2+z^2)^\frac{5}{2}}\,dx\wedge dy\wedge dz + \frac{-2y^2+x^2+z^2}{(x^2 + y^2+z^2)^\frac{5}{2}}\,dy\wedge dz\wedge dx + \frac{-2z^2+x^2+y^2}{(x^2 + y^2+z^2)^\frac{5}{2}}\,dz\wedge dx\wedge dy \\
    &= 0
\end{align*}


\textcolor{red}{Exercise 6.7.4:}
Let $\varphi$ be the $2$-form on $\RR^4$ given
by
\begin{equation*}
    \varphi = x_1^2 x_3 \, dx_2 \wedge dx_3 + x_1 x_3 \, dx_1 \wedge dx_4 .
\end{equation*}
Compute $d\varphi$.
\smallskip

\textbf{Solution:}
\begin{align*}
    d(x_1^2 x_3 \, dx_2 \wedge dx_3 + x_1 x_3 \, dx_1 \wedge dx_4)
    &= d(x_1^2 x_3 \, dx_2 \wedge dx_3) + d(x_1 x_3 \, dx_1 \wedge dx_4) \\
    &= (D_1(x_1^2 x_3)\, dx_1) \wedge dx_2 \wedge dx_3 +
        (D_3(x_1 x_3)\, dx_3) \wedge dx_1 \wedge dx_4\\
    &= 2x_1 x_3 \,dx_1 \wedge dx_2 \wedge dx_3 -
        x_1 \,dx_1 \wedge dx_3 \wedge dx_4\\
\end{align*}

\textcolor{red}{Exercise 6.7.6:}
Let $f$ be a function from $\RR^3$ to $\RR$. Compute the exterior derivatives:

\textbf{a.} $d(f\, dx\wedge dz)$.

\textbf{b.} $d(f\, dy\wedge dz)$.
\smallskip

\textbf{Part A:}
\begin{align*}
    d(f\, dx\wedge dz)
    &= (D_1(f) dx + D_2(f) dy + D_3(f) dz) \wedge dx\wedge dz \\
    &= -D_2(f) \,dx \wedge dy\wedge dz
\end{align*}

\textbf{Part B:}
\begin{align*}
    d(f\, dy\wedge dz)
    &= (D_1(f) dx + D_2(f) dy + D_3(f) dz) \wedge dy\wedge dz \\
    &= D_1(f) \,dx \wedge dy\wedge dz
\end{align*}

\textcolor{red}{Exercise 6.7.9:}
Find all the $1$-forms $\omega = p(y,z) \, dx + q(x,z) \, dy$ such that
\begin{equation*}
    d\omega = x\, dy\wedge dz + y \, dx \wedge dz.
\end{equation*}
\smallskip

\textbf{Solution:}
\begin{align*}
    d(p\,dx + q \,dy)
    &= (D_2 p \,dy + D_3 p \,dz)\wedge dx + (D_1 q \,dx + D_3 q \,dz)\wedge dy \\
    &= (-D_3 q)\,dy\wedge dz + (-D_3 p)\,dx\wedge dz + (D_1 q - D_2 p)\,dx\wedge dy \\
    &= x \,dy\wedge dz + y \,dx\wedge dz + 0 \,dx\wedge dy \\
\end{align*}
From this,
\[
    D_3 q = -x
\]
\[
    D_3 p = -y
\]
\[
    D_1 q = D_2 p
\]
Thus, $\omega$ must be in the form 
\[
    \omega = (-yz + F(y))\, dx + (-xz + G(x)) \, dy
\]
where $F$ and $G$ are arbitrary differentiable functions.

\newpage
\begin{center}
   \textbf{EXERCISES FOR SECTION 6.8}
\end{center}

\textcolor{red}{Exercise 6.8.6:}
Show that $df = W_{ \operatorname{grad} f}$ when $f \begin{pmatrix}x\\y\\z\end{pmatrix}
=xyz$ by computing both from the definitions and evaluating on a vector $\vec v = \begin{pmatrix}a\\b\\c\end{pmatrix}$.
\smallskip

\textbf{Solution:} 
The exterior derivative of $f$ is 
\begin{align*}
    df(\vec v) 
    &= D_1(xyz) \,dx + D_2(xyz) \,dy + D_3(xyz) \,dz \\
    &= yz \,dx + xz \,dy + xy \,dz
\end{align*}
    
The work form of the gradient is 
\[
    W_{ \operatorname{grad} f}
    = W_{\begin{bmatrix} yz \\ xz \\ xy \end{bmatrix}} 
    = yz \,dx + xz \,dy + xy \,dz
\]
Thus $df = W_{ \operatorname{grad} f}$.

\textcolor{red}{Exercise 6.8.7:}
Let $\varphi = xy \, dx + z\, dy + yz \, dz$
be a $1$-form on $\RR^3$.
For what vector field $\vec F$ can $\varphi$ be written $W_{\vec F}$? Show the equivalence of $dW_{\vec F}$ and $\Phi_{\vec \nabla \times \vec F}$
by computing both from the definitions.
\smallskip

\textbf{Solution: } 
The 1-form can be rewritten as a work form with vector field
\[
    \vec F = \begin{bmatrix} xy \\ z \\ yz \end{bmatrix}
\]
The exterior derivative of the work form is 
\begin{align*}
    d(xy \, dx + z\, dy + yz \, dz)
    &= d(xy \, dx) + d(z\, dy) + d(yz \, dz) \\
    &= (D_2(xy) \,dy) \wedge \,dx + (D_3(z)\,dz)\wedge dy + (D_2(yz)\,dy)\wedge dz \\
    &= -x \,dx \wedge dy + (z-1) \,dy \wedge dz
\end{align*}

The curl is 
\[
    \vec \nabla \times \vec F
    = \begin{bmatrix}
        D_1 \\ D_2 \\ D_3
    \end{bmatrix} \times
    \begin{bmatrix}
        xy \\ z \\ yz
    \end{bmatrix}
    = \begin{bmatrix}
        z-1 \\ 0 \\ -x
    \end{bmatrix}
\]

The flux of the curl is equal to the exterior derivative of the work form
\[
    \Phi_{\vec \nabla \times \vec F}
    = (z-1)\,dy \wedge dz - 0 \,dx \wedge dz - x\,dx \wedge dy
    = W_{\vec F}
\]
\textcolor{red}{Exercise 6.8.12:}
\textbf{a.} What is the divergence of
$\vec F \begin{pmatrix}
x\\y\\ z
\end{pmatrix} = \begin{bmatrix}
 x^2 \\ y^2 \\yz
\end{bmatrix}$?
\smallskip

\textbf{b.} Use part (a) to compute $d\Phi_{\vec F} P_{\begin{pmatrix}
1\\1\\2
\end{pmatrix}} (\vec e_1 , \vec e_2 , \vec e_3)$.

\textbf{c.} Compute it again, directly form the definition of exterior derivative.

\textbf{Part A: }
\[
    \operatorname{div} \vec F =
    \begin{bmatrix}
        D_1 \\ D_2 \\ D_3
    \end{bmatrix}
    \cdot 
    \begin{bmatrix}
        x^2 \\ y^2 \\ z^2
    \end{bmatrix} =
    2x + 3y 
\]

\textbf{Part B: }
Using part A, the derivative of the flux form of $\vec F$ is
\[
    d\Phi_{\vec F}
    = M_{\operatorname{div} \vec F} \\ 
    = (2x + 3y) \,dx \wedge dy \wedge dz
\]
Therefore,
\[
    d\Phi_{\vec F}P_{\begin{pmatrix}
        1\\1\\2
    \end{pmatrix}} (\vec e_1 , \vec e_2 , \vec e_3)
    = (2+3) \det(\vec e_1 , \vec e_2 , \vec e_3)
    = 5
\]

\textbf{Part C: }
The exterior derivative is
\begin{align*}
    d\Phi_{\vec F} 
    &= d(x^2 \,dy \wedge dz + y^2 \,dz \wedge dx + yz \,dx \wedge dy) \\
    &= (2x \,dx) \wedge dy \wedge dz + (2y \,dy) \wedge dz \wedge dx + (y \,dz) \wedge dx \wedge dy \\
    &= (2x+3y) \,dx \wedge dy \wedge dz
\end{align*}
This is the same form as from part b, so evaluating it at the same 
parallelogram would also yield 5.





















\end{document} 