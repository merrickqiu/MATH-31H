\documentclass[12pt]{article}
\usepackage[utf8]{inputenc}
\usepackage[english]{babel}
\usepackage[letterpaper,top=2cm,bottom=2cm,left=3cm,right=3cm,marginparwidth=1.75cm]{geometry}
\usepackage{amsmath}
\usepackage{tikz}
\usepackage{graphicx}
\usepackage{bigints}
\usepackage{amssymb}
\usepackage[colorlinks=true, allcolors=blue]{hyperref}

\newcommand{\RR}{\mathbb{R}}

\title{Math 31CH HW 5\\ Due May 10 at 11:59 pm by Gradescope Submission\\
6.2.1, 6.2.2, 6.2.3, 6.3.1, 6.3.3, 6.3.4, 6.3.5, 6.3.6, 6.3.7, 6.3.8, 6.3.11, 6.3.12}
\author{Professor Bennett Chow}
\date{}

\begin{document}
\maketitle
% % \newpage


\begin{center}
   \textbf{EXERCISES FOR SECTION 6.2}
\end{center}

\textcolor{red}{Exercise 6.2.1:}
Set up each of the following integrals of form fields over parametrized domains as an ordinary multiple integral, and compute it.

\textbf{a.} $\displaystyle \int_{[\gamma(I)]} x \, dy + y \, dz$,
where $I=[-1,1]$, and $\displaystyle \gamma(t) = \begin{pmatrix} \sin (t) \\ \cos (t) \\ t \end{pmatrix}$.

\textbf{b.} $\int_{[\gamma(U)]} x_1\, dx_2 \wedge dx_3 + x_2\, dx_3 \wedge dx_4$, where
$\displaystyle U = \left\{ \left. \begin{pmatrix} u \\ v \end{pmatrix} \right| \, 0\leq u, v\,; \; u+ v \leq 2 \right\}$,
$\gamma \begin{pmatrix} u \\ v \end{pmatrix} =\begin{pmatrix} uv \\ u^2+v^2 \\ u-v \\ \ln(u+v+1) \end{pmatrix}$.
\smallskip

\textcolor{blue}{Solution to 6.2.1:}
For part a
\begin{align*}
    \int_{[\gamma(U)]} x \,dy + y \,dz
    &= \int_{-1}^1 x \,dy + y \,dz
        \left(P_{\begin{pmatrix} \sin t \\ \cos t \\ t \end{pmatrix}}
            \begin{bmatrix}
                \cos t \\ -\sin t \\ 1
            \end{bmatrix}
        \right) \,dt \\ 
    &= \int_{-1}^1 \sin t (-\sin t) + \cos t (1) \,dt \\
    &= \int_{-1}^1 \frac{1}{2}\cos 2t - \frac{1}{2} + \cos t \,dt\\
    &= \left[\frac{\sin 2t}{4} - \frac{1}{2}t + \sin t \right]_{-1}^{1} \\
    &= \frac{\sin 2}{2} + 2\sin 1 - 1
\end{align*}

For part b 
\begin{align*}
    &\int_{[\gamma(U)]} x_1\,dx_2 \wedge \,dx_3 +x_2\,dx_3 \wedge \,dx_4 \\
    =& \int_0^2 \int_0^{2-u} x_1\,dx_2 \wedge \,dx_3 +x_2\,dx_3 \wedge \,dx_4
        \left(P_{\begin{pmatrix} uv \\ u^2 + v^2 \\ u-v \\ \ln(u+v+1) \end{pmatrix}}
            \begin{bmatrix}
                v & u \\
                2u & 2v \\
                1 & -1 \\
                \frac{1}{u+v+1} & \frac{1}{u+v+1}
            \end{bmatrix}
        \right) \,dv \,du \\ 
    =& \int_0^2 \int_0^{2-u} uv(-2u-2v) + (u^2+v^2)\frac{2}{u+v+1} \,dv \,du \\
    =& \frac{64}{45} - \frac{4}{3}\ln 3
\end{align*}
The last step was calculated using Matlab.
\begin{verbatim}
    >> syms u v
    >> f = -2*u*v*(u+v) + 2*(u^2+v^2)/(1+u+v)
    >> int(int(f,v,0,2-u),u,0,2)
       ans = 64/45 - (4*log(3))/3
\end{verbatim}
\newpage

\textcolor{red}{Exercise 6.2.2:}
Repeat Exercise 6.2.1, for the following.

\textbf{a.}
$\displaystyle \int_{[\gamma(U)]} x\, dy \wedge dz$, where
$U=[-1,1]\times [-1,1]$, and
$\displaystyle \gamma
\begin{pmatrix}
u\\v
\end{pmatrix} =
\begin{pmatrix}
u^2 \\u+v\\v^3
\end{pmatrix}
$.

\textbf{b.}
$\displaystyle \int_{[\gamma(U)]} x_2\, dx_1 \wedge dx_3 \wedge dx_4$, where
$U= \left\{ \left. \begin{pmatrix}
u\\v\\w
\end{pmatrix}\right| \,
0 \leq u,v,w\, ; \; u+v+w \leq 3
\right\}$, and
$\displaystyle \gamma
\begin{pmatrix}
u\\v\\w
\end{pmatrix} =
\begin{pmatrix}
uv \\u^2+w^2\\u-v\\w
\end{pmatrix}
$.

\smallskip

\textcolor{blue}{Solution to 6.2.2:}
For part a
\begin{align*}
    \int_{[\gamma(U)]} x\, dy \wedge dz
    &= \int_{-1}^1 \int_{-1}^1 x\, dy \wedge dz
        \left(
            P_{\begin{pmatrix}
                u^2 \\ u+v \\ v^3
            \end{pmatrix}}
            \begin{bmatrix}
                2u & 0 \\
                1 & 1 \\
                0 & 3v^2
            \end{bmatrix}
        \right) \,dv \,du \\ 
    &= \int_{-1}^1 \int_{-1}^1 3u^2v^2 \,dv \,du \\
    &= \int_{-1}^1 2u^2 \,du \\
    &= \frac{4}{3}
\end{align*}
For part b
\begin{align*}
    &\int_{[\gamma(U)]} x_2\, dx_1 \wedge dx_3 \wedge dx_4 \\
    =& \int_0^3 \int_0^{3-u} \int_0^{3-u-v} x_2\, dx_1 \wedge dx_3 \wedge dx_4 
        \left(
            P_{\begin{pmatrix}
                uv \\ u^2+w^2 \\ u-v \\ w
            \end{pmatrix}}
            \begin{bmatrix}
                v & u & 0\\
                2u & 0 & 2w\\
                1 & -1 & 0\\
                0 & 0 & 1
            \end{bmatrix}
        \right)
    \,dw \,dv \,du \\
    =& \int_0^3 \int_0^{3-u} \int_0^{3-u-v} (u^2+w^2)(-v-u) \,dw \,dv \,du \\
    =& -12.15
\end{align*}
The last step was solved using \href[pdfnewwindow=true]{https://www.wolframalpha.com/input?i=%5Cint_0%5E3+%5Cint_0%5E%7B3-u%7D+%5Cint_0%5E%7B3-u-v%7D+%28u%5E2%2Bw%5E2%29%28-v-u%29+%5C%2Cdw+%5C%2Cdv+%5C%2Cdu}{Wolfram Alpha}.
\newpage

\textcolor{red}{Exercise 6.2.3:}
Set up each of the following integrals of form fields over parametrized domains as an ordinary multiple integral.
\smallskip

\textbf{a.}
$\displaystyle \int_{[\gamma(U)]} (x_1+x_4)\, dx_2 \wedge dx_3 $, where
$U= \left\{ \left. \begin{pmatrix}
u\\v
\end{pmatrix}\right| \;
|v| \leq u \leq 1
\right\}$, and where
\newline \bigskip
$\displaystyle \gamma
\begin{pmatrix}
u\\v
\end{pmatrix} =
\begin{pmatrix}
e^u \\ e^{-v} \\ \cos (u) \\ \sin (v)
\end{pmatrix}
$.
\medskip

\textbf{b.}
$\displaystyle \int_{[\gamma(U)]} x_2\, x_4\, dx_1 \wedge dx_3 \wedge dx_4$, where
$U= \left\{ \left. \begin{pmatrix}
u\\v\\w
\end{pmatrix}\right| \;
(w-1)^2 \geq u^2+v^2, \;
0 \leq w \leq 1
\right\}$, and where
\newline \bigskip
$\displaystyle \gamma
\begin{pmatrix}
u\\v\\w
\end{pmatrix} =
\begin{pmatrix}
u+v\\u-v\\w+v\\w-v
\end{pmatrix}
$.

\medskip

\textcolor{blue}{Solution to 6.2.3:}
For part a
\begin{align*}
    \int_{[\gamma(U)]} (x_1+x_4)\, dx_2 \wedge dx_3
    &= \int_0^1 \int_{-u}^u (x_1+x_4)\, dx_2 \wedge dx_3
        \left(
            P_{\begin{pmatrix}
                e^u \\ e^{-v} \\ \cos u \\ \sin v
            \end{pmatrix}}
            \begin{bmatrix}
                e^u & 0 \\
                0 & -e^{-v} \\
                -\sin u & 0\\
                0 & \cos v
            \end{bmatrix}
        \right) \,du \,dv\\
    &= -\int_0^1 \int_{-u}^u (e^u+\sin v)(e^{-v}\sin u) \,du \,dv
\end{align*}
For part b
\begin{align*}
    &\int_{[\gamma(U)]} x_2\, x_4\, dx_1 \wedge dx_3 \wedge dx_4 \\
    =& \int_0^1 \int_{w-1}^{1-w} \int_{-\sqrt{(w-1)^2-v^2}}^{-\sqrt{(w-1)^2-v^2}} 
        x_2\, x_4\, dx_1 \wedge dx_3 \wedge dx_4 
        \left(
            P_{\begin{pmatrix}
                u+v \\ u-v \\ w+v \\ w-v
            \end{pmatrix}}
            \begin{bmatrix}
                1 & 1 & 0\\
                1 & -1 & 0\\
                0 & 1 & 1\\
                0 & -1 & 1
            \end{bmatrix}
        \right) \,du \,dv \,dw \\
    =& \int_0^1 \int_{w-1}^{1-w} \int_{-\sqrt{(w-1)^2-v^2}}^{-\sqrt{(w-1)^2-v^2}} 
        2(u-v)(w-v) dx_1 \,du \,dv \,dw
\end{align*}
\newpage

\textcolor{red}{Exercise 6.3.1:}
Is the constant vector field $\begin{bmatrix}1\\1\end{bmatrix}$
a tangent vector field defining an orientation of the line of equation $x+y=0$?
How about the line of equation $x-y=0$?
\smallskip

\textcolor{blue}{Solution to 6.3.1:}
    The line of equation $x+y=0$ has a tangent vector of 
    $\begin{bmatrix}
        1 \\ -1
    \end{bmatrix}$
    so the constant vector field
    $\begin{bmatrix}
        1 \\ 1
    \end{bmatrix}$
    does not define an orientation.

    The line of equation $x-y=0$ has a tangent vector of 
    $\begin{bmatrix}
        1 \\ 1
    \end{bmatrix}$
    so the constant vector field
    $\begin{bmatrix}
        1 \\ 1
    \end{bmatrix}$
    does define an orientation.
\newpage

\textcolor{red}{Exercise 6.3.3:}
Does any constant vector field define an orientation of the unit sphere in $\mathbb{R}^3$?
\smallskip

\textcolor{blue}{Solution to 6.3.3:}
    No constant vector fields defines an orientation of the unit sphere since
    a constant vector field is not transversal to the unit sphere.
    This is because there exists a point in the unit sphere 
    such that $x \in v^\perp$ for all $v$.

\newpage

\textcolor{red}{Exercise 6.3.4:}

Find a vector that orients the curve given by
$x+x^2+y^2 =2$.
\smallskip



\textcolor{blue}{Solution to 6.3.4:}
    Using implicit differentiation,
    \begin{align*}
        x + x^2 + y^2 = 2
        &\implies dx + 2x\,dx + 2y\,dy = 0 \\
        &\implies 2y\,dy = (1+2x)\,dx \\
        &\implies \frac{dy}{dx} = -\frac{1+2x}{2y}
    \end{align*}
    Therefore the following orients the curve in the counterclockwise direction.
    \[
        t(x,y) = 
        \begin{bmatrix}
            -2y \\
            1+2x
        \end{bmatrix}
    \]

\newpage

\textcolor{red}{Exercise 6.3.5:}

Which of the vector fields
\begin{equation*}
    \begin{bmatrix}
    1\\1\\1
    \end{bmatrix}, \quad
      \begin{bmatrix}
   - 1\\1\\1
    \end{bmatrix}, \quad
      \begin{bmatrix}
   - 1\\-1\\1
    \end{bmatrix}, \quad
      \begin{bmatrix}
    -1\\-1\\-1
    \end{bmatrix}
\end{equation*}
define an orientation of the plane $P\subset \mathbb{R}^3$ of equation
$x+y+z=0$, and among these, which pairs define the same orientation?\medskip
\smallskip

\textcolor{blue}{Solution to 6.3.5:}
    None of the four vectors solve the equation, 
    so all of the vector fields orient the plane.
    Choosing the basis vectors we can calculate the orientation of the 
    vector fields.
    \[
        \begin{bmatrix}
            0 \\ 1 \\ -1
        \end{bmatrix},
        \begin{bmatrix}
            1 \\ 0 \\ -1
        \end{bmatrix}
    \]
    \[
        \det \begin{bmatrix}
            1 & 0 & 1 \\
            1 & 1 & 0 \\
            1 & -1 & -1
        \end{bmatrix} = (-1) + (-2) = -3
    \]
    \[
        \det \begin{bmatrix}
            -1 & 0 & 1 \\
            1 & 1 & 0 \\
            1 & -1 & -1
        \end{bmatrix} = (1) + (-2) = -1
    \]
    \[
        \det \begin{bmatrix}
            -1 & 0 & 1 \\
            -1 & 1 & 0 \\
            1 & -1 & -1
        \end{bmatrix} = (1) + (0) = 1
    \]
    \[
        \det \begin{bmatrix}
            -1 & 0 & 1 \\
            -1 & 1 & 0 \\
            -1 & -1 & -1
        \end{bmatrix} = (1) + (2) = 3
    \]
    So
    $
        \begin{bmatrix}
            1 \\ 1 \\ 1
        \end{bmatrix},
        \begin{bmatrix}
            -1 \\ 1 \\ 1
        \end{bmatrix}
    $
    are the same orientation and
    $
        \begin{bmatrix}
            -1 \\ -1 \\ 1
        \end{bmatrix},
        \begin{bmatrix}
            -1 \\ -1 \\ -1
        \end{bmatrix}
    $
    are another orientation.
\newpage

\textcolor{red}{Exercise 6.3.6:}
Find a vector field that orients the surface $S \subset \mathbb{R}^3$ given
by $x^2+y^3+z=1$.
\smallskip

\textcolor{blue}{Solution to 6.3.6:}
    The gradient of the locus orients the surface
    \[
        \nabla f(x, y, z) =
        \begin{bmatrix}
            2x \\ 3y^2 \\ 1
        \end{bmatrix} 
    \]
\newpage

\textcolor{red}{Exercise 6.3.7:}
Let $V$ be the plane of the equation $x+2y-z=0$. Show that the bases
\begin{equation*}
    \vec v_1 = \begin{bmatrix}
        1\\0\\1
    \end{bmatrix},\;
    \vec v_2 = \begin{bmatrix}
        0\\1\\2
    \end{bmatrix}
    \quad \text{and} \quad
    \vec w_1 = \begin{bmatrix}
        2\\-3\\-4
    \end{bmatrix},\;
    \vec w_2 = \begin{bmatrix}
        1\\2\\5
    \end{bmatrix}
\end{equation*}
give the same orientation
\smallskip

\textcolor{blue}{Solution to 6.3.7:}
    The change of basis matrix is 
    \[
        P_{w \to v} = 
        \begin{bmatrix}
            2 & 1 \\
            -3 & 2
        \end{bmatrix}
    \]
    This matrix has a determinant of $7$, 
    so both basis have the same orientation.
\newpage

\textcolor{red}{Exercise 6.3.8:}
Let $P$ be the plane of equation $x+y+z=0$.
\smallskip

\textbf{a.} Of the three bases
\begin{equation*}
        \begin{bmatrix}
        1\\0\\-1
    \end{bmatrix},\,
     \begin{bmatrix}
        0\\1\\-1
    \end{bmatrix} ,\qquad
        \begin{bmatrix}
        -1\\0\\1
    \end{bmatrix},\,
     \begin{bmatrix}
        -1\\1\\0
    \end{bmatrix} ,\qquad
            \begin{bmatrix}
        1\\-1\\0
    \end{bmatrix},\,
     \begin{bmatrix}
        0\\-1\\1
    \end{bmatrix}
\end{equation*}
which gives a different orientation than the other two? \medskip

\textbf{b.}
Find a normal vector to $P$ that gives the same orientation as that basis.\medskip

\textcolor{blue}{Solution to 6.3.8:} \\
    The change of base matrix from the first to the second base is 
    \[
        P_{1 \to 2} = \begin{bmatrix}
            -1 & -1\\
            0 & 1
        \end{bmatrix}
    \]
    The change of base matrix from the second to the third base is 
    \[
        P_{2 \to 3} = \begin{bmatrix}
            -1 & -1\\
            1 & 0
        \end{bmatrix}
    \]
    $\det P_{1\to 2} = -1$ and $\det P_{2\to 3} = 1$, so the first basis 
    has a different orientation than the other two.
    The orientation 
    $\begin{bmatrix}
        1 \\ 1 \\ 1
    \end{bmatrix}$
    has the same orientation as the first base since
    \[
        \det \begin{bmatrix}
            1 & 1 & 0 \\
            1 & 0 & 1 \\
            1 & -1 & -1
        \end{bmatrix} 
        = (1) - (-2)
        = 3 > 0
    \]
\newpage

\textcolor{red}{Exercise 6.3.11:}
Let $S \subset \mathbb{R}^4$
be the locus given by the equations
$x_1^2-x_2^2 =x_3$
and $2x_1x_2 = x_4$.
\smallskip

\textbf{a.}
Show that $S$ is a surface.\medskip \\
\textbf{b.}
Find a basis for the tangent space to $S$ at the origin that is direct for the orientation given by Proposition 6.3.9.
\smallskip

\textcolor{blue}{Solution to 6.3.11:} \\
\textbf{a.}
The locus of the surface is
\[
    f(x_1, x_2, x_3, x_4) = \begin{bmatrix}
        x_1^2-x_2^2-x_3 \\ 2x_1x_2 - x_4
    \end{bmatrix}
\].
The derivative is 
\[
    Df(x_1, x_2, x_3, x_4) =
    \begin{bmatrix}
        2x_1 & -2x_2 & -1 & 0 \\
        2x_2 & 2x_1 & 0 & -1
    \end{bmatrix}
\]
From Theorem 3.1.10, $S$ is a smooth manifold since the derivative is onto.

\textbf{b.}
The tangent space consists of points where the derivative is zero,
so the following work as basis vectors .
\[
    \begin{bmatrix}
        1 \\ 0 \\ 0 \\ 0
    \end{bmatrix},
    \begin{bmatrix}
        0 \\ 1 \\ 0 \\ 0
    \end{bmatrix}
\]
At the origin,
\[
    Df(0) =
    \begin{bmatrix}
        0 & 0 & -1 & 0 \\
        0 & 0 & 0 & -1
    \end{bmatrix}
\]
Using the orientation from Proposition 6.3.9,
\[
    \det 
    \begin{bmatrix}
        0 & 0 & 1 & 0 \\
        0 & 0 & 0 & 1 \\
        -1 & 0 & 0 & 0 \\
        0 & -1 & 0 & 0
    \end{bmatrix}
    = 1
\]
Thus this basis is direct at the origin.

\newpage

\textcolor{red}{Exercise 6.3.12:}
Consider the manifold $M\subset \mathbb{R}^4$ of equation
$x_1^2 +x_2^2 +x_3^2 -x_4=0$.
Find a basis for the tangent space to $M$ at the point
$\displaystyle \begin{pmatrix}
1\\0\\0\\1
\end{pmatrix}$
that is direct for the orientation given by Proposition 6.3.9.
\smallskip

\textcolor{blue}{Solution to 6.3.12:}
The derivative of the locus is 
\[
    Df(x_1,x_2,x_3,x_4) = 
    \begin{bmatrix}
        2x_1 & 2x_2 & 2x_3 & -1
    \end{bmatrix}
\]
At the point it is 
\[
    Df(1, 0, 0, 1) = 
    \begin{bmatrix}
        2 & 0 & 0 & -1
    \end{bmatrix}
\]
Thus the tangent space has a basis of 
\[
    \begin{bmatrix}
        -1 \\ 0 \\ 0 \\ 2
    \end{bmatrix},
    \begin{bmatrix}
        0 \\ 1 \\ 0 \\ 0
    \end{bmatrix},
    \begin{bmatrix}
        0 \\ 0 \\ 1 \\ 0
    \end{bmatrix}
\]
The basis is direct since 
\[
    \det 
    \begin{bmatrix}
        2 & -1 & 0 & 0 \\
        0 & 0 & 1 & 0 \\
        0 & 0 & 0 & 1 \\
        -1 & 2 & 0 & 0
    \end{bmatrix}
    = 2(2) - (1) = 3
\]















\end{document}
