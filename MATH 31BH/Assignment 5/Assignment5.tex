%Set document class
\documentclass{report}

%Load math symbol packages
\usepackage{amsmath}
\usepackage{amssymb}

%New commands
\newcommand{\solution}{\textbf{Solution: }}
\newcommand{\inner}[2]{\langle #1, #2 \rangle}

%Declare beginning of document
\begin{document}

%Center to make a document title
\begin{center}
	\huge{\bf Math 31BH: Assignment 5} \\
	Due 02/06 at 23:59 \\
	Merrick Qiu
\end{center}

\bigskip

%Start a numbered list (there are other list formats, such as itemize and description}
\begin{enumerate}

	\item
	Let $f,g \colon \mathbb{R} \to \mathbb{R}^2$ be differentiable functions, and define
	$D\colon \mathbb{R} \to \mathbb{R}$ by $D(t) = \det(f(t),g(t)).$ Prove that
	
		\begin{equation*}
			D'(t)= \det(f'(t),g(t)) + \det(f(t),g'(t)).
		\end{equation*}				
	
	\solution
	Since $D(t) = \det(f(t), g(t))$, we can write using component functions that 
	\[
		D(t) = \det( 
		\begin{bmatrix}
			f_1(t) & g_1(t) \\
			f_2(t) & g_2(t)
		\end{bmatrix}) =
		f_1(t)g_2(t) - f_2(t)g_1(t)
	\]
	Taking the derivative we have that 
	\begin{align*}
		D'(t) 
		&= (f_1(t)g_2(t) - f_2(t)g_1(t))' \\
		&= f_1(t)g_2'(t) + f_1'(t)g_2(t) - f_2(t)g_1'(t) - f_2'(t)g_1(t) \\
		&= (f_1'(t)g_2(t) - f_2'(t)g_1(t)) + (f_1(t)g_2'(t) - f_2(t)g_1'(t)) \\
		&= \det(f'(t),g(t)) + \det(f(t),g'(t))
	\end{align*}

	\medskip
	\item
	Consider a particle moving in $\mathbb{R}^2$ such that its position at time 
	time $t \in \mathbb{R}$ is given by $f(t)=(t^2,t^3).$ 
	
		\begin{enumerate}
		
			\smallskip
			\item
			Calculate the velocity and speed of the particle at time $t$.
			
			\smallskip
			\item
			Show that there is a unique 
			time at which the particle has zero velocity, and calculate its acceleration vector
			at this time.
			
			\smallskip
			\item
			Write down an integral whose value is the distance traveled by the 
			particle between time $t=-1$ and time $t=1.$ (You need not evaluate 
			your integral, but I will be impressed if you do).
			
		\end{enumerate}
	
	\solution
	\begin{enumerate}
		\item 
		The velocity is $f'(t) = (2t, 3t^2)$ 
		and the speed is 
		\[
			\|f'(t)\| =
			\sqrt{(2t)^2 + (3t^2)^2} =
			\sqrt{4t^2 + 9t^4}
		\]

		\item 
		$2t = 0$ and $3t^2 = 0$ only have a solution at $t=0$,
		so the velocity is only zero when $t=0$.
		Since $f''(t) = (2, 6t)$,
		the acceleration is $f''(0) = (2, 0)$.

		\item 
		The integral is 
		\[
			\int_{-1}^1 \sqrt{4t^2 + 9t^4} \,dt
		\]
	\end{enumerate}

	\medskip
	\item
	Consider a particle traveling along a helix in $\mathbb{R}^3$ such that its position 
	at time $t\in \mathbb{R}$ is 
	
		\begin{equation*}
			f(t) = (at,b\cos \omega t, b\sin \omega t),
		\end{equation*}
	
	
	\noindent
	where $a,b,\omega$ are positive constants.
	
		\begin{enumerate}
		
			\smallskip
			\item
			Show that the particle is moving at constant speed.
			
			\smallskip
			\item
			Show that for all times $t$ the acceleration vector $f''(t)$
			is a linear combination of the position vector $f(t)$ and the 
			constant vector $(1,0,0).$
			
			\smallskip
			\item
			Show that the velocity vector $f'(t)$ and the acceleration 
			vector $f''(t)$ are orthogonal for all times $t$.
		
		\end{enumerate}
	\solution 
	\begin{enumerate}
		\item 
		The velocity of the particle is
		$f'(t) = (a, -b\omega\sin\omega t, b\omega\cos\omega t)$
		so the speed is 
		\begin{align*}
			\|f'(t)\|^2
			&= a^2 + b^2\omega^2\sin^2\omega t + b^2\omega^2\cos^2\omega t \\
			&= a^2 + b^2\omega^2(\sin^2\omega t + \cos^2\omega t) \\
			&= a^2 + b^2\omega^2
		\end{align*}
		Since $a,b,\omega$ are constants, speed is constant.

		\item 
		Since 
		$f''(t) = (0, -b\omega^2\cos\omega t, -b\omega^2\sin\omega t)$, 
		we can write
		\[f''(t) = at\omega^2(1,0,0)-\omega^2f(t)\]

		\item The scalar product of $f'(t)$ and $f''(t)$ is 
		\[
			f'(t) \cdot f''(t) =
			0 +
			b^2\omega^3\sin\omega t \cos\omega t - 
			b^2\omega^3\sin\omega t \cos\omega t
			= 0
		\]
		Therefore, the velocity and acceleration are always orthogonal.
	\end{enumerate}
\end{enumerate}
%Declare end of document
\end{document}