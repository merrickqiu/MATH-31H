%Set document class
\documentclass{report}

%Load math symbol packages
\usepackage{amsmath}
\usepackage{amssymb}

%New commands
\newcommand{\solution}{\textbf{Solution: }}
\newcommand{\inner}[2]{\langle #1, #2 \rangle}

%Declare beginning of document
\begin{document}

%Center to make a document title
\begin{center}
	\huge{\bf Math 31BH: Assignment 1} \\
	Due 01/09 at 23:59 \\
	Merrick Qiu
\end{center}

\bigskip

%Start a numbered list (there are other list formats, such as itemize and description}
\begin{enumerate}

	% QUESTION 1
	\item
	Let $(\mathbf{V},\langle \cdot, \cdot\rangle)$ be a Euclidean space, and let 
	$\mathcal{B}=\{\mathbf{e}_1,\dots,\mathbf{e}_n\}$ be an orthonormal basis
	of $\mathbf{V}$. Show that the $n \times 1$ coordinate matrix of any $\mathbf{v} \in \mathbf{V}$ 
	relative to $\mathcal{B}$ is given by
	
		\begin{equation*}
			[\mathbf{v}]_\mathcal{B} = \begin{bmatrix}
				\langle \mathbf{e}_1, \mathbf{v} \rangle \\
				\vdots \\
				\langle \mathbf{e}_n, \mathbf{v} \rangle 
				\end{bmatrix}.
		\end{equation*}
		
	\solution
	Let $v = a_1e_1 + \cdots + a_ne_n$. 
	Using the bilinearity of the inner product 
	and the definition of an orthonormal basis, we have that for a arbitrary index $i$
	\begin{align*}
		\inner{e_i}{v} 
		&= \inner{e_i}{a_1e_1 + \cdots + a_ie_i + \cdots + a_ne_n} \\
		&= a_1\inner{e_i}{e_1} + \cdots + a_i\inner{e_i}{e_i} + \cdots + a_n\inner{e_i}{e_n} \\
		&= a_i
	\end{align*}
	Since $\inner{e_i}{v} = a_i$ for all indexes,
	\[
		[\mathbf{v}]_\mathcal{B} = 
		\begin{bmatrix}
			a_1 \\
			\vdots \\
			a_n
		\end{bmatrix} =
		\begin{bmatrix}
			\langle \mathbf{e}_1, \mathbf{v} \rangle \\
			\vdots \\
			\langle \mathbf{e}_n, \mathbf{v} \rangle 	
		\end{bmatrix}
	\]
	
	% QUESTION 2
	\medskip	
	\item
	With the same notation as in the previous problem, let $A \in \mathrm{End} \mathbf{V}$ be 
	a linear operator. Show that the $n \times n$ matrix of $A$ relative to $\mathcal{B}$ is 
	
		\begin{equation*}
			[A]_\mathcal{B} = \begin{bmatrix}
			\langle \mathbf{e}_1,A\mathbf{e}_1 \rangle & \dots & \langle \mathbf{e}_1,A\mathbf{e}_n \rangle \\
			\vdots & {} & \vdots \\
			\langle \mathbf{e}_n,A\mathbf{e}_1 \rangle & \dots & \langle \mathbf{e}_n,A\mathbf{e}_n \rangle
			\end{bmatrix}.
		\end{equation*}
	
	\solution 
	Let the $i$th row and $j$th column of $[A]_\mathcal{B}$ be $a_{ij}$.
	\begin{align*}
		\inner{e_i}{Ae_j} 
		&= \inner{e_i}{a_{1j}e_1 + \cdots + a_{ij}e_i + \cdots + a_{nj}e_n} \\
		&= a_{1j}\inner{e_i}{e_1} + \cdots + a_{ij}\inner{e_i}{e_i} + \cdots + a_{nj}\inner{e_i}{e_n} \\
		&= a_{ij}
	\end{align*}
	Since $\inner{e_i}{Ae_j} = a_{ij}$ for all indexes, 
	\begin{equation*}
		[A]_\mathcal{B} = 
		\begin{bmatrix}
			a_{11} & \dots & a_{1n} \\
			\vdots & {} & \vdots \\
			a_{n1} & \dots & a_{nn}
		\end{bmatrix} =
		\begin{bmatrix}
			\langle \mathbf{e}_1,A\mathbf{e}_1 \rangle & \dots & \langle \mathbf{e}_1,A\mathbf{e}_n \rangle \\
			\vdots & {} & \vdots \\
			\langle \mathbf{e}_n,A\mathbf{e}_1 \rangle & \dots & \langle \mathbf{e}_n,A\mathbf{e}_n \rangle
		\end{bmatrix}
	\end{equation*}

	% QUESTION 3
	\medskip
	\item
	With the same notation as in the previous problems, for each $1 \leq i,j \leq n$ let $E_{ij} \in \mathrm{End}\mathbf{V}$ 
	be the linear operator defined by 
	
		\begin{equation*}
			E_{ij} \mathbf{e}_k = \langle \mathbf{e}_j,\mathbf{e}_k \rangle \mathbf{e}_i, \quad 1 \leq k \leq n.
		\end{equation*}
			
	\noindent
	What is the matrix of $E_{ij}$ relative to $\mathcal{B}$?
	
	\solution 
	Since the basis is orthonormal,
	the expression $E_{ij}e_k$ is equal to $e_i$ when $j=k$ and 0 when $j \neq k$.
	In other words, $E_{ij}$ sends the basis vector $e_j$ to the basis vector $e_i$.
	Thus the matrix of $E_{ij}$ will contain a 1 at the index $(i,j)$ and 0 everywhere else. 
	% QUESTION 4
	\medskip
	\item
	With the same notation as in the previous problems, prove that 
	
		\begin{equation*}
			E_{ij} E_{kl} = \langle \mathbf{e}_j,\mathbf{e}_k \rangle E_{il}
		\end{equation*}
		
	\noindent
	Deduce from this that $\mathcal{E} = \{ E_{ij} \colon 1 \leq i , j \leq n\}$
	is an orthonormal basis of $\mathrm{End} \mathbf{V}$, where by definition the scalar product of
	two operators $A,B \in \mathrm{End}\mathbf{V}$ is
	$\langle A,B \rangle = \operatorname{Tr} A^*B$, with $\operatorname{Tr}$ the trace and $A^*$ the
	adjoint (aka transpose) of $A$. What is $\dim \mathrm{End}\mathbf{V}$?

	\solution
	Let $m$ be a arbitrary index.
	We have that 
	\begin{align*}
		E_{ij}E_{kl}e_m
		&= E_{ij}\inner{e_l}{e_m}e_k \\
		&= \inner{e_l}{e_m}E_{ij}e_k \\
		&= \inner{e_l}{e_m}\inner{e_j}{e_k}e_i \\
		&= \inner{e_j}{e_k}\inner{e_l}{e_m}e_i \\
		&= \inner{e_j}{e_k}E_{il}e_m
	\end{align*}
	Since $E_{ij}E_{kl}$ is a linear transformation and it sends all
	basis vectors to the same element as $\inner{e_j}{e_k}E_{il}$,
	we have that
	$E_{ij}E_{kl} = \inner{e_j}{e_k}E_{il}$.

	The elements in $\mathcal{E}$ can be represented by 
	$n \times n$ matrices with one 1 and 0 everywhere else;
	these matrices are linearly independent and also span
	all endomorphisms of $V$ since $\mathrm{End} \mathbf{V}$ 
	can be represented as all $n \times n$ matrices.
	Since 
	\begin{align*}
		 \inner{E_{ij}}{E_{kl}} 
		 &= \operatorname{Tr} E_{ij}^* E_{kl} \\
		 &= \operatorname{Tr} E_{ji} E_{kl} \\
		 &= \operatorname{Tr} \inner{e_i}{e_k}E_{jl}
	\end{align*}
	$\inner{E_{ij}}{E_{kl}}$ is equal to 1 iff $i = k$ and $j = l$
	and 0 otherwise. Therefore $\mathcal{E}$ is an orthonormal basis.
	The dimension of $\mathrm{End} \mathbf{V}$ is $n \cdot n = n^2$.

	% QUESTION 5
	\medskip
	\item
	With the same notation as in the previous problems, prove that 
	$\mathcal{S} = \{ E_{ij} + E_{ji} \colon 1 \leq i \leq j \leq n\}$ is an orthogonal basis of the subspace
	$\mathrm{Sym}\mathbf{V}$ of $\mathrm{End}\mathbf{V}$ consisting of symmetric operators. 
	What is $\dim \mathrm{Sym}\mathbf{V}$?

	\solution 
   The elements in $\mathcal{S}$ can be represented by symmetric
   $n \times n$ matrices with two 1s off the diagonal and 0 everywhere else
   or a 2 on the diagonal and 0 everywhere else.
   Every $n \times n$ symmetric matrix $A \in \mathrm{Sym}\mathbf{V}$ with entries $a_{ij}$
   can then be written as 
   \[
	   \sum_{i=1}^n \sum_{j=i+1}^n a_{ij}(E_{ij}+E_{ji}) + \sum_{i=1}^n \frac{a_{ii}}{2} (E_{ii}+E_{ii})
   \]
   Therefore, $\mathcal{S}$ spans $\mathrm{Sym}\mathbf{V}$.
   
   Since
   \begin{align*}
	   \inner{E_{ij}+E_{ji}}{E_{kl}+E_{lk}} 
	   &= \operatorname{Tr} (E_{ij}+E_{ji})^* (E_{kl}+E_{lk}) \\
	   &= \operatorname{Tr} (E_{ij}+E_{ji}) (E_{kl}+E_{lk}) \\
	   &= \operatorname{Tr} E_{ij}E_{kl} + E_{ij}E_{lk} + E_{ji}E_{kl} + E_{ji}E_{lk} \\
	   &= \operatorname{Tr} \inner{e_j}{e_k}E_{il} + \inner{e_j}{e_l}E_{ik} +\inner{e_i}{e_k}E_{jl} + \inner{e_i}{e_l}E_{jk}
  \end{align*}
  $\inner{E_{ij}+E_{ji}}{E_{kl}+E_{lk}}$ is nonzero iff 
  $i=k$ and $j=l$ or $i=l$ and $j=k$, 
  both of which would indicated that $E_{ij}+E_{ji} = E_{kl}+E_{lk}$.
  Therefore $\mathcal{S}$ is orthogonal, which implies it is also linearly independent.
  Since, $\mathcal{S}$ also spans $\mathrm{Sym}\mathbf{V}$, 
  $\mathcal{S}$ is an orthogonal basis for $\mathrm{Sym}\mathbf{V}$.
  
  The dimension of $\mathrm{Sym} \mathbf{V}$ is 
  the number of upper triangular elements in the $n \times n$ matrix representation,
  so the dimension is $1 + \dots + n = \frac{n(n+1)}{2}$.
   
\end{enumerate}

%Declare end of document
\end{document}