%Set document class
\documentclass{report}

%Load math symbol packages
\usepackage{amsmath}
\usepackage{amssymb}

%New commands
\newcommand{\solution}{\textbf{Solution: }}
\newcommand{\inner}[2]{\langle #1, #2 \rangle}
\newcommand{\pdiv}[1]{\frac{\partial}{\partial #1}}

%Declare beginning of document
\begin{document}

\bigskip

%Start a numbered list (there are other list formats, such as itemize and description}
For all $a \in \mathbb{R}$ and $n \in \mathbb{Z^+}$, the function must have
\[
	f'(a) 
	= \frac{f(a+n)-f(a)}{n} 
	= \frac{\int_a^{a+n} f'(x) \,dx}{n}
\]
Thus, the derivative must be periodic with interval 1.
Assume that the derivative is not constant,
meaning there exists $b$ such that $f'(b) \neq f'(a)$.
This implies
\[
	f'(a) 
	= \frac{\int_a^{a+n} f'(x) \,dx}{n} 
	\neq \frac{\int_b^{b+n} f'(x) \,dx}{n}
	= f'(b)
\]
However this cannot be true since $f'$ is periodic and 
the size of the intervals of integration are equal and a multiple of the period of $f'$.
Thus the derivative must be constant, 
implying that the only class of functions that satisfy the condition are linear functions.
%Declare end of document
\end{document}