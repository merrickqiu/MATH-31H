%Set document class
\documentclass{report}

%Load math symbol packages
\usepackage{amsmath}
\usepackage{amssymb}

%New commands
\newcommand{\solution}{\textbf{Solution: }}
\newcommand{\inner}[2]{\langle #1, #2 \rangle}

%Declare beginning of document
\begin{document}

%Center to make a document title
\begin{center}
	\huge{\bf Math 31BH: Midterm} \\
	Due 02/07 at 16:00 \\
	Merrick Qiu
\end{center}

\bigskip

%Start a numbered list (there are other list formats, such as itemize and description}
\begin{enumerate}
	\item 
	\begin{enumerate}
		\item 
		For any arbitrary $t \in \mathbb{R}$,
		\begin{align*}
			f(t)
			&= (2\cos t, \sin t) \\
			&= (2\cos (t+2\pi), \sin (t+2\pi)) \\
			&= f(t+2\pi)
		\end{align*}
		Therefore, $f(t) = f(t+2\pi)$ for all $t \in \mathbb{R}$.

		\item 
		Let $v \in \operatorname{Im} f$ be arbitrary.
		Therefore $v = (2\cos t, \sin t)$ for some $t$.
		Since 
		\[
			\frac{1}{4}(2\cos t)^2 + \sin^2 t
			= \cos^2 t + \sin^2 t 
			= 1
		\]
		$v \in C$, so $\operatorname{Im} f \subseteq C$
		since $v$ was arbitrary. \\

		Let $v \in C$ be arbitrary in the form $ v=(x, y)$.
		Let $t = \arccos \frac{x}{2}$;
		$t$ exists since $\|\frac{x}{2}\|$ cannot be greater than 1.
		Since  $\frac{1}{4}x^2 + y^2 = 1$, then 
		$\frac{1}{2}x = \sqrt{1-y^2}$ .
		Using the identity that $\cos(\arcsin(x)) = \sqrt{1-x^2}$,
		\[
			\arccos \frac{x}{2} 
			= \arccos \sqrt{1-y^2}
			= \arcsin y
		\] 
		Therefore $t = \arccos \frac{x}{2} = \arcsin y$ meaning that $f(t) = (x, y)$
		and $v$ is in the image of $f$.
		Since $v \in \operatorname{Im} f$ and $v$ was arbitrary, 
		$C \subseteq \operatorname{Im} f$.
		Since $\operatorname{Im} f \subseteq C$ and $C \subseteq \operatorname{Im} f$,
		$\operatorname{Im} f = C$. The curve of $f$ is an ellipse.

		\item 
		The function $g(t) = (t, \sqrt{1-\frac{1}{4}t^2})$
		is a non-periodic parameterization of $C$ since 
		\[
			\frac{1}{4}x^2 + \sqrt{1-\frac{1}{4}x^2}^2 = 1
		\]
		and
		\[
			(t, \sqrt{1-\frac{1}{4}x^2}) \neq (t+p, \sqrt{1-\frac{1}{4}(x+p)^2})
		\]
		for any nonzero $p \in \mathbb{R}$.
	\end{enumerate}
	
	\item 
	\begin{enumerate}
		\item 
		The component functions for the basis of elementary matrices are 
		\begin{align*}
			f_{11}(t) &= e^t    & f_{12}(t) &= e^{2t} \\
			f_{21}(t) &= e^{3t} & f_{22}(t) &= e^{4t}
		\end{align*}

		\item 
		$f(t)$ is smooth if all of its component functions are smooth.
		Since all the component functions are in the form $e^{kt}$ for some constant $k$,
		the component functions have derivative $ke^{kt}$. 
		Therefore the derivative exists for all $t \in \mathbb{R}$
		and the derivative is continuous since the second derivative, $k^2e^{kt}$, exists.
		Thus, the component functions of $f$ are continuously differentiable. \\

		The component function derivatives are nonvanishing since 
		$ke^{kt}$ does not ever equal zero. 
		Therefore the component functions are smooth and thus $f(t)$ is a smooth curve.

		\item 
		\[
			L(f) 
			= \int_0^1 \|f'(t)\| \,dt
			= \int_0^1 \sqrt{(e^t)^2 + (2e^{2t})^2 + (3e^{3t})^2 + (4e^{4t})^2} \,dt
		\]
	\end{enumerate}

	\item 
	\begin{enumerate}
		\item 
		Using the chain rule on the component functions, 
		\begin{align*}
			g'(t)
			&= (f(t) \cdot f(t))' \\
			&= (\sum_{i=1}^m f_i(t)^2)' \\
			&= \sum_{i=1}^m 2f_i(t) f_i'(t)\\
			&= 2f(t) \cdot f'(t) 
		\end{align*}

		\item 
		Let $f: \mathbb{R} \rightarrow \mathbb{R}^3$ 
		represent the position of a particle on a sphere through time.
		Let $g(t) = f(t) \cdot f(t)$.
		Since the particle is moving in a sphere, 
		the norm of the position must be a constant,
		so  $g'(t) = 0$.
		Since $g'(t) = 2f(t) \cdot f'(t)$,
		then $f(t) \cdot f'(t) = 0$, so position and velocity are always orthogonal.

		\item 
		Let $f: \mathbb{R} \rightarrow \mathbb{R}^3$ 
		represent the velocity of a particle moving at a constant speed.
		Let $g(t) = f(t) \cdot f(t)$.
		Since the speed is constant,
		the norm of the velocity must be a constant,
		so $g'(t) = 0$.
		Since $g'(t) = 2f(t) \cdot f'(t)$,
		then $f(t) \cdot f'(t) = 0$, so velocity and acceleration are always orthogonal.
		 
	\end{enumerate}
\end{enumerate}
%Declare end of document
\end{document}