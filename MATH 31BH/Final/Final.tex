%Set document class
\documentclass{report}

%Load math symbol packages
\usepackage{amsmath}
\usepackage{amssymb}

%New commands
\newcommand{\solution}{\textbf{Solution: }}
\newcommand{\inner}[2]{\langle #1, #2 \rangle}
\newcommand{\pdiv}[1]{\frac{\partial}{\partial #1}}

%Declare beginning of document
\begin{document}

%Center to make a document title
\begin{center}
	\huge{\bf Math 31BH: Final} \\
	Due 03/16 at 18:30 \\
	Merrick Qiu
\end{center}

\bigskip

%Start a numbered list (there are other list formats, such as itemize and description}
\begin{enumerate}
	\item 
	Let $\epsilon > 0$ be arbitrary.
	Let $\delta = \frac{\epsilon}{C}$.
	For $v \in V$ with $\|v\| < \delta$, 
	then $\|f(v)\| < \epsilon$ since
	\[
		\|f(v)\| \leq 
		C\|v\| <
		C \delta =
		C \frac{\epsilon}{C} =
		\epsilon
	\]
	Therefore, $f$ is continuous because for all $\epsilon > 0$,
	there exists a $\delta > 0$ such that 
	$\|v\| < \delta$ implies $\|f(v)\| < \epsilon$.

	\item 
	The value of $f$ at $\frac{\pi}{3}$ is
	\[
		f(\frac{\pi}{3}) = (\frac{1}{2}, \frac{\sqrt{3}}{2})
	\]
	
	The derivative of $f$ at $\frac{\pi}{3}$ is 
	\[
		f'(t) = (-\sin(t), \cos(t))
	\]
	\[
		f'(\frac{\pi}{3}) = (-\frac{\sqrt{3}}{2}, \frac{1}{2})
	\]
	Using $r$ as a parameter, a tangent line at $t$ can be written in form 
	$\{f(t) + rf'(t) : r \in \mathbb{R}\}$. 
	Therefore the tangent line is 
	$\{(\frac{1}{2}, \frac{\sqrt{3}}{2}) + r(-\frac{\sqrt{3}}{2}, \frac{1}{2}) : r \in \mathbb{R}\}$.
	Written as an equation it is
	$g(r) = (\frac{1}{2}-\frac{\sqrt{3}}{2}r, \frac{\sqrt{3}}{2} + \frac{1}{2}r)$.

	\item 
	\begin{enumerate}
		\item 
		The velocity is the derivative of the position.
		Therefore the velocity at time $t$ is 
		\[
			f'(t) = 
			(a, -b\omega\sin\omega t, b\omega\cos\omega t)
		\]

		\item 
		The speed of a particle is the magnitude of its veloicty.
		Therefore the speed at time $t$ is 
		\[
			\|f'(t)\| = 
			\sqrt{a^2 + b^2\omega^2\sin^2\omega t + b^2\omega^2\cos^2\omega t}
		\]

		\item 
		The acceleration is the derivative of the velocity.
		Therefore the acceleration at time $t$ is 
		\[
			f''(t) = 
			(0, -b\omega^2\cos\omega t, -b\omega^2\sin\omega t)
		\]
		The dot product of the acceleration and velocity is 
		\begin{align*}
			f'(t) \cdot f''(t)  
			&= (-b\omega\sin\omega t)(-b\omega^2\cos\omega t) + 
				(b\omega\cos\omega t)(-b\omega^2\sin\omega t) \\
			&= b^2\omega^3 \sin\omega t\cos\omega t - 
				b^2\omega^3 \sin\omega t\cos\omega t \\ 
			&= 0
		\end{align*}
		Since the dot product of acceleration and velocity is zero 
		for $t$, the acceleration and velocity are orthogonal at time $t$.
	\end{enumerate}

	\item 
	The gradient of $f_1$ is a vector of the partial derivatives of $f_1$.
	Using the chain rule to find the partial derivatives, 
	\begin{align*}
		\nabla f_1(x,y) 
		&= (\pdiv{x} g_1(x+y), \pdiv{y} g_1(x+y)) \\
		&= (g_1'(x+y)\pdiv{x} (x+y), g_1'(x+y)\pdiv{y} (x+y)) \\
		&= (g_1'(x+y), g_1'(x+y))
	\end{align*}

	Similarly for $f_2$, 
	\begin{align*}
		\nabla f_2(x,y) 
		&= (\pdiv{x} g_2(x-y), \pdiv{y} g_2(x-y)) \\
		&= (g_2'(x-y)\pdiv{x} (x-y), g_2'(x+y)\pdiv{y} (x-y)) \\
		&= (g_2'(x-y), -g_2'(x-y))
	\end{align*}

	The dot product of the gradients is 
	\[
		\nabla f_1(x,y) \cdot \nabla f_2(x,y)
		= g_1'(x+y)g_2'(x-y) - g_1'(x+y)g_2'(x-y)
		= 0
	\]
	Since the dot product of the gradients is 0 
	for all $x$ and $y$,
	the gradient of $f_1$ is orthogonal to the gradient of $f_2$
	at every point in $\mathbb{R}^2$.

	\item 
	Writting v as its components, the function is 
	\[
		f(v_1,...,v_n) = (v_1^2 + ... + v_n^2)^a
	\]
	The partial derivative for an arbitrary component variable is 
	\begin{align*}
		\pdiv{v_i} f(v_1,..., v_i,...,v_n) 
		&= \pdiv{v_1} (v_1^2 + ... + v_i^2 + ...+ v_n^2)^a \\
		&= a(v_1^2 + ... + v_i^2 + ...+ v_n^2)^{a-1}2v_i \\
		&= 2a(v\cdot v)^{a-1}v_i
	\end{align*}
	The gradient vector is therefore 
	\[
		\nabla f(v) = 
		(2a(v\cdot v)^{a-1}v_1,...,2a(v\cdot v)^{a-1}v_n) =
		2a(v\cdot v)^{a-1}v
	\]

	\item 
	\begin{enumerate}
		\item 
		Since a convex hull is compact, 
		the extreme value theorem says that a maximizer
		exists in $S$.

		\item 
		The gradient of $f$ is
		\[
			\nabla f(x, y) = (3x^2+y, x)
		\]
		The gradient is only zero at the point $(0, 0)$.
		Therefore $(0, 0)$ is the only critical point, which is on the boundary.
		Therefore it is sufficient to only check the boundaries for a maximum.

		For the left side $f(0, y) = 0$,
		which achieves a maximum of $0$ for all points.

		For the right side $f(1, y) = 1+y$,
		which achieves a maximum of $2$ at $y=1$.

		For the bottom side $f(x, 0) = x^3$, 
		which achieves a maximum of $1$ at $x=1$.

		For the top side $f(x, 1) = x^3 + 1$,
		which achieves a maximum of $2$ at $x=1$.

		Therefore, $f$ has a maximum value of $2$ 
		at the point $(1, 1)$ on $S$. 
	\end{enumerate}
		
	\item 
	\begin{enumerate}
		\item 
		Since $f(0,0) = (1, 0)$ and $f(0, 2\pi) = (1, 0)$,
		the function is not injective, and therefore it is not invertible.

		\item 
		The Jacobian matrix is the matrix of the possible derivatives 
		on the component functions.
		\[
			J_f(x,y) 
			= \begin{bmatrix}
				\pdiv{x} e^x\cos y & \pdiv{y} e^x\cos y \\
				\pdiv{x} e^x\sin y & \pdiv{y} e^x\sin y
			\end{bmatrix} \\
			= \begin{bmatrix}
				e^x\cos y & -e^x\sin y \\
				e^x\sin y & e^x\cos y
			\end{bmatrix}
		\]

		\item 
		From the inverse function theorem,
		$f$ is locally invertible if $\det(f'(v,\cdot)) \neq 0$
		for a point $v \in \mathbb{R}^2$.
		The determinant is 
		\begin{align*}
			\det(f'(v,\cdot))  
			&= \det (J_f(x,y)) \\
			&= (e^x\cos y)(e^x\cos y) - (-e^x\sin y)(e^x\sin y) \\
			&= e^{2x}\cos^2y + e^{2x}\sin^2y \\
			&= e^{2x}(\cos^2y + \sin^2y) \\
			&= e^{2x} \\
			&\neq 0 \text{ for all $x$}
		\end{align*}		
		Since the determinant of the derivative is never zero,
		$f$ is locally invertible for any point $v \in \mathbb{R}^2$.
	\end{enumerate}
\end{enumerate}
%Declare end of document
\end{document}