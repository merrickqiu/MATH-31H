%Set document class
\documentclass{report}

%Load math symbol packages
\usepackage{amsmath}
\usepackage{amssymb}

%New commands
\newcommand{\solution}{\textbf{Solution: }}
\newcommand{\inner}[2]{\langle #1, #2 \rangle}
\newcommand{\pdiv}[1]{\frac{\partial}{\partial #1}}

%Declare beginning of document
\begin{document}

%Center to make a document title
\begin{center}
	\huge{\bf Math 31BH: Assignment 8} \\
	Due 03/06 at 23:59 \\ 
	Merrick Qiu
\end{center}

\bigskip

%Start a numbered list (there are other list formats, such as itemize and description}
\begin{enumerate}

	\medskip
	\item
	Let $C$ be a plane curve, i.e. the image of a continuous function 
	$f \colon \mathbb{R} \to \mathbb{R}^2$. What are the boundary points of $C$?
	Is $C$ open, closed, neither, both?
	
	\solution 
	The boundary points of $C$ are the points, $v$, for which 
	the ball $B_\epsilon(v)$ contains both points in the image and outside the image 
	for every $\epsilon > 0$.
	Unless $C$ is space filling, $C$ will be a closed curve.
  
	\medskip
	\item
	Maximize the function $f(x,y) = x^2e^{-(x^4+y^2)}$ over $\mathbb{R}^2$.
	
	\solution 
	For $v = (x, y)$
	\[
		x^2e^{-(x^4+y^2)} \leq (x^2+y^2)e^{-(x^2+y^2)} = \|v\| e^{-\|v\|}
	\]
	The function is also always nonnegative 
	since $x^2$ and $e^{-(x^4+y^2)}$ are nonnegative.
	Therefore
	\[
		0 \leq 
		\lim_{\|v\| \to \infty} x^2e^{-(x^4+y^2)} \leq  
		\lim_{\|v\| \to \infty} \|v\| e^{-\|v\|} = 0
	\]
	Since $\lim_{\|v\| \to \infty} x^2e^{-(x^4+y^2)} = 0$ and the function is smooth,
	the maximum of the function must be a critical point. \\

	The partial derivative with respect to $x$ is
	\[
		\pdiv{x} x^2e^{-(x^4+y^2)} = 
		x^2 e^{-x^4-y^2} (-4x^3) + 2xe^{-x^4-y^2)} =
		2xe^{-x^4-y^2}(1-2x^4)
	\]
	The left-hand term is only zero when $x=0$ and 
	the right hand term is only zero when $x= \pm \sqrt[4]{\frac{1}{2}}$.
	Therefore, the partial derivative with respect to $x$ is only zero for 
	points in the form $(0, y)$ or $(\pm \sqrt[4]{\frac{1}{2}}, y)$.
	The partial derivative with respect to $y$ is
	\[
		\pdiv{y} x^2e^{-(x^4+y^2)} = 
		x^2e^{-x^4-y^2}(-2y) =
		-2x^2y e^{-x^4-y^2}
	\]
	This is only zero when $x=0$ or $y=0$.
	Therefore, the partial derivative with respect to $y$ is only zero for 
	points in the form $(0, y)$ or $(x, 0)$. 
	The critical points are thus points in the form 
	$(0, y)$ or  $(\pm \sqrt[4]{\frac{1}{2}}, 0)$.

	\begin{align*}
		f(0, y) &= 0 \\
		f(\sqrt[4]{\frac{1}{2}}, 0) &= \sqrt{\frac{1}{2}}e^{-\frac{1}{2}} \\
		f(-\sqrt[4]{\frac{1}{2}}, 0) &= \sqrt{\frac{1}{2}}e^{-\frac{1}{2}} \\
	\end{align*}

	Therefore the global maximizer of $f$ 
	are the points $(\pm \sqrt[4]{\frac{1}{2}}, 0)$
	with a value of $\sqrt{\frac{1}{2}}e^{-\frac{1}{2}}$ each.

	\medskip
	\item
	Find the maximum of the function $f(x,y) = x^3+xy$ on the unit square, 
	and on the square with vertices $(\pm 1, \pm 1)$ and $(\pm 1, \mp 1)$.

	\solution 
	I will find the maximum on the square with vertices 
	$(\pm 1, \pm 1)$ and $(\pm 1, \mp 1)$ first.
	Since the square is a compact set, 
	the maximizer is either a critical point or a point on the boundary.
	The partial derivative with respect to $x$ is 
	\[
		\pdiv{x} x^3 + xy = 3x^2 + y
	\]
	This is zero for points in the form $(x, -3x^2)$.
	The partial derivative with respect to $y$ is 
	\[
		\pdiv{y} x^3 + xy = x
	\]
	This is zero for points in the form $(0, y)$
	The only critical point is therefore $(0, 0)$
	This critical point has a value of $f(0, 0) = 0$.

	Points on the top edge take the form $(x, 1)$.
	The expression $x^3 + x$, corresponding to the top edge values,
	is maximized when $x=1$ with a value of 2.

	Points on the bottom edge take the form $(x, -1)$.
	The expression $x^3 - x$, corresponding to the bottom edge values,
	is maximized when $x= -\frac{1}{\sqrt{3}}$ 
	with a value of $\frac{1}{3}^\frac{1}{2} - \frac{1}{3}^\frac{3}{2}$.

	Points on the right edge take the form $(1, y)$.
	The expression $1 + y$, corresponding to the right edge values,
	is maximized when $y = 1$ with a value of 2 

	Points on the left edge take the form $(-1, y)$.
	The expression $-1-y$, corresponding to the left edge values,
	is maximized when $y=-1$ with a value of 0.

	From this, we can see that the maximizer is the point $(1, 1)$ 
	with a value of 2. 
	If only the unit square is considered, 
	$(1, 1)$ is still the maximizer since 
	the unit square is a subset of the square centered at the origin,
	and $(1, 1)$ is in the unit square.
	\medskip
	\item
	Find the terms of order at most two in the Taylor expansion of $f(x,y)=\log(1+xy)$ at the point
	$(0,0)$.
	
	\solution 
	The value at $(0, 0)$ is $f(0, 0) = 0$
	The gradient is $G_f(x, y) = [\frac{y}{1+xy}, \frac{x}{1+xy}]$
	from taking the partial derivatives. 
	The gradient at $(0, 0)$ is thus $G_f(0, 0) = [0, 0]$.
	The Hessian is the matrix of the partial derivatives of the partial derivatives.
	\begin{align*}
		H_f(x, y) 
		= 	\begin{bmatrix}
				\pdiv{x}\pdiv{x} f & \pdiv{x}\pdiv{y} f \\
				\pdiv{y}\pdiv{x} f & \pdiv{y}\pdiv{y} f
			\end{bmatrix} \\
		= 	\begin{bmatrix}
				\pdiv{x}\frac{y}{1+xy} & \pdiv{x}\frac{x}{1+xy} \\
				\pdiv{y}\frac{y}{1+xy} f & \pdiv{y}\frac{x}{1+xy}
			\end{bmatrix} \\
		= 	\begin{bmatrix}
			-\frac{y^2}{(1+xy)^2} & \frac{1}{(1+xy)^2} \\
			\frac{1}{(1+xy)^2} & -\frac{x^2}{(1+xy)^2}
		\end{bmatrix} \\
	\end{align*}
	Therefore,
	\[
		H_f(0, 0) =
		\begin{bmatrix}
			0 & 1 \\
			1 & 0
		\end{bmatrix}
	\]

	Therefore, 
	\begin{align*}
		P_2(x, y) 
		&= f(0, 0) + \inner{G_f(0, 0)}{v} + \frac{1}{2}\inner{H_f(0, 0)v}{v} \\
		&= 0 + 0 + \frac{1}{2}\inner{[y, x]}{[x, y]} \\
		&= xy
	\end{align*}
	\medskip
	\item
	Repeat the previous problem with $f(x,y) = e^{x+y}$.

	\solution
	The value at $(0,0)$ is $f(0,0) = 1$
	The gradient is $G_f(x, y) = [e^{x+y}, e^{x+y}]$ 
	since the partial derivatives of $f$ are the function itself. 
	Therefore the gradient at at $(0,0)$ is $G_f(0, 0) = [1, 1]$.
	Because the partial derivatives are still $f$,
	the hessian of $f$ is the matrix 
	\[
	H_f(x, y) =
	\begin{bmatrix}
		e^{x+y} & e^{x+y} \\
		e^{x+y} & e^{x+y}
	\end{bmatrix}
	\]
	Therefore the hessian at $(0, 0)$ is
	\[
	H_f(0, 0) =
	\begin{bmatrix}
		1 & 1 \\
		1 & 1
	\end{bmatrix}
	\]

	Therefore 
	\begin{align*}
		P_2(x, y) 
		&= f(0, 0) + \inner{G_f(0, 0)}{v} + \frac{1}{2}\inner{H_f(0, 0)v}{v} \\
		&= 1 + (x + y) + \frac{1}{2}\inner{[x+y, x+y]}{[x, y]} \\
		&= 1 + (x + y) + \frac{1}{2}(x^2+2xy+y^2)
	\end{align*}
	\end{enumerate}
	
%Declare end of document
\end{document}