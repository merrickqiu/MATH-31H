%Set document class
\documentclass{report}

%Load math symbol packages
\usepackage{amsmath}
\usepackage{amssymb}

%New commands
\newcommand{\solution}{\textbf{Solution: }}
\newcommand{\inner}[2]{\langle #1, #2 \rangle}
\newcommand{\pdiv}[1]{\frac{\partial}{\partial #1}}

%Declare beginning of document
\begin{document}

%Center to make a document title
\begin{center}
	\huge{\bf Math 31BH: Assignment 7} \\
	Due 02/27 at 23:59 \\
	Merrick Qiu
\end{center}

\bigskip

%Start a numbered list (there are other list formats, such as itemize and description}
\begin{enumerate}

	\medskip
	\item
	Consider the function of three variables defined by $f(x,y,z) = x^2y\sin(yz)$.
	
		\begin{enumerate}
		
		\smallskip
		\item
		Show that $f$ is differentiable on $\mathbb{R}^3$.
		
		\smallskip
		\item
		Calculate the gradient vector $\nabla f(1,-1,\pi)$.
		
		\smallskip
		\item
		Write down a formula for the derivative 
		$f'((1,-1,\pi),(x,y,z))$.
		
		\end{enumerate}
		
	\solution 
	\begin{enumerate}
		\item 
		$f$ is differentiable on $\mathbb{R}^3$
		since the partial derivatives exist.
		\begin{align*}
			&\pdiv{x} x^2y\sin(yz) = 2xy\sin(yz) \\
			&\pdiv{y} x^2y\sin(yz) = x^2yz\cos(yz) + x^2\sin(yz) \\
			&\pdiv{z} x^2y\sin(yz) = x^2y^2\cos(yz) 
		\end{align*}

		\item 
		The gradient is the vector of the partial derivatives.
		\begin{align*}
			&\nabla f(x,y,z) 
			(2xy\sin(yz), x^2yz\cos(yz) + x^2\sin(yz), x^2y^2\cos(yz)) \\
			&\nabla f(1,-1, \pi) = 
			(-2\sin(-\pi), -\pi\cos(-\pi) + \sin(-\pi), \cos(-\pi)) =
			(0, \pi, -1)
		\end{align*}
			
		\item 
		The derivative can be represented as the scalar product between 
		the gradient and $(x,y,z)$. 
		\[
			f'((1, -1, \pi),(x, y, z)) =
			\nabla f(1, -1, \pi) \cdot (x, y, z) =
			\pi y-z
		\]
	\end{enumerate}

	\medskip
	\item
	Find the partial derivatives of $f(x,y) = x^y$.

	\solution 
	The partial derivative is computed by treating other variables as constants.
	\begin{align*}
		&\pdiv{x} x^y = yx^{y-1} \\
		&\pdiv{y} x^y = \ln(x) x^y
	\end{align*}
	
	\medskip
	\item
	Let $f(x,y) = x^2+y^3$. Find the directional derivative of 
	$f$ at $\mathbf{v}=(-1,3)$ in the direction of maximal increase
	of $f$.

	\solution 
	The partial derivatives are 
	\begin{align*}
		&\pdiv{x} x^2+y^3 = 2x \\
		&\pdiv{y} x^2+y^3 = 3y^2
	\end{align*}
	Therefore, the gradient is $w=(-2, 27)$
	Therefore, 
	\[
		f'(v,w) = \nabla f(v) \cdot w = (-2, 27) \cdot (-2,27) = 733
	\]
	Since the question is finding the directional derivative,
	we divide by the norm of the gradient so,
	$f'(v, e) =  \sqrt{733}$.

	\medskip
	\item
	Let $f$ be a differentiable function defined on an open set $D$ 
	in a Euclidean space $\mathbf{V}$. Suppose that $\mathbf{m} \in D$
	is a maximum of $f$, i.e. $f(\mathbf{m}) \geq f(\mathbf{v})$ for all
	$\mathbf{v} \in D$. Prove that $\nabla f(\mathbf{m}) = \mathbf{0}_\mathbf{V}$,
	the zero vector in $\mathbf{V}.$

	\solution
	Suppose that $\nabla f(m) \neq 0_v$.
	This means that 
	\[
		f'(m, \nabla f(m)) =
		\nabla f(m) \cdot \nabla f(m) 
		> 0
	\]
	Since
	\[
		f'(m, \nabla f(m)) = 
		\lim_{h \to 0} \frac{f(m+h\nabla f(m))-f(m)}{h} =
		\lim_{h \to 0^+} \frac{f(m+h\nabla f(m))-f(m)}{h} 
		> 0
	\]
	\begin{align*}
		\lim_{h \to 0^+} \frac{f(m+h\nabla f(m))-f(m)}{h} > 0
		&\implies \lim_{h \to 0^+} f(m+h\nabla f(m))-f(m) > 0 \\ 
		&\implies \lim_{h \to 0^+} f(m+h\nabla f(m)) > f(m) \\ 
	\end{align*}
	there exists an $h > 0$ such that $f(m+h\nabla f(m)) > f(m)$.	
	This contradicts the fact that $f(m)$ is the maximum, so 
	$\nabla f(m)$ must be $0_v$.
	
	\end{enumerate}
	
%Declare end of document
\end{document}