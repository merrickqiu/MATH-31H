%Set document class
\documentclass{report}

%Load math symbol packages
\usepackage{amsmath}
\usepackage{amssymb}

%New commands
\newcommand{\solution}{\textbf{Solution: }}
\newcommand{\inner}[2]{\langle #1, #2 \rangle}
\newcommand{\pDiv}[1]{\frac{\partial}{\partial #1}}

%Declare beginning of document
\begin{document}

%Center to make a document title
\begin{center}
	\huge{\bf Math 31BH: Assignment 6} \\
	Due 02/20 at 23:59 \\ 
	Merrick Qiu
\end{center}

\bigskip

%Start a numbered list (there are other list formats, such as itemize and description}
\begin{enumerate}

	\medskip
	\item
	Given an example of a function $f \colon \mathbb{R}^2 \to \mathbb{R}$ which 
	is differentiable at $(0,0)$ in the direction $(1,0),$ but not in the direction $(0,1).$
	
	\solution 
	The function $f(x_1, x_2) = \|x_2\|$ is differentiable at 
	$(0, 0)$ in direction $(1, 0)$ with derivative $\pDiv{x_1} = 0$, but 
	it is not differentiable in direction $(0, 1)$ since $\|x_2\|$ does not have
	a derivative relative to $x_2$ at $(0, 0)$.

	This is because the left-hand limit of the newton quotient of $\|x_2\|$
	equals $-1$, but the right-hand limit of the newton quotient equals $1$
	at $(0, 0)$.

	\medskip
	\item
	Let $D \subseteq \mathbf{V}$ be an open set in a Euclidean space, and suppose 
	$f \colon D \to \mathbb{R}$ is differentiable at $\mathbf{v} \in D$ with respect to 
	$\mathbf{w} \in \mathbf{V}.$ Prove that $f$ is $\mathbf{v}$ with respect to any 
	scalar multiple of $\mathbf{w}.$

	\solution 
	Using the newton quotient, $f'(v,aw) = af'(v,w)$ for any scalar $a$ since 
	\begin{align*}
		f'(v, aw)
		&= \lim_{h\to0} \frac{f(v+haw) - f(v)}{h} \\
		&= a \lim_{ha\to0} \frac{f(v+haw) - f(v)}{ha} \\
		&= af'(v, w)
	\end{align*}
	If $a=0$, then the newton quotient trivially ends up equalling $0$, 
	so the above proof works for all scalars.
	Therefore f is differentiable with respect to any scalar multiple of $w$.
		
	\medskip
	\item
	Let $f \colon \mathbb{R}^n \to \mathbb{R}$ be the function defined by 
	$f(\mathbf{v}) = \mathbf{v} \cdot \mathbf{v}.$
	
		\begin{enumerate}
		
			\smallskip
			\item
			Compute the partial derivatives of $f$ at a given point 
			$\mathbf{v}\in \mathbb{R}^n.$
			
			\smallskip
			\item
			Prove that $f$ is continuously differentiable on $\mathbb{R}^n.$
			
			\smallskip
			\item
			Compute the gradient of $f$ at a given $\mathbf{v} \in \mathbb{R}^n$.
		\end{enumerate}

	\solution 
	\begin{enumerate}
		\item 
		Decomposing v into its $n$ components,
		\[
			\pDiv{x_i} f(v) = 
			\pDiv{x_i} (v \cdot v) = 
			\pDiv{x_i} \sum_{j=1}^n x_j^2 = 
			2x_i
		\]

		\item 
		$f$ is differentiable since partial derivative exist for all basis vectors,
		and the partial derivatives, $2x_i$, are all continuous.

		\item 
		The gradient is a vector composed of all the partial derivatives so,
		\[
			\nabla f(v) = (2x_1, ..., 2x_n) = 2v
		\]
	\end{enumerate}
		
	\medskip
	\item
	Let $A$ be a linear operator on a Euclidean space $\mathbf{V},$ 
	and define a function $f \colon \mathbf{V} \to \mathbb{R}$ by 
	$f(\mathbf{v})=\langle \mathbf{v},A\mathbf{v}\rangle$. Prove
	that $f$ is continuously differentiable on $\mathbf{V}$, and compute 
	the gradient $\nabla f(\mathbf{v})$ for each $\mathbf{v} \in \mathbf{V}.$

	\solution 
	Using the newton quotient and the bilinearity of the scalar product,
	\begin{align*}
		f'(v, w)
		&= \lim_{h\to0} \frac{f(v+hw) - f(v)}{h} \\
		&= \lim_{h\to0} \frac{\inner{v+hw}{A(v+hw)} - \inner{v}{Av}}{h} \\
		&= \lim_{h\to0} \frac{\inner{v+hw}{Av+Ahw)} - \inner{v}{Av}}{h} \\
		&= \lim_{h\to0} 
			\frac{\inner{v}{Av} + \inner{v}{Ahw} + \inner{hw}{Av} + \inner{hw}{Ahw} 
			- \inner{v}{Av}}{h} \\
		&= \lim_{h\to0} \frac{h\inner{v}{Aw} + h\inner{Av}{w} + h^2\inner{w}{Aw}}{h} \\
		&= \lim_{h\to0} \inner{v}{Aw} + \inner{Av}{w} + h\inner{w}{Aw} \\
		&= \inner{A^*v}{w} + \inner{Av}{w} \\
		&= \inner{(A+A^*)v}{w} 
	\end{align*}
	Therefore f is always differentiable.
	Using the Cauchy-Schwartz inequality, 
	the derivative is Lipschitz continuous since
	\[
		\|f'(v, w)\| =
		\|\inner{(A+A^*)v}{w}\| =
		\|\inner{v}{(A+A^*)w}\| \leq
		\|v\| \|(A+A^*)w\|
	\]
	Since $f'(v, w) = \inner{(A+A^*)v}{w}$, the gradient is $(A+A^*)v$.
			
\end{enumerate}
	
%Declare end of document
\end{document}