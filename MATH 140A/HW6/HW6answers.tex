\documentclass{article}

\usepackage{amsmath}
\usepackage{amssymb}
\usepackage{hyperref}
\usepackage{indentfirst}

%User defined commands
\newcommand{\interior}[1]{
  {\kern0pt#1}^{\mathrm{o}}
}

\begin{document}
\begin{center}
	\huge{\bf Math 140A: Homework 6} \\
	Merrick Qiu
\end{center}

\section*{A}
(\textbf{Connected implies only $\emptyset$ and $X$ clopen})
Suppose that there existed a set $A$ 
that was clopen that was not $\emptyset$ or $X$.
Let $B = X \setminus A$, which is also clopen since $A$ is clopen.
Since $A = \overline{A}$ and $B = \overline{B}$ and $A \cap B = \emptyset$,
we have that $A \cap \overline{B} = \emptyset$ and $\overline{A} \cap B = \emptyset$.
Thus $X$ is not connected, which is a contradiction. \\

(\textbf{Only $\emptyset$ and $X$ clopen implies connected})
Supposed that only $\emptyset$ and $X$ were clopen but that 
$X$ was not connected.
Then there exists two subsets $A, B \subset X$ such that
$A \cap \overline{B} = \emptyset$, $\overline{A} \cap B = \emptyset$,
and $X = A \cup B$. This would imply that $A$ is closed 
since if there was $x \in \overline{A}$ and $x \not\in A$,
then $x \in B$, but we have that $\overline{A} \cap B = \emptyset$. \\

This would also imply $A$ is open since 
if there was a point $x \in A$ and $x \not\in \interior{A}$,
then every neighborhood of $x$ would contain a point in $B$,
so $x$ would be a limit point of $B$ but this contradicts 
$A \cap \overline{B} = \emptyset$.
Thus $A$ is a set that is clopen that is neither $\emptyset$
or $X$, which is a contradiction.
\newpage 

\section*{B}
($\implies$) If $\ell \in \mathbb{R}$ is a subsequential limit of $\{a_n\}$,
then there exists a subsequence $\{a_{n_i}\}$ that converges to $\ell$.
By the definition of convergence, for all $\epsilon$,
we can choose $N$ such that for all $n_i \geq N$, $|a_{n_i} - \ell| < \epsilon$.
Thus there are infinite elements with $|a_{n_i} - \ell| < \epsilon$ 
in the subsequence, and there are also infinite elements in $\{n \in \mathbb{N} : |a_n - \ell| < \epsilon\}$.

($\impliedby$) If we choose a subsequence by
discarding elements $a_n$ where $|a_n - \ell| \geq |a_{n-1} - \ell|$,
then we get a sequence that converges to $\ell$.
This is because for all $\epsilon > 0$, there must exist an index $N$ where $|a_N - \ell| < \epsilon$
(by the hypothesis that for every $\epsilon > 0$, $\{n \in \mathbb{N} : |a_n - \ell| < \epsilon\}$ is infinite)
and all indicies $n > N$ have it so that $|a_n - \ell| < |a_N - \ell| < \epsilon$
by the construction of this subsequence.

\newpage

\section*{Rudin 19}
\begin{enumerate}
	\item Since $A$ and $B$ are disjoint, $A \cap B = \emptyset$.
	Since $A$ and $B$ are closed, $A = \overline{A}$ and $B = \overline{B}$.
	Thus $A \cap \overline{B} = \emptyset$ and $\overline{A} \cap B = \emptyset$,
	so $A$ and $B$ are separated.
	\item Supposed that $A \cap \overline{B} \neq \emptyset$.
	Then there exists $x \in A$ such that x is a limit point of $B$, $x \in B'$.
	Since $A$ is an open set, $x$ is an interior point of $A$,
	but this contradicts the fact that $x$ is a limit point of $B$
	since every neighborhood around $x$ will contain a point in $B$.
	Thus $A \cap \overline{B} = \emptyset$, and we can repeat
	the argument to also show that $\overline{A} \cap B = \emptyset$.
	Thus disjoint open sets are separated.
	\item $A$ is simply the open neighborhood around $p$, so it is open.
	$B$ is also an open set, because for all $x \in B$, you can choose 
	a neighborhood of radius $r < d(p,x) - \delta$ which is a subset of $B$
	by the triangle inequality.
	Since by the previous part disjoint open sets are separated,
	$A$ and $B$ are also separated.
	\item If $X$ is a connected metric space, then it cannot be split
	into two separated sets.
	If for some $\delta$ there did not exist any points $q$ such that 
	$d(p,q) = \delta$, then $A$ and $B$ would be separate sets whose union is $X$,
	which contradicts the fact that $X$ is connected.
	Thus for all $\delta$, there exists some $q$ such that $d(p,q) = \delta$,
	and since there is uncountable number of possible values of $\delta$,
	$X$ must also be uncountable.
\end{enumerate}
\newpage 

\section*{Rudin 20}
The closure of a connected set is indeed connected.
Supposed that the closure of a set $X$ was not connected.
Then there exists $A$ and $B$ where $A \cap \overline{B} = \emptyset$ or $\overline{A} \cap B = \emptyset$.
If we remove the points that are in the closure but not in $X$,
then we get two sets which are also separated.
This is a contradiction, so the closure of a connected set is connected.

The interior of a connect set is not always connected.
Let $A \subset \mathbb{R}^2$ be the closed ball around $(1,0)$ of radius $1$ 
and $B \subset \mathbb{R}^2$ be the closed ball around $(-1,0)$ of radius $1$.
The set $A \cup B$ is connected but its interior is not.
\newpage 

\section*{Rudin 21}
\begin{enumerate}
	\item $A_0$ and $B_0$ are disjoint since $A$ and $B$ are disjoint.
	Suppose that there was a $t_0 \in A_0$ that was a limit point of $B_0$.
	Since $t_0$ is a limit point of $B_0$, 
	there exists a point, $x \in B_0$, for all
	balls around $t_0$ of radius $r$.

	All points $p(x) \in B$ are also
	distance $<r(|a| + |b|)$ from $p(t_0)$, meaning $p(t_0)$ is a limit point of $B$.
	Since $p(t_0) \in A$ is also a limit point of $B$, 
	$A \cap \overline{B} \neq \emptyset$ so $A$ and $B$ would no longer be separated,
	which is a contradiction so we know that $A_0 \cap \overline{B_0} = \emptyset$.
	We can repeat the above argument to also show that
	$\overline{A_0} \cap B_0 = \emptyset$ thus $A_0$ and $B_0$ are separated.
	
	\item Supposed that $t \in A \cup B$ for all $t \in [0,1]$.
	$[0,1]$ can then be written as the union of the separate sets $[0,1] \cap A_0$
	and $[0,1] \cap B_0$, but this would imply that $[0,1]$ is not connected, which is 
	not the case. Thus there must exist some $t_0$ such that $t_0 \in A \cup B$
	\item If it wasn't connected, then there would exist
	 separated subsets $A$ and $B$ whose union is the entire convex subset,
	but there would be a $t_0$ such that  $p(t_0) \not\in A \cup B$
	which is a contradiction.
\end{enumerate}
\newpage

\section*{Rudin 1}
If $\{s_n\}$ converges to some point $\ell$, then that means that
for all $\epsilon > 0$, there exists $N$ such that for all $n \geq N$,
$|s_n - \ell| < \epsilon$.
By the reverse triangle inequality we have that 
\[
	 ||s_n| - |\ell|| \leq |s_n - \ell| < \epsilon.
\]
Thus $\{|s_n|\}$ converges to $|\ell|$.
The converse is not true as $s_n = i^n$ does not converge but 
$\{|s_n|\}$ converges to 1.
\newpage 

\section*{Rudin 3}
Since $s_1 < 2$ and if $s_n < 2$, then $\sqrt{2 + \sqrt{s_n}} < \sqrt{2 + 2} = 2$
$s_n < 2$ for all $n$ by induction.
This sequence is also strictly increasing since 
\[
	\frac{s_{n+1}}{s_n} = \sqrt{\frac{2+\sqrt{s_n}}{s_n^2}} > \sqrt{1+ \frac{\sqrt{s_n}}{s_n^2}} > 1
\]
Since it is strictly increasing and bounded, this sequence converges.
\newpage 

\section*{Rudin 5}
If $\lim_{n \to \infty} \sup a_n = \infty$, then the inequality is clearly true.
Otherwise $\{a_n\}$ is bounded above. Let $\{a_{n_i}\}$ and $\{b_{n_i}\}$ be subsequences such that 
$\lim_{i \to \infty} (a_{n_i} + b_{n_i}) = \lim_{n \to \infty} \sup (a_n + b_n)$.
Then choose a subsequence $\{a_{n_{i_j}}\}$ such that $\lim_{j \to \infty}  a_{n_{i_j}} = \lim_{i \to \infty}  \sup a_{n_i}$.
$a_{n_{i_j}} + b_{n_{i_j}}$ still converges to  $\lim_{n \to \infty}  \sup (a_n + b_n)$
and since $a_{n_k}$ is bounded above we have that $b_{n_{i_j}}$ converges to the difference
\[
	\lim_{j \to \infty} b_{n_{i_j}} = \lim_{j \to \infty}  (a_{n_{i_j}} + b_{n_{i_j}}) - \lim_{j \to \infty} a_{n_{i_j}}.
\]
We've shown that there are subsequences whose sum converges to the limit of the supremum
of the sum of the sequences, and since the limit of these two sequences 
are less than or equal to the limit of the supremum of the original two sequences,
we've shown the inequality is true.


\end{document}