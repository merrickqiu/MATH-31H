\documentclass{article}

\usepackage{amsmath}
\usepackage{amssymb}
\usepackage{hyperref}
\usepackage{indentfirst}

%User defined commands
\newcommand{\interior}[1]{
  {\kern0pt#1}^{\mathrm{o}}
}

\begin{document}
\begin{center}
	\huge{\bf Math 140A: Homework 5} \\
	Merrick Qiu
\end{center}
\section*{A}
$(\implies)$ \textbf{Prove compactness implies finiteness.}
Assume by contradiction that $A$ is infinite.
Choose the open cover of $A$ that is made up of the open 
neighborhoods around all the points in $A$ with radius $r < 1$.
Since each of these neighborhoods only covers the point 
that is is centered on,
a finite subcover does not exist since 
removing any one of these open neighborhoods would result 
in $A$ not being covered.
This is a contradiction so compact sets must be finite 
for metric spaces with the discrete metric.

$(\impliedby)$ \textbf{Prove finiteness implies compactness.}
Suppose there is an open cover of $A$.
For each point in $A$, choose a open set from the open cover 
that covers that point.
Since there is a finite number of points in $A$,
these open sets constitue a finite subcover of $A$.

\section*{Rudin 12}
Let $\{G_{\alpha}\}$ be an open covering of $K$.
Let $G_{\alpha_0}$ be the open set that covers $0$.
Include this set in the finite subcover.
Since $G_{\alpha_0}$ is open, $0$ is an interior point 
of $G_{\alpha_0}$ so there exists a radius $r > 0$
where $N_r(0) \subset G_{\alpha_0}$.

The set $K$ only contains the limit point $0$,
so there is a finite number of points not within the 
neighborhood of $0$ and so there is a finite number of points 
not covered by $G_{\alpha_0}$.
For each point not covered by $G_{\alpha_0}$,
pick an open set in the open covering that covers that point 
and add it into the finite subcover.
We have constructed a finite subcover so $K$ is compact.

\section*{Rudin 23}
Since $X$ is separable, let 
$Y \subset X$ be a countable dense subset of $X$.
Let $\{V_{\alpha}\}$ be a collection of all the 
open neighborhoods centered around points in $Y$
with rational radiuses.
Since $Y$ is countable and rational numbers are countable,
$\{V_{\alpha}\}$ is also countable.

For every $x \in X$ and open set $G \subset X$ with $x \in G$,
we can find an open neighborhood around $x$ with rational radius 
such that $x \in N_r(x) \subset G$, so $\{V_{\alpha}\}$ 
acts as a countable base.
\newpage 

\section*{Rudin 24}
Let $\delta > 0$ and pick $x_1 \in X$.
Choose $x_1, \cdots x_j \in X$
such that $d(x_i, x_{i+1}) \geq \delta$ if possible.
This process must stop after a finite number of steps,
because if it was infinite, then this sequence would have 
a limit point and all the points could not be $\delta$
apart from each other.

Thus, $X$ can therefore be covered by finitely
many neighborhoods of radius $\delta$
since if it couldn't, 
we could add another point $x_{j+1}$ to the sequence.
We can take the union of all 
$\delta = \frac{1}{n}$ for all $n \in \mathbb{N}$
in order to generate a set of countable points
that are dense in $X$.
Thus $X$ is separable.


\section*{Rudin 25}
We can choose a cover of $K$ composing of open neighborhoods of radius $\frac{1}{n}$,
and since $K$ is compact, we can find a finite subcover of this cover.
If we union together the finite subcovers we get for all $n = 1,2,3,\cdots$,
we get a set that also covers $K$, and the centers of all these neighborhoods
forms a countable base for $K$. 
This is a countable base because for every $x \in K$ and every open set $G \subset K$ with $x \in G$,
$x$ is an interior point of $G$ and so we can find an open neighborhood that is 
from one of the finite subcovers where $x \in N_{\frac{1}{n}}(x) \subset G$.

Since every point in $K$ is arbitrarily close to some point in this 
countable base, $K$ is also separable.

\section*{Rudin 26}
Exercise 24 implies $X$ has a separable base, and exercise 23 implies that 
since $X$ has a separable base, it has a countable base.
It follows that every open cover of $X$ has a countable subcover $\{G_n\}$,
which we can obtain as a union of a subcollection of the countable base.
Assume by contradiction that there did not exist a finite subcollection of 
$\{G_n\}$ that covered $X$. 
Then the complement, $F_n$ of $G_1 \cup \cdots \cup G_n$,
must be nonempty for each $n$ since no set of $G_1 \cup \cdots \cup G_n$
completely covers $X$.

However $\cap F_n$ is empty because $\cup G_n$ covers $X$.
Let $E$ is a set which contains a point from each $F_n$.
$E$ has a limit point, $p \in E$, since every infinite subset of $X$ has a limit point.
$p$ must also belong to one of the sets $G_n$, and since $G_n$ is open,
there is a neighborhood around $p$ that is contained by $G_n$. 
But this neighborhood around $p$ cannot contain any point in $F_m$
where $m > n$ which is a contradiction.









\end{document}