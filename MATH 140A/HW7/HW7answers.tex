\documentclass{article}

\usepackage{amsmath}
\usepackage{amssymb}
\usepackage{hyperref}
\usepackage{indentfirst}

%User defined commands
\newcommand{\interior}[1]{
  {\kern0pt#1}^{\mathrm{o}}
}

\begin{document}
\begin{center}
	\huge{\bf Math 140A: Homework 7} \\
	Merrick Qiu
\end{center}

\section*{A}
\begin{enumerate}
	\item $\{x_n\}$ is bounded so $s_N$ and $r_n$ exists for all $N \in \mathbb{N}$.
	For all $N \in \mathbb{N}$, $s_N$ cannot exceed $s_1$ since the sup of a subsequence cannot be larger
	than the sup of the whole sequence(since the existence of the larger sup contradicts
	the fact that the sup is supposed to be the best upper bound),
	and $s_N$ is bounded below by $r_1$ since the sup of a subsequence cannot be 
	smaller than the inf of the whole sequence.
	Likewise, $r_N$ is bounded below by $r_1$ and bounded above by $s_1$ so 
	both $\{s_N\}$ and $\{r_n\}$ are bounded.
	\item The sup of a sequence cannot be smaller than the sup of a subsequence
	and the inf of a sequence cannot be larger than the inf of a subsequence.
	This means that $s_K \geq s_{K+1}$  and $r_K \leq r_{K+1}$ for all $K \in \mathbb{N}$
	so $\{s_N\}$ is non-increasing and $\{r_n\}$ is non-decreasing.
	\item It cannot be that $s < \lim \sup x_n$. There must exist a 
	subsequence $\{x_{n_k}\}$
	with $s +c <\lim \{x_{n_k}\}$ for all $c > 0$.
	However, this subsequence must have infinitely many elements that are greater 
	than $s+c$, but this contradicts the fact that there exists some$s_N$ with $s < s_N < s+c$
	and $x_n < s_N < s+c$ for all $n > N$.

	It also cannot be that $s > \lim \sup x_n$.
	Suppose that $s - \lim \sup x_n = c$.
	Choose a subsequence $x_{n_k}$ so that $|s_k - x_{n_k}| < \frac{c}{4}$ for all $k$.
	Then choose $K$ such that for all $k > K$, $|s - s_{k}| < \frac{c}{4}$.
	Thus $|s - x_{n_k}| = |s - s_{k} + s_{k} - x_{n_k}| \leq  |s - s_{k}| + |s_{k} - x_{n_k}| < \frac{c}{2}$
	for all $k > K$, so this subsequence is bounded by $\lim \sup x_n + \frac{c}{2}$ from below.
	Thus there must exist a subsequence of this subsequence that converges to something 
	$\geq \lim \sup x_n + \frac{c}{2}$ which contradicts the definition of $\lim \sup$.
\end{enumerate}
\newpage 

\section*{B}
Let $\{G_{\alpha, \epsilon}\}$ be an open cover of $K$ of open neighborhoods of radius $<\epsilon $
and let $\{G_{i, \epsilon}\}$ be a finite subcover of $K$ chosen from $\{G_{\alpha, \epsilon}\}$.
Create a sequence by appending all the centers of the neighborhoods of $\{G_{i, \epsilon}\}$ into the sequence
for all $\epsilon = \frac{1}{n}$ where $n = 1,2,3, \dots$.
Thus for every point in $p \in K$, we are able to choose a subsequence that contains
one point from each of the open covers whose neighborhood contains $p$
and get a subsequence that converges to $p$.
\newpage 

\section*{C}
$a_n \to a$ implies that for all $\epsilon > 0$, 
there exists some $N$ such that for all $n > N$,
$|a-a_n| < \epsilon$.
We need to prove that for all $\epsilon' > 0$,
there exists some $N'$ such that for all $n > N'$,
$|\sqrt{a}-\sqrt{a_n}| < \epsilon'$.
Since square roots are always positive,
\[
	|\sqrt{a}-\sqrt{a_n}| = 
	|\frac{a-a_n}{\sqrt{a}+\sqrt{a_n}}| = 
	\frac{1}{\sqrt{a}+\sqrt{a_n}}|a-a_n| \leq 
	\frac{1}{\sqrt{a}}|a-a_n| <
	\frac{1}{\sqrt{a}} \epsilon
\]
Thus we can choose $N'$ such that for all $n>N'$,
$|a-a_n| < \sqrt{a}\epsilon'$, and this implies that 
$|\sqrt{a}-\sqrt{a_n}| < \epsilon'$,
so we have shown that $\sqrt{a_n} \to \sqrt{a}$.
\newpage 

\section*{Rudin Question 5}
If $\lim \sup a_n + b_n = -\infty$, then the statement trivially true.
If $\lim \sup a_n + b_n = \infty$, then $\lim \sup a_n = \infty$
or $\lim \sup b_n = \infty$ since if they were both not infinity,
then both sequences would be bounded above and so $\lim \sup a_n + b_n = \infty$ couldn't be true.

Thus consider the case where all the terms are finite.
Let $\{a_{n_k} + b_{n_k}\}$ be a subsequence such that converges to $\lim \sup a_n + b_n$.
Choose $\{a_{n_{k_m}}\}$  such that it converges to $\lim \sup a_{n_k}$
which means that $\{a_{n_{k_m}} + b_{n_{k_m}}\}$ converges to $\lim \sup a_n + b_n$.
Since $\lim \sup a_n$ is greater than or equal to what $\{a_{n_{k_m}}\}$ converges to 
and likewise for $\lim \sup a_n$ for $\{b_{n_{k_m}}\}$, the inequality holds.
\newpage 

\section*{Rudin Question 20}
Since $p_{n_i} \to p$, then for some $K$, for all $\epsilon>0$,
and for $i > K$, $d(p_{n_i}, p) < \epsilon$.
Since $\{p_n\}$ is cauchy, there exists $N$ such that for all $m, n > N$,
$d(p_n, p_m) < \epsilon$.
Choose $M\geq N$ large enough so that $k > K$ and $n_k > N$ for all $k > M$. 
Then $d(p_n, p) \leq d(p_n, p_{n_{K+1}}) + d(p_{n_{K+1}}, p) < 2\epsilon$,
so the original sequence also converges to $p$.
\newpage

\section*{Rudin Question 21}
Suppose the intersection contained more than one point.
Pick two points $p_1$ and $p_2$ from the interesection.
Since $\lim \operatorname{diam} E_n = 0$,
there exists $E_n$ with $\operatorname{diam} E_n < d(p_1, p_2)$.
However both these points cannot exist inside of $E_n$ anymore
by the definition of diameter, which is a contradiction.
\newpage



\end{document}