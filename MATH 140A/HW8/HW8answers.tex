\documentclass{article}

\usepackage{amsmath}
\usepackage{amssymb}
\usepackage{hyperref}
\usepackage{indentfirst}

%User defined commands
\newcommand{\interior}[1]{
  {\kern0pt#1}^{\mathrm{o}}
}

\begin{document}
\begin{center}
	\huge{\bf Math 140A: Homework 8} \\
	Merrick Qiu
\end{center}

\section*{A}
\begin{enumerate}
	\item From theorem 3.37 we have that 
	\[	
		\infty = \lim \inf n+1 = \lim \inf \frac{(n+1)!}{n!} \leq \lim \inf \sqrt[^n]{n!}
	\]
	Thus this sequence diverges and the limit is $\infty$.
	\item
	The term can be rewritten as
	\[
		\frac{\sqrt[^n]{n!}}{n} = \sqrt[^n]{\frac{n!}{n^n}}
	\]
	From theorem 3.37,
	\[
		\lim \inf \frac{c_{n+1}}{c_n} \leq \lim \inf \sqrt[^n]{c_n} \leq \lim \sup \sqrt[^n]{c_n} \leq \lim \sup \frac{c_{n+1}}{c_n}
	\]
	Using $c_n = \frac{n!}{n}$,
	\[
		\frac{c_{n+1}}{c_n} = \frac{\frac{(n+1)!}{(n+1)^{n+1}}}{\frac{n!}{n^n}} = \frac{(n+1)n^n}{(n+1)^{n+1}}=\left(\frac{n}{n+1}\right)^n
	\]
	\[
		 \frac{1}{e} = \lim \inf \left(\frac{n}{n+1}\right)^n \leq \lim \inf \sqrt[^n]{\frac{n!}{n^n}} \leq \lim \sup \sqrt[^n]{\frac{n!}{n^n}} \leq \lim \sup \left(\frac{n}{n+1}\right)^n = \frac{1}{e}
	\]
	Thus 
	\[
		\lim_{n \to \infty} \frac{\sqrt[^n]{n!}}{n} = \frac{1}{e}
	\]
\end{enumerate}
\newpage 

\section*{B}
Let $\epsilon > 0$ and choose $N$ such that $c^Nd(x, f(x)) < \epsilon$.
Then for all $n\geq N$ ,
\begin{align*}
	d(f^n(x), f^{n+1}(x)) &= cd(f^{n-1}(x), f^{n}(x))\\
	&= c^2d(f^{n-2}(x), f^{n-1}(x))\\
	&\vdots \\
	&= c^n d(x, f(x)) \\
	&< \epsilon
\end{align*}
By the triangle inequality, for all $m \geq n \geq N$ 
\begin{align*}
	d(f^n(x), f^{m}(x)) &\leq d(f^{n}(x), f^{n+1}(x)) + d(f^{n+1}(x), f^{n+2}(x)) + \cdots + d(f^{m-1}(x), f^{m}(x))\\
	&= c^n d(x, f(x)) + c^{n+1}d(x, f(x)) + \cdots + c^{m-1}d(x, f(x))\\
	&< \epsilon + c\epsilon + \cdots +c^{m-n}\epsilon \\
	&< \frac{\epsilon}{1-c} \\
\end{align*}

We can readjust our choice of $N$ such that $c^{N}d(x, f(x)) < (1-c)\epsilon$,
and then for all $n,m \geq N$
\[
	d(f^n(x), f^{m}(x)) < \epsilon
\]
Thus the sequence is cauchy.
\newpage 

\section*{Rudin 7}
Since $(\sqrt{a_n}-\frac{1}{n})^2 \geq 0$,
this implies that $\frac{1}{2}(a_n + \frac{1}{n^2}) \geq \frac{\sqrt{a_n}}{n}$.
Since $\sum a_n$ converges and $\sum \frac{1}{n^2}$ converges,
$\frac{\sqrt{a_n}}{n}$ must also converge by the comparison test.
\newpage 

\section*{Rudin 8}
Since $\{b_n\}$ is monotonic and bounded, there exists $\sup |b_n|$.
Because $\sup |b_n| \sum a_n$ converges since it is just multiplying everything by a constant,
and $|b_n| < \sup |b_n|$ for all $n$,
$\sum a_n b_n$ converges as well.
\newpage 

\section*{Rudin 10}
When $z>1$ all the terms in the series are $\geq a_n$.
Since all coefficients must be nonzero integers, $|a_n| \geq 1$ for all $a_n$.
Thus $\lim_{n \to \infty} a_nz^n \neq 0$ so $\sum a_nz^n$ diverges.
Thus the radius of convergence is at most 1.
\newpage 

\section*{Rudin 11}
\begin{enumerate}
	\item If $\sum a_n$ is unbounded then $\frac{a_n}{1+a_n}$ does not approach $0$ so $a_n$ diverges,
	and if $a_n$ is bounded by $M$ then $\frac{1}{1+M}a_n \leq \frac{a_n}{1+a_n}$,
	so by comparison, $a_n$ diverges.
	\item \begin{align*}
		\frac{a_{N+1}}{S_{N+1}} + \cdots + \frac{a_{N+k}}{S_{N+k}} 
		&\geq \frac{a_{N+1}}{S_{N+k}} + \cdots + \frac{a_{N+k}}{S_{N+k}} \\
		&= \frac{S_{N+k} - S_{N}}{S_{N+k}} \\
		&= 1 - \frac{S_{N}}{S_{N+k}}
	\end{align*}
	Thus this series cannot be cauchy convergent since the RHS
	can be made arbitrarily close to 1 by taking sufficiently large $k$.
	\item 
	\[
		\frac{1}{s_{n-1}} - \frac{1}{s_{n}} = \frac{s_n - s_{n-1}}{s_{n-1}s_n} = \frac{a_n}{s_{n-1}s_n} \geq \frac{a_n}{s_n^2}
	\]
	Since $\sum_{n=2}^N \frac{1}{s_{n-1}} - \frac{1}{s_{n}} = \frac{1}{a_1}- \frac{1}{s_n}$,
	this sequence approaches $\frac{1}{a_1}$ and so by comparison, $\frac{a_n}{s_n^2}$ converges.
	\item Since $\frac{a_n}{1+n^2a_n} < \frac{1}{n^2}$, $\sum \frac{a_n}{1+n^2a_n}$ converges.
	However $\sum \frac{a_n}{1+na_n}$ may converge or diverge.
	
\end{enumerate}




\end{document}