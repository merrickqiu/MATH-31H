\documentclass{article}

\usepackage{amsmath}
\usepackage{amssymb}
\usepackage{hyperref}
\usepackage{indentfirst}

%User defined commands
\newcommand{\interior}[1]{
  {\kern0pt#1}^{\mathrm{o}}
}

\begin{document}
\begin{center}
	\huge{\bf Math 140A: Homework 10} \\
	Merrick Qiu
\end{center}

\section*{A}
Choose $\epsilon > 0$. From the limits we know that there exist
$c_1, c_2$ such that $x > c_1$ implies that $|x-\ell_1| < \frac{\epsilon}{2}$
and $x < c_2$ implies that $|x-\ell_2| < \frac{\epsilon}{2}$.
For all $c_2-1 \leq x \leq c_1+1$, $f$ is uniformly continuous since 
$[c_2-1, c_1+1]$ is a compact set. Assume that we choose a $\delta < 1$.
$[c_1, \infty]$ and $[-\infty, c_2]$
are uniformly continuous for any $\delta$ since we chose $c_1$ and $c_2$
such that all points are within $\epsilon$ of each other.
Thus, the function is uniformly continuous.
\newpage 

\section*{B}
Choose $\epsilon > 0$.
If we choose $\delta < \left(\frac{\epsilon}{c}\right)^{\frac{1}{\alpha}}$,
Then 
\[
	d(f(x), f(y)) \leq Cd(x,y)^{\alpha} < \epsilon
\]
so $f$ is uniformly continuous.

\newpage 

\section*{C}
$f$ is a homeomorphism iff the image of an open set is an open set as well.
All connected sets in $\mathbb{R}$ are intervals, and the image of $f$ 
on a connected set must also be connected, so $f((a,b))$ must also be an interval for all $a,b \in \mathbb{R}$. 
This interval must be open as well since 
continuous bijections map open intervals to open intervals.
Since all open sets can be expressed as the countable union of open intervals,
$f$ maps open sets to open sets.
Thus f is a homeomorphism.
\newpage 

\section*{Rudin 2}
Let $y \in f(\overline{E})$ and 
$x \in \overline{E}$ with $f(x) = y$.
Since $\overline{E}$ is a closure,
there must exist a sequence $x_i \to x$
where $x_i \in E$ for all $i$.
Thus all the points in the sequence $f(x_i)$ are in 
the image, $f(E)$, and $\lim f(x_i) = f(\lim x_i) = y$ due to the continuity of $f$,
so $y \in \overline{f(E)}$.
Since $y$ was arbitrary, we have that 
$f(\overline{E}) \subset \overline{f(E)}$. \\

If $f: [1, \infty) \to \mathbb{R}$, $f(x) = \frac{1}{x}$
and $E = [1, \infty)$, then 
$f(\overline{E}) = (0,1]$ but $\overline{f(E)} = [0,1]$
\newpage

\section*{Rudin 4}
Since $E$ is dense in $X$, every point $x\in X$
is the limit of some sequence of points, $x_i \to x$ 
where $x_i \in E$, for all $i$.
Since $f$ is continuous we have that 
$\lim f(x_i) = f(\lim x_i) = f(x)$,
so every point $f(x)$ can also be represented as a 
sequence of points from $f(E)$.
Thus $f(E)$ is dense in $f(X)$.

If $g(x_i) = f(x_i)$ for all $x_i \in E$,
then $f(x) = \lim f(x_i) = \lim g(x_i) = g(x)$,
so $f(x) = g(x)$.
\newpage 

\section*{Rudin 6}
($\implies$)
Take a sequence of points in the graph $(x_n, f(x_n))$.
Since $E$ is compact, there is a subsequence $x_{n_k} \to x$
for some $x$.
Since $f$ is continuous, we also have that $f(x_{n_k}) \to f(x)$,
so any sequence of points in the graph has a convergent subsequence,
so the graph is compact.

($\impliedby$)
Let $x \in E$.
Take a sequence of points in the graph $(x_n, f(x_n))$ such that
$x_n \to x$.
By compactness, there exists a convergent subsequence 
$(x_{n_k}, f(x_{n_k})) \to (x, f(x))$.
It is the case that $f(x_{n_k}) \to f(x)$
because if it did not, then 
the graph would fail to contain the limit point $(x, f(x))$,
which would contradict the fact that the graph is compact.
\newpage 

\section*{Rudin 8}
Let $a = \inf E$ and $b = \sup E$
For some $\epsilon > 0$,
there exists $\delta$ such that all points 
$p,q \in E$ with $|p-q| < \delta$ implies that 
$|f(p) - f(q)| < \epsilon$.
We can "divide up" $E$ into $N = \lceil \frac{b-a}{\delta} \rceil$
sections.
We have that $\inf f(E) \geq f(a) - N*\epsilon$
and $\sup f(E) \leq f(a) + N*\epsilon$
since the maximum amount of change that can occur within
a delta neighborhood of a point is $\epsilon$.
Thus, $f(E)$ is bounded.
\newpage 

\section*{Rudin 14}
If we define $g(x) = f(x) - x$,
$0 \leq g(0) \leq 1$ and 
$-1 \leq g(1) \leq 0$. 
By the intermediate value theorem,
$g(x) = 0$ at some point,
so there exists some point where $f(x) = x$.
\newpage 

\section*{Rudin 18}
At every point $p$, the limit is $0$.
For any $\epsilon > 0$ we can find $n$ such that  $\frac{1}{n} < \epsilon$
by the archimedes principle.
Then we can choose $\delta$ such that all rational numbers with 
denominator $<n$ are not in the delta neighborhood around $p$.
This is possible because there is a finite number of rational numbers
with denominators from $1$ to $n$.

All the points $x$ in this $\delta$ neighborhood have $|x - 0| < \epsilon$,
so the function converges to $0$ at all points.
However only irrational points evaluate to $0$
so the function is only continuous at irrational points, but 
it has simple discontinuity at every rational point.













\end{document}