\documentclass{article}

\usepackage{amsmath}
\usepackage{amssymb}
\usepackage{hyperref}
\usepackage{indentfirst}

%User defined commands
\newcommand{\interior}[1]{
  {\kern0pt#1}^{\mathrm{o}}
}

\begin{document}
\begin{center}
	\huge{\bf Math 140A: Homework 4} \\
	Merrick Qiu
\end{center}
\section*{A}
\begin{enumerate}
    \item The statement is false. 
    In $\mathbb{R}$, the intervals $(0,1)$ and $(1,2)$ are disjoint but 
    their distance is $0$.
    \item The statement is false.
    The intervals $(0,1)$ and $(1,2)$ are separate since 
    $[0,1] \cap (1,2) = \emptyset$ and $(0,1) \cap [1,2] = \emptyset$ but 
    their distance is $0$.
    \item The statement is false.
    The intervals $(0,1)$ and $(1,2)$ are disjoint open sets but 
    their distance is $0$.
    \item The statement is false.
    Let $A = \mathbb{N}$ and let $B = \{n + \frac{1}{n+1} : n \in \mathbb{N}\}$.
    $A$ and $B$ both only have isolated points so they are both closed.
    They are also both disjoint.
    However $d(A,B) = 0$, so this statement is false.
\end{enumerate}
\newpage 

\section*{B}
\begin{enumerate}
    \item We need to show that all points in the closed neighborhood
    are limit points and all points not in the closed neighborhood 
    are not limit points.
    If $y \in \overline{N}_r(x)$, then we need to show that $y$ is a limit point by
    finding a point in $N_{r'}(y)$ for an arbitrary $r'$.
    If $r' > r - d(x,y)$, then choose $r' \leq r - d(x,y)$ since a point in 
    a smaller neighborhood will also be in the larger neighborhood.
    Any point $z \in N_{r'}(y)$ will also be in $\overline{N}_r(x)$ since 
    by the triangle inequality, 
    \[
        d(x, z) \leq d(x,y) + d(y,z) \leq r
    \]
    If $y \not\in \overline{N}_r(x)$,  then we need to show that $y$ is not a limit point
    by finding a $r'$ such that $\overline{N}_r(x) \cap N_{r'}(y) = \emptyset$.
    Choose $r' < d(x,y) - r$. 
    For all points $z \in N_{r'}(y)$
    \[
        r < d(x,y) - d(y,z) \leq d(x,z)
    \]
    by the triangle inequality(with $d(y,z)$ subtracted from both sides).
    Thus the closed neighborhood is a closed set.
    \item We need to show that points inside the closed neighborhood are limit points 
    of the open neighborhood
    and points outside the closed neighborhood are not limit points.
    If $y \in \overline{N}_r(x)$ then $d(x,y) \leq r$.
    For all $r'$ we need to find a point $z\in N_{r'}(y)$
    such that $z \in N_r(x)$ as well.
    Choose $c$ such that $ \frac{r - r'}{r} < c < 1$ 
    and let $z = cx + (1-c)y$.
    This point is both in the neighborhood of $y$ as well as in 
    neighborhood $x$ so $y$ is a limit point.

    If $y \not\in \overline{N}_r(x)$ then $d(x,y) > r$.
    We need to find a $r'$ such that $N_{r'}(y) \cap N_r(x) = \emptyset$.
    Choose $r' < d(x,y) - r$.
    For all points $z \in N_{r'}(y)$
    \[
        r < d(x,y) - d(y,z) \leq d(x,z)
    \]
    by the triangle inequality, so $y$ is not a limit point.
    \item It is not true in general.
    For the discrete metric, a $N_1(x)$ only contains $x$ and its closure
    also just contains $x$. 
    However, $\overline{N}_1(x)$ contains all the points in the metric space.
\newpage
\section*{C}
    ($\implies$)All non-empty open subsets $O$
    must contain a point $x \in O$,
    and this point is an interior point because the set is open.
    For some neighborhood $N_r(x)$, $N_r(x) \subset O$
    since $x$ is an interior point, but since $A$ is dense,
    $x$ is also a limit point of $A$ and $a \in N_r(x)$
    for some $a \in A$.
    Thus $A \cap O \neq \emptyset$ for all $O$.

    ($\impliedby$)
    Since all open neighborhoods are open and $A \cap O \neq \emptyset$ for all open sets,
    we know that every open neighborhood contains a point in $A$, which
    is the definition of $A$ being dense.
\newpage
\section*{Rudin Question 9}
\begin{enumerate}
    \item For all $x \in \interior{E}$, we know that there exists $r$
    such that  $N_r(x) \subset E$ by the definition of an interior point. 
    It is sufficient to show that $x$ is an interior point of $\interior{E}$
    by showing that all points $y \in N_r(x)$ are also in $\interior{E}$.
    If we choose $r' < r-d(x,y)$, then $N_{r'}(y)\subset N_r(x) \subset E$ 
    (since by the triangle inequality, a point $z \in N_{r'}(y)$
    will have $d(x,z) \leq d(x,y) + d(y,z)$)
    so we know that $y$ is also an interior point of $E$.
    Thus $\interior{E}$ is always open.
    \item If $\interior{E} = E$ then every point of $E$ is an interior point 
    so $\interior{E} = E$ implies that $E$ is open.
    If $E$ is open, then every point of $E$ is an interior point of $E$
    so $E \subset \interior{E}$.
    Likewise, if a point is an interior point, it must be in $E$.
    So a set being open implies $E = \interior{E}$.
    \item This is true since $G \subset E = \interior{E}$.
    \item Since $E$ is open, its complement is closed.
    This means that $E^c = \overline{E^c}$.
    Also since $E = \interior{E}$ we have that 
    \begin{align*}
        \interior{E} &= E\\
        \implies {\interior{E}}^c &= E^c \\
        \implies {\interior{E}}^c &= \overline{E^c} \\
    \end{align*}
    \item No, the set $\mathbb{R} \setminus \{0\}$ does not have an interior point 
    at 0 but its closure does.
    \item No the rational numbers have a closure that is the real numbers,
    but the rational numbers have no interior points.
\end{enumerate}
\newpage
\section*{Rudin Question 22}
The set of points which have only rational coordinates are dense in $R^k$.
Let $(x_1, x_2, \cdots, x_k) \in \mathbb{R}^k$, with a neighborhood around that point 
of radius $r$.

We can choose $(y_1, y_2, \cdots, y_k) \in \mathbb{Q}^k$
such that $|x_i - y_i|^2 < \frac{r^2}{k}$ for all $i$
by the density of the rationals in the reals,.
Under the standard metric this point in $\mathbb{Q}^k$ will
be at most $r$ distance away from the original point in $\mathbb{R}^k$,
thus $\mathbb{R}^k$ is separable.
\newpage
\end{enumerate}









\end{document}