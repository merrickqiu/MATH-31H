\documentclass{article}

\usepackage{amsmath}
\usepackage{amssymb}
\usepackage{hyperref}
\usepackage{indentfirst}

%User defined commands
\newcommand{\sgn}{\operatorname{sgn}}

\begin{document}
\begin{center}
	\huge{\bf Math 140A: Homework 1} \\
	Merrick Qiu
\end{center}
\section*{A.}
    \[
        \sup E = \frac{13}{11}
    \]
    $\frac{13}{11}$ is an upper bound since all elements in $E$ are less than or equal to it.
    For any $c<\frac{13}{11}$, choose $b=\frac{13}{11}$ from $E$.
    Since $c < b \leq \frac{13}{11}$, $\frac{13}{11}$ is the supremum.

    \[
        \inf E = \frac{5}{11}
    \]
    $\frac{5}{11}$ is a lower bound since all elements in $E$ are greater than it.
    Let $c>\frac{5}{11}$ where $c = \frac{5}{11} + \epsilon$ for some $\epsilon > 0$.
    Choose $b \in E$ where $b = \frac{5n+8}{11n}$ and $n > \frac{8}{11\epsilon}$.
    Since $\frac{5}{11} \leq b < c$, $\frac{5}{11}$ is the infimum.   
    \newpage
\section*{B.}
    Let $a = \max(\sup T, \sup S)$.
    For all $x \in T \cup S$, $x \in T$ or $x \in S$.
    Thus $x < \sup T$ or $x < \sup S$ and so $x < \max(\sup T, \sup S)$.
    Therefore $a$ is an upperbound.

    \textbf{Case 1} If $\sup T \leq \sup S$, then $a=\sup S$.
    For any $c$ we choose, there exists some $b \in S$ such that $c < b \leq a$
    by the definition of the supremum.

    \textbf{Case 2} Similarly if $\sup T > \sup S$, we can choose some $b \in T$ such that $c < b \leq a$.
    Therefore, $a$ is the best upperbound and so it is the supremum.
    \newpage
\section*{C.}
    Let $a=(\sup S)*(\sup T)$.
    For all $x \in ST$, $x=st$ for some $s \in S$ and $t \in T$.
    Since $s \leq \sup S$, $t \leq \sup T$, and s and t are positive, 
    $st \leq (\sup S)(\sup T)$.
    Thus, $a$ is an upperbound for $ST$.

    Let $c < a$ where $c = c_Sc_T$ for some $c_S < \sup S$ and $c_T < \sup T$.
    Choose some $b_S$ and $b_T$ such that
    $c_S< b_S \leq \sup S$ and $c_T< b_T \leq \sup T$.
    Let $b = b_Sb_T$.
    Since these are subsets of the positive real numbers, this implies that 
    $c < b \leq (\sup S)(\sup T)$, meaning that $a$ is the supremum of $ST$.\\

    Let $a=\sup S + \sup T$.
    For all $x \in S + T$, $x=s + t$ for some $s \in S$ and $t \in T$.
    Since $s \leq \sup S$ and  $t \leq \sup T$, 
    $s + t \leq \sup S + \sup T$.
    Thus, $a$ is an upperbound for $S + T$.

    Let $c < a$ where $c = c_S + c_T$ for some $c_S < \sup S$ and $c_T < \sup T$.
    Choose some $b_S$ and $b_T$ such that
    $c_S< b_S \leq \sup S$ and $c_T< b_T \leq \sup T$.
    Let $b = b_S + b_T$.
    This implies that 
    $c < b \leq \sup S + \sup T$, meaning that $a$ is the supremum of $S+T$.
\newpage
\section*{D}
\begin{enumerate}
    \item The addition of two rational functions is rational
    since the functions can be rewritten with a common denominator and then added to yield another rational function.
    Since addition of functions is commutative and associative, so is the addition of rational functions.
    0 is a rational function and adding 0 to any rational function yields that rational function.
    The inverse of every rational function can be found by negating the coefficients of the numerator and
    adding a rational function to its inverse yields 0.

    The multiplication of two rational functions is rational
    since the multiplication of the numerators is a polynomial and the multiplication of the denominators is a polynomial.
    Since multiplication of functions is commutative and associative, so is the multiplication of rational functions.
    1 is a rational function and multiplying 1 to any rational function yields that rational function.
    The inverse of every rational function can be found by swapping the numerator and the denominator, and 
    multiplying a rational function to its inverse yields 1.

    Since the distributive law holds for functions, it also holds for rational functions.
    \item For any two rational functions $\frac{p}{q}$ and $\frac{f}{g}$, $\frac{p}{q} - \frac{f}{g}$ 
    either has $a_nb_m > 0$, $a_nb_m = 0$, or $a_nb_m < 0$.
    This means that either $\frac{p}{q} > \frac{f}{g}$, 
    $\frac{p}{q} = \frac{f}{g}$, or $\frac{p}{q} < \frac{f}{g}$.
    Also for any three rational functions $\frac{p}{q}$, $\frac{f}{g}$, and $\frac{a}{b}$,
    if $\frac{p}{q} > \frac{f}{g}$ and $\frac{f}{g} > \frac{a}{b}$, then
    $\frac{p}{q} - \frac{f}{g} > 0$ and $\frac{f}{g} - \frac{a}{b} > 0$. This would imply that 
    $\frac{p}{q} - \frac{a}{b} > 0$ and $\frac{p}{q} > \frac{a}{b}$.
    Thus F is an ordered set.

    Let $\frac{p}{q}$, $\frac{f}{g}$, and $\frac{a}{b}$ be three rational functions.
    $\frac{p}{q} > \frac{f}{g}$ implies that $\frac{p}{q} - \frac{f}{g} > 0$, which implies
    $(\frac{p}{q} + \frac{a}{b}) - (\frac{f}{g} + \frac{a}{b}) > 0$, which is equivalent to
    $\frac{p}{q} + \frac{a}{b} > \frac{f}{g} + \frac{a}{b}$.

    For rational functions $\frac{p}{q} > 0$ and $\frac{f}{g} > 0$,
    $a_n$ of $\frac{p}{q}\cdot\frac{f}{q}$ is the product of $a_n$ of $\frac{p}{q}$
    and $a_n$ of $\frac{f}{q}$.
    Similarly, $b_m$ of $\frac{p}{q}\cdot\frac{f}{q}$ is the product of $b_m$ of $\frac{p}{q}$
    and $b_m$ of $\frac{f}{q}$.
    Since the sign $a_n$ and $b_m$ match for $\frac{p}{q}$ and $\frac{f}{g}$
    the sign of $a_n$ and $b_m$ in the product must match, 
    so $\frac{p}{q} \cdot \frac{f}{g} > 0$.
    Since all the axioms of an ordered field have been met, $F$ is an ordered field.
    \item The order defined in (ii) is equivalent to the dictionary ordering since 
    we only check the sign of the most significant terms in the numerator and denominator.
    \[
        -x^5, 3-2x, 2, x+6, x^2
    \]
    \item For all $a \in \mathbb{R}$, $x-a > 0$ since $a_nb_m = 1 > 0$, so $x > a$.
\end{enumerate}
\newpage

\section*{E.}
    \textbf{Exercise 1}
    Since $r$ is rational it can be written as $r = \frac{f}{g}$.
    Assume that $r+x$ is rational. 
    Then there exists integers $p$ and $q$ such that $\frac{p}{q} = r+x$.
    Then $x$ could be written as $\frac{p}{q} - \frac{f}{g} = \frac{pg-fq}{qg}$,
    which is a contradiction.
    Therefore, $r+x$ must be irrational.

    Also assume that $rx$ is rational.
    Then there exists integers $p$ and $q$ such that $\frac{p}{q} = rx$.
    Then x could be written as $\frac{\frac{p}{q}}{\frac{f}{g}} = \frac{pg}{qf}$,
    which is a contradiction.
    Therefore, $rx$ must be irrational.\\

    \textbf{Exercise 2}
    Assume that there exists a rational number $\frac{p}{q}$ in simplified form
    such that $\frac{p^2}{q^2} = 12$.
    $p^2 = 12q^2$ implies that $p^2$ and $p$ are a multiple of 3,
    so $p=3k$ for some $k$.
    Substituting this in for $p$ yields $3k^2 = 4q^2$, which implies that $q^2$
    and $q$ are a multiple of 3.
    This is a contradiction since we assumed that 
    $\frac{p}{q}$ was already simplified, so $12$ has no rational root.\\

    \textbf{Exercise 5}
    Let $a = -\sup(-A)$.
    For all $x\in A$, $-x$ is an element of $-A$.
    By the definition of the supremum $-a \geq -x$ for all $-x$,
    which implies $a \leq x$ for all $x$ and so $a$ is a lower bound for $A$.

    From the definition of the supremum, for all $c < -a$, there exists 
    $b$ such that $c < b \leq -a$.
    Therefore for all $-c > a$, you can choose 
    $-b$ such that  $a \leq -b < -c$, so $a$ is the infimum.

    \textbf{Exercise 8}
    The square of a number is always nonnegative in an ordered field,
    but $i^2 = -1$, so the complex numbers cannot be an ordered field.





\newpage




\end{document}