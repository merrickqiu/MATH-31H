\documentclass{article}

\usepackage{amsmath}
\usepackage{amssymb}
\usepackage{hyperref}
\usepackage{indentfirst}

%User defined commands
\newcommand{\sgn}{\operatorname{sgn}}

\begin{document}
\begin{center}
	\huge{\bf Math 140A: Homework 3} \\
	Merrick Qiu
\end{center}
\section*{A}
\begin{enumerate}
    \item \textbf{Closed: }The integers have no limit points because
    any sufficiently small neighborhood around a point 
    will have no neighboring integers.
    There are also no interior points since any neighborhood around 
    an integer will have non-integer numbers.
    Vacuously, the integers are closed.
    \item \textbf{Neither: }The limit points are $[a,b]$ and the set is neither open nor closed.
    This set is not closed since $a$ is a limit point
    that is not in the set.
    This set is also not open since $b$ is in a point in the set 
    that is not an interior point.
    \item \textbf{Neither: } Since the rationals are dense in the reals,
    every real number is a limit point.
    The rationals are not closed since nonrational reals 
    are limit points.
    The rationals are also not open since the set has no interior points.
    \item \textbf{Neither: } This set has limit points $-1$ and $1$ since the 
    $\frac{1}{m}$ term can get arbitrarily close to 0.
    The set is not closed since both limit terms are not in the set.
    The set has no interior points, so it is not open either.
    \item \textbf{Neither: }$0$ is a limit point since $\frac{1}{n} + \frac{1}{m}$
    can get arbitrarily close to $0$ and numbers of the form $\frac{1}{n}$
    are limit points since $\frac{1}{m}$ can get arbitrarily close to 0 as well.
    The set is not closed since $0$ is not in the set.
    The set is also not open since $2$ is in the set but it is not an interior point.
    \item \textbf{Neither: } The limit points are $1$ and $-1$.
    The set is not closed since the limit points are not in the set.
    The set is not open since none of the points are interior points.
\end{enumerate}
\newpage 

\section*{B}
\begin{enumerate}
    \item Let $a + bi \in \mathbb{C}$ be a complex number with a neighborhood of radius $r$.
    Since $\mathbb{Q}$ is dense in $\mathbb{R}$, we can find a rational number $a'$ 
    that is in the neighborhood around $a$ of radius $\frac{r}{\sqrt{2}}$
    and a rational number $b'$ that is in the neighborhood around $b$ of radius $\frac{r}{\sqrt{2}}$.
    By the usual metric, $a' + b'i$ is guaranteed to be in the neighborhood around $a+bi$ of radius $r$.
    $a'+b'i$ is also in the neighborhood of $a+bi$ so $A$ is dense in $\mathbb{C}$.
    \item If there is a neighborhood around $(a_1, a_2,\cdots, a_n) \in \mathbb{R}^n$ with radius $r$,
    then we can choose $(a_1', a_2', \cdots, a_n') \in \mathbb{Q}^n$ 
    where for all $i$, $a_i'$ is a rational number in the neighborhood around $a_1$ of radius $\frac{r}{\sqrt{n}}$.
    Thus $\mathbb{Q}$ is dense in $\mathbb{R}$.
    \item If there is a neighborhood around $(c_1, c_2,\cdots, c_n) \in \mathbb{C}^n$ with radius $r$,
    then we can choose $(c_1', c_2', \cdots, c_n') \in \mathbb{A}^n$ ,
    where for all $i$, $c_i'\in A$ is in the neighborhood around 
    $c_i$ of radius $\frac{r}{\sqrt{n}}$ by using (1).
\end{enumerate}
\newpage 

\section*{C}
Either $A$ has limit points or it has no limit points.
If it has no limit points, then by definition it is a discrete set.
If $A$ has a limit point $p$, then we can show every interval 
$(x,x + \epsilon)$ for $x, \epsilon \in \mathbb{R}$ 
contains a point in $A$ to show that $A$ is dense. \\

Since $p$ is a limit point, we can find some $q \in A$ within the 
neighborhood of radius of $\epsilon$.
Let $a = p-q$, and since $A-A=A$, we know that $a \in A$.
If we add a multiple of $a$ to $l$, we are guaranteed
to find some $l + na \in (x, x + \epsilon)$ for $n \in \mathbb{Z}$ since $|a| < \epsilon$.
So $A$ is dense in this case and every additive subgroup must be either discrete or dense in the reals.
\newpage 

\section*{D Problem 2}
Since the union of a sequence of 
at most countable sets is also at most countable,
we just need to show that the set of algebraic numbers with 
positive integer $N$ such that
\[
    n + |a_0| + \cdots + |a_n| = N
\]
is at most countable to show that the algebraic numbers are countable.
This is because the set of algebraic numbers is the union 
of these sets over all $N$.\\

Since there are only a finite number of integer coefficients that satisfy that equation 
for a given $N$ and each set of coefficients corresponds to a finite number of 
complex numbers, we know that the set of algebraic numbers for a given $N$
is finite.\\

Thus the algebraic numbers are countable.
(We know that it is countable instead of just being at most countable 
because the natural numbers are a subset of the algebraic numbers).
\newpage 

\section*{D Problem 5}
All numbers of the form $a + \frac{1}{n}$ where $a \in \{0, 1, 2\}$ and $n \in \mathbb{N}$
has only three limit points, which are 0, 1, and 2.
\newpage 

\section*{D Problem 6}
\textbf{E' is closed} \\
Let $p$ be a limit point of $E'$.
Within the neighborhood of radius $r$, 
we can find a point $q \in E'$ that is a limit point for $E$.
Choosing $r' < r - d(p, q)$ as the radius for a neighborhood 
around $q$, we can choose a point $s \in E$ in this neighborhood.
Since $d(p, s) \leq d(p, q) + d(q,s)$, we know that 
$s$ is also in the neighborhood of $p$, so $p$ is also a 
limit point of $E$.
Thus every limit point of $E'$ is in $E'$ so $E'$ is closed.

\textbf{E and E' have the same limit points} \\
The neighborhoods of any limit point $p \in E'$, 
contain some $q \neq p$ with $q \in E$.
Since $\overline{E} = E \cup E'$,
$q \in \overline{E}$ as well so 
all limit points of $E'$ are limit points of $\overline{E}$.

For a limit point $p \in \overline{E}'$, 
all the neighborhoods of $p$ will at least contain 
either a point in $E$ or a point in $E'$.
If it is a point $q \in E'$ then we can choose a sufficiently
small radius(like we did in the first part of this problem) 
around $q$ such that this neighborhood is a subset of the neighbhorhood 
around $p$ (by the triangle inequality).
Thus either way, an arbitrary neighborhood around $p$ will contain a point in $E$
so all limit points of $\overline{E}$ are limit pionts of $E$.

Thus $E$ and $\overline{E}$ have the same limit points.
$E$ and $E'$ have the same limit points as well, 
which can be shown by the same argument.
\newpage 

\section*{D Problem 8}
Since every point in an open set is an interior point,
and interior points are limit points in $\mathbb{R}^2$,
every point in every open set is a limit point in $\mathbb{R}^2$.
By definition every point in a closed set is limit point as well.






\end{document}