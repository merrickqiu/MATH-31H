\section{Adjoint matrixes}
\begin{proof} $Av$ is surjective.\gap

    Let $A = \{a_1,...,a_n\}^T$ be an $n \times m$ complex matrix.
    Then the $ij$th entry of $AA^*$ would be $a_i \cdot \overline{a_j}$.
    Let $B = \{b_1,...,b_n\}$ be an $r \times n$ complex matrix.
    Then the $ij$th entry of $B^*B$ would be $\overline{b_i} \cdot b_j$.
    Since $AA^* - B^*B = I_n$, we have that
    \[
        (AA^*)_{ij} - (B^*B)_{ij} =
        a_i \cdot \overline{a_j} - b_j \cdot \overline{b_i} = 
        \inner{a_i}{a_j} - \inner{b_j}{b_i}=
        \left\{
        \begin{array}{ll}
            0 & i \neq j \\
            1 & i = j \\
        \end{array} 
        \right.
    \]
    Since the standard inner product of a vector to itself is real and nonnegative, 
    we have that $(B^*B)_{ii} \geq 0$ and $(AA^*)_{ii} = (B^*B)_{ii} + 1 \geq 1$ on the diagonals.\gap

    Assume that there exists a nonzero $v \in \complex^n$ 
    such that $AA^*v = 0$.
    Therefore we have that
    \begin{align*}
        AA^*v - B^*Bv = I_nv
        &\implies -B^*Bv = I_nv\\
        &\implies B^*Bv = -v\\
    \end{align*}
    This contradicts the fact that the diagonal entries of $AA^*$
    are greater than or equal to 1, so $AA^*$ must have a kernel of zero.
    Since $AA^*$ is square, $AA^*$ is invertible and 
    there exists an inverse $AA^*(AA^*)\inv = I$.
    Therefore, $A^*(AA^*)\inv$ is the right inverse of $A$ 
    and so $A$ is surjective.
\end{proof}