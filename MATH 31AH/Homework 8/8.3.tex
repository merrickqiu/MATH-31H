\section{Quotients and Duals}
\begin{proof} $U$ is a subspace of $V^*$\gap

    The zero functional has $\lambda_0(w) = 0$ 
    for all $w \in W$, so $\lambda_0 \in U$.\gap

    For all $\lambda, \mu \in U$ and for all $w \in W$,
    \begin{align*}
        (\lambda + \mu)(w)
        &= \lambda(w) + \mu(w)\\
        &= 0 + 0\\
        &= 0
    \end{align*}
    Therefore, $\lambda + \mu \in U$.\gap

    For all $\lambda \in U$, for all $c \in \field$, and for all $w \in W$,
    \begin{align*}
        (c\lambda)(w)
        &= c\lambda(w)\\
        &= c \cdot 0\\
        &= 0
    \end{align*}
    Therefore, $c\lambda \in U$.\gap

    Since $U$ has a zero, it is closed under addition, 
    and it is closed under scalar multiplication, $U$ is a subspace of $V^*$.
\end{proof}
\begin{proof} $W^*$ and $V^*/U$ are isomorphic.\gap
    
    Let $T: W^* \rightarrow V^*/U$ with $T(w) := w + U$ be a function.
    From the definition of $U$, 
    $W^*$ and $U$ span $V$ and $W^* \cap U = 0$
    so $W^* \oplus U = V^*$.
    Therefore each coset in $V^*/U$ can be uniquely written 
    as $(w+u) + U = w+U$ with $w \in W^*$ and $u \in U$, so $T$ is surjective.
    Let $w_1 \in W^*$ and $w_2 \in W^*$ be different functionals.
    \[
        T(w_1) = w_1 + U
        \text{ and }
        T(w_2) = w_2 + U
    \]
    Since $w_2-w_1 \notin U$, $T$ is injective.
    Since $T$ is injective and surjective, $T$ is bijective.\gap

    $T$ is linear since
    \begin{align*}
        T(c_1w_1 + c_2w_2) 
        &=(c_1w_1 + c_2w_2) + U\\
        &= c_1(w_1 + U) + c_2(w_2 + U)\\
        &= c_1T(w_1) + c_2T(w_2)
    \end{align*}
    Since $T$ is bijective and linear, it is an isomorphism.
    Therefore, $W^* \cong V^*/U$.
\end{proof}
\begin{proof}$(V/W)^*$ and $U$ are isomorphic.\gap

    Let $T: U \rightarrow (V/W)^*$ with $T(\lambda)(v+W) := \lambda(v)$ 
    with $v \in V$ be a function.
    $T$ is well-defined since for $v = v' + w$.
    \begin{align*}
        T(\lambda)(v+W)
        &= \lambda(v)\\
        &= \lambda(v' + w)\\
        &= \lambda(v') + \lambda(w)\\
        &= \lambda(v')\\
        &= T(\lambda)(v'+W)
    \end{align*}
    $T$ has an inverse $T\inv: (V/W)^* \rightarrow U$ with
    $T\inv(\lambda)(v) := \lambda(v+W)$ for $\lambda \in (V/W)^*$ and $v \in V$,
    so $T$ is bijective.\gap

    $T$ is also linear since
    \begin{align*}
        T(c_1\lambda + c_2\mu)(v+W)
        &= (c_1\lambda + c_2\mu)(v)\\
        &= c_1\lambda(v) + c_2\mu(v)\\
        &= c_1T(\lambda)(v+W) + c_2T(\mu)(v+W)
    \end{align*}
    Since $T$ is bijective and linear, $T$ is a isomorphism.
    Therefore, $(V/W)^* \cong U$.
\end{proof}