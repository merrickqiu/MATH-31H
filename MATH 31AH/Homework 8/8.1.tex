\section{Quotients and matrices}
\begin{proof}$\tbar: V/W \rightarrow V/W$ with $\tbar(v+W) = T(v) + W$ is well defined. \gap
    
    Let $\ttilde: V \rightarrow V/W$ with $\ttilde(v) := T(v) + W$.
    If $w \in W$, then $T(w) \in W$ since $W$ is T-invariant.
    For all $w \in W$, $\ttilde(w) = T(w) + W = 0 + W$, so $W \subseteq \ker \ttilde$.
    Because of the universal property of quotient spaces,
    the function $\tbar: V/W \rightarrow V/W$ with 
    $\tbar(v+W) = \ttilde(v) = T(v) + W$ is well-defined.\gap

    Since the first $m$ indexes are an ordered basis for $W$
    and $W$ is T-invariant, $A$ represents $T$ restricted to $W$.
    $\basisb - \basisc$ is a basis of $V/W$ and 
    $\tbar(v+W) = T(v) + W$, so
    $C$ represents what $\tbar$ does to $v \in V-W$.
\end{proof}