\section{Quotients and Direct Sums}
\begin{proof}$(V \oplus W)/\mathcal{W} \cong V$.\gap
    
    To avoid notational abuse, we will use $\mathcal{W} :=\{(0,w):w \in W\}$.
    Let $T: V \rightarrow (V \oplus W)/\mathcal{W}$ 
    with $T(v) := (v,0) + \mathcal{W}$ be a function.
    Every coset in $(V \oplus W)/\mathcal{W}$ can uniquely
    be written as $(v,w) + \mathcal{W} = (v,0) + \mathcal{W}$,
    so $T$ is surjective.
    Let $v_1 \in V$ and $v_2 \in V$ be different vectors.
    \[
        T(v_1) = (v_1,0) + \mathcal{W}
        \text{ and }
        T(v_2) = (v_2,0) + \mathcal{W}
    \]
    Since $(v_2-v_1,0) \notin \mathcal{W}$, $T$ is injective.
    Since $T$ is injective and surjective, $T$ is bijective.\gap

    $T$ is linear since
    \begin{align*}
        T(c_1v_1 + c_2v_2) 
        &=(c_1v_1 + c_2v_2,0) + \mathcal{W}\\
        &= c_1((v_1,0) + \mathcal{W}) + c_2((v_2,0) + \mathcal{W})\\
        &= c_1T(v_1) + c_2T(v_2)
    \end{align*}
    Since $T$ is bijective and linear, it is an isomorphism.
    Therefore, $(V \oplus W)/\mathcal{W} \cong V$.
\end{proof}