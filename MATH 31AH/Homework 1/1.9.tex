\section{$\mathbb{R}^2$ and $\mathbb{C}$ (Optional)}

x is analogous to the real component of a complex number,
and y is analogous to the imaginary component of a complex number.
When viewed in this way, the arithmetic operations for $\mathbb{R}^2$ and $\mathbb{C}$
are isomorphic. These rules can be used to define a field structure $\mathbb{F}^2$ for every $\mathbb{F}$:
\begin{enumerate}
    \item Addition is commutative.
    \begin{enumerate}
        \item $(a, b) + (c, d)$
        \item $(a+c, b+d)$
        \item $(c+a, d+b)$
        \item $(c, d) + (a, b)$
    \end{enumerate}   
    \item Addition is associative.
    \begin{enumerate}
        \item $((a, b) + (c, d)) + (e, f)$
        \item $(a+c, b+d) + (e, f)$
        \item $(a+c+e, b+d+f)$
        \item $(a+(c+e), b+(d+f))$
        \item $(a, b) + (c+e, d+f)$
        \item $(a, b) + ((c, d) + (e, f))$
    \end{enumerate}
    \item Multiplication is distributive.
    \begin{enumerate}
        \item $(a, b) \cdot ((c, d) + (e, f)) =$
        \item $(a, b) \cdot (c+e, d+f) =$
        \item $(ac+ae-bd-bf, ad+af+bc+be) =$
        \item $(ac-bd, ad+bc) + (ae-bf, af+be) =$
        \item $(a, b) \cdot (c, d) + (a, b) \cdot (e, f)$
    \end{enumerate}
    \item Zero is $(0, 0)$.
    \item One is $(1, 0)$.
    \item Additive inverse: $-(x,y) = (-x, -y)$
    \item Multiplicative inverse: $(x,y)^-1 = (\frac{a}{a^2+b^2}, \frac{-b}{a^2+b^2})$
    \item $(0, 0) \neq (1, 0)$.
\end{enumerate}