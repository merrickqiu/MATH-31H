\section{Projections and trace}

\begin{proof} $\tr(P)$ is well defined.\gap

    The trace of all similar matrices are the same, so
    $\tr(P)$ is the same for all choices of $\mathcal{B}$.
    Therefore, $\tr(P)$ is well defined 
    when $V$ is finite dimensional.
\end{proof}

\begin{proof}$\tr(P) = \dim \Image(P)$\gap

    Let $\basisb = \{e_1,...,e_s\}$ be an ordered basis for $\Image(P)$.
    Since $\Image(P) \subseteq V$, 
    $\basisb$ can be completed to form an ordered basis for $V$,
    $\basisc = \{e_1,...,e_s,e_{s+1},...,e_n\}$.
    Let $[P]_\basisc^\basisc$ be the matrix representation 
    of $P$ in basis $\basisc$.
    A vector is in $\Image(P)$ iff the associated vector in $\mathbb{F}^n$
    has zero in entries from $s+1$ through $n$; 
    otherwise, $\basisb$ would no longer be a basis for $\Image(P)$.\gap

    Since $P \circ P = P$, 
    if a vector $v \in \Image(P)$, then $P(v) = v$.
    Therefore the top-left $s \times s$ block of $[P]_\basisc^\basisc$
    would be the identity matrix.
    Since all $v \in \Image(P)$ have zeros in entries from $s+1$ through $n$,
    the bottom $n-s$ rows of $[P]_\basisc^\basisc$ are zeros.
    Therefore the block matrix form of $[P]_\basisc^\basisc$ is
    \[ 
    [P]_\basisc^\basisc =
    \begin{bmatrix}
        I_{s \times s} & B \\
        0 & 0
    \end{bmatrix}
    \]

    The trace of $[P]_\basisc^\basisc$ would therefore be $s$, which is simply
    the dimension of the image.
    Since the choice of basis does not change the trace of a transformation,
    we have that $\tr(P) = \dim \Image(P)$. 
\end{proof}