\section{Trace}
\begin{proof} $\tr(AB) = \tr(BA)$.\gap

    Let $A$ and $B$ be $n \times n$ matrices.
    The trace of a matrix is the sum of the diagonal entries,
    so 
    \[
        \tr(AB) =\sum_{i=1}^n \sum_{j=1}^n a_{ij} b_{ji}
        \text{ and }
        \tr(BA) =\sum_{i=1}^n \sum_{j=1}^n b_{ij} a_{ji}
    \]
    Since both traces sum up all term of the form $a_{ij} b_{ji}$,
    $\tr(AB) = \tr(BA)$.
\end{proof}

\begin{proof} Similar matricies have the same trace.\gap

    Let $A$ and $B$ be similar matricies with $A = PBP\inv$.
    Since $\tr(AB) = \tr(BA)$, we have that
    \begin{align*}
        \tr(A)
        &= \tr(P(BP\inv))\\
        &= \tr((BP\inv)P)\\
        &= \tr(BI)\\
        &= \tr(B)
    \end{align*}
    Therefore $\tr(A) = \tr(B)$ if $A \cong B$.
\end{proof}

\begin{proof} 
    $\tr(A) = \dim E_{\lambda_1} \cdot \lambda_1 + ... + \dim E_{\lambda_r} \cdot \lambda_r$\gap
    
    Since $A$ is diagonalizable with eigenvalues $\lambda_1,...,\lambda_r$,
    $A$ is similar to a diagonal matrix, $D$, with the eigenvalues of $A$ on the diagonal.
    Similar matricies have the same trace, so
    the trace of $A$ is the trace of $D$,
    which is the sum of the eigenvalues on the diagonal matrix.\gap

    Since the dimension of an eigenspace is equal to the number of eigenvectors
    associated with the eigenvalue of the eigenspace,
    the dimension of an eigenspace is equal to the number of times that 
    the eigenvalue appears on the diagonal of the diagonal matrix.
    The trace of $D$ is equal to the 
    sum of each eigenvalue multiplied by how many times it appears on the diagonal, 
    so therefore,
    $\tr(A) = \tr(D) = \dim E_{\lambda_1} \cdot \lambda_1 + ... + \dim E_{\lambda_r} \cdot \lambda_r$.

\end{proof}