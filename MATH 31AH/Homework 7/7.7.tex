\section{Matrices and Inner Products}

\begin{proof}$\inner{v}{w} = v^TA^TAw$ iff $A$ is invertible\gap

    $v^TA^TAw$ is an inner product iff 
    it satisfies the properties of inner products.\\
    For addition
    \begin{align*}
        \inner{u + v}{w}
        &= (u+v)^T A^T Aw\\
        &= (u^T +v^T) A^T Aw\\
        &= u^T A^T Aw + v^T A^T Aw\\
        &= \inner{u}{w} + \inner{v}{w}
    \end{align*}
    For scalar multiplication
    \begin{align*}
        \inner{cv}{w}
        &= (cv)^T A^T Aw\\
        &= cv^T A^T Aw\\
        &= c\inner{v}{w}
    \end{align*}
    For symmetry, the transpose of a $1 \times 1$ matrix is itself. 
    \begin{align*}
        \inner{v}{w}
        &= v^T A^T A w\\
        &= ((v^T A^T A w)^T)^T\\
        &= (w^T A^T A v)^T\\
        &= w^T A^T A v\\
        &= \inner{w}{v}
    \end{align*}
    For positive-definiteness
    \begin{align*}
        \inner{v}{v}
        &= v^TA^T Av\\
        &= (Av)^T Av\\
        &= Av \cdot Av\\
        &\geq 0
    \end{align*}
    For $\inner{v}{v} = 0$ iff $v = 0$
    \begin{align*}
        \inner{v}{v} = 0
        &\iff v^TA^T Av = 0\\
        &\iff Av \cdot Av = 0\\
        &\iff Av = 0
    \end{align*}
    If A is invertible, then $Av = 0$ iff $v = 0$.
    If A is not invertible, then there exists $v \neq 0$ with $Av = 0$.
    Therefore, $\inner{v}{w} = v^TA^TAw$ iff $A$ is invertible.
\end{proof}