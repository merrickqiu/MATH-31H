\section{Induced maps}
\begin{proof} If $T$ is injective, then $T^*$ is surjective. \gap
    
    The definition of $T^*$ on some $\mu \in W^*$ 
    and some $v \in V$ is
    $(T^*(\mu))(v) = \mu(T(v))$.
    $T^*$ is surjective if and only if for every  $\varphi \in V^*$,
    there exists some $\mu \in W^*$ such that $T^*(\mu) = \varphi$ or
    $(T^*(\mu))(v) = \mu(T(v)) = \varphi(v)$
    for all $v \in V$. \gap

    
    % Let $\mathcal{B} = \{w_1,w_2,...,w_s\}$ be a basis for $Image(T)$, and
    % let $\mathcal{C} = \{w_1,...,w_s,w_{s+1},...,w_n\}$ 
    % be $\mathcal{B}$ extended to be a basis for $W$. 
    Let $\varphi \in V^*$ be arbitrary. 
    Let $\mathcal{B}$ be a basis for $W$.
    Since T is injective, there exists a left-inverse $T\inv$ for it. 
    We can uniquely define a linear transformation by where it maps
    the basis vectors in $\mathcal{B}$. 
    For all $b \in \mathcal{B}$,
    let $\mu \in W^*$ such that
    if $b \in Image(T)$, then
    $\mu(b) = \varphi(T\inv(b))$.
    Otherwise,
    $\mu(b) = 0$.
    In other words,
    \[\mu(b) = 
    \begin{dcases}
        \varphi(T\inv(b)) & b \in Image(T)\\
        0 & b \notin Image(T)
    \end{dcases}
    \]\gap

    % For all $w \in W$, $w$ can be written 
    % as a linear combination of the vectors in $\mathcal{C}$,
    % $c_1v_1 + ... + c_sv_s + c_{s+1}v_{s+1} + ... + c_nv_n$,
    % for some $c_1,c_2,...c_n \in \mathbb{F}$.
    % Let $\mu \in W^*$ such that
    %     $\mu(w) 
    %     = \mu(c_1v_1 + ... + c_sv_s + c_{s+1}v_{s+1} + ... c_nv_n)
    %     := \varphi(T\inv(c_1v_1 + ... + c_sv_s))$. \gap

    If $w \in Image(T)$, then
    $\mu(w) = \varphi(T\inv(w))$.
    Since
        $\mu(T(v))
        = \varphi(T\inv(T(v)))
        = \varphi(v)$
    for all $v \in V$, 
    there exists a $\mu$ where $\mu(T(v)) = \varphi(v)$,
    and thus $T^*$ is surjective.
\end{proof}\vspace{4mm}

\begin{proof} If $T$ is surjective, then $T^*$ is injective \gap

    The definition of $T^*$ on some $\mu \in W^*$ 
    and some $v \in V$ is
    $(T^*(\mu))(v) = \mu(T(v))$.
    $T^*$ is injective if and only if for all $\varphi,\psi \in W^*$,
    $\varphi \neq \psi$ implies $T^*(\varphi) \neq T^*(\psi)$, or 
    $\varphi(T(v)) \neq \psi(T(v))$ for some $v \in V$.\gap

    Let $\varphi,\psi \in W^*$ be arbitrary linear functionals 
    such that $\varphi \neq \psi$.
    Let $\mathcal{B}$ be a basis for $W$.
    Since $\varphi \neq \psi$, 
    there exists a $b \in \mathcal{B}$ such that
    $\varphi(b) \neq \psi(b)$.
    Since T is surjective,
    there exists a $v \in V$ such that
    $T(v) = b$.\gap

    Since $\varphi(T(v)) = \varphi(b) \neq \psi(b) = \psi(T(v))$,
    we know that $\varphi \neq \psi$ implies $T^*(\varphi) \neq T^*(\psi)$,
    so $T^*$ is injective.
\end{proof}