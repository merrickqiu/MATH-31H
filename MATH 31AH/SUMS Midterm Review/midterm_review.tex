\documentclass[12pt, letterpaper, twoside]{article}
\usepackage[utf8]{inputenc}
\usepackage[margin=2cm]{geometry}
\usepackage{amsmath}
\usepackage{amssymb}
\usepackage{amsthm}
\usepackage{gauss}

\newcommand{\gap}{\vspace{2mm}}
\newcommand{\inv}{^{-1}}

\title{SUMS 31AH - Midterm Review}
\author{Merrick Qiu}
\date\today

\begin{document}

\maketitle
\section{Find the change of basis matrix P, from $\beta$ to $\alpha$}

$\alpha = \{
    \begin{bmatrix}
        1 & -1\\
        0 & 0
    \end{bmatrix},
    \begin{bmatrix}
        1 & 1\\
        0 & 0
    \end{bmatrix},
    \begin{bmatrix}
        0 & 0\\
        1 & -1
    \end{bmatrix},
    \begin{bmatrix}
        0 & -1\\
        0 & -1
    \end{bmatrix}
\}$ \\
$\beta = \{
    \begin{bmatrix}
        1 & -2\\
        1 & -2
    \end{bmatrix},
    \begin{bmatrix}
        1 & 1\\
        -1 & 1
    \end{bmatrix},
    \begin{bmatrix}
        2 & -1\\
        1 & -2
    \end{bmatrix},
    \begin{bmatrix}
        -2 & 0\\
        1 & -1
    \end{bmatrix}
\}$\gap

$\begin{bmatrix}
    1 & -2\\
    1 & -2
\end{bmatrix} =
\begin{bmatrix}
    1 & -1\\
    0 & 0
\end{bmatrix} + 
\begin{bmatrix}
    0 & 0\\
    1 & -1
\end{bmatrix} + 
\begin{bmatrix}
    0 & -1\\
    0 & -1
\end{bmatrix}$

$\begin{bmatrix}
    1 & 1\\
    -1 & 1
\end{bmatrix} =
\begin{bmatrix}
    1 & 1\\
    0 & 0
\end{bmatrix} -
\begin{bmatrix}
    0 & 0\\
    1 & -1
\end{bmatrix}$

$\begin{bmatrix}
    2 & -1\\
    1 & -2
\end{bmatrix} =
\begin{bmatrix}
    1 & -1\\
    0 & 0
\end{bmatrix} + 
\begin{bmatrix}
    1 & 1\\
    0 & 0
\end{bmatrix} +
\begin{bmatrix}
    0 & 0\\
    1 & -1
\end{bmatrix} +
\begin{bmatrix}
    0 & 0\\
    1 & -1
\end{bmatrix}$

$\begin{bmatrix}
    -2 & 0\\
    1 & -1
\end{bmatrix} =
-\begin{bmatrix}
    1 & -1\\
    0 & 0
\end{bmatrix} -
\begin{bmatrix}
    1 & 1\\
    0 & 0
\end{bmatrix} +
\begin{bmatrix}
    0 & 0\\
    1 & -1
\end{bmatrix}$\gap

$P = 
\begin{bmatrix}
    1 & 0 & 1 & -1\\
    0 & 1 & 1 & -1\\
    1 & -1 & 1 & 1\\
    1 & 0 & 1 & 0
\end{bmatrix}$

\section{Union and Intersection of two subspaces a subspace?}\gap

The union of two subspaces is not necessarily a subspace(for $dim(v) >= 2$).
For example, the x and y axis.
The intersection of two subspaces is a subspace.
Zero is in $S_1$ and $S_2$, so zero is in $S_1 \cap S_2$.
Let $c \in \mathcal{F}$ and $v \in S_1 \cap S_2$.
Since $cv \in S_1$ and $cv \in S_2$, we know that
$cv \in S_1 \cap S_2$.
Let $v,w \in S_1 \cap S_2$.
Since $v+w \in S_1$ and $v+w \in S_2$,
$v+w \in S_1 \cap S_2$.

\section{Linear transformation $T: \mathbb{R} \rightarrow \mathbb{R}$ from [-1, 1] to [1,3]?}

Since [-1, 1] contains zero but [1,3] doesn't,
and a linear transformation always maps zero to zero,
the statement is false.
Think of linear transformations from $\mathbb{R}$ to $\mathbb{R}$
as scalar multiplication.

\section{If A and AB are invertible, then B is invertible.}

Socks and shoes principle: 
If A and B are invertible, then $(AB)\inv = B\inv A\inv$
Hint: If T invertible, then exists U such that $I = TU = UT$.

\begin{align}
    (AB) \inv AB = I 
    &\implies ((AB) \inv A)B = I \\
    &\implies (AB)\inv A = B\inv (Left)
\end{align}

\begin{align}
    AB(AB)\inv = I
    &\implies B (AB)\inv = A\inv \\
    &\implies B (AB)\inv A = A\inv A = I \\
    &\implies (AB)\inv A = B\inv (Right) 
\end{align}



\section{Does there exist 2024 independent vectors in $\mathbb{R}^{2021}$?}

No, a basis is the maximal linearly independent set, 
but since the basis has 2021 elements,
there cannot be a linearly independent set with 2024 elements.
\end{document}

