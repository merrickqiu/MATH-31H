\section{Invariant subspaces and block matrices}

\begin{proof} There exists an ordered basis of V,
    $\mathcal{B} = (v_1,v_2,...,v_m,v_{m+1},v_{m+2},...,v{n})$,
    such that
    $(v_1,v_2,...,v_m)$
    is an ordered basis of W.\gap

    Let $(v_1,v_2,...,v_m)$ be an ordered basis for $W$.
    Since W is a finite linearly independent subset of $V$
    ,and using problem 7 from problem set 3, 
    $(v_1,v_2,...,v_m)$ can be completed to formed
    an ordered basis of $V$.
\end{proof}

\begin{proof} T is invariant iff the lower-left block of the representing matrix is zeros.\gap
    
    Since $\mathcal{B}$ is linearly independent, 
    none of the vectors $(v_{m+1},v_{m+2},...,v_n)$
    can be formed as a linear combination 
    of the vectors from $(v_1,v_2,...,v_m)$.
    Since $(v_1,v_2,...,v_m)$ is a basis for $W$,
    the vectors $(v_{m+1},v_{m+2},...,v_n)$ 
    are not in $W$. 
    Therefore, the corresponding vector in $\mathbb{F}^n$
    for a vector $w \in W$ must 
    have zeros in indexes $m+1$ to $n$.\gap

    If a vector, $w$, has zeros in indexes $m+1$ to $n$,
    then it will be a linear combination of 
    the vectors in $(v_1,v_2,...,v_m)$.
    Since $(v_1,v_2,...,v_m)$ is a basis for $W$, 
    then $w \in W$.
    Therefore a vector is in $W$ 
    if and only if it has zeros 
    in indexes $m+1$ to $n$.\gap

    $W$ is invariant under $T$ if and only if
    the any vector with zeros in indexes $m+1$ to $n$
    maintains zeros in indexes $m+1$ to $n$ after the transformation $T$.
    If the bottom left $(n-m)\times(m)$ block of $[T]_\mathcal{B}^\mathcal{B}$
    is not all zeros, then it is possible for $T(w)$ to have a nonzero value
    in indexes $m+1$ to $n$ when $w$ has zero values for these indexes.
    Therefore W is only invariant under $T$ if and only if 
    the bottom left $(n-m)\times(m)$ block of 
    $[T]_\mathcal{B}^\mathcal{B}$ is all zeros.
\end{proof}
