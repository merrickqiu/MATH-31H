\section{Linear maps and independence}
\begin{proof} $T(I) := \{T(v):v \in I\}$ is linearly independent.\gap

    Since T is injective and linear, we have that 
    \begin{align*}
        c_1T(v_1) + c_2T(v_2) + ... + c_nT(v_n) = 0 &\xrightarrow{Linear} \\ 
        T(c_1v_1 + c_2v_2 + ... + c_nv_n) = 0 &\xrightarrow[T\inv(0)=0]{Injective} \\
        c_1v_1 + c_2v_2 + ... + c_nv_n = 0
    \end{align*}
     

    Therefore, if a nontrivial linear combination of the vectors in $T(I)$ exists,
    it would imply that a nontrivial linear combination of vectors in $I$ exists,
    which contradicts the fact that $I$ is linearly independent.
    Therefore $T(I)$ is linearly independent.\gap

    If $T$ was not injective, then it would be possible for 
    two vectors in $V$ to map to the same vector, 
    which would make $T(I)$ not linearly independent.
    Therefore, it doesn't hold if the assumption of injectivity is removed.
\end{proof}