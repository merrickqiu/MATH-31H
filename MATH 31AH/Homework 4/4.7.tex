\section{Polynomial change of basis}

These are the transformations of the basis vectors
from  $\mathcal{B}$ to $\mathcal{C}$:\\ 
$1 = 1$\\
$x+1 = 2(1) + 1(x-1)$\\
$x^2+x+1 = 2(1) + 2(x-1) + (x^2-x+1)$\\
$x^3+x^2+x+1 = 2(1) + 2(x-1) + 2(x^2-x+1) + (x^3-x^2+x-1)$ \gap \\
Therefore, the transition matrix can be written as 
$[T]_\mathcal{C}^\mathcal{B} =
\begin{bmatrix} 
    1 & 2 & 2 & 2 \\
    0 & 1 & 2 & 2 \\
    0 & 0 & 1 & 2 \\
    0 & 0 & 0 & 1
\end{bmatrix}$. \gap \\
These are the transformations of the basis vectors
from $\mathcal{C}$ to $\mathcal{B}$:\\
$1 = 1$\\
$x-1 = -2(1) + 1(x+1)$\\
$x^2-x+1 = 2(1) - 2(x+1) + 1(x^2+x+1)$\\
$x^3-x^2+x-1 = -2(1) + 2(x+1) - 2(x^2+x+1) + 1(x^3+x^2+x+1)$

Therefore, the transition matrix can be written as 
$[T]_\mathcal{B}^\mathcal{C} =
\begin{bmatrix} 
    1 & -2 & 2 & -2 \\
    0 & 1 & -2 & 2 \\
    0 & 0 & 1 & -2 \\
    0 & 0 & 0 & 1
\end{bmatrix}$.

