\section{Completing a basis}
\begin{proof}There exists vectors $v_{s+1},v_{s+2},...,v_n$ such that 
            $\{v_1,...,v_s,v_{s+1},...,v_n\}$ is a basis of $V$.\gap

    Let $\mathcal{S}=\{v_1,v_2,...,v_s\}$ be a 
    linearly independent subset of the $n$-dimensional vector space $V$.
    From lecture 8, the basis for an $n$-dimensional vector space must be have $n$ elements.
    Since the dimension of $\mathcal{S}$ is less than $n$, 
    it cannot be a basis for $V$, and it cannot span $V$.
    Therefore there exists some vector $v_{s+1} \in V$ that is not in the span of 
    $\mathcal{S}$, and is independent from all other vectors in $\mathcal{S}$.
    This vector can be added to $\mathcal{S}$ and it will still be linearly independent.
    A total of $n-s$ vectors can be added to $\mathcal{S}$ in a simmilar fashion
    until the size of $\mathcal{S}$ is $n$-dimensions.
    At this point, since $\mathcal{S}$ will have $n$ vectors,
    and it is linearly independent,
    from the theorem of lecture 8, $\mathcal{S}$ will be a basis for $V$.
    Thus there exists vectors $v_{s+1},v_{s+2},...,v_n$ such that
    $\{v_1,...,v_s,v_{s+1},...,v_n\}$ is a basis of $V$.
\end{proof}