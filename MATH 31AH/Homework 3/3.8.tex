\section{Trimming down to a basis}
\begin{proof} A subset $\{v_1,...,v_m\}$ can be "trimmed down" to a basis $\{v_{i_1},...,v_{i_n}\}$ of $V$.\gap

    Let $\mathcal{S} = \{v_1,...,v_m\}$ be a subset of $V$ that spans $V$. 
    If $\mathcal{S}$ is linearly independent, then it is already a basis for $V$,
    and it must have $n$-elements as shown in lecture 8.
    This means that $\mathcal{S}$ is a basis $\{v_{i_1},...,v_{i_n}\}$ of $V$.
    If $\mathcal{S}$ is linearly dependent,
    then that means a nontrivial linear combination 
    of vectors in $\mathcal{S}$ is zero, meaning
    $a_1v_1 + a_2v_2 + ... + a_mv_m = 0$.
    This means that some vector $v_i$ with a nonzero coefficient $a_i$
    can be formed as a linear combination of the other vectors, i.e.
    $-\frac{a_1}{a_i}v_1 - \frac{a_2}{a_i}v_2 - ... - \frac{a_m}{a_i}v_m = v_i$.\gap

    If $v_i$ is removed from $\mathcal{S}$, then it will still span $V$,
    since any linear combination with $v_i$ can substituted in with the linear combination
    $-\frac{a_1}{a_i}v_1 - \frac{a_2}{a_i}v_2 - ... - \frac{a_m}{a_i}v_m$.
    Vectors can be removed from $\mathcal{S}$ until 
    $\mathcal{S}$ is linearly independent.
    At this point, $\mathcal{S}$ will be a basis for $V$ since
    it is linearly independent and it still spans $V$.
    $\mathcal{S}$ cannot have more than $n$ elements at this point 
    because that would imply the existence of a basis 
    with more elements than the vector space, 
    which was shown in lecture 8 to be impossible.
    Thus, $\mathcal{S}$ would have to have $n$ elements, and so
    a basis $\{v_{i_1},...,v_{i_n}\}$ of $V$ exists.
\end{proof}