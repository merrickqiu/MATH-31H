\section{Direct Sum}

\begin{proof} $\mathcal{B} \oplus \mathcal{C}$ is a basis for $V \oplus W$.\gap

    Let $(a,b) \in V \oplus W$. 
    $(a,0)$ can be written as a linear combination
    of vectors in $\{(v,0) : v \in \mathcal{B}\}$ 
    since $\mathcal{B}$ spans V.
    Simmilarly, $(0,b)$ can be written as a linear combinations 
    of vectors in $\{(0,w) : w \in \mathcal{C}\}$ 
    since $\mathcal{C}$ spans W.
    Since $(a,0) + (0,b) = (a,b)$, 
    $(a,b)$ can be written as a linear combination of vectors in
    $\{(v,0) : v \in \mathcal{B}\} \cup \{(0,w) : w \in \mathcal{C}\}$.
    Therefore, $\mathcal{B} \oplus \mathcal{C}$ spans $V \oplus W$.\gap

    
    The only way for a linear combination of $\mathcal{B} \oplus \mathcal{C}$
    to be zero is if the linear combination of the vectors in 
    $\{(v,0) : v \in \mathcal{B}\}$ is zero 
    and the linear combination of vectors in 
    $\{(0,w) : w \in \mathcal{C}\}$ is zero.
    Since $\mathcal{B}$ is linearly independent, 
    and $\mathcal{C}$ is linearly independent,
    The only linear combination of vectors in $\mathcal{B} \oplus \mathcal{C}$ 
    that are zero is the trivial linear combination.
    Therefore, $\mathcal{B} \oplus \mathcal{C}$  is linearly independent.\gap

    Since $\mathcal{B} \oplus \mathcal{C}$ is linearly independent,
    and it spans $V \oplus W$,
    $\mathcal{B} \oplus \mathcal{C}$ is a basis for $V \oplus W$.
\end{proof}