\section*{Problem 4}
\begin{proof} There exists $\lambda$ with $\lambda(v) \neq 0$ if $v \neq 0$.\gap

    Let $\mathcal{B} = \{e_1,...,e_n\}$ be 
    the canonical basis of $\mathbb{R}^n$.
    Let $\mathcal{B}^* = \{\lambda_1,...,\lambda_n\}$ be 
    the dual basis of $\mathbb{R}^n$.
    Since $v \neq 0$, there exists a linear combination of the basis vectors 
    such that $v = c_1e_1 + ... + c_ie_i + ... + c_ne_n$ with some $c_i \neq 0$.
    We have that for $\lambda_i$,
    \begin{align*}
        \lambda_i(v)
        &= \lambda_i(c_1e_1 + ... + c_ie_i + ... c_ne_n)\\
        &= c_i\\
        &\neq 0
    \end{align*}
    Therefore, there exists $\lambda$ with $\lambda(v) \neq 0$ if $v \neq 0$.
\end{proof}