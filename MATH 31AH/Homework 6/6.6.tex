\newcommand{\eigenvector}[1]{
    \begin{bmatrix} 
        \frac{1}{#1}, 
        \frac{1}{(#1)^2}, 
        \frac{1}{(#1)^3},
        \frac{1}{(#1)^4},
        \frac{1}{(#1)^5},
        1,
     \end{bmatrix}
}
\section{A 6x6 example}

We have that
\[
A-\lambda I =
\begin{bmatrix}
    -\lambda & 0 & 0 & 0 & 0 & 1\\
    1 & -\lambda & 0 & 0 & 0 & 0\\
    0 & 1 & -\lambda & 0 & 0 & 0\\
    0 & 0 & 1 & -\lambda & 0 & 0\\
    0 & 0 & 0 & 1 & -\lambda & 0\\
    0 & 0 & 0 & 0 & 1 & -\lambda\\
\end{bmatrix}
\]

There are only two possible permutations,
or "rook placements" that result in non-negative terms, so
\begin{align}
    \det(A-\lambda I)
    &= (-\lambda)^6 - 1^6 \\
    &= \lambda^6 - 1
\end{align}

$\lambda^6 - 1 = 0$ when $\lambda = 
1, \eulerID{\pi}{3}, \eulerID{2\pi}{3},
-1, \eulerID{4\pi}{3}, \eulerID{5\pi}{3}$, 
the 6 roots of unity.
The row reduced matrix of $A-\lambda I$ is
\[
\begin{bmatrix}
    -\lambda & 0 & 0 & 0 & 0 & 1\\
    0 & -\lambda & 0 & 0 & 0 & \frac{1}{\lambda}\\
    0 & 0 & -\lambda & 0 & 0 & \frac{1}{\lambda^2}\\
    0 & 0 & 0 & -\lambda & 0 & \frac{1}{\lambda^3}\\
    0 & 0 & 0 & 0 & -\lambda & \frac{1}{\lambda^4}\\
    0 & 0 & 0 & 0 & 0 & \frac{-\lambda^6+1}{\lambda^5}
\end{bmatrix}
\]
Since  $\lambda^6 - 1 = 0$, 
we have that $\frac{-\lambda^6+1}{\lambda^5} = 0$.
The rref of $A-\lambda I$ is thus
\[
\rref(A-\lambda I) =
\begin{bmatrix}
    1 & 0 & 0 & 0 & 0 & -\frac{1}{\lambda}\\
    0 & 1 & 0 & 0 & 0 & -\frac{1}{\lambda^2}\\
    0 & 0 & 1 & 0 & 0 & -\frac{1}{\lambda^3}\\
    0 & 0 & 0 & 1 & 0 & -\frac{1}{\lambda^4}\\
    0 & 0 & 0 & 0 & 1 & -\frac{1}{\lambda^5}\\
    0 & 0 & 0 & 0 & 0 & 0
\end{bmatrix}
\]

Using the rref, we can determine the eigenvectors of the eigenspace.
\begin{align}
    v_1 &= \{\eigenvector{1}\}\\
    v_{\eulerID{\pi}{3}} &= \{\eigenvector{\eulerID{\pi}{3}}\}\\
    v_{\eulerID{2\pi}{3}} &= \{\eigenvector{\eulerID{2\pi}{3}}\}\\
    v_{-1} &= \{\eigenvector{-1}\}\\
    v_{\eulerID{4\pi}{3}} &= \{\eigenvector{\eulerID{4\pi}{3}}\}\\
    v_{\eulerID{5\pi}{3}} &=\{\eigenvector{\eulerID{5\pi}{3}}\}
\end{align}
