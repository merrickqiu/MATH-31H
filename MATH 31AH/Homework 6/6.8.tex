\section{Commuting operators and eigenspaces}
\begin{proof} $E_\lambda$ is $U$-invariant when $T$ and $U$ are commutative. \gap
    
    Let $T, U: V \rightarrow V$ be two linear transformations that commute.
    Let $v \in  E_\lambda$ be an eigenvector.
    Since it is a T-eigenvector, we have that $T(v) = \lambda(v)$.
    Since $T$ and $U$ commute, we have that
    \begin{align}
        U(T(v)) 
        &= U(\lambda v) \\
        &= \lambda U(v) \\
        &= T(U(v))
    \end{align}
    Since $T(U(v)) = \lambda U(v)$, 
    we know that $U(v) \in E_\lambda$,
    and so $E_\lambda$ is $U$-invariant.\gap
\end{proof}

\begin{proof}
    $E_\lambda$ is not necessarily $U$-invariant 
    when $T$ and $U$ are not commutative.\gap

    Let $T, U: \reals^2 \rightarrow \reals^2$ such that
    $T$ projects the vector onto the x-axis 
    and $U$ rotates the vector by $\pi/2$.
    $E_\lambda$ would clearly not be $U$-invariant since 
    the only eigenspace of T would be the x-axis, but
    $U$ rotates vectors off of the x-axis.
\end{proof}