\section{$\reals$ and $\complex$ and diagonalizability}

The matrix has eigenvalues when
$\det(R_\theta - \lambda I) = 0$.
We have that the characteristic polynomial is
\begin{align}
    (\cos(\theta)-\lambda)^2 + \sin^2(\theta)
    &= \cos^2(\theta) - 2\lambda + \lambda^2 + \sin^2(\theta)\\
    &= 1 - 2\lambda\cos(\theta) + \lambda^2
\end{align}
Using the quadratic formula, we get that our eigenvalues are
\begin{align}
    \lambda
    &= \frac{2\cos(\theta) \pm \sqrt{4\cos^2(\theta) - 4}}{2}\\
    &= \cos(\theta) \pm \sqrt{\cos^2(\theta) - 1}\\
    &= \cos(\theta) \pm i \sin(\theta)\\
    &= e^{\pm i\theta}
\end{align}

Our eigenspace matrices are
\[
E_{e^{-i\theta}} =
\begin{bmatrix}
    i \sin(\theta) & -\sin(\theta)\\
    \sin(\theta) & i \sin(\theta)
\end{bmatrix} 
\]
\[
E_{e^{i\theta}} =
\begin{bmatrix}
    -i \sin(\theta) & -\sin(\theta)\\
    \sin(\theta) & -i \sin(\theta)
\end{bmatrix}
\]

So the corresponding eigenvectors are 
$v_{e^{-i\theta}} = (-i, 1)$ and
$v_{e^{i\theta}} = (i, 1)$
Over $\reals$, $R_\theta$ is only diagonalizable when $\theta = 0, \pi$
since these are the only two angles that result in real eigenvalues.
Over $\complex$, $R_\theta$ is diagonalizable for all $\theta$
because there are always two eigenvalues.
