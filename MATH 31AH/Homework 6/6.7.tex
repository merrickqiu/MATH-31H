\section{A real sequence}

\begin{proof} $A^n = PD^nP\inv$ if $A = PDP\inv$.\gap

    We will induct on n.
    For $n=1$, we have that $A^1 = PD^1P\inv$.
    For $n>1$, assume that $A^{n-1} = PD^{n-1}P\inv$.
    We have that
    \begin{align}
        A^n
        &= A^{n-1} A \\
        &= (PD^{n-1}P\inv) (PDP\inv) \\
        &= PD^nP\inv
    \end{align}
    This completes the inductive step, 
    so $A^n = PD^nP\inv$.
\end{proof}

\begin{proof} 
    When $a_1 = a_2 = 1$ and $a_n = 4a_{n-1} - 2a_{n-2}$,
    $a_n =  \frac{(2+\sqrt{2})^{n-2} + (2-\sqrt{2})^{n-2}}{2}$ for $n>2$. \gap

    The nth number in the sequence can be found using the following equation:
    \[
    \begin{bmatrix} a_n \\ a_{n-1} \end{bmatrix} =
    \begin{bmatrix}
        4 & -2 \\
        1 & 0
    \end{bmatrix}^{n-2}
    \begin{bmatrix} 1 \\ 1 \end{bmatrix}
    \]

    The matrix has characteristic polynomial of
    $(4-\lambda)(-\lambda) + 2 = \lambda^2 - 4\lambda + 2$.
    Using the quadratic formula, we can find that
    the matrix has eigenvalues of $\lambda_1 = 2+\sqrt{2}$, $\lambda_2 = 2-\sqrt{2}$.
    The eigenvectors for these eigenvalues are
    $v_1 = (2+\sqrt{2}, 1)$ and $v_2 = (2-\sqrt{2}, 1)$
    since they solve the corresponding homologous system of equations for the eigenspace.\gap

    Since there are two eigenvectors, 
    we can diagonalize the matrix and 
    simplify the expression to
    \begin{align}
        \begin{bmatrix} a_n \\ a_{n-1} \end{bmatrix} 
        &=
        \begin{bmatrix}
            2+\sqrt{2}& 2-\sqrt{2} \\
            1 & 1
        \end{bmatrix}
        \begin{bmatrix}
            2+\sqrt{2} & 0 \\
            0 & 2-\sqrt{2}
        \end{bmatrix}^{n-2}
        \begin{bmatrix}
            \frac{1}{2\sqrt{2}} & \frac{\sqrt{2}-2}{2\sqrt{2}} \\
            -\frac{1}{2\sqrt{2}} & \frac{\sqrt{2}+2}{2\sqrt{2}}
        \end{bmatrix}
        \begin{bmatrix} 1 \\ 1 \end{bmatrix}\\
        &=
        \frac{1}{2\sqrt{2}}
        \begin{bmatrix}
            2+\sqrt{2}& 2-\sqrt{2} \\
            1 & 1
        \end{bmatrix}
        \begin{bmatrix}
            (2+\sqrt{2})^{n-2} & 0 \\
            0 & (2-\sqrt{2})^{n-2}
        \end{bmatrix}
        \begin{bmatrix} \sqrt{2}-1 \\ \sqrt{2}+1 \end{bmatrix}\\
        &=
        \frac{1}{2\sqrt{2}}
        \begin{bmatrix}
            2+\sqrt{2}& 2-\sqrt{2} \\
            1 & 1
        \end{bmatrix}
        \begin{bmatrix} 
            (\sqrt{2}-1)(2+\sqrt{2})^{n-2}\\ 
            (\sqrt{2}+1)(2-\sqrt{2})^{n-2}
        \end{bmatrix}\\
        &=
        \frac{1}{2\sqrt{2}}
        \begin{bmatrix} 
            \sqrt{2}(2+\sqrt{2})^{n-2} + \sqrt{2}(2-\sqrt{2})^{n-2}\\ 
            (\sqrt{2}-1)(2+\sqrt{2})^{n-2} + (\sqrt{2}+1)(2-\sqrt{2})^{n-2}
        \end{bmatrix}\\
    \end{align}
    Therefore, $a_n =  \frac{(2+\sqrt{2})^{n-2} + (2-\sqrt{2})^{n-2}}{2}$ for $n>2$.
    % This simplifies to
    % \begin{align}
    %     \begin{bmatrix} a_n \\ a_{n-1} \end{bmatrix} 
    %     &=
    %     \begin{bmatrix}
    %         2+\sqrt{2}& 2-\sqrt{2} \\
    %         1 & 1
    %     \end{bmatrix}
    %     \begin{bmatrix}
    %         2+\sqrt{2} & 0 \\
    %         0 & 2-\sqrt{2}
    %     \end{bmatrix}^{n-2}
    %     \begin{bmatrix}
    %         \frac{1}{2\sqrt{2}} & \frac{\sqrt{2}-2}{2\sqrt{2}} \\
    %         -\frac{1}{2\sqrt{2}} & \frac{\sqrt{2}+2}{2\sqrt{2}}
    %     \end{bmatrix}
    %     \begin{bmatrix} 1 \\ 1 \end{bmatrix}\\
    %     &=
    %     \begin{bmatrix}
    %         2+\sqrt{2}& 2-\sqrt{2} \\
    %         1 & 1
    %     \end{bmatrix}
    %     \begin{bmatrix}
    %         (2+\sqrt{2})^{n-2} & 0 \\
    %         0 & (2-\sqrt{2})^{n-2}
    %     \end{bmatrix}
    %     \begin{bmatrix}
    %         1 & \sqrt{2}-2 \\
    %         -1 & \sqrt{2}+2
    %     \end{bmatrix}
    %     \begin{bmatrix} \frac{1}{2\sqrt{2}} \\ \frac{1}{2\sqrt{2}} \end{bmatrix}\\
    %     &=
    %     \begin{bmatrix}
    %         2+\sqrt{2}& 2-\sqrt{2} \\
    %         1 & 1
    %     \end{bmatrix}
    %     \begin{bmatrix}
    %         (2+\sqrt{2})^{n-2} & 2(2+\sqrt{2})^{n-3} \\
    %         -(2-\sqrt{2})^{n-2} & 2(2+\sqrt{2})^{n-3}
    %     \end{bmatrix}
    %     \begin{bmatrix} \frac{1}{2\sqrt{2}} \\ \frac{1}{2\sqrt{2}} \end{bmatrix}\\
    %     &=
    %     \begin{bmatrix}
    %         (2+\sqrt{2})^{n-1} - (2-\sqrt{2})^{n-1} & 2(2+\sqrt{2})^{n-2} + 2(2-\sqrt{2})^{n-2}\\
    %         (2+\sqrt{2})^{n-2} - (2-\sqrt{2})^{n-2} & 2(2+\sqrt{2})^{n-3} + 2(2-\sqrt{2})^{n-3}
    %     \end{bmatrix}
    %     \begin{bmatrix} \frac{1}{2\sqrt{2}} \\ \frac{1}{2\sqrt{2}} \end{bmatrix}\\
    %     &=
    %     \begin{bmatrix}
    %         \frac{(2+\sqrt{2})^{n-1} - (2-\sqrt{2})^{n-1} + 2(2+\sqrt{2})^{n-2} + 2(2-\sqrt{2})^{n-2}}{2\sqrt{2}}\\
    %         \frac{(2+\sqrt{2})^{n-2} - (2-\sqrt{2})^{n-2} + 2(2+\sqrt{2})^{n-3} + 2(2-\sqrt{2})^{n-3}}{2\sqrt{2}}
    %     \end{bmatrix}
    % \end{align}
    
    
\end{proof}