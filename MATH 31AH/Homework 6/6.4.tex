\section{Polynomials and diagonalizability}

\begin{proof}$f(A)$ is diagonalizable when $A$ is diagonalizable. \gap
       
    \textbf{Base Case:} Since $A$ is diagonalizable, 
    let $P$ be a matrix such that $A = PDP\inv$ for some $D$.
    $A^n$ can thus be written as $A = PD^nP\inv$.\footnote{See problem 7 for proof}
    Therefore $c_nA^n = P(c_nD^n)P\inv$, 
    so each of the terms in $f(A)$ is diagonalizable,
    including when $n=0$.\gap

    \textbf{Inductive Step:} For $n > 0$,
    assume that the sum of the first $n-1$ terms 
    is diagonalizable with $f_{n-1}(A) = PBP\inv$
    for some matrix $B$.
    Since the $nth$ term can be written as $c_nA^n = P(c_nD^n)P\inv$,
    the sum of the first $n$ terms is
    \begin{align}
        f_{n-1}(A) + c_nA^n
        &= PBP\inv +  P(c_nD^n)P\inv\\
        &= P(BP\inv + c_nD^nP\inv)\\
        &= P(B + c_nD^n)P\inv
    \end{align}
    This completes the inductive step, and so 
    $f(A)$ is diagonalizable when $A$ is diagonalizable.
\end{proof}

