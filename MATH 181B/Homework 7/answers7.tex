%Set document class
\documentclass{article}

%Load math symbol packages
\usepackage{amsmath}
\usepackage{amssymb}
\usepackage{mathtools}
\usepackage{indentfirst}
\usepackage{listings}

%User defined commands
\newcommand{\var}{\operatorname{Var}}
\newcommand{\cov}{\operatorname{Cov}}
\newcommand{\sumN}{\sum_{i=1}^{n}}
\newcommand{\sumM}{\sum_{i=1}^{m}}
\newcommand{\prodN}{\prod_{i=1}^{n}}
\newcommand{\prodM}{\prod_{i=1}^{m}}

\begin{document}
\begin{center}
	\huge{\bf Math 181B: Homework 7} \\
	Merrick Qiu 
\end{center}
\subsection*{Exercise 1}
	\begin{align*}
		\sum_{i=1}^{b} k(\bar{Y}_{i.} - \bar{Y}_{..})^2 
		&= \sum_{i=1}^{b} k(\bar{Y}_{i.} - \mu -\beta + \mu + \beta - \bar{Y}_{..})^2 \\
		&= \sum_{i=1}^{b} k(\bar{Y}_{i.} - \mu -\beta)^2
			+ \sum_{i=1}^{b} k(\bar{Y}_{..} - \mu -\beta)^2
			- 2 \sum_{i=1}^{b} k(\bar{Y}_{i.} - \mu -\beta)(\bar{Y}_{..} - \mu -\beta) \\
		&= \sum_{i=1}^{b} k(\bar{Y}_{i.} - \mu -\beta)^2
			+ bk(\bar{Y}_{..} - \mu -\beta)^2
			- 2bk(\bar{Y}_{..} - \mu -\beta)^2 \\
		&= \sum_{i=1}^{b} k(\bar{Y}_{i.} - \mu -\beta)^2 - bk(\bar{Y}_{..} - \mu -\beta)^2 \\
	\end{align*}
	\begin{align*}
		E\left[\sum_{i=1}^{b} k(\bar{Y}_{i.} - \mu -\beta)^2 - bk(\bar{Y}_{..} - \mu -\beta)^2\right]
		&= \sum_{i=1}^{b} kE[(\bar{Y}_{i.} - \mu -\beta)^2] - bkE[(\bar{Y}_{..} - \mu -\beta)^2] \\
		&= \sum_{i=1}^{b} k(\var(\bar{Y}_{i.}-\mu-\beta) + (E[\bar{Y}_{i.}-\mu-\beta])^2) -bk\var(\bar{Y}_{..}) \\
		&= \sum_{i=1}^{b} k\left(\frac{\sigma^2}{k} + (\beta_i -\beta + \mu - \mu)^2\right) - bk\frac{\sigma^2}{bk} \\
		&= \sigma^2(b-1) + \sum_{i=1}^{b} k(\beta_i -\beta)^2 \\
	\end{align*}
\newpage 

\subsection*{Exercise 2}
\begin{lstlisting}
score = c(5, 6.1, 7.2, 6, 5.9, 7, 6.5, 6.3, 7.3)
data = data.frame(score=score,
                  block = c(rep('Sit', 3), rep('Wait', 3), rep('Drop it',3)),
                  treatment = c(rep(c('Dog 1', 'Dog 2', 'Dog 3'), 3)))
summary(aov(score ~treatment + block, data))

#             Df Sum Sq Mean Sq F value Pr(>F)  
# treatment    2 2.9867  1.4933   8.145 0.0389 *
# block        2 0.5600  0.2800   1.527 0.3215  
# Residuals    4 0.7333  0.1833       
\end{lstlisting}

The dogs are the treatment groups and the commands are the blocks.
We have H0: $\mu_1 = \mu_2 = \mu_3$ and 
H1: the dogs take different amount of time to respond.
Since the p value of $0.0389$ is less than $0.05$,
we reject the null and the dogs do take different amounts of time to respond to commands. \\

To test if the commands have an effect, we have the hypothesis
We have H0: $\beta_1 = \beta_2 = \beta_3$ and 
H1: the commands have different response times.
Since the p value for the commands is $0.3215$ is greater than $0.05$,
we fail to reject the null and the commands do not have an important impact.
\newpage

\subsection*{Exercise 3}
\begin{lstlisting}
mu = c(1, 2, 3);
findRejections = function(beta) {
  df = data.frame();
  for (i in 1:3) {
    for (j in 1:3) {
      newRow = c(
        mu[i] + beta[j] + rnorm(1, 0,sqrt(0.4)),
        toString(i),
        toString(j)
      )
      df = rbind(df, newRow);
    }
  }
  colnames(df) = c('score','treatment','block');
  anova = summary(aov(score ~treatment, df));
  rbd = summary(aov(score ~treatment + block, df));

  anovaValue = anova[[1]][["Pr(>F)"]][1];
  rbdValue = rbd[[1]][["Pr(>F)"]][1];
  return(c(anovaValue < 0.05, rbdValue < 0.05))
}
# PART A
rejectionsA = matrix(NA, nrow=1000, ncol=2);
for (i in 1:1000) {
  rejectionsA[i,] = findRejections(c(1,1,1));
}
colSums(rejectionsA)/1000
# POWER OF ANOVA IS SLIGHTLY BETTER
# > colSums(rejectionsA)/1000
# [1] 0.761 0.627
# PART B
rejectionsB = matrix(NA, nrow=1000, ncol=2);
for (i in 1:1000) {
  rejectionsB[i,] = findRejections(c(1,2,3));
}
colSums(rejectionsB)/1000
# POWER OF ANOVA IS TERRIBLE
# > colSums(rejectionsB)/1000
# [1] 0.100 0.605
\end{lstlisting}

When the block effects are the same, the power of the anova test is slightly better,
but when the block effects are different, the rbd power is unaffected while the
anova test power suffers greatly. 

\end{document}






