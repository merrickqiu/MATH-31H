%Set document class
\documentclass{article}

%Load math symbol packages
\usepackage{amsmath}
\usepackage{amssymb}
\usepackage{mathtools}
\usepackage{indentfirst}
\usepackage{listings}

%User defined commands
\newcommand{\var}{\operatorname{Var}}
\newcommand{\cov}{\operatorname{Cov}}
\newcommand{\sumN}{\sum_{i=1}^{n}}
\newcommand{\sumM}{\sum_{i=1}^{m}}
\newcommand{\prodN}{\prod_{i=1}^{n}}
\newcommand{\prodM}{\prod_{i=1}^{m}}

\begin{document}
\begin{center}
	\huge{\bf Math 181B: Homework 6} \\
	Merrick Qiu 
\end{center}
\subsection*{Exercise 1}
\begin{align*}
	E[SSTOT] &= E[SSTR + SSE] \\
	&= E[SSTR] + E[SSE] \\
	&= E[SSTR] + E\left[\sum_{j=1}^k (n_j-1)S_j^2\right]  \\
	&= E[SSTR] +  \sum_{j=1}^k (n_j-1)\sigma^2 \\
	&= E[SSTR] +  (n-k)\sigma^2 \\
	&= (k-1)\sigma^2 + \sum_{j=1}^k n_j(\mu_j-\mu)^2 + (n-k)\sigma^2 \\
\end{align*}
\newpage

\subsection*{Exercise 2}
\begin{lstlisting}
setwd("C:/Users/merri/Documents/MATH-31H/MATH 181B/Homework 6")
teen = read.csv("Adolescent.csv")$x
adult = read.csv("Adult.csv")$x
middle = read.csv("MiddleAge.csv")$x

People = data.frame(times = c(teen, adult, middle), 
		Age= c(rep('Teen', length(teen)), rep('Adult', length(adult)),
		rep('Middle', length(middle))))
summary(aov(times~Age, People))

# > summary(aov(times~Age, People))
#           Df Sum Sq Mean Sq F value   Pr(>F)    
# Age       `2  173.6   86.80   22.46 3.08e-09 ***
# Residuals 147 568.1    3.86                     
# ---

# H0: mu_teen = mu_adult = mu_middle, H1: At least one mu is different
# Assume that Yij are independent, normal, and 
# have the same variation for each j
# Since the p-value of 3.08e-09 is <0.05, we reject the null hypothesis and 
# the mean screen times of all the age groups are not the same.
\end{lstlisting}

No, you cannot conclude that average adolescents spend more time in front of the screen
than the other two groups because the conclusion only says that at least one group
has a different mean. 
For the same reason, you cannot conclude that adults spend more time than middle aged adults.

\begin{lstlisting}
t.test(teen, adult, var.equal = T) # p-value = 1.667e-05
t.test(teen, middle, var.equal = T) # p-value = 2.448e-09
t.test(adult, middle, var.equal = T) # p-value = 0.04485

# Using the Bonferroni correction we use alpha/n = 0.01667
# Since mean of teen is greater than adult and middle and 
# the p-value is <alpha/n, we conclude that teens have more 
# screen time on average than adults and middle.
# Since 0.04485 > alpha/n, we cannot conclude that adults 
# watch more tv than middle.
\end{lstlisting}
\newpage 

\subsection*{Exercise 3}
\begin{lstlisting}
reject <- function(p) {
  first = rnorm(50, 1, 1);
  second = rnorm(50, 1, 1);
  third = rnorm(50, 1, 1);
  p1 = t.test(first, second, var.equal = T)$p.value;
  p2 = t.test(second, third, var.equal = T)$p.value; 
  p3 = t.test(first, third, var.equal = T)$p.value;
  return (p1<p | p2<p | p3<p);
}

regular_errors = rep(0, 1000);
bonferroni_errors = rep(0, 1000);
for(i in 1:1000) {
  regular_errors[i] = reject(0.05)
  bonferroni_errors[i] = reject(0.05/3)
}
sum(regular_errors)/1000
sum(bonferroni_errors)/1000
# > sum(regular_errors)/1000
# [1] 0.118
# 
# > sum(bonferroni_errors)/1000
# [1] 0.042

# The proportion of type 1 errors is indeed <0.05 for the bonferroni correction,
# but it is more than double 0.05 without the correction.
\end{lstlisting}
\end{document}






