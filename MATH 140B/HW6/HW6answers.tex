\documentclass{article}

\usepackage{amsmath}
\usepackage{amssymb}
\usepackage{hyperref}
\usepackage{indentfirst}
\usepackage{mathrsfs}
\usepackage{enumitem}

%User defined commands
\def\upint{\mathchoice%
    {\mkern13mu\overline{\vphantom{\intop}\mkern7mu}\mkern-20mu}%
    {\mkern7mu\overline{\vphantom{\intop}\mkern7mu}\mkern-14mu}%
    {\mkern7mu\overline{\vphantom{\intop}\mkern7mu}\mkern-14mu}%
    {\mkern7mu\overline{\vphantom{\intop}\mkern7mu}\mkern-14mu}%
  \int}
\def\lowint{\mkern3mu\underline{\vphantom{\intop}\mkern7mu}\mkern-10mu\int}

\begin{document}
\begin{center}
	\huge{\bf Math 140B: Homework 6} \\
	Merrick Qiu
\end{center}

\section*{Problem 1}
Note that $C_M^\alpha([0,1])$ is the set of all Holder continuous
functions on $[0,1]$ with exponent $\alpha$.
Thus, $C_M^\alpha([0,1])$ is equicontinuous since 
for all $\epsilon > 0$, we can choose $\delta < \left(\frac{\epsilon}{M}\right)^\frac{1}{\alpha}$
to bound $|f(x) - f(y)| < \epsilon$ for all $f \in C_M^\alpha([0,1])$ and $x, y \in [0,1]$.
Subsets of $C_M^\alpha([0,1])$ will also be equicontinuous, and if these
subsets are also closed and bounded, they are compact due to Arzela-Ascoli theorem.
\newpage 

\section*{Rudin 7.20}
Since the integral is linear operator and $\int_0^1 f(x)x^n\, dx = 0$ for all $n \in \mathbb{Z}$,
$\int_0^1 f(x)P(x)\, dx = 0$ for any polynomial $P(x)$ as well.
By the Weierstrass theorem, there exist a sequence of polynomials $P_n$ such that 
\[
  \lim_{n\to \infty} P_n(x) = f(x).
\]
If we take the sequence of the integrals of the product of $f$ with these polynomials,
then by Theorem $7.16$,
\[
  \lim_{n\to \infty} \int_0^1 f(x)P_n(x)\, dx = \int_0^1 f^2(x)\, dx = 0.
\]
Thus, $f(x) = 0$ on $[0,1]$.
\newpage 

\section*{Rudin 7.25}
Fix $n$. For $i = 0, \cdots, n$ put $x_i = i/n$. 
Let $f_n$ be a continuous function on $[0,1]$ such that $f_n(0) = c$,
\begin{equation}
  f_n'(t) = \phi(x_i, f_n(x_i)) \qquad \text{if } x_i < t < x_{i+1}
\end{equation}
and put 
\begin{equation}
  \Delta_n(t) = f'_n(t) - \phi(t,f_n(t))
\end{equation}
except at points $x_i$ where $\delta_n(t) = 0$. Then 
\begin{equation}
  f_n(x) = c \int_0^x [\phi(t,f_n(t)) + \Delta_n(t)] \,dt
\end{equation}
\begin{enumerate}[label=(\alph*)]
  \item Choose $M < \infty$ so that $|\phi| \leq M$, 
        which implies that $|f_n'| \leq M$ from $(1)$
        and $|\Delta_n| \leq 2M$ from $(2)$.
        $\Delta_n \in \mathscr{R}$ is Riemann integrable because
        $f_n'$ is a step function and $\phi$ is continuous, so from $(2)$
        $\Delta_n$ has finitely many discontinuities.
        Since $\Delta_n$ is Riemann-integrable over $[0,1]$, 
        $|f_n| \leq |c| + M = M_1$.
  \item Since $|f_n'| \leq M$ and $|f_n(x) - f_n(y)| =\leq\int_x^y |f_n'(t)| \,dt \leq M|x-y|$,
        $(f_n)$ is equicontinuous with $\delta = \frac{\epsilon}{M}$.
  \item Since $(f_n)$ is equicontinuous and bounded uniformly by $M_1$,
        by Arzela-Ascoli theorem there exists a subsequence that converges to some $f$ uniformly.
  \item Since $\phi$ is uniformly continuous, the sequence $f_{n_k} \to f$ converging uniformly
        implies that 
        \[
          \phi(t, f_{n_k}(t)) \to \phi(t,f(t))
        \]
        converges uniformly as well since $f_{n_k} - f$ can be arbitrarily small.
  \item Since $\phi$ converges uniformly and 
  \[
    \Delta_n(t) = \phi(x_i, f_n(x_i)) - \phi(t, f_n(t))
  \]
  $\Delta_n(t) \to 0$ since $f_{n_k} \to f$ uniformly and $x_i \to t$ as well.
  \item Thus the solution to the problem exists and is
  \[  
      f(x) = c + \int_0^x \phi(t, f(t)) \,dt
  \]
\end{enumerate}
\newpage 

\section*{Rudin 8.1}
Note that from induction,
\[
  \frac{d^n}{dx^n} e^{-1/x^2} = \frac{P_n(x)}{Q_n(x)} e^{-1/x^2}.
\]
In the base case where $n=1$,
\[
  \frac{d}{dx} e^{-1/x^2} = \frac{2}{x^3}e^{-1/x^2}
\]
When the statement is true for $n=k$ then for $n=k+1$
\begin{align*}
  \frac{d^{k+1}}{dx^{k+1}} e^{-1/x^2} 
  &= \frac{d}{dx} \left(\frac{P_k(x)}{Q_k(x)} e^{-1/x^2}\right) \\
  &= \frac{P_k(x)}{Q_k(x)} \frac{2}{x^3}e^{-1/x^2} + \frac{P_k'(x)Q_k(x) - P_k(x)Q_k'(x)}{Q_k(x)^2}e^{-1/x^2} \\
  &=\left( \frac{P_k(x)}{Q_k(x)} \frac{2}{x^3} + \frac{P_k'(x)Q_k(x) - P_k(x)Q_k'(x)}{Q_k(x)^2} \right)e^{-1/x^2} \\
\end{align*}


For all $n \in \mathbb{Z}$ if $y = \frac{1}{x}$ then from Theorem 8.6(f)
\[
  \lim_{x^+\to 0}\frac{e^{-1/x^2}}{x^n} 
  = \lim_{y\to \infty}y^ne^{-y^2}
  = 0
\]
\[
  \lim_{x^-\to 0}\frac{e^{-1/x^2}}{x^n} 
  = \lim_{y\to -\infty}y^ne^{-y^2}
  = 0
\]
Thus, the derivative evaluated at zero exists and is 
\[
  f^(n)(0) = \lim_{x\to 0} \frac{P_n(x)}{Q_n(x)} e^{-1/x^2} = 0
\]
\newpage 

\section*{Rudin 8.4}
\begin{enumerate}[label=(\alph*)]
  \item \[
    \lim_{x\to 0} \frac{b^x - 1}{x}
    = \frac{d}{dx}_{x=0} e^{x\log b}
    = [\log (b) e^{x\log b}]_{x=0}
    =\log b
  \]
  \item \[
    \lim_{x\to 0} \frac{\log(1+x)}{x}
    = \frac{d}{dx}_{x=0} \log(1+x)
    = \left[\frac{1}{1+x}\right]_{x=0}
    = 1
  \]
  \item \[
    \lim_{x\to 0} (1+x)^{1/x}
    = \lim_{x\to 0} e^\frac{\log(1+x)}{x}
    = e
  \]
  \item \[
    \lim_{x\to 0} \left(1+\frac{x}{n}\right)^n 
    = \left(\left(1 + \frac{x}{n}\right)^{1/(x/n)}\right)^n
    = e^x
  \]
\end{enumerate}
\newpage 

\section*{Rudin 8.5}
\begin{enumerate}[label=(\alph*)]
  \item \[
    \lim_{x\to 0} \frac{e-(1+x)^{1/x}}{x} 
    = \frac{d}{dx}_{x=0} (1+x)^{1/x}
    = \lim_{x\to 0} (1+x)^{1/x}\left(\frac{(1+x)\log(1+x) - x}{x^2(x+1)}\right)
    = \frac{e}{2}
  \]
  \item \[
  \lim_{n\to \infty} \frac{n}{\log n}[n^{1/n}-1]
  = \lim_{n\to \infty} \frac{e^{\frac{\log n}{n}}-1}{\frac{\log n}{n}}
  = \frac{d}{dx}_{x=0} e^x
  = 1
  \]
  \item \[
    \lim_{x\to 0} \frac{\tan x - x}{x(1-\cos x)} =
    \lim_{x\to 0} \frac{\sin x - x\cos x}{x\cos x (1-\cos x)} = 
    \lim_{x\to 0} \frac{x\sin x}{x\sin x -\cos x + 1} = 
    \lim_{x\to 0} \frac{\sin x + x\cos x}{2\sin x + x\cos x} = 
    \frac{2}{3}
  \]
  \item \[
    \lim_{x\to 0} \frac{x-\sin x}{\tan x - x} =
    \lim_{x\to 0} \frac{(x-\sin x)\cos x}{\sin x - x\cos x} = 
    \lim_{x\to 0} \frac{1-\cos x}{x\sin x} = 
    \lim_{x\to 0} \frac{\sin x}{x\cos x + \sin x} =
    \lim_{x\to 0} \frac{\cos x}{2\cos x} =
    \frac{1}{2}
  \]
\end{enumerate}
\newpage

\section{Rudin 8.6}
For $x\neq 0$ and $y=0$,  $f(x)f(0) = f(x+0)$ implies $f(0) = 1$.
We have that 
\[
  f'(x) = \lim_{h\to 0} \frac{f(x+h) - f(x)}{h}
  = \lim_{h\to 0} \frac{f(x)f(h) - f(x)}{h}
  = f(x) f'(0)
\]
If $c= f'(0)$ then the function $g(x) = e^{cx}$
satisfies the given conditions.
$f(x) = g(x)$ since both these functions have $f(0) = g(0) = 1$
and $\frac{d}{dx} \frac{f}{g} = 0$.
\newpage 

\section*{Rudin 8.7}
$\frac{\sin x}{x} < 1$ since $\sin x$ achieves a supremum of $1$ over the interval $[0, \frac{\pi}{2}]$.
$\frac{2}{\pi} < \frac{\sin x}{x}$ since over the interval $[0, \frac{\pi}{2}]$,
\[
  \frac{d}{dx} \frac{\sin x}{x}  = \frac{x\cos x -\sin x}{x^2} < 0
\]
so $\frac{\sin x}{x}$ is strictly decreasing.
Since $\frac{\sin(\pi/2)}{\pi/2} = \frac{2}{\pi}$, the inequality holds.
\newpage 
\end{document}