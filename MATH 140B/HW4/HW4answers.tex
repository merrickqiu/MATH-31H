\documentclass{article}

\usepackage{amsmath}
\usepackage{amssymb}
\usepackage{hyperref}
\usepackage{indentfirst}
\usepackage{mathrsfs}

%User defined commands
\def\upint{\mathchoice%
    {\mkern13mu\overline{\vphantom{\intop}\mkern7mu}\mkern-20mu}%
    {\mkern7mu\overline{\vphantom{\intop}\mkern7mu}\mkern-14mu}%
    {\mkern7mu\overline{\vphantom{\intop}\mkern7mu}\mkern-14mu}%
    {\mkern7mu\overline{\vphantom{\intop}\mkern7mu}\mkern-14mu}%
  \int}
\def\lowint{\mkern3mu\underline{\vphantom{\intop}\mkern7mu}\mkern-10mu\int}

\begin{document}
\begin{center}
	\huge{\bf Math 140B: Homework 4} \\
	Merrick Qiu
\end{center}

\section*{Rudin 19}
If $\gamma_1$ is an arc, it is 1-1.
Since $\phi$ is 1-1, $\gamma_2$ must also be 1-1
and so it is an arc as well.
Similarly since $\phi^{-1}$ exists and is 1-1,
$\gamma_1$ must be 1-1 if $\gamma_2$ is.

If $\gamma_1$ is a closed curve, then $\gamma_1(a) = \gamma_1(b)$.
Since $\phi$ is 1-1, continuous, and $\phi(c) = a$, it must be that $\phi(d) = b$.
Thus $\gamma_2(c) = \gamma_2(d)$ and $\gamma_2$ is also a closed curve.
Similarly since $\phi^{-1}$ exists and is 1-1,
$\gamma_2(c) = \gamma_2(d)$ implies $\gamma_1(a) = \gamma_1(b)$.

Since $\phi$ and $\phi^{-1}$ create corresponding partitions between
$[a,b]$ and $[c,d]$, one curve must be rectifiable if the other is,
and they must have the same length.
This is clear since if $\{x_i\}$ is a partition of $[c,d]$ then
\[
	\sum |\gamma_1(\phi(x_i)) - \gamma_1(\phi(x_{i-1}))| = \sum |\gamma_2(x_i) - \gamma_2(x_{i-1})|
\]
\newpage 

\section*{Rudin 1}
By definition, a sequence of functions uniformly converges $f_n \to f$
if for all $\epsilon > 0$ there exists $N$ such that $n \geq N$ implies 
\[
	|f_n(x) - f(x)| \leq \epsilon
\]
We can bound functions with $n \geq N$ below by $\inf f(x) - \epsilon$
and above by $\sup f(x) + \epsilon$.
Since there is a finite number of functions where $n < N$,
we can bound it below by $\inf_{n < N} \inf_{x \in \mathbb{R}} f_n(x)$
and above by $\sup_{n < N} \sup_{x \in \mathbb{R}} f_n(x)$.
Thus every uniformly convergent sequence of bounded functions is uniformly bounded.
\newpage 

\section*{Rudin 2}
If $(f_n)$ converges uniformly then there exists $N_1$ such that $n \geq N_1$ implies 
\[
	|f_n(x) - f(x)| < \frac{\epsilon}{2}.
\]
Similarly there exists $N_2$ such that for $n \geq N_2$
\[
	|g_n(x) - g(x)| < \frac{\epsilon}{2}.
\]
Thus for $n \geq \max (N_1,N_2)$,
\[
	|(f_n + g_n)(x) - (f+g)(x)| < \epsilon.
\]
Since $f_n$ and $g_n$ are uniformly bounded, there exists $M$
such that $|f_n(x)| < M$ and $|g_n(x)| < M$ for all $n$ and $x$.
Then choose $N_1$  so that for $n > N_1$
\[
	|f_n(x) - f(x)| < \frac{\epsilon}{2M}
\]
and choose $N_2$ so that for $n > N_2$
\[
	|g_n(x) - g(x)| < \frac{\epsilon}{2M}.
\]
Thus, 
\begin{align*}
	|f_n(x)g_n(x) - f(x)g(x)| &\leq |f_n(x)g_n(x) - f_n(x)g(x)| + |f_n(x)g(x) - f_n(x)g(x)| \\
	&\leq M |g_n(x) - g(x)| + M|f_n(x) - f(x)| \\
	&< M\frac{\epsilon}{2M} + M\frac{\epsilon}{2M} \\
	&= \epsilon
\end{align*}
\newpage 

\section*{Rudin 3}
$f_n(x) = x$ converges uniformly to $f(x) = x$ and 
$g_n(x) = \frac{1}{n}$ converges uniformly to $g(x) = 0$
but $f_n(x)g_n(x)$ does not converge because when $x=n$,
$f_n(x)g_n(x) = 1$.
\newpage 

\section*{Rudin 4}
The series diverges when $x=0$ since all the terms
are $1$ and the series diverges when $x = -\frac{1}{n^2}$
for all natural numbers $n$ since there is division by $0$
on the $n$th term.
For intervals $[\delta, \infty)$ for $\delta > 0$ the series converges uniformly
since we can bound each term with
\[
	\frac{1}{n^2x} \leq \frac{1}{n^2\delta}.
\]
For intervals $(-\infty, -\delta]$ and $x\neq -\frac{1}{n^2}$ the interval also converges uniformly
since for a sufficiently large $n$ such that $n^2 \geq \frac{2}{\delta}$ 
\[
	\left|\frac{1}{1+n^2x} \right| = \frac{1}{n^2}\frac{1}{\frac{1}{n^2}+x} 
	\leq \frac{1}{n^2}\frac{1}{\frac{1}{n^2}+\delta}
	\leq \frac{2}{\delta n^2} 
\]
\newpage 

\section*{Rudin 5}
$f_n$ converges to 0 since all points $x \leq 0$ always evaluate to $0$
and all positive points are $\frac{1}{n} < x$ for a sufficiently large $n$.
However it does not converge uniformly since for every $f_n$,
$f_n\left(\frac{1}{n+\frac{1}{2}}\right) = 1$.
\newpage 

\section*{Rudin 6}
The series can be rewritten as 
\[
	x^2 \sum \frac{(-1)^n}{n^2} + \sum \frac{(-1)^n}{n}
\]
The left series converges since $x^2\frac{(-1)^n}{n^2} \leq \frac{M}{n^2}$
if $M$ is the sup over an interval for that term.
The right sequence also converges, so the overall sequence converges uniformly.
However the series does not converge absolutely since 
each term is greater than $\frac{1}{n}$, which diverges.
\newpage 

\section*{Rudin 8}
Since each term is $< |c_n$ and $c_n$ converges,
$f_n$ converges uniformly.
Since each individual term is continuous for $x \neq x_n$,
the overall function is continuous as well.
\newpage 

\section*{Rudin 9}
Since $f_n$ converges uniformly, there exists $N_1$ such that
\[
	|f_n(x_n) - f_n(x)| < \frac{\epsilon}{2}.
\]
Since $f$ is continuous, there exists $N_2$ such that
\[
	|f(x_n) - f_n(x)| < \frac{\epsilon}{2}.
\] 
By triangle inequality,
\[
	|f_n(x_n) - f(x)| \leq |f_n(x_n) - f(x_n)| + |f(x_n) - f(x)| < \epsilon
\]
The converse is not true, as exemplified by $f_n = x^n$ for points approaching $1$.










\end{document}