\documentclass{article}

\usepackage{amsmath}
\usepackage{amssymb}
\usepackage{hyperref}
\usepackage{indentfirst}
\usepackage{mathrsfs}
\usepackage{enumitem}

%User defined commands
\def\upint{\mathchoice%
    {\mkern13mu\overline{\vphantom{\intop}\mkern7mu}\mkern-20mu}%
    {\mkern7mu\overline{\vphantom{\intop}\mkern7mu}\mkern-14mu}%
    {\mkern7mu\overline{\vphantom{\intop}\mkern7mu}\mkern-14mu}%
    {\mkern7mu\overline{\vphantom{\intop}\mkern7mu}\mkern-14mu}%
  \int}
\def\lowint{\mkern3mu\underline{\vphantom{\intop}\mkern7mu}\mkern-10mu\int}

\begin{document}
\begin{center}
	\huge{\bf Math 140B: Homework 7} \\
	Merrick Qiu
\end{center}

\section*{Problem 8.11}
Let $\epsilon > 0$.
Since $f(x) \to 1$ as $x \to \infty$, we can choose $A$ so that $1-\epsilon < f(x) < 1+\epsilon$
when $x \geq A$. \\

Since
\[
  (1-\epsilon)e^{-tx} = t\int_A^\infty e^{-tx}(1-\epsilon) \,dx \leq 
  t\int_A^\infty e^{-tx}f(x) \,dx \leq 
  t\int_A^\infty e^{-tx}(1+\epsilon) \,dx = (1+\epsilon)e^{-tx}
\]
we have that 
\[
  \lim_{t\to 0} t\int_A^\infty e^{-tx}f(x) \,dx = 1.
\]
Since $e^{-tx} \leq 1$ when $x \geq 0$
\[
  \lim_{t\to 0} \left| t\int_0^A e^{-tx}f(x) \,dx\right| \leq 
  \lim_{t\to 0} t\int_0^A |f(x)|\,dx = 0
\]
Thus finally we have that
\[
  \lim_{t\to 0} t\int_0^\infty e^{-tx}f(x) \,dx = 
  \lim_{t\to 0} t\int_0^A e^{-tx}f(x) \,dx +
  \lim_{t\to 0} t\int_A^\infty e^{-tx}f(x) \,dx= 1
\]
\newpage 

\section*{Problem 8.12}
\begin{enumerate}
  \item If the Fourier series is
  \[
      \frac{1}{2}a_0 + \sum_{n=1}^N(a_n\cos (nx) + b_n \sin (nx)) 
  \]
  then
  \[
    a_0 = \frac{1}{\pi}\int_{-\pi}^{\pi} f(x) = \frac{2\delta}{\pi}
  \]
  \[
    a_n = \frac{1}{\pi}\int_{-\pi}^{\pi} f(x)\cos(nx) 
    = \frac{1}{\pi}\int_{-\delta}^{\delta} \cos(nx)
    = \frac{2\sin(n\delta)}{\pi n}
  \]
  \[
    b_n = \frac{1}{\pi}\int_{-\pi}^{\pi} f(x)\sin(nx) = 0
  \]
  \item When $f(x)$ is Lipschitzs continuous, we have that
  \[
    f(x) = \frac{\delta}{\pi} + \frac{2}{\pi}\sum_{n=1}^N\frac{\sin(n\delta)}{n} \cos (nx).
  \]
  Since $f(x)$ is Lipschitzs at $x=0$ we get that,
  \[
    \frac{\delta}{\pi} + \frac{2}{\pi}\sum_{n=1}^N\frac{\sin(n\delta)}{n} = 1.
  \]
  Solving for the sum yields
  \[
    \sum_{n=1}^N\frac{\sin(n\delta)}{n} = \frac{\pi -\delta}{2}
  \]
  \item From Parseval's theorem,
  \[  
     \frac{2\delta}{\pi} = \frac{1}{\pi} \int_{-\pi}^{\pi} |f(x)|^2 \,dx = 
     \sum_{n=0}^{\infty} |a_n|^2 = \frac{2\delta^2}{\pi^2} + \sum_{n=1}^{\infty} \frac{4\sin^2(n\delta)}{\pi^2 n^2}
  \]
  Solving for the sum yields 
  \[
    \sum_{n=1}^{\infty} \frac{\sin^2(n\delta)}{n^2\delta} = \frac{\pi - \delta}{2}
  \]
  \item If we let $\delta \to 0$, then this is simply the Riemann integral of $\frac{\sin x}{x}$
  with $\Delta x = \delta$.
  Thus 
  \[
    \int \left(\frac{\sin x}{x} \right)^2 \,dx = \frac{\pi}{2} = \frac{\pi}{2}
  \]
  \item Putting $\delta = \frac{\pi}{2}$ yields 
  \[  
      \frac{2}{\pi} \sum_{n=1}^{\infty} \frac{\sin^2 (n\pi /2)}{n^2}= \frac{\pi}{4}
      \implies \sum_{n=1}^{\infty} \frac{1}{(2n-1)^2}  = \frac{\pi^2}{8}
  \]
\end{enumerate}
\newpage 

\section*{Rudin 8.13}
Note that the fourier series of $f(x) = x$ is 
\[
  f(x) = c_0 + \sum_{n=-\infty}^{\infty} c_n e^{inx} = \pi + \sum_{n=-\infty}^{\infty} \frac{i}{n} e^{inx}
\]
since 
\[
  c_0 = \frac{1}{2\pi} \int_{0}^{2\pi}x \,dx = \pi
\]
\begin{align*}
  c_n &= \frac{1}{2\pi} \int_{0}^{2\pi} xe^{-inx} \\
  &=\frac{1}{2\pi} \left[-\frac{1}{in}xe^{-inx} + \frac{1}{n^2} e^{-inx}\right]_0^{2\pi} \\
  &= \frac{1}{2\pi}\left[\left(-\frac{2\pi}{in} + \frac{1}{n^2}\right) - \left(\frac{1}{n^2}\right)\right]\\
  &=-\frac{1}{in} = \frac{i}{n}
\end{align*}

From Parseval's theorem 
\[
  \frac{4\pi^2}{3}= \frac{1}{2\pi} \int_{0}^{2\pi} |x|^2 \,dx = \sum_{-\infty}^{\infty} |c_n|^2 = \pi^2 + 2\sum_{n=1}^{\infty} \frac{1}{n^2}
\]
Solving for the sum of inverse squares yields 
\[
  \sum_{n=1}^{\infty} \frac{1}{n^2} = \frac{\pi^2}{6}
\]
\newpage 
\section*{Rudin 8.14}
Note that the fourier series of $f(x) = (\pi - |x|)^2$ is 
\[
    f(x) = \frac{1}{2}a_0 + \sum_{n=1}^N(a_n\cos (nx) + b_n \sin (nx))
    = \frac{\pi^2}{3} +  \sum_{n=1}^N \frac{4}{n^2} \cos nx
\]
since 
\[
  a_0 = \frac{1}{\pi} \int_{-\pi}^{\pi} (\pi - |x|)^2\,dx = 
  \frac{2}{\pi} \int_{0}^{\pi} (\pi - x)^2\,dx = \frac{2\pi^2}{3}
\]
\[
  a_n = \frac{2}{\pi} \int_{0}^{\pi} (\pi - x)^2 \cos(nx)\,dx
  = (-1)^n \frac{2}{\pi} \int_{0}^{\pi} x^2 \cos(nx)\,dx
  = \frac{4}{n^2}
\]
\[
  b_n = \frac{1}{\pi} \int_{-\pi}^{\pi} (\pi - |x|)^2 \sin(nx)\,dx
  = 0
\]
Since $f(x)$ is Lipschitzs continuous, we can evaluate at $x=0$ to get 
\[
  f(0) = \pi^2 = \frac{\pi^2}{3} +  \sum_{n=1}^N \frac{4}{n^2} \implies \sum_{n=1}^{\infty} \frac{1}{n^2} = \frac{\pi^2}{6}
\]
From Parseval's theorem,
\[
  \frac{2\pi^4}{5} = \frac{1}{\pi}\int_{-\pi}^{\pi} (\pi - |x|)^4 = \frac{2}{9}\pi^4 + 16\sum_{n=1}^\infty \frac{1}{n^4}
\]
This implies that 
\[
  \sum_{n=1}^\infty \frac{1}{n^4} = \frac{\pi^4}{90}
\]
\end{document}