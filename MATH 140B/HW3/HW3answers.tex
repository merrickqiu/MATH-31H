\documentclass{article}

\usepackage{amsmath}
\usepackage{amssymb}
\usepackage{hyperref}
\usepackage{indentfirst}
\usepackage{mathrsfs}

%User defined commands
\def\upint{\mathchoice%
    {\mkern13mu\overline{\vphantom{\intop}\mkern7mu}\mkern-20mu}%
    {\mkern7mu\overline{\vphantom{\intop}\mkern7mu}\mkern-14mu}%
    {\mkern7mu\overline{\vphantom{\intop}\mkern7mu}\mkern-14mu}%
    {\mkern7mu\overline{\vphantom{\intop}\mkern7mu}\mkern-14mu}%
  \int}
\def\lowint{\mkern3mu\underline{\vphantom{\intop}\mkern7mu}\mkern-10mu\int}

\begin{document}
\begin{center}
	\huge{\bf Math 140B: Homework 3} \\
	Merrick Qiu
\end{center}

\section*{Rudin 1}
Let $P$ be a partition on $[a,b]$.
If $x_0 \in [x_{i-1},x_i]$ then $m_i = 0$ and $M_i = 1$ and 
$m_i = M_i = 0$ for all other intervals.
Since $\Delta x_i$ can be chosen to be arbitrarily small
and $\alpha$ is continuous at $x_0$,
\[
	\lowint_a^b f(x) \,d\alpha = \sup L(P,f,\alpha) = \sup 0 = 0
\]
\[
	\upint_a^b f(x) \,d\alpha = \inf U(P,f,\alpha) = \inf \alpha(x_i) - \alpha(x_{i-1}) = 0
\]
Thus $f \in \mathscr{R}(\alpha)$ and $\int_a^b f(x) \,d\alpha = 0$.

\section*{Rudin 2}
Suppose $f(x_0) \neq 0$ for some $x_0$.
Since $f$ is continuous at $x_0$, there exists $\delta > 0$
such that 
\[
	|x-x_0| < \delta \implies|f(x) - f(x_0)| < \frac{f(x_0)}{2}.
\]
Since $f$ is continuous, nonnegative, and bounded, it is Riemann integrable.
Since all $x \in B_\delta(x_0)$ is positive,
\[
	\int_a^b f(x) \,dx \geq \int_{\max(a, x-\delta)}^{\min(b, x+\delta)} f(x) \,dx > 0
\]

\section*{Rudin 4}
Since the rationals and the irrationals are both dense in the reals,
$M_i = 1$ and $m_i = 0$ for all intervals for all partitions.
Thus every lower Riemann sum equals $0$ and every upper Riemann sum equals $b-a$,
so $f \notin \mathscr{R}$.
\newpage 

\section*{Rudin 5}
The rational indicator function from problem 4 is a counterexample. 
It is not Riemann integrable but its square is just the constant function.
However the integrability of $f^3$ does imply the integrability of $f$
by theorem 6.11.

\section*{Rudin 6}
Cover $P$ with open intervals $(u_j, v_j)$ where each interval has length 
$\alpha(v_j) - \alpha(u_j) <\epsilon$.
If we remove these open intervals from $[0,1]$ we get another compact set $K$,
which $f$ is uniformly continuous on, meaning
$|s-t| < \delta \implies |f(s) - f(t)| < \epsilon$.
If we form a partition where each $u_j,v_j$ occurs in $P$, no point of 
any segument $(u_j,v_j)$ occurs in $P$, and each $x_{i-1} \neq u_j$ has $\Delta x_i < \delta$
then 
\[
	U(P,f,\alpha) - L(P,f,\alpha) \leq [\alpha(b) - \alpha(b)]\epsilon + 2M\epsilon
\]
where $m \leq f(x) \leq M$ as bounds.
Thus since $\epsilon$ was arbitrary we have that $f \in \mathscr{R}$ on $[0,1]$.
\newpage

\section*{Rudin 7}
Let $\epsilon > 0$ and $M = \sup |f(x)|$.
Let $P$ be a partition that contains $c$ when
$0 < c \leq \frac{\epsilon}{4M}$ and 
the difference between the upper and lower Riemann sums is $<\frac{\epsilon}{4}$.
Then the upper and lower Riemann sums on $[c,1]$ using the points of $P$
inside that intervals are also $<\frac{\epsilon}{4}$.
Finally note that the value of the upper and lower Riemann sums in $[0,c]$
are also $<\frac{\epsilon}{4}$.
Thus it must be that 
\[
	\left|\int_0^1 f(x) \,dx -  \int_c^1 f(x) \,dx\right| < \epsilon
\]
The function $f(x) = (-1)^n(n+1)$ for $\frac{1}{n+1} < x \leq \frac{1}{n}$
has a limit 
\[
	\int_c^1 f(x) \,dx = (-1)^N(N+1)(\frac{1}{N}-c) + \sum_{k=1}^{N-1} \frac{(-1)^k}{k}
\]
where $\frac{1}{N+1} < c \leq \frac{1}{N}$.
However this limit does not exist for $|f|$ since 
\[
	\int_c^1 |f(x)| \,dx = (N+1)(\frac{1}{N}-c) + \sum_{k=1}^{N-1} \frac{1}{k}
\]

\section*{Rudin 8}
Note that for a partition of $[0,n]$ with the points $0,1,2,\cdots,n$,
If the sum does not converge, then the integral does not converge since
\[
	 \sum_{n=1}^n f(n) = L(f,P) \leq  \int_0^n f(x) \,dx
\]
Likewise if the integral does not converge then the sum does not converge since
\[
	\int_0^n f(x) \,dx \leq  U(f,P) =\sum_{n=0}^n f(n)
\]
\newpage 

\section*{Rudin 11}
Define
\[
	P(\lambda) = \int_a^b (\lambda u(x) + v(x))^2 \,d\alpha
	= \lambda^2 \int_a^b u^2(x)\,d\alpha + 2\lambda \int_a^b u(x)v(x) \,d\alpha + \int_a^b v^2(x)\,d\alpha \geq 0
\]
This implies a negative determinant so 
\begin{align*}
	&\left(2 \int_a^b u(x)v(x) \,d\alpha \right)^2 - 4\int_a^b u^2(x)\,d\alpha  \int_a^b v^2(x)\,d\alpha \leq 0 \\
	\implies& 4\left(\int_a^b u(x)v(x) \,d\alpha \right)^2 \leq 4\int_a^b u^2(x)\,d\alpha  \int_a^b v^2(x)\,d\alpha \\
	\implies & \int_a^b u(x)v(x) \,d\alpha  \leq \left(\int_a^b u^2(x)\,d\alpha\right)^\frac{1}{2}  \left(\int_a^b v^2(x)\,d\alpha\right)^\frac{1}{2}  \\
	\implies & \int_a^b |u(x)||v(x)| \,d\alpha  \leq ||u||_2||v||_2 \\
\end{align*}
Expanding out the definition yields 
\begin{align*}
	||f-h||_2^2 &= \int_a^b |f-h|^2 \,d\alpha \\
	&=\int_a^b |(f-g) + (g-h)|^2 \,d\alpha \\
	&= \int_a^b |f-g|^2 \,d\alpha  + \int_a^b |f-g||g-h| \,d\alpha + \int_a^b |g-h|^2 \,d\alpha\\
	&\leq ||f-g||_2^2 + 2||f-g||_2||g-h||_2 + ||g-h||_2^2 \\
	&= (||f-g||_2 +||g-h||_2)^2
\end{align*}

\section*{Rudin 12}
Define for a partition $P = \{x_0,\cdots,x_n\}$
\[
	g(t) = \frac{x_i-t}{\Delta x_i}f(x_{i-1}) + \frac{t-x_{i-1}}{\Delta x_i}f(x_i) 
\]
Since $g$ is continuous, and $|f-g|$ is bounded by $M_i - m_i$.
We can choose a partition such that the upper riemann sum of $f$ is bounded by $\frac{\epsilon^2}{2M}$
where $M$ is the max of $|f(x)|$,
which implies that 
\[
	\sum (M_i - m_i)^2[\alpha(x_i) - \alpha(x_{i-1})] < \epsilon^2
\]
so $||g-f||_2 < \epsilon$.








\end{document}