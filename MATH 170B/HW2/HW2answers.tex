\documentclass{article}

\usepackage{amsmath}
\usepackage{amssymb}
\usepackage{hyperref}
\usepackage{indentfirst}
\usepackage{matlab-prettifier}
\usepackage[shortlabels]{enumitem}
\usepackage{graphicx}
\usepackage{physics}
\usepackage[margin=1in]{geometry}

%User defined commands
\newcommand{\inv}[1]{#1^{-1}}
\newcommand{\cond}{\kappa_{||\cdot||}}

\begin{document}
\begin{center}
	\huge{\bf Math 170B: Homework 2} \\
	Merrick Qiu
\end{center}

\section*{Fixed Point Method}
Since $-x\sin^2(1/x)$ is continuous, we just need to prove 
that $g$ is continuous at $0$.
Since $-1 \leq \sin^2 (1/x)\leq 1$,
by the squeeze theorem
\[
	\lim_{x \to 0} -x\sin^2(1/x) = 0.
\]
Thus $g$ is continuous everywhere including $0$.
If there did exist a point $x_0 \in (0,1]$ then it must
be that $\sin^2 (1/x_0) = -1$. However this is impossible
since a square cannot be negative. Thus $x = 0$ is the only fixed point.
	
\section*{Interpolation 1}
\begin{align*}
	p_3(x) &= \sum_{k=0}^3 L_k(x)y_k \\
	&= 10\frac{(x-7)(x-1)(x-2)}{-8} +  146\frac{(x-3)(x-1)(x-2)}{120} + 2\frac{(x-3)(x-7)(x-2)}{-12} + 1\frac{(x-3)(x-7)(x-1)}{5} \\
	&= -\frac{5(x-7)(x-1)(x-2)}{4} +  \frac{73(x-3)(x-1)(x-2)}{60} - \frac{(x-3)(x-7)(x-2)}{6} + \frac{(x-3)(x-7)(x-1)}{5}
\end{align*}

\section*{Interpolation 2}
When $x = x_0$ we have that $x_0 - x_0 = 0$ so
\[
	g(x_0) + \frac{x_0 - x_0}{x_n - x_0}[g(x) - h(x)] =  g(x_0) = f(x_0) 
\]
When  $x = x_i$ for $i = 1,\dots,n-1$ we have that $[g(x_i) - h(x_i)] = 0$ so
\[
	g(x_i) + \frac{x_0 - x_i}{x_n - x_0}[g(x_i) - h(x_i)] = g(x_i) = f(x_i)
\]
When $x = x_n$ 
\[
	g(x_n) + \frac{x_0 - x_n}{x_n - x_0}[g(x_n) - h(x_n)] = g(x_i) - [g(x_n) - h(x_n)] = h(x_n) = f(x_n)
\]
Thus the function interpolates $f$ at $x_0,\dots,x_n$.

\section*{Interpolation 3}
Since $f^{(n)}(\eta_x) = \sinh(\eta_x)$ or $f^{(n)}(\eta_x) = \cosh(\eta_x)$
and both functions are less than 2 in the interval $[-1,1]$, we have that 
$f^{(n)}(\eta_x)x \leq 2$.
The error term is 
\begin{align*}
	E(x) &= \frac{f^{(n)}(\eta_x)}{n!}\prod_{j=0}^{n-1}(x-x_j) \\
	&\leq \frac{f^{(n)}(\eta_x)}{n!}2^{n-1}(x-0) \\
	&\leq \frac{2^{n}}{n!}
\end{align*}

\section*{Interpolation 4}
The Lagrange form is 
\begin{align*}
	p_2(x) &= \sum_{k=0}^3 L_k(x)y_k \\
	&= 0L_0(x) + 1\frac{(x+2)(x-1)}{-2} - 1\frac{(x+2)x}{3} \\
	&= -\frac{(x+2)(x-1)}{2} - \frac{(x+2)x}{3} \\
	&= -\frac{5}{6}x^2 - \frac{7}{6}x + 1
\end{align*}
The Newton form is 
\[
	p_0 = 0
\]
\[
	C_1 = \frac{1}{x_1+2} = \frac{1}{2}
\]
\[
	p_1 = 0 + C_1(x+2) = \frac{1}{2}(x+2)
\]
\[
	C_2 = \frac{-1 - \frac{1}{2}(x_2+2)}{(x_2+2)x_2} = -\frac{5}{6}
\]
\[
	p_2 = \frac{1}{2}(x+2) + C_2(x+2)x = \frac{1}{2}(x+2) -\frac{5}{6} (x+2)x = -\frac{5}{6}x^2 - \frac{7}{6}x + 1
\]

\section*{Interpolation 5}
The left hand side is the Lagrange interpolant and the right hand side is 
the Newton interpolant written using the divided differences.
By the uniqueness of the interpolating polynomial, both sides must be equal.

\section*{Interpolation 6}
Since both polynomials are equal, they must have equal coefficients.
The coefficient for $x^n$ in the Netwon interpolant is $f[x_0,\dots,x_n]$.
The coefficient for $x^n$ in the Lagrange interpolant is 
$\sum_{i=0}^{n} f(x_i) \prod_{\substack{j=0\\ j\neq i}}^{n} (x_i - x_j)^{-1}$.
Thus the equality holds.

\section*{Interpolation 7}
\begin{center}
	\begin{tabular}{ c|c|c|c|c} 
	$x_i$ & $f[x_i]$ & $f[x_i,x_{i+1}]$ & $f[x_i,x_{i+1},x_{i+2}]$ & $f[x_i,x_{i+1},x_{i+2},x_{i+3}]$\\
	 \hline
	 4 & 63 & $\frac{11-63}{2-4} = 26$ & $\frac{2-26}{0-4} = 6$ & $\frac{5-6}{3-4} = 1$\\ 
	 2 & 11 & $\frac{7-11}{0-2} = 2$ & $\frac{7-2}{3-2} = 5$\\ 
	 0 & 7 & $\frac{28-7}{3-0} = 7$  \\ 
	 3 & 28 \\ 
	\end{tabular}
\end{center}

The final polynomial is 
\[
	63 + 26(x-4) + 6(x-4)(x-2) + (x-4)(x-2)x
\]

\section*{Bisection Method}
Applying this function to $f(x) = x - 2e^{-x}$ on $[0,1]$ yields $x=0.8526$
\lstinputlisting[style=Matlab-editor]{BisectionRoot.m}
\end{document}
