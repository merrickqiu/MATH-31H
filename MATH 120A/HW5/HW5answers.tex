\documentclass{article}

\usepackage{amsmath}
\usepackage{amssymb}
\usepackage{hyperref}
\usepackage{mathrsfs}
\usepackage{enumerate}
\usepackage{bm}
\usepackage{physics}
\usepackage{graphicx}
\setlength{\parindent}{0pt}
\usepackage[parfill]{parskip}
\usepackage[margin=1in]{geometry}

\newcommand{\cjg}[1]{\overline{#1}}
\DeclareMathOperator\Arg{Arg}

\begin{document}
\begin{center}
	\huge{\bf Math 120A: Homework 5} \\
	Merrick Qiu
\end{center}
\section*{Problem 1: Page 71 Question 4(a)}
\begin{align*}
	f(z) &= \frac{1}{z^4} \\
	&= \frac{1}{(re^{i\theta})^4} \\
	&= \frac{1}{r^4}e^{-i4\theta} \\
	&= \frac{1}{r^4}(\cos 4\theta - i\sin 4\theta)
\end{align*}
\[
	u = \frac{\cos 4\theta}{r^4}, \quad v = -\frac{\sin 4\theta}{r^4}
\]
\[
	ru_r = -\frac{4\cos 4\theta}{r^4} = v_\theta
\]
\[
	u_\theta = -\frac{4\sin 4\theta}{r^4} = -rv_r
\]
By the polar form of the Cauchy-Riemann equations,
$f$ is differentiable when $z \neq 0$.
\begin{align*}
	f'(z) = e^{-i\theta}(u_r + iv_r) &= e^{-i\theta}\left(-\frac{4\cos 4\theta}{r^5} + i \frac{4\sin 4\theta}{r^5}\right) \\
	&= -\frac{4}{r^5}e^{-i\theta}e^{-i4\theta} \\
	&= -\frac{4}{(re^{i\theta})^5} \\
	&= -\frac{4}{z^5}
\end{align*}
\newpage 

\section*{Problem 2: Page 76 Question 1(c)(d)}
\[
	f(z) = e^{-y} \sin x - i e^{-y}\cos x
\]
\[
	u = e^{-y} \sin x, \quad v = -e^{-y}\cos x
\]
\begin{equation*}
	\begin{aligned}
		u_x &= e^{-y}\cos x & u_y &= -e^{-y}\sin x \\
		v_x &= e^{-y}\sin x & v_y &= e^{-y}\cos x 
	\end{aligned}
\end{equation*}

Since $u_x = v_y$ and $u_y = -v_x$ everywhere,
the function is entire.
\begin{align*}
	f(z) &= (z^2 - 2)e^{-x}e^{-iy} \\
	&= ((x+iy)^2 - 2)e^{-x}(\cos y - i\sin y) \\
	&= ((x^2 - y^2 - 2) + 2ixy)e^{-x}(\cos y - i\sin y) \\
	&= e^{-x}((x^2-y^2-2)\cos y - i(x^2-y^2-2)\sin y + 2ixy\cos y +2xy\sin y) \\
	&= e^{-x}((x^2-y^2-2)\cos y + 2xy\sin y + i (2xy\cos y - (x^2-y^2-2)\sin y))
\end{align*}
\[
	u =  e^{-x}(x^2-y^2-2)\cos y + 2e^{-x}xy\sin y, \quad v = 2e^{-x}xy\cos y - e^{-x}(x^2-y^2-2)\sin y
\]
\begin{align*}
	u_x &= 2e^{-x}x\cos y - e^{-x}(x^2-y^2-2)\cos y + 2e^{-x}y\sin y - 2e^{-x}xy\sin y \\
	u_y &= -e^{-x}(x^2-y^2-2)\sin y -2e^{-x}y\cos y + 2e^{-x}xy\cos y + 2e^{-x}x\sin y\\
	v_x &= 2e^{-x}y\cos y - 2e^{-x}xy\cos y - 2e^{-x}x\sin y + e^{-x}(x^2-y^2-2)\sin y\\
	v_y &= -2e^{-x}xy\sin y + 2e^{-x}x\cos y - e^{-x}(x^2-y^2-2)\cos y + 2e^{-x}y\sin y
\end{align*}

Since $u_x = v_y$ and $u_y = -v_x$ everywhere,
the function is entire.
\newpage
\section*{Problem 3: Page 76 Question 2(a)(c)}
\[
	f(z) = xy + iy
\]
\[
	u = xy, \quad v = y
\]
\begin{equation*}
	\begin{aligned}
		u_x &= y & u_y &= x \\
		v_x &= 0 & v_y &= 1
	\end{aligned}
\end{equation*}
By the Cauchy-Riemann equations, the function is only differentiable at 
the point $(0,1)$.
No point has a neighborhood of complex differentiable points so 
the function is nowhere analytic.
\[
	f(z) = e^ye^{ix} = e^y\cos(x) + ie^y\sin(x) 
\]
\[
	u = e^y\cos(x), \quad v = e^y\sin(x) 
\]
\begin{equation*}
	\begin{aligned}
		u_x &= -e^y\sin(x) & u_y &= e^y\cos(x) \\
		v_x &= e^y\cos(x)  & v_y &= e^y\sin(x) 
	\end{aligned}
\end{equation*}
By the Cauchy-Riemann equations, the function is only differentiable on 
the line $(0, y)$.
No point has a neighborhood of complex differentiable points so 
the function is nowhere analytic.
\newpage
\section*{Problem 4: Page 76 Question 4(c)}
\[
	f(z) = \frac{z^2}{(z+2)(z^2+2z+2)}
\]
Since $f$ is the quotient of two polynomials,
the function is analytic everywhere except
for when the quotient is zero, which are the singular points.
Factoring the quotient yields
\[
	(z+2)(z^2 + 2z+2) = (z+2)(z+1+i)(z+1-i).
\]
Therefore the singular points are 
$z = -2, -1 \pm i$.
\newpage
\section*{Problem 5: Page 77 Question 7}
Since the function is analytic on a domain, it must follow
the Cauchy-Riemann equations hold for each point.
However since $v = 0$, this implies that 
$u_x = v_y = 0$ and $u_y = -v_x = 0$, which means that $f$ must be constant.

\newpage

\end{document}