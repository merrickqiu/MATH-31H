\documentclass{article}

\usepackage{amsmath}
\usepackage{amssymb}
\usepackage{hyperref}
\usepackage{mathrsfs}
\usepackage{enumerate}
\usepackage{bm}
\usepackage{physics}
\usepackage{graphicx}
\setlength{\parindent}{0pt}
\usepackage[parfill]{parskip}
\usepackage[margin=1in]{geometry}

\newcommand{\cjg}[1]{\overline{#1}}
\DeclareMathOperator\Arg{Arg}

\begin{document}
\begin{center}
	\huge{\bf Math 120A: Homework 8} \\
	Merrick Qiu
\end{center}
\section*{Problem 1}
We can parameterize $C = \{|z-2| = 2\}$ as 
\[
	z(t) = 2e^{it} + 2 \qquad (0 \leq t < 2\pi)
\]
\begin{align*}
	\int_C \frac{1}{z-2} \,dz 
	&= \int_0^{2\pi} \frac{1}{2e^{it}} \left(2ie^{it}\right) \,dz  \\
	&= \int_0^{2\pi} o \,dz  \\
	&= 2\pi i
\end{align*}
\newpage 

\section*{Problem 2: Page 138, 1(a)}
If $z$ is on $C$ then 
\[
	|z+4| \leq |z| + |4| = 6
\]
\[
	|z^3-1| \geq ||z|^3-|1|| = 7.
\]
Therefore 
\[
	\left|\frac{z+4}{z^3-1}\right| \leq \frac{6}{7}
\]
so $M = \frac{6}{7}$ and the length of $C$ is 
$L=\pi$.

\[
	\left|\int_C \frac{z+4}{z^3-1} \,dz\right|
	\leq ML = \frac{6\pi}{7}.
\]
\newpage

\section*{Problem 3: Page 139, 4}
If $z$ is on $C_R$ then 
\[
	|2z^2-1| \leq 2|z|^2 + |-1| = 2R^2+1
\]
\[
	|z^4 + 5z^2 + 4| = 
	|(z^2-1)||(z^2-4)| \geq
	||z|^2 - |1||||z|^2 - |4|| =
	(R^2-1)(R^2-4)
\]
Therefore
\[
	\left|\frac{2z^2-1}{z^4 + 5z^2 + 4}\right| \leq \frac{2R^2+1}{(R^2-1)(R^2-4)}
\]
so $M_R = \frac{2R^2+1}{(R^2-1)(R^2-4)}$ and the length of $C$ is 
$L = \pi R$.

\[
	\left|\int_C \frac{2z^2-1}{z^4 + 5z^2 + 4}\right| \leq M_RL = 
	\frac{\pi R(2R^2+1)}{(R^2-1)(R^2-4)}
\]

If we divide the numerator and denominator by $R^4$,
we see that the integral goes to $0$ as $R \to \infty$ since $M_RL \to 0$.

\[
	\frac{\pi R(2R^2+1)}{(R^2-1)(R^2-4)}
	\cdot \frac{\frac{1}{R^4}}{\frac{1}{R^4}}
	=
	\frac{\pi (\frac{2}{R}+\frac{1}{R^3})}{(1-\frac{1}{R^2})(1-\frac{4}{R^2})}
\]
\newpage 
\section*{Problem 4: Page 147, 1}
\begin{align*}
	\int_C z^n \,dz &=
	\left[\frac{1}{n+1}z^{n+1}\right]_{z_1}^{z_2} \\
	&= \frac{1}{n+1}z_2^{n+1} - \frac{1}{n+1}z_1^{n+1} \\
	&= \frac{1}{n+1}(z_2^{n+1} - z_1^{n+1})
\end{align*}
\newpage

\section*{Problem 5: Page 147, 2(b)(c)}
\begin{align*}
	\int_0^{\pi+2i} \cos\left(\frac{z}{2}\right) \,dz 
	&= \left[2\sin\left(\frac{z}{2}\right)\right]_0^{\pi+2i}\\
	&= 2\sin\left(\frac{\pi}{2} + i\right) \\
	&= \frac{e^{i\frac{\pi}{2}-1} - e^{-i\frac{\pi}{2}+1}}{i} \\
	&= \frac{e^{-1}i + ei}{i} \\
	&= e + \frac{1}{e}
\end{align*}

\begin{align*}
	\int_1^3 (z-2)^3 \,dz 
	&= \left[\frac{1}{4}(z-2)^4\right]_1^3 \\
	&= \frac{1}{4} - \frac{1}{4} \\
	&= 0
\end{align*}
\newpage  

\section*{Problem 6}
Note that the term we are integrating has an antiderivative.
\[
	\frac{d}{dz} \left(-\frac{1}{3}(z-5)^{-3}\right) = (z-5)^{-4}
\]

Therefore for any closed contour which does not pass through $5$,
the integral is $0$.
\end{document}