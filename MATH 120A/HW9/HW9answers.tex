\documentclass{article}

\usepackage{amsmath}
\usepackage{amssymb}
\usepackage{hyperref}
\usepackage{mathrsfs}
\usepackage{enumerate}
\usepackage{bm}
\usepackage{physics}
\usepackage{graphicx}
\setlength{\parindent}{0pt}
\usepackage[parfill]{parskip}
\usepackage[margin=1in]{geometry}

\newcommand{\cjg}[1]{\overline{#1}}
\DeclareMathOperator\Arg{Arg}

\begin{document}
\begin{center}
	\huge{\bf Math 120A: Homework 9} \\
	Merrick Qiu
\end{center}
\section*{1. Page 159: Problem 1(a)(b)}
\[
	f(z) = \frac{z^2}{z+3}
\]
Since this function is the quotient of two polynomials,
it is only not analytic at $z=-3$ which is outside of the 
region bound by the contour.
Therefore the function is analytic everywhere interior to 
and on $C$ so the integral is zero by the Cauchy-Goursat theorem.
\[
	f(z) = ze^{-z}
\]
Since $z$ is analytic and $e^{-z}$ is analytic,
their product is analytic everywhere.
Therefore the integral is zero by the Cauchy-Goursat theorem.
\newpage
\section*{2. Page 159: Problem 2(a)}
\[
	f(z) = \frac{1}{3z^2+1}
\]
Since this function is the quotient of two polynomials,
it is analytic everywhere except when $3z^2+1 = 0$, where 
$z = \pm \frac{1}{\sqrt{3}}i$, which is inside of the square
and outside of the region bound by the circle and the square.
Therefore by the corollary, the two integrals are equal.
\newpage 

\section*{3. Page 170: Problem 1(a)(b)}
Since $e^{-z}$ is analytic, by the cauchy integral formula
\[
	\int_C \frac{e^{-z} \,dz}{z-(\pi i/2)} = 2\pi i e^{-\pi i/2} = 2\pi
\]
Since $\frac{\cos z}{z^2+8}$ is analytic interior to $C$,
\[
	\int_C \frac{\cos z}{z(z^2+8)} \,dz =  2\pi i \frac{\cos 0}{0^2+8} = \frac{2\pi i}{8}
\]
\newpage 

\section*{4. Page 170: Problem 2}
Since $\frac{1}{z+2i}$ is analytic inside the circle,
\[
	\int_C \frac{1}{(z-2i)(z+2i)}\,dz = 2\pi i \frac{1}{2i+2i} = \frac{\pi}{2}
\]
By the cauchy extension,
\[
	\int_C \frac{1}{(z-2i)^2(z+2i)^2}\,dz = \frac{2\pi i}{1!} \left(-\frac{2}{(2i+2i)^3}\right) = \frac{\pi}{16}
\]
\newpage 
\section*{5. Page 170: Problem 3}
Since $2s^2-s-2$ is analytic everywhere,
\[
	g(2) = \int_C \frac{2s^2-s-2}{s-2} \,ds = 2\pi i (2(2)^2 - 2 - 2) = 8\pi i
\]
If $|z| > 3$ the cauchy-goursat theorem says that $g(z) = 0$.

\newpage
\section*{6. Page 170: Problem 4}
If $z$ is inside then by the cauchy extension 
\[
	\frac{d^2}{ds^2} s^3+2s
	= \frac{d}{ds} 3s^2 + 2
	= 6s
\]
\begin{align*}
	\int_C \frac{s^3+2s}{(s-z)^3}\,ds &= \frac{2\pi i}{2!} \left(6z\right) \\
	&= 6\pi i z \\
\end{align*}
If $z$ is outside then the cauchy-goursat theorem says that $g(z) = 0$.

\end{document}