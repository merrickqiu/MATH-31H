%Set document class
\documentclass{article}

%Load math symbol packages
\usepackage{amsmath}
\usepackage{amssymb}
\usepackage{tikz} 
\usepackage{hyperref}
\usepackage{mathtools}
\usepackage{indentfirst}
\usepackage{graphicx}

%User defined commands
\newcommand{\var}{\operatorname{Var}}

\begin{document}
\begin{center}
	\huge{\bf Math 181A: Homework 3} \\
	Merrick Qiu 
\end{center}

\subsection*{Problem 1: 5.3.2 and 5.3.10}
\begin{enumerate}
	\item Taking the average of all the subjects yields that $\bar{x} = 0.77$. 
	The critical value for $95\%$ is $Z_{\alpha/2} = 1.96$.
	The margin of error is $ME = 1.96\cdot\frac{0.09}{\sqrt{19}} = 0.04$.
	We have that the confidence interval is $(0.73, 0.81)$.
	Since $0.80$ is in the confidence interval, 
	it is likely that the detergent does not cause respiratory illness.
	\item Modeling his performance as a Bernoulli distribution,
	he has standard deviation $\sigma = \sqrt{0.356\cdot(1-0.356)} = 0.479$.
	Thus he has confidence interval
	\[
		(0.356-1.96\cdot\frac{ 0.479}{\sqrt{540}}, 0.356+1.96\cdot\frac{ 0.479}{\sqrt{540}})
		= (0.316, 0.396)
	\]
\end{enumerate}
\newpage

\subsection*{Problem 2: 5.3.26}
Assuming the upper bound of $\hat{p} = 0.4$ gives
\[
	n \geq \frac{2.58^2(0.4)(0.6)}{0.05^2} = 639.01.
\]
Thus $n=640$ is a lower-bound.
\newpage

\subsection*{Problem 3}
The margin of error of the first sample is 
\[
	ME = 1.96\cdot\frac{0.45\cdot0.55}{n} = \frac{0.485}{n}
\]
The margin of error of the second sample is
\[
	ME_{new} = 1.645\cdot\frac{0.48\cdot0.52}{n_{new}} = \frac{0.411}{n_{new}}
\]
Setting $ME_{new}$ to be a third of $ME$ yields 
\begin{align*}
	ME_{new} = \frac{1}{3}ME 
	&\implies \frac{0.411}{n_{new}} = \frac{0.162}{n} \\
	&\implies n_{new} = 2.54n.
\end{align*}
\newpage

\subsection*{Problem 4: 5.4.6}

In order for $Y_{min} = y$, one of the $\binom{n}{1}$ variables must have value 
$y$ with density $\frac{1}{\theta}$ and the other $n-1$ variables must have value
$\geq y$ with probability $\frac{\theta-y}{\theta}$.
Therefore, $f_{Y_{min}}(y) = n\frac{1}{\theta}\left(\frac{\theta-y}{\theta}\right)^{n-1}$.
Finding the expected value of $Y_{min}$ yields
\begin{align*}
	\int_0^\theta n\frac{1}{\theta}\left(\frac{\theta-y}{\theta}\right)^{n-1} \,dy
	&=\frac{n}{\theta} \int_0^\theta \left(\frac{\theta-y}{\theta}\right)^{n-1} \,dy \\
	&= \frac{n}{\theta} \left[-\frac{\theta}{n(n+1)}(ny+\theta)\left(\frac{\theta-y}{\theta}\right)^n\right]_0^\theta \\
	&= - \frac{1}{n+1}(0-\theta) \\
	&= \frac{\theta}{n+1}.
\end{align*}
Therefore an unbiased estimator would be $\hat{\theta} = (n+1)Y_{min}$.
\newpage 

\subsection*{Problem 5: 5.4.9}

The expected value of Y is 
\begin{align*}
	E[Y] &= \int_0^\frac{1}{\theta} 2y^2\theta^2 \\
	&= \left[\frac{2}{3}y^3\theta^2\right]_0^\frac{1}{\theta} \\
	&= \frac{2}{3}\frac{1}{\theta}
\end{align*}
We have that 
\begin{align*}
	E[c(Y_1+2Y_2)] &= 3cE[Y] \\
	&= 2c \frac{1}{\theta}
\end{align*}
Therefore, $c=\frac{1}{2}$.
\newpage 

\subsection*{Problem 6: R Simulation}
\begin{enumerate}
	\item Here is the plot of the likelihood function, with the red line
	representing the sample mean.

	\includegraphics*[scale=0.35]{hw3likelihood.png}

	\item Here is the plot of the log likelihood function, with the red line
	representing the sample mean.

	\includegraphics*[scale=0.35]{hw3loglikelihood.png}
\end{enumerate}



\end{document}

