%Set document class
\documentclass{article}

%Load math symbol packages
\usepackage{amsmath}
\usepackage{amssymb}
\usepackage{tikz} 
\usepackage{hyperref}
\usepackage{mathtools}
\usepackage{indentfirst}
\usepackage{graphicx}

%User defined commands
\newcommand{\var}{\operatorname{Var}}
\newcommand{\cov}{\operatorname{Cov}}

\begin{document}
\begin{center}
	\huge{\bf Math 181A: Homework 9} \\
	Merrick Qiu 
\end{center}

\subsection*{Problem 1}
\begin{enumerate}
	\item The likelihood ratio is 
	\[
		\frac{L(1)}{L(2)} = \frac{1}{2x}
	\]
	This is a monotonically decreasing function of $x$.
	Thus we need to find $x \geq c'$ with probability 0.05.
	We have that 
	\[
		0.05 = \int_{c'}^1 1 \,dx = 1-c'.
	\]
	This implies that $c' = 0.95$ so $x \geq 0.95$ is the critical region.
	\item Assuming that $\theta = 2$, the probability that $x \geq 0.95$ is
	\[
		\int_{0.95}^1 2x \,dx = 1 - 0.95^2 = 0.0975.
	\]
\end{enumerate}
\newpage 

\subsection*{Problem 2}
\begin{enumerate}
	\item The likelihood ratio is 
	\[
		\frac{L(4)}{L(\lambda_1)} 
		= \frac{4e^{-4x}}{\lambda_1e^{-\lambda_1x}}
		= \frac{4}{\lambda_1}e^{(\lambda_1-4)x}
	\]
	This is a monotonically increasing function of $x$.
	Thus we need to find $x \leq c'$ with probability 0.05.
	We have that 
	\[
		0.05 = \int_0^{c'} 4e^{-4x} \,dx = -e^{-4c'} + 1.
	\]
	This implies that $c' = -\frac{\ln(0.95)}{4} = 0.0128$.
	The critical region is therefore $x \leq 0.0128$.
	\item The likelihood ratio is the same, 
	except it is now a monotonically decreasing function of $x$.
	Thus we need to find $x \geq c'$ with probability 0.05.
	We have that 
	\[
		0.05 = \int_{c'}^{\infty} 4e^{-4x} \,dx = 0 + e^{-4c'}
	\]
	This implies that $c' = -\frac{\ln(0.05)}{4} = 0.749.$
	The critical region is therefore $x \geq 0.749$.
\end{enumerate}
\newpage 

\subsection*{Problem 3}
The likelihood ratio is 
\[
	\frac{L(\lambda_0)}{L(\lambda_1)} 
	= \frac{\prod_{i=1}^n \frac{\lambda_0^{x_i}e^{\lambda_0}}{x_i!}}{\prod_{i=1}^n \frac{\lambda_1^{x_i}e^{\lambda_1}}{x_i!}}
	= \prod_{i=1}^n \frac{\lambda_0^{x_i}e^{\lambda_0}}{\lambda_1^{x_i}e^{\lambda_1}}
	= e^{\lambda_0 - \lambda_1} \frac{\lambda_0}{\lambda_1}^{n\bar{x}}
\]
Since $\lambda_1 > \lambda_0$, the likelihood ratio is a monotonically decreasing function of $\bar{x}$
and so the likelihood ratio test rejects the null hypothesis when $\bar{X} \geq c'$.
\newpage 

\subsection*{Problem 4}

Since $\Omega = \mathbb{R}$, the generalized likelihood ratio is 
\begin{align*}
	\frac{L(\lambda_0)}{L(\hat{\lambda}_{MLE})}
	= \frac{\prod_{i=1}^n \lambda_0 e^{-\lambda_0x_i}}{\prod_{i=1}^n \frac{1}{\bar{x}}e^{-\frac{1}{\bar{x}}x_i}}
	= \lambda_0^n e^{-\lambda_0 n\bar{x}} \bar{x}^n e^n 
	= (\lambda_0e \bar{x} e^{-\lambda_0 \bar{x}}   )^n
\end{align*}
Since the likelihood ratio is a monotonically increasing function of $\bar{X}e^{-\lambda_0\bar{X}}$,
it can be rewritten as $\bar{X}e^{-\lambda_0\bar{X}} \leq c'$.
\newpage 

\subsection*{Problem 5: 6.5.1}
If $k$ is the sum of all the values of $k_i$, the generalized likelihood ratio is 
\begin{align*}
	\frac{L(p_0)}{L(\hat{p}_{MLE})}
	&= \frac{\prod_{i=1}^n (1-p)^{k_i-1}p}{\prod_{i=1}^n (1-n/k)^{k_i-1}n/k} \\
	&= \frac{p^n(1-p)^{k - n}}{(n/k)^n(1-n/k)^{k - n}} \\
	&=\left(\frac{p}{nk}\right)^n \left(\frac{1-p}{1-n/k}\right)^{n-k}
\end{align*}
\newpage

\subsection*{Problem 6}
\begin{enumerate}
	\item The ratio is 
	\begin{align*}
		\frac{L(0.5)}{L(\hat{p}_{MLE})}
		&= \frac{\binom{n}{x}0.5^n}{\binom{n}{x}(x/n)^x(1-x/n)^{n-x}} \\
		&= 0.5^n \binom{n}{x}(x/n)^{-x}(1-x/n)^{x-n}
	\end{align*}
	\item The critical region is $x = \{0, 1,11,12\}$
	The significance level is therefore 
	\[
		\binom{12}{0} 0.5^{12} + \binom{12}{1} 0.5^{12} + \binom{12}{11} 0.5^{12} + \binom{12}{12} 0.5^{12}
		= 26\cdot 0.5^{12}
		= 0.00635
	\]
\end{enumerate}
\newpage 

\subsection*{R Problem}
My code yielded a confidence interval for the mean of $(202.3892, 215.7476)$ and
a confidence interval for the median of $(235, 250)$.
The estimated probability of being 5 units away from the mean is $0.236$.
\end{document}




