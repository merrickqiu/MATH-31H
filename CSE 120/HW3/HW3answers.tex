\documentclass{article}

\usepackage{amsmath}
\usepackage{amssymb}
\usepackage{hyperref}
\usepackage{indentfirst}
\usepackage{listings}
\usepackage{xcolor}
\usepackage{graphicx}
\lstset{language=C++,
                basicstyle=\ttfamily,
                keywordstyle=\color{blue}\ttfamily,
                stringstyle=\color{red}\ttfamily,
                commentstyle=\color{green}\ttfamily,
                morecomment=[l][\color{magenta}]{\#}
}

%User defined commands
\newcommand{\sgn}{\operatorname{sgn}}

\begin{document}
\begin{center}
	\huge{\bf CSE 120: Homework 3} \\
	Merrick Qiu
\end{center}

\section*{Nachos VM Worksheet}
\begin{enumerate}
    \item 4
    \item 8
    \item \begin{enumerate}
        \item pageTable[0].vpn = 0
        \item pageTable[0].ppn = 2
        \item pageTable[1].vpn = 1
        \item pageTable[1].ppn = 0
        \item pageTable[2].vpn = 2
        \item pageTable[2].ppn = 6
        \item pageTable[3].vpn = 3
        \item pageTable[3].ppn = 4
    \end{enumerate}
    \item 128*2 = 256 bytes
    \item 128*6 = 768 bytes
    \item It is in virtual page 2, which is in physical page 6.
    \item 298-256 = 42 bytes
    \item 128*6 + 42 = 810 bytes
    \item Since the virtual and physical pages aren't always the same,
    memory that is contiguous in the virtual address space might not be contiguous in the physical address space,
    so we would need to check if there are page boundaries for the array we are copying.
\end{enumerate}

\section*{Question 2}
\begin{enumerate}
    \item 200 ns 
    \item 0.75*100 + 0.25*200 = 125 ns 
    \item 0.995*100 + 0.005*200 = 100.5 ns
\end{enumerate}

\section*{Question 3}\
\begin{enumerate}
    \item 22 bits are left.
    \item We can take the first 22 bits and then shift by 10 bits(page size is $2^{10}$)
    \begin{align*}
        0x0000FFFF &= 0000 0000 0000 0000 1111 1111 1111 1111 \\
        &\implies     0000 0000 0000 0000 1111 1100 0000 0000 \\
        &\implies     0000 0000 0000 0000 0000 0000 0011 1111 \\
        &= 0x3F = 63
    \end{align*}
    \item The offset is the last 10 bits which is 0b1111111111 = 0x3FF = 1023.
    \item 4*1024 = 4096 bytes 
    \item 4096 + 1023 = 5119 bytes
\end{enumerate}

\section*{Question 4}
\begin{enumerate}
    \item $2^{44-16} = 2^{28}$
    \item A page can have $2^14$ entries so we need 14 bits to index into a page table.
    The offset will be 16 bits since that is the size of a page.
    Since there are two tables, we have a total of 14 + 14 + 16 = 44 bits.
    \item 4 gigabytes is $2^32$ bytes which is $2^16$ pages.
    Since page can hold $2^14$ entries, we need 4 pages, plus 1 root page table.
    This is $2^16+5$ page frames.
\end{enumerate}

\section*{Question 5}
For a 2 nanosecond instruction, it would be $2+10000000*\frac{1}{20000000} = 2.5 ns$.
For a 1 nanosecond instruction, it would be 1.5 ns

\section*{Question 6}
There are a total of 8 page faults for FIFO

\begin{tabular}{c c c}
    Reference History & Queue & Faults\\
    4 & 4 & 1 \\
    42 & 42 & 2 \\
    427 & 427 & 3 \\
    4272 & 427 & 4 \\
    42725 & 4275 & 4 \\
    427253 & 2753 & 5 \\
    4272533 & 2753 & 5 \\
    42725332 & 2753 & 5 \\
    427253323 & 2753 & 5 \\
    4272533231 & 7531 & 6 \\
    42725332312 & 5312 & 7 \\
    427253323126 & 3126 & 8
\end{tabular} \\

There are a total of 7 page faults for LRU

\begin{tabular}{c c c}
    Reference History & Queue &Faults\\
    4 & 4 & 1 \\
    42 & 42 & 2 \\
    427 & 427 & 3 \\
    4272 & 472  & 4 \\
    42725 & 4725 & 4 \\
    427253 & 7253 & 5 \\
    4272533 & 7253 & 5 \\
    42725332 & 7532 & 5 \\
    427253323 & 7523 & 5 \\
    4272533231 & 5231 & 6 \\
    42725332312 & 5312 & 6 \\
    427253323126 & 3126 & 7
\end{tabular}

\section*{Question 7}
The working set page replacement algorithm tries
to keep the pages that the processes regularly need, the working set,
so that page faults are not too burdensome.

\section*{Question 8}
\begin{enumerate}
    \item The CPU speed is not what is stopping the utilization from increasing.
    \item Probably wont help since the storage of the disk is not the issue
    \item Multiprogramming will probably decrease utilization since they will require more memory
    \item Decreasing multiprogramming will free up memory and increase cpu utilization
    \item Increasing main memory will increase cpu utilization by decreasing the amount of page faults 
    \item It will help
    \item Can help if the algorithm works correctly
    \item Doesn't change much.
\end{enumerate}





\end{document}