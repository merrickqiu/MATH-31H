\documentclass{article}

\usepackage{amsmath}
\usepackage{amssymb}
\usepackage{hyperref}
\usepackage{indentfirst}

%User defined commands
\newcommand{\sgn}{\operatorname{sgn}}

\begin{document}
\begin{center}
	\huge{\bf CSE 120: Homework 1} \\
	Merrick Qiu
\end{center}

\begin{enumerate}
    \item If we didn't have privileged instructions, then users could run code that could halt the whole machine
    and do many other actions that they should not be able to do. 

    If we didn't have memory protection, then users could modify and delete memory from anywhere in the machine,
    which could break the normal functioning of the computer or be used for malicious purposes.
    There would be no privacy between applications as well.

    If we didn't have timer interrupts the OS would not be able to gain control of non-cooperative processes,
    and a program with an infinite loop could make the computer get stuck.
    \item Calling exit is useful to signal to the parent process whether or not
    the program ran successfully or if some sort of error occured.
    \item \begin{enumerate}
        \item Setting the value of the timer should be priviledged since
        a program could maliciously prevent timer interrupts.
        \item The clock value is not private, so reading it is not an issue
        and it shouldn't be a privileged operation.
        \item Clearing memory can be dangerous, so it should be priviledged.
        \item Turning off interrupts should be priviledged since a program could 
        use it maliciously to avoid handing control to the OS.
        \item Switching to kernel mode should be priviledged since it could be maliciously
        used by a program to take control of the machine.
    \end{enumerate}
    \item open can fail if permission to read the file is denied.
    read can fail if it is given a file descriptor to a directory.
    fork can fail if there is not enough memory for the child.
    exec can fail if there is not enough memory for the new process.
    unlink can fail if the file is being used by another program.
    \item \begin{enumerate}
        \item If a program gets stuck in an infinite loop and cannot exit,
        SIGKILL might be the only way to terminate the program.
        If it could be caught, then it would be possible to write programs that cannot
        ever be terminated.
        \item As long as the state of the program is saved while it is paused,
        the programmed can be continued later at any time without any issues.
    \end{enumerate}
    \item The interval timer could be set to 1ms, and everytime there is a timer interrupt,
    the OS could update a variable that kept track of the number of miliseconds that have 
    elapsed in order to keep track of the time.
    \item I could create an exception for each system call and execute 
    the trap instruction during the exception handling.
    \item \begin{enumerate}
        \item There are 6 total processes.
        \item /bin/ls runs 2 times.
    \end{enumerate}
\end{enumerate}




\end{document}