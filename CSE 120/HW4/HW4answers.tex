\documentclass{article}

\usepackage{amsmath}
\usepackage{amssymb}
\usepackage{hyperref}
\usepackage{indentfirst}
\usepackage{listings}
\usepackage{xcolor}
\usepackage{graphicx}
\lstset{language=C++,
                basicstyle=\ttfamily,
                keywordstyle=\color{blue}\ttfamily,
                stringstyle=\color{red}\ttfamily,
                commentstyle=\color{green}\ttfamily,
                morecomment=[l][\color{magenta}]{\#}
}

%User defined commands
\newcommand{\sgn}{\operatorname{sgn}}

\begin{document}
\begin{center}
	\huge{\bf CSE 120: Homework 4} \\
	Merrick Qiu
\end{center}
\section*{Question 1}
A total of 10 reads are necessary.
There are 5 directories/files,
and there has to be a read for the inode and 
a read for the data block.

\section*{Question 2}
The 10 direct pointers can point to 
$40K$ of data. The indirect pointer 
points to a block of entirely direct pointers,
so there are $1K$ direct pointers in here. That is $4M$ of data.
The double indirect pointer has $1M$ of direct pointers,
which stores $4G$ of data.
In total that is $4,004,040K$ bytes of data.

\section*{Question 3}
\begin{enumerate}
    \item There is a total of $2KB * 1M = 2GB$ of waste.
    \item There is a total of $256B * 1M = 256MB$ of waste.
    \item Unless I had a very large amount of storage space,
    I would probably want this benefit.
\end{enumerate}

\section*{Question 4}
It does not necessarily fit in the same inode.
All zip can do is access files and directories,
but it is up to the operating system file system 
to create the inodes.

\section*{Question 5}
\begin{enumerate}
    \item In Unix I could individually
    give each 4990 users permission, or I could make a group.
    \item I could make a blacklist of the users that do not 
    have access to the file.
\end{enumerate}

\section*{Question 6}
A file system cache helps improve performance by
storing block location pointers in physical memory instead
of disk, which is much faster.
However physical memory is limited, so systems cannot 
use incredibly large caches.

\section*{Question 7}
\begin{tabular}{c||c|c|c|c}
    I\textbackslash X & 100ms & 10 ms & 1 ms & 0.1 ms \\
    \hline \hline
    25 ms & 20\% & 71\% & 96\% & 99.6\% \\
    5 ms & 4.8\% & 33\% & 83\% & 98\% \\
    0.1 ms & 0.0999\% & 0.99\% & 9.09\% & 50\% \\
    0.005 ms & 0.005\% & 0.05\% & 0.498\% & 4.76\% \\
    0.001 ms & 0.001\% & 0.01\% & 0.1\% & 0.99\% \\
    
\end{tabular}







\end{document}