%Set document class
\documentclass{report}

%Load math symbol packages
\usepackage{amsmath}
\usepackage{amssymb}
\usepackage{hyperref}

%User defined commands
\newcommand{\sgn}{\operatorname{sgn}}
% \newcommand{\max}{\operatorname{max}}
\setlength{\parindent}{0pt}
\begin{document}
\begin{center}
	\huge{\bf Math 188: Homework 1} \\
	Merrick Qiu
\end{center}

\subsection*{Problem 1: Closed Form of a Recurrence Relation.}

The characteristic equation of $a_n = 5a_{n-1}-8a_{n-2}+4a_{n-3}$ is 
\[
    t^3-5t^2+8t-4 = (t-2)^2(t-1) 
\]

The closed form solution is in the form 
\[
    a_n = \alpha_1 2^n + \alpha_2 n 2^n + \alpha_3
\]

Plugging in the values of the starting conditions yields
\begin{align*}
    1 &= \alpha_1 + \alpha_3 \\
    1 &= 2\alpha_1 + 2\alpha_2 + \alpha_3 \\
    2 &= 4\alpha_1 + 8\alpha_2 + \alpha_3
\end{align*}

Solving for the constants yields 
$\alpha_1 = -1, \alpha_2 = \frac{1}{2}, \alpha_3 = 2$.
The closed formula for the recurrence relation is
\[
    a_n = -2^n + \frac{n2^n}{2} + 2
\]
I used Wolfram Alpha for factoring the characteristic equation
and solving the system of equations for the constants.
\newpage

\subsection*{Problem 2: Vandermonde Matrix Determinant is Nonzero.}
% The matrix has the form

We will induct on $d$ to prove that the determinant is nonzero.
For $d = 1$, the determinant is $1$.
For $d=n$, suppose that $(r_i^{j-1})_{i,j=1,\hdots,n}$ has nonzero determinant 
for all $r_1,\hdots,r_n$ distinct.
Let $V = (r_i^{j-1})_{i,j=1,\hdots,n+1}$ for $r_1,\hdots,r_{n+1}$ distinct.
\[
    V = \begin{bmatrix}
        1 & r_1 & r_1^2 & \hdots & r_1^n \\
        1 & r_2 & r_2^2 & \hdots & r_2^n \\
        1 & r_3 & r_3^2 & \hdots & r_3^n \\
        \vdots & \vdots & \vdots & \ddots & \vdots \\
        1 & r_{n+1} & r_{n+1}^2 & \hdots & r_{n+1}^n
    \end{bmatrix}
\]
Starting from the last column and ending at the second column, 
subtract each column by $r_1$ times the previous column.
Then use the laplace expansion formula along the first row, and
then factor $(r_i - r_1)$ out every $i$th row.
\begin{align*}
    \det \begin{bmatrix}
        1 & r_1 & r_1^2 & \hdots & r_1^n \\
        1 & r_2 & r_2^2 & \hdots & r_2^n \\
        1 & r_3 & r_3^2 & \hdots & r_3^n \\
        \vdots & \vdots & \vdots & \ddots & \vdots \\
        1 & r_{n+1} & r_{n+1}^2 & \hdots & r_{n+1}^n
    \end{bmatrix}
    &= \det \begin{bmatrix}
        1 & 0 & 0  & \hdots & 0 \\
        1 & r_2 - r_1 & r_2^2 - r_2r_1& \hdots & r_2^n - r_2^{n-1}r_1 \\
        1 & r_3 - r_1 & r_3^2  - r_3r_1& \hdots & r_3^n - r_3^{n-1}r_1 \\
        \vdots & \vdots & \vdots & \ddots & \vdots \\
        1 & r_{n+1} - r_1 & r_{n+1}^2  - r_{n+1}r_1& \hdots & r_{n+1}^n -  r_{n+1}^{n-1}r_1
    \end{bmatrix} \\
    &= \det \begin{bmatrix}
        r_2 - r_1 & r_2(r_2 - r_1)& \hdots & r_2^{n-1}(r_2 - r_1) \\
        r_3 - r_1 & r_3(r_3 - r_1)& \hdots & r_3^{n-1}(r_3 - r_1) \\
        \vdots & \vdots & \ddots & \vdots \\
        r_{n+1} - r_1 & r_{n+1}(r_{n+1} - r_1)& \hdots & r_{n+1}^{n-1}(r_{n+1} - r_1)
    \end{bmatrix} \\
    &= (r_2 - r_1)(r_3-r_1)\hdots(r_{n+1}-r_1)
    \det \begin{bmatrix}
        1 & r_2 & \hdots & r_2^{n-1} \\
        1 & r_3 & \hdots & r_3^{n-1} \\
        \vdots & \vdots & \ddots & \vdots \\
        1 & r_{n+1}& \hdots & r_{n+1}^{n-1}
    \end{bmatrix} \\
    &\neq 0
\end{align*}

Since all the $r_i$ terms are distinct, the $(r_i-r_1)$ terms are nonzero.
By our inductive hypothesis, the determinant of the matrix on the right must be nonzero.
Thus, the determinant of $(r_i^{j-1})_{i,j=1,\hdots,d}$ is nonzero. \\

A nonzero determinant implies that each of the rows are linearly independent,
meaning that the sequences $(r_1^n)_{n \geq 0}, \hdots, (r_d^n)_{n \geq 0}$ 
that make up the rows of the matrix are linearly independent. \\
    
Proof adapted from \href{https://en.wikipedia.org/wiki/Vandermonde_matrix#By_row_and_column_operations}{Wikipedia}.

\newpage

\subsection*{Problem 3: Recurrence Relation as Piecewise Polynomials}
The characacteristic polynomial, $(t^2-1)^d = (t-1)^d(t+1)^d$, 
implies that for constants $\alpha$ and $\beta$, 
the closed form solution of the recurrence relation is 
\[
    a_n = \sum_{i = 0}^{d-1} \alpha_i n^{i} (-1)^n 
        + \sum_{i = 0}^{d-1} \beta_i n^{i}
\]
$(-1)^n = 1$ for even $n$ and $(-1)^n = -1$ for odd $n$ so 
\[
    a_n = 
    \begin{cases} 
        \sum_{i = 0}^{d-1} (\alpha_i + \beta_i) n^{i} & \text{n is even} \\
        \sum_{i = 0}^{d-1} (\beta_i - \alpha_i) n^{i} & \text{n is odd}
     \end{cases}
\]
Since the equations for each case are polynomials of degree $d-1$, 
the statement is true.
\newpage

\subsection*{Problem 4: Equal Linear Recurrence Sequences}
\begin{enumerate}
    \item Since both sequences are of order $d$ and share the same $d$ terms after $k$,
     all the terms $n \geq k+d$ can be uniquely determined through the linear recurrence relation equation.
    The $a_{k-1}$th term is determined through the equation, 
    \[
        a_{k+d-1} = c_1a_{k+d-2} + c_2a_{k+d-3} + \hdots + c_da_{k-1}
    \] 
    Solving out for $a_{k-1}$ yields an equation that only depends on constants 
    and $a_k$ through $a_{k+d-1}$.
    \[
        a_{k-1} = \frac{a_{k+d-1} - c_1a_{k+d-2} - c_2a_{k+d-3} - \hdots - c_{d-1}a_k}{c_d}
    \]
    Repeating this proceedure uniquely determines all terms from $0$ to $k-1$ to be the same
    for both sequences. Since both sequences have the same terms, they are equal.
    \item The $-1$st term can be found similarly to the $(k-1)$ term 
    from the previous part by the equation 
    \[
        b_{-1} = \frac{b_{d-1} - c_1b_{d-2} - c_2b_{d-3} - \hdots - c_{d-1}b_0}{c_d}
    \]
    Repeating this process uniquely determines the values of the sequence for negative indexes
    since the next negative index solely depends on the $d$ larger values in the sequence.
    \item The next negative fibonacci number can be determined using
    \[
        f_{n-2} = f_{n} - f_{n-1}
    \]
    This gives the value that are the same as the fibonacci values but with alternating signs.
    \begin{align*}
        f_{-1} &= 1 \\
        f_{-2} &= -1 \\
        f_{-3} &= 2 \\
        f_{-4} &= -3 \\
        f_{-5} &= 5 \\
        f_{-6} &= -8 \\
        \vdots
    \end{align*}
\end{enumerate}
\newpage 

\subsection*{Problem 5: Limits of Sums and Products of Power Series}
\begin{enumerate}
    \item For all $n$, if $([x^n]A_i(x))_i = [x^n]A(x)$ for some $i \geq N_a(n)$ and 
    $([x^n]B_i(x))_i = [x^n]B(x)$ for some $i \geq N_b(n)$, then 
    \[
        ([x^n](A_i(x) + B_i(x)))_i = [x^n](A(x)+B(x))
    \] 
    is true for $i \geq \max(N_a(n), N_b(n))$. 
    Thus, $\lim_{i \to \infty}(A_i(x) +B_i(x)) = A(x) + B(x)$.
    \item For all $n$, if $([x^n]A_i(x))_i = [x^n]A(x)$ for some $i \geq N_a(n)$ and 
    $([x^n]B_i(x))_i = [x^n]B(x)$ for some $i \geq N_b(n)$, then 
    \[
        A(x)B(x) = \sum_{n \geq 0} \left(\sum_{i=0}^n a_i b_{n-i}\right)x^n
    \]
    Thus the coefficient in the $n$th term depends on the coefficients of
    powers from $0$ to $n$ from $A$ and $B$. So the $n$th term in the product will converge
    at $i \geq \max_{0 \leq m \leq n}(\max(N_a(m), N_b(m)))$.
    Thus, $\lim_{i \to \infty}(A_i(x)B_i(x)) = A(x)B(x)$.
\end{enumerate}
\end{document}