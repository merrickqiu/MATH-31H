%Set document class
\documentclass{article}

%Load math symbol packages
\usepackage{amsmath}
\usepackage{amssymb}
\usepackage{hyperref}
\usepackage{indentfirst}

%User defined commands
\newcommand{\sgn}{\operatorname{sgn}}

\begin{document}
\begin{center}
	\huge{\bf Math 188: Homework 3} \\
	Merrick Qiu
\end{center}

\section{Another Catalan Interpretation}
   Let $T_n$ be the number of ways of triangulating 
   a polygon with $(n+2)$ vertices.
   Consider an arbitrary edge of this polygon.
   The number of ways of triangulating that polygon
   can be divided into disjoint subsets based on which triangle
   that edge is a part of. 

   Moving clockwise from the chosen edge, 
   number the other vertexes from $0$ to $n-1$.
   The triangle consisting of the chosen edge and the $i$th vertex 
   divides the polygon into a $(i+2)$-gon and a $(n-i+1)$-gon.
   Thus the number of ways of triangulating the polygon 
   with the chosen edge connected to the $i$th vertex
   is $T_iT_{n-i-1}$. Summing over all possible values of $i$ yields
   \[
      T_n = \sum_{i=0}^{n-1} T_i T_{n-i-1}.
   \]

   Since $T_0 = 1$ and $T_n$ has 
   the same recurrence relation as the Catalan numbers,
   it follows that $T_n = C_n$.
\newpage

\section{Balanced Parenthesis with Stars}
   There are $a_{n-1}$ strings of length $n$ that start with "*".
   For strings that begin with a "(", consider the ")" that pairs with it.
   If there is a string of length $i$ inside the parentheses,
   then there is a string of length $n-i-2$ to the right of the parenthesis.
   Since both of these strings must contain balanced parenthesis as well,
   there are $a_ia_{n-i-2}$ strings of length $n$ that start with "(", so
   \[
      a_n = a_{n-1} + \sum_{i=0}^{n-2} a_ia_{n-i-2}
   \]
   The right term is the coefficient of $x^{n-2}$ in $A(x)^2$, so 
   \begin{align*}
      A(x) &= 1 + x + \sum_{n \geq 2} a_n x^n \\
           &= 1 + x + \sum_{n \geq 2} 
            \left( a_{n-1} + \sum_{i=0}^{n-2} a_ia_{n-i-2}\right) x^n \\
           &= 1 + x + \sum_{n \geq 2} a_{n-1}x^n + 
            \sum_{n \geq 2}\left(\sum_{i=0}^{n-2} a_ia_{n-i-2}\right) x^n \\
           &= 1 + x + x\sum_{n \geq 2} a_{n-1}x^{n-1} + 
            x^2\sum_{n \geq 2}\left(\sum_{i=0}^{n-2} a_ia_{n-i-2}\right) x^{n-2} \\
           &= 1+ x + x(A(x)-1) + x^2 A(x)^2 \\
           &= 1 + xA(x) + x^2 A(x)^2.
   \end{align*}
   Rearranging the above equation yields 
   \[
      x^2A(x)^2 + (x-1)A(x) + 1 = 0.
   \]
   Thus, 
   \begin{align*}
      a(x) &= x^2 \\
      b(x) &= x-1 \\
      c(x) &= 1.
   \end{align*}
\newpage
\section{Compositions of 2n into 8 Parts}
      \begin{enumerate}
         \item Each $x_i$ can be written as 
            $x_i = 2k_i$ for some $k_i \geq 0$, so
            \begin{align*}
               x_1 + x_2 + \hdots + x_8 = 2n 
               &\implies 2k_1 + 2k_2 + \hdots + 2k_8 = 2n \\
               &\implies k_1 + k_2 + \hdots + k_8 = n.
            \end{align*}
            This is a weak composition of $n$ with 8 parts so
            \[ 
               \binom{8+n-1}{n} = \binom{n+7}{n}.
            \]
         \item Each $x_i$ can be written as 
            $x_i = 2k_i+1$ for some $k_i \geq 0$, so
            \begin{align*}
               x_1 + x_2 + \hdots + x_8 = 2n 
               &\implies (2k_1+1) + (2k_2+1) + \hdots + (2k_8+1) = 2n \\
               &\implies k_1 + k_2 + \hdots + k_8 = n-4
            \end{align*}
            This is a weak composition of $n-4$ with 8 parts so
            \[ 
               \binom{8+(n-4)-1}{n-4} = \binom{n+3}{n-4}.
            \]
         \item For a given $x_8$, the ways of choosing the 
         values for the other $x_i$ is a weak composition 
         of $2n-x_8$ with 7 parts.
         Summing over the possible values of $x_8$ yields 
         \[
            \sum_{i=0}^9 \binom{7+(2n-i)-1}{2n-i} = \sum_{i=0}^9 \binom{2n-i+6}{2n-i}.
         \]
      \end{enumerate}
\newpage 
\section{Sums over Compositions}
      \begin{enumerate}
         \item Let $D$ be the derivative.
            Consider the product 
            \begin{align*}
               P(x) &= \left(\sum_{a_1 \geq 1} a_1x^{a_1}\right)
                       \left(\sum_{a_2 \geq 2} a_2x^{a_2}\right)\hdots 
                       \left(\sum_{a_n \geq 1} a_nx^{a_n}\right) \\
                    &= \left(xD\left(\frac{1}{1-x}\right)\right)^n 
                     = \left(\frac{x}{(1-x)^2}\right)^n
            \end{align*}
            Note that the $[x^k]$ term encodes the sum of products of
            compositions of $k$ into $n$ parts, so
            \[
               [x^k]P(x) = \sum_{(a_1,\hdots,a_n)} a_1a_2\hdots a_n.
            \]   
            Using the binomial theorem yields and the fact that
            $\binom{-n}{k} = (-1)^k\binom{n+k-1}{k}$,
            \begin{align*}
               P(x) &=  x^n(1-x)^{-2n} = x^n \sum_{k \geq 0} (-1)^{k} \binom{-2n}{k} x^k \\
                    &= \sum_{k \geq n} (-1)^{k-n}\binom{-2n}{k-n} x^k = \sum_{k \geq n} (-1)^{k-n}\left((-1)^{k-n} \binom{2n+(k-n)-1}{k-n}\right) x^k \\
                    &= \sum_{k \geq n} \binom{n+k-1}{k-n} x^k.
            \end{align*}
            Thus,
            \[
               \sum_{(a_1,\hdots,a_n)} a_1a_2\hdots a_n 
               = [x^k]P(x)
               = \binom{n+k-1}{k-n}
            \]
         \item Using the same idea as part a, consider the product 
         \begin{align*}
            P(x) &= \left(\sum_{a_1 \geq 1} 1^{a_1-1}x^{a_1}\right)
                    \left(\sum_{a_2 \geq 2} 2^{a_2-1}x^{a_2}\right)\hdots 
                    \left(\sum_{a_n \geq 1} n^{a_n-n}x^{a_n}\right) \\
                 &= \left(\frac{x}{1-x}\right)
                    \left(\frac{x}{1-2x}\right)\hdots 
                    \left(\frac{x}{1-nx}\right) \\ 
                 &= \frac{x^k}{(1-x)(1-2x)\hdots(1-kx)}.
         \end{align*}

         This is the generating function for the stirling numbers, so
         \[
            \sum_{(a_1,\hdots,a_n)}2^{a_2-1}3^{a_3-1}\hdots n^{a_n-1}
            = [x^k]P(x) = S(k,n).
         \] 
      \end{enumerate}
\newpage 
\section{Tuples of subsets}
   \begin{enumerate}
      \item Each element of $[n]$ can either be in $S$ and $T$,
         be in $T$ but not $S$, or not be in $T$ or $S$, so
         there are $3^n$  pairs of subset such that $S \subseteq T$.
         However there are $2^n$ ways in which $S=T$ so there are $3^n - 2^n$
         pairs of subsets such that $S \subsetneqq T$.
      \item Each element of $[n]$ has $2^k$ ways of appearing in $k$ subsets,
         but since the element must appear in at least one subset, there are $2^k-1$ ways 
         for the element to appear in the subsets $(S_1,\hdots,S_k)$.
         Thus, there are $(2^k-1)^n$ $k$-tuples of subsets of $[n]$.

   \end{enumerate}
      

\end{document}