%Set document class
\documentclass{article}

%Load math symbol packages
\usepackage{amsmath}
\usepackage{amssymb}
\usepackage{tikz} 
\usepackage{hyperref}
\usepackage{mathtools}
\usepackage{indentfirst}

%User defined commands
\newcommand{\sgn}{\operatorname{sgn}}
\newcommand{\qbinom}[2]{\binom{#1}{#2}_{q}}
\newcommand{\floor}[1]{\lfloor #1 \rfloor}
\newcommand{\Cn}{\textbf{C}_n}
\newcommand{\Dn}{\textbf{D}_n}
\newcommand{\inv}{^{-1}}


\begin{document}
\begin{center}
	\huge{\bf Math 188: Homework 7} \\
	Merrick Qiu 
\end{center}

\section{Necklaces in the Dihedral Group}
	The Dihedral group is composed of rotational symmetries and
	reflection symmetries. 
	The number of cycles of a rotation is 
	$\gcd(n, i)$ if we rotate by $i$ places,
	and the number of cycles of a reflection is $\lceil \frac{n}{2}\rceil$.
	Thus, 
	\[
		|Y^X/G| 
		= \frac{1}{|G|}\sum_{g\in G} |Y|^{c_X(g)} 
		= \frac{1}{2n}  
		\left(
			nk^{\lceil \frac{n}{2}\rceil} +
			\sum_{i=1}^n k^{\gcd(n, i)} 
		\right)		
	\]
	\newpage 
\section{Coloring a Matrix}
	There are four elements in the group of rotations of the matrix.
	The identity element has 9 1-cycles.
	A rotation clockwise and a rotation counter-clockwise have
	1 1-cycle and 2 4-cycles.
	A 180 degree rotation has 1 1-cycle and 4 2-cycles.
	Thus the number of colorings is,
	\[
		|Y^X/G| 
		= \frac{1}{|G|}\sum_{g\in G} |Y|^{c_X(g)}
		= \frac{1}{4} (k^9 + k^5 + 2k^3)
	\].
	The cyclic indicator is
	\[
		Z_X(G;t_1,\hdots,t_9)
		= \frac{1}{4}(t_1^9 + t_1t_4^2 + t_1t_2^4)
	\].
	The number of ways to color the matrix with three colors 
	for three entries each is 
	\begin{align*}
		&[y_1^3y_2^3y_3^3]\frac{1}{4}
		\left(\left(\sum_{i=1}^k y_i\right)^9 + 
		\left(\sum_{i=1}^k y_i\right)\left(\sum_{i=1}^k y_i^4\right)^2 + 
		\left(\sum_{i=1}^k y_i\right)\left(\sum_{i=1}^k y_i^2\right)^4\right)\\
		=& \frac{1}{4}\left( \binom{9}{3,3,3,0,\hdots} + 0 + 0\right) \\
		=& 420
	\end{align*}
	\newpage
\section{Theorem 7.9}
	For each orbit $\alpha \in Y^X / G$, 
	let $f \in \alpha$ be a representative.
	A bijection between these orbits and 
	weak compositions 
	of $n$ with $d$ parts 
	exists where $a_i = |f\inv[\{i\}]|$
	for a weak composition $(a_1,\hdots,a_d)$. 
	The number of orbits is therefore 
	$\binom{d+n-1}{n}$.
	Applying Theorem 7.9 yields 
	\begin{align*}
		\binom{d+n-1}{n} = \frac{1}{n!}\sum_{g\in G} d^{c_X(g)}
		&\implies (d+n-1)_n = \sum_{g\in G} d^{c_X(g)} \\
		&\implies (d+n-1)_n = \sum_{k=0}^n c(n, k)d^k \\
	\end{align*}
\end{document}

