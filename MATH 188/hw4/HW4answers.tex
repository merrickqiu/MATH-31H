%Set document class
\documentclass{article}

%Load math symbol packages
\usepackage{amsmath}
\usepackage{amssymb}
\usepackage{hyperref}
\usepackage{indentfirst}

%User defined commands
\newcommand{\sgn}{\operatorname{sgn}}
\newcommand{\qbinom}[2]{\binom{#1}{#2}_{q}}

\begin{document}
\begin{center}
	\huge{\bf Math 188: Homework 4} \\
	Merrick Qiu
\end{center}

\section{Stirling Numbers as Polynomials}
   \begin{enumerate}
      \item \textbf{Lemma: }
      If a sequence is a polynomial of degree $0$, 
      its first difference is of degree $0$.
      If a sequence is a polynomial of degree $d+1>0$,
      the first difference is a polynomial of degree $d$ since
      \begin{align*}
         \sum_{k=0}^{d+1} a_k(x+1)^k - \sum_{k=0}^{d+1} a_kx^k
         &= \left(\sum_{k=0}^{d+1} a_kx^k + a_k\binom{k}{1}x^{k-1} + \hdots + a_k\right) - \sum_{k=0}^{d+1} a_kx^k \\
         &=  \left(\sum_{k=0}^{d+1} a_k\binom{k}{1}x^{k-1} + a_k\binom{k}{2}x^{k-2} + \hdots + a_k\right).
      \end{align*}
      This implies that if the first difference of a sequence 
      is a polynomial of degree $d$ where $d>0$,
      then the sequence must be a polynomial of degree $d+1$ 
      since polynomials of other degrees cannot have a first difference of degree $d$.

      \textbf{Proof: }
      We can induct on $r$ to show that 
      $S(n+r, n)$ is a polynomial of $n$ of degree $2r$. 
      For $r=0$, $S(n, n) = 1$ .
      Assume that $S(n+r-1, n)$ is of degree $2r-2$.
      For $r > 0$, applying the recursive formula yields
      \begin{align*}
         &S(n+r, n) = S(n+r-1, n-1) + nS(n+r-1, n) \\
         \implies& S(n+r, n) - S(n+r-1, n-1) = nS(n+r-1, n). 
      \end{align*}
      
      By the inductive hypothesis, $nS(n+r-1, n)$ is of degree $2r-1$.
      Since the first difference of the sequence $(S(n+r, n))_{n \geq 0}$
      is a polynomial of degree $2r-1$ and $2r-1 > 0$, 
      $S(n+r, n)$ is a polynomial of degree $2r$. 
      Similarly for $c(n+k, n)$, we have that $c(n, n) = 1$ for $r=0$
      and for $r > 0$,
      \begin{align*}
         &c(n+r, n) = c(n+r-1, n-1) + (n+r-1)c(n+r-1, n) \\
         \implies& c(n+r, n) - c(n+r-1, n-1) = (n+r-1)c(n+r-1, n). 
      \end{align*}
      The first difference of $(c(n+r, n))_{n \geq 0}$
      is a polynomial of degree $2r-1>0$, so 
      $c(n+r, n)$ is a polynomial of degree $2r$. 

      \item Repeatedly applying the recursive formula for $S(n, k)$ yields 
      \begin{align*}
         &S(n+r, n) = S(n+r-1, n-1) + nS(n+r-1, n) \\
         = &S(n+r-2, n-2) + (n-1)S(n+r-2, n-1) + nS(n+r-1, n) \\
         \vdots \\
         = &\sum_{i=1}^n i S(i+r-1 , i).
      \end{align*}

      For $r=2$ we have that 
      \begin{align*}
         S(n+2, n) 
         &= \sum_{i=1}^n i S(i+1 , i) 
            = \sum_{i=1}^n i \binom{i+1}{2} \\
         &= \frac{1}{2}\sum_{i=1}^n i^3 + \frac{1}{2} \sum_{i=1}^n i^2 
            = \frac{n^2(n+1)^2}{8} +  \frac{n(n+1)(2n+1)}{12} \\
         &= \frac{1}{24}n(n+1)(n+2)(3n+1).
      \end{align*}

      For $r=3$ we have that 
      \begin{align*}
         S(n+3, n) 
         &= \sum_{i=1}^n i S(i+2 , i) 
            = \sum_{i=1}^n \frac{1}{24}i^2(i+1)(i+2)(3i+1) \\
         &= \frac{1}{8}\sum_{i=1}^n i^5 + \frac{5}{12} \sum_{i=1}^n i^4
            + \frac{3}{8}\sum_{i=1}^n i^3 + \frac{1}{12} \sum_{i=1}^n i^2 \\
         &= \frac{1}{48}n^2(n+1)^2(n+2)(n+3)
      \end{align*}

      Repeatedly applying the recursive formula for $c(n, k)$ yields 
      \begin{align*}
         &c(n+r, n) = c(n+r-1, n-1) + (n+r-1)c(n+r-1, n) \\
         = &c(n+r-2, n-2) + (n+r-2)c(n+r-2, n-1) + (n+r-1)c(n+r-1, n) \\
         \vdots \\
         = &\sum_{i=1}^{n} (i+r-1) c(i+r-1 , i)
      \end{align*}

      For $r=2$ we have that 
      \begin{align*}
         c(n+2, n) &= \sum_{i=1}^{n} (i+1) c(i+1, i) \\
         &= \sum_{i=1}^{n} (i+1) \binom{i+1}{2} \\
         &= \frac{1}{24}n(n+1)(n+2)(3n+5)
      \end{align*}

      For $r=3$ we have that 
      \begin{align*}
         c(n+3, n) &= \sum_{i=1}^{n} (i+2)c(i+2 , i) \\
         &= \sum_{i=1}^{n} \frac{1}{24}i(i+1)(i+2)^2(3i+5) \\
         &= \frac{1}{48}n(n+1)(n+2)^2(n+3)^2
      \end{align*}
   \end{enumerate}
\newpage

\section{Equality of Two Integer Partitions}
The generating function for $a_n$ is 
\[
   \sum_{n \geq 0} a_n x^n
   = \prod_{i \geq 0} (1+x^i+x^{2i})
   = (1+x+x^2)(1+x^2+x^4)(1+x^3+x^6)\hdots .
\]
The generating function for $b_n$ is 
\[
   \sum_{n \geq 0} b_n x^n 
   = \prod_{i \geq 0} \frac{1-x^{3i}}{1-x^i}
   = \frac{1}{(1-x)(1-x^2)(1-x^4)(1-x^5)\hdots}.
\]
Since $1+x^i+x^{2i} = \frac{1-x^{3i}}{1-x^i}$,
\[
   \sum_{n \geq 0} a_n x^n
   = \prod_{i \geq 0} (1+x^i+x^{2i})
   = \prod_{i \geq 0} \frac{1-x^{3i}}{1-x^i}
   = \sum_{n \geq 0} b_n x^n.
\]
Since the generating functions for both sequences are equal,
we have that $a_n = b_n$ for all $n$.
\newpage

\section{Generalization of Example 3.27}
The coefficient of $y^kx^n$ in $\prod_{i \geq 0}(1+x^{2i+1}y)$ counts 
the number of partitions of $n$ that have $k$ parts.
Under the bijection, the number of parts in the odd partitions 
is equal to the number of "bends"
(the number of pairs of columns and rows of the same index),
in the Young Diagram. 
Since a partition with a size $r$ Durfee square has $r$ bends,
\[
   \prod_{i \geq 0}(1+x^{2i+1}y)
   = \sum_{r \geq 0} x^{r^2}y^r\sum p_{\leq r}(n)x^2n 
   = \sum \frac{x^{r^2}y^r}{(1-x^2)(1-x^4)\hdots(1-x^{2r})}.
\]
\newpage
\section{q-Binomial Constants}
\begin{enumerate}
   \item 
      We can induct on d to show the identity.
      For $d=0$ we have 
      \[
         \sum_{n \geq 0} \qbinom{n}{n} x^n 
         = \sum_{n \geq 0} x^n
         = \frac{1}{1-x}
      \]
      Assume that $\sum_{n \geq 0} \qbinom{n+d-1}{n} x^n 
      = \prod_{i=0}^{d-1} (1-q^ix)^{-1}$.
      Using the symmetry of the q-binomial coefficient
      and the q-Pascal's identity yields
      \begin{align*}
         \sum_{n \geq 0} \qbinom{n+d}{n} x^n 
         &= \sum_{n \geq 0} \qbinom{n+d}{d} x^n \\
         &= q^d\sum_{n \geq 0} \qbinom{n+d-1}{d} x^n 
            + \sum_{n \geq 0} \qbinom{n+d-1}{d-1} x^n \\
         &= q^dx\sum_{n \geq 0} \qbinom{n+d-1}{n-1} x^{n-1} 
            + \sum_{n \geq 0} \qbinom{n+d-1}{n} x^n \\
         &= q^dx \sum_{n \geq 0} \qbinom{n+d}{n} x^n
            + \prod_{i=0}^{d-1} (1-q^ix)^{-1}.
      \end{align*}

      Solving out for $\sum_{n \geq 0} \qbinom{n+d}{n} x^n$ shows that 
      \begin{align*}
         &\sum_{n \geq 0} \qbinom{n+d}{n} x^n 
            = q^dx \sum_{n \geq 0} \qbinom{n+d}{n} x^n
            + \prod_{i=0}^{d-1} (1-q^ix)^{-1} \\
         \implies &(1-q^dx)\sum_{n \geq 0} \qbinom{n+d}{n} x^n 
            = \prod_{i=0}^{d-1} (1-q^ix)^{-1} \\
         \implies &\sum_{n \geq 0} \qbinom{n+d}{n} x^n 
            = \prod_{i=0}^{d} (1-q^ix)^{-1}.
      \end{align*}
   \item 
      The generating function $\prod_{i=0}^{d} (1-q^ix)^{-1}$
      represents making a choice of integers $m_1, \hdots, m_d$ 
      and selecting the terms $x^{m_i}q^{im_i}$, 
      which corresponds to the partition where $i$ gets used $m_i$ times.

      Therefore a term of degree $x^n$ corresponds
      to a partition with $\leq n$ parts, and
      a term of degree $q^s$ corresponds
      to a partition of size $s$.
      Since all the parts in the partition have size $\leq d$,
      the coefficient of $x^n$ is the sum $\sum_{\lambda} q^{|\lambda|}$
      over all integer partitions, $\lambda$, whose Young diagram fits in 
      a $n \times d$ rectangle.
\end{enumerate}
\newpage 

\section{Counting Linear Maps}
   \begin{enumerate}
      \item 
         A linear map $V \to M$ can be uniquely represented as a $m \times n$ matrix. 
         Thus, there are $q^{mn}$ linear maps.
      \item 
         The linear map is surjective if each of its $m$ rows are linearly independent.
         Thus there are $\prod_{i=0}^{m-1} (q^n-q^i)$ surjective linear maps $V \to W$.
      \item 
         A rank k linear map $V \to W$ is equivalent to a surjective linear map 
         onto a k-dimensional subspace of W.
         Thus there are $\prod_{i=0}^{k-1} (q^n-q^i)$ rank $k$ linear maps.
   \end{enumerate}

\end{document}