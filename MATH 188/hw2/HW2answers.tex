%Set document class
\documentclass{report}

%Load math symbol packages
\usepackage{amsmath}
\usepackage{amssymb}
\usepackage{hyperref}

%User defined commands
\newcommand{\sgn}{\operatorname{sgn}}

\setlength{\parindent}{0pt}
\begin{document}
\begin{center}
	\huge{\bf Math 188: Homework 2} \\
	Merrick Qiu
\end{center}

\subsection*{Problem 1: Formal Power Series Composite Inverse}
\begin{enumerate}
    \item ($\implies$) Let $F(x) = \sum_{i=1}^{\infty} a_n x^n$ and
         let $G(x) = \sum_{i=1}^{\infty} b_n x^n$ 
         be some formal power series with $b_1 \neq 0$.
         If $F(G(x)) = x$, then $[x^1]F(G(x)) = a_1b_1 = 1$ meaning $a_1 \neq 0$. \\
         ($\impliedby$) If $[x^1]F(x) \neq 0$, then the constants $b_i$ can be computed recursively 
         given that $F(G(x)) = x$. This is possible because each $b_i$ only depends on coefficients of smaller indexes.
         \begin{align*}
            a_1b_1 &= 1 \\
            a_1b_2 + a_2b_1^2 &= 0 \\
            a_1b_3 + 2a_2b_1b_2 + a_3b_1^3 &= 0 \\
            a_1b_4 + a_2(b_2^2 + 2b_1b_3) + 3a_3b_1^2b_2 + a_4b_1^4 &= 0 \\
            \vdots
         \end{align*}
         Note that the $G(x)$ constructed here has $G(0) = 0$, meaning that $G(x)$
         satisfies $ F(G(x)) \iff G(0) = 0$.
         % The general form of the $n$th coefficient of the composition is 
         % \[
         %    [x^n]F(G(x)) = \sum_{k \in \mathbb{N}, j_1+\hdots+j_k = n}
         %        a_kb_{j_1}b_{j_2}\hdots b_{j_k}
         % \]
    \item Let $G^{-1}$ be the right composite inverse of $G$ (it can be calculated
         in the same way that $G$ was calculated from $F$).
         Let $I = x$ be the identity power series.
         Using the associativity of power series composition,
         \begin{align*}
            (G \circ F)(x) &= (G \circ (F \circ I))(x) \\
            &= (G \circ (F \circ G \circ G^{-1}))(x) \\
            &= (G \circ (F \circ G) \circ G^{-1})(x) \\
            &=(G \circ G^{-1})(x) \\
            &= x
         \end{align*} 

         Suppose that there exists some other power series $G'$ such that $F(G'(x)) = x$.
         \begin{align*}
            (G \circ F \circ G') (x) 
            &= (G \circ (F \circ G')) (x)  = G(x) \\
            &= ((G \circ F) \circ G') (x) = G'(x)
         \end{align*}
         Thus, $G(x)$ is unique.
\end{enumerate}
\newpage 

\subsection*{Problem 2: Binomial Theorem}
\begin{enumerate}
   \item Using the binomial theorem,
   \[
      \sum_{i=0}^n \binom{n}{i} \frac{1}{2^i} 
      = \left(1+\frac{1}{2}\right)^n
      = \left(\frac{3}{2}\right)^n
   \]
   \item Adding $x^2$ times the second derivative of the binomial theorem to $x$ times the first derivative yields
   \begin{align*}
      (n(n-1)(1+x)^{n-2})x^2 + (n(1+x)^{n-1})x 
      &= \sum_{i=0}^{n} i(i-1)\binom{n}{i}x^{i} + \sum_{i=0}^{n} i\binom{n}{i}x^{i}\\
      &= \sum_{i=0}^{n} i^2\binom{n}{i}x^{i}
   \end{align*}
      

   With $x=3$,
   \begin{align*}
      \sum_{i=0}^{n} i^2\binom{n}{i}3^{i} 
      &= (n(n-1)(4)^{n-2})3^2 + (n(4)^{n-1})3 \\
      &= 3n(1+3n)4^{n-2}
   \end{align*}
\end{enumerate}
\newpage 

\subsection*{Problem 3: Choosing Cats and Dogs}
\begin{enumerate}
   \item $(1+x)^{a+b}$ can be expanded using the binomial theorem.
      \[
         (1+x)^{a+b} = \sum_{n=0}^{\infty} \binom{a+b}{n} x^i
      \]
      $(1+x)^{a}(1+x)^{b}$ can be expanded using the binomial theorem 
      and combined using the definition of products of power series.
      \begin{align*}
         (1+x)^{a}(1+x)^{b} 
         &= \left(\sum_{n=0}^{\infty} \binom{a}{n} x^n\right) 
         \left(\sum_{n=0}^{\infty} \binom{b}{n} x^n\right) \\
         &= \sum_{n=0}^{\infty} \left(\sum_{i=0}^n \binom{a}{i}\binom{b}{n-i}\right) x^n
      \end{align*}
         
      
      Comparing the coefficients term-by-term yields 
      \[
         \binom{a+b}{n} = \sum_{i=0}^n \binom{a}{i}\binom{b}{n-i}
      \]
   \item The number of ways of choosing $n$ animals from $a$ dogs and $b$ cats
      is equivalent to the number of ways of choosing $n$ animals with exactly
      0 dogs plus choosing $n$ animals with exactly 1 dog all the way to choosing 
      $n$ animals with exactly $n$ dogs.
\end{enumerate}
\newpage 

\subsection*{Problem 4: Arranging "MISSISSIPPI"}
The letters count in "MISSISSIPPI" is one of 'M', two of 'P', four of 'I',
 and four of 'S' with eleven total letters.
\[
   \binom{11}{1, 2, 4, 4} = \frac{11!}{1!2!4!4!} = 34650
\]
Using the multinomial coefficient to count, 
there are $34650$ total ways of arranging the letters in "MISSISSIPPI".
\newpage 

\subsection*{Problem 5: Rational Generating Functions}
   Using the binomial theorem, doing a change of variables, 
   utilizing the fact that $k<d+1$ to set $n=0$,
   \begin{align*}
      \sum_{n \geq 0} f(n) x^n 
      &= \frac{g_0 + g_1x + \hdots + g_dx^d}{(1-x)^{d+1}} \\
      &= \sum_{k=0}^{d} \frac{g_kx^k}{(1-x)^{d+1}} \\
      &= \sum_{k=0}^{d} \sum_{n=0} g_k\binom{d+n}{n} x^{n+k} \\
      &= \sum_{k=0}^{d} \sum_{n=k} g_k\binom{d+n-k}{n-k} x^{n} \\
      &= \sum_{k=0}^{d} \sum_{n=0} g_k\binom{d+n-k}{n-k} x^{n} \\
   \end{align*}
   For a given $n=t$ ,
   \[
      f(t) = \sum_{k=0}^{d} g_k\binom{d+t-k}{t-k}  = \sum_{k=0}^{d} g_k\binom{d+t-k}{d}
   \]
   Plugging in $t=0,\hdots,d$ yields the following system of equations in matrix form,
   \[
      \begin{bmatrix}
         f(0) \\ f(1) \\ f(2) \\ \vdots \\ f(d)
      \end{bmatrix}
      = \begin{bmatrix}
         \binom{d}{d} & \binom{d-1}{d} & \binom{d-2}{d} &\hdots & \binom{0}{d} \\
         \binom{d+1}{d} & \binom{d}{d} & \binom{d-1}{d} &\hdots & \binom{1}{d} \\
         \binom{d+2}{d} & \binom{d+1}{d} & \binom{d}{d} &\hdots & \binom{2}{d} \\
         \vdots & \vdots & \vdots & \ddots & \vdots \\
         \binom{2d}{d} & \binom{2d-1}{d} & \binom{2d-2}{d} & \hdots & \binom{d}{d}
      \end{bmatrix}
      \begin{bmatrix}
         g_0 \\ g_1 \\ g_2 \\ \vdots \\ g_d
      \end{bmatrix}
      = \begin{bmatrix}
         1 & 0 & 0 &\hdots & 0 \\
         \binom{d+1}{d} & 1 & 0 &\hdots & 0 \\
         \binom{d+2}{d} & \binom{d+1}{d} & 1 &\hdots & 0 \\
         \vdots & \vdots & \vdots & \ddots & \vdots \\
         \binom{2d}{d} & \binom{2d-1}{d} & \binom{2d-2}{d} & \hdots & 1
      \end{bmatrix}
      \begin{bmatrix}
         g_0 \\ g_1 \\ g_2 \\ \vdots \\ g_d
      \end{bmatrix}
   \]
   Working inductively, we can see that all $g_k$ are integers.
   Since $f(0)$ is an integer and $f(0) = g_0$, $g_0$ is an integer.
   If $g_0, \hdots, g_k$ are integers then 
   \[
      f(k+1) = \sum_{i=0}^{k} \binom{d+k+1-i}{d} g_i + g_{k+1}
   \]
   Since $f(k+1)$ is an integer and $\sum_{i=0}^{k} \binom{d+k+1-i}{d} g_i$
   is a sum of products of integers, then $g_{k+1}$ is also an integer. 
   $f(a)$ is also an integer for all integer $a$ since 
   $f(t)  = \sum_{k=0}^{d} g_k\binom{d+t-k}{d}$, and
   $g_k$ and the binomial coefficients are integers for all $t \in \mathbb{Z}$.
\end{document}