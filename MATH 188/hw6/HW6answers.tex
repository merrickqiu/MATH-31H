%Set document class
\documentclass{article}

%Load math symbol packages
\usepackage{amsmath}
\usepackage{amssymb}
\usepackage{tikz} 
\usepackage{hyperref}
\usepackage{mathtools}
\usepackage{indentfirst}

%User defined commands
\newcommand{\sgn}{\operatorname{sgn}}
\newcommand{\qbinom}[2]{\binom{#1}{#2}_{q}}
\newcommand{\floor}[1]{\lfloor #1 \rfloor}
\newcommand{\Cn}{\textbf{C}_n}
\newcommand{\Dn}{\textbf{D}_n}


\begin{document}
\begin{center}
	\huge{\bf Math 188: Homework 6} \\
	Merrick Qiu
\end{center}

\section{Another Proof of Cayley's Formula}
\begin{enumerate}
   \item Let $T_n$ be all labeled trees with verticies $1,\hdots, n$.
      The partial derivative of $\Cn$ is
      \[
         \frac{\partial \Cn}{\partial x_n} 
         = \sum_{T_n} d_n x_1^{d_1}\hdots x_{n-1}^{d_{n-1}}x_n^{d_n-1}.
      \]
      Substituting in $x_n = 0$ leaves only trees in $T_n$ such that $d_n = 1$.
      These trees can be generated by taking all trees in $T_{n-1}$ and choosing
      a vertex to attach the $n$th vertex to.
      Since attaching the $n$th vertex to the $i$th vertex increases $d_i$ by one,
      we have that 
      \begin{align*}
         \Cn^{(n)}
         &= \sum_{i=1}^{n-1} \left(x_i\sum_{T_{n-1}} x_1^{d_1}\hdots x_{n-1}^{d_{n-1}}\right) \\
         &= (x_1 + x_2 + \hdots + x_{n-1}) \textbf{C}_{n-1}.
      \end{align*}

      The partial derivative of $\Dn$ is
      \[
         \frac{\partial \Dn}{\partial x_n} 
         = (n-2)(x_1\hdots x_n)(x_1 + \hdots + x_n)^{n-3} + 
         (x_1\hdots x_{n-1})(x_1 + \hdots + x_n)^{n-2}.
      \]
      Substituting in $x_n = 0$ yields
      \begin{align*}
         \Dn^{(n)}
         &= (x_1\hdots x_{n-1})(x_1 + \hdots + x_{n-1})^{n-2} \\
         &= (x_1\hdots x_{n-1})\textbf{D}_{n-1}
      \end{align*}
   \item Since the variables are symmetric with respect to each other,
      the proof from part (a) can be repeated to show 
      that for all $i=1,\hdots,n$
      \[
         \Cn^{(i)} = \left(\sum_{\substack{j=1 \\ j\neq i}}^n x_j\right) \textbf{C}_{n-1} 
         \quad \text{and} \quad
         \Dn^{(i)} = \left(\sum_{\substack{j=1 \\ j\neq i}}^n x_j\right) \textbf{D}_{n-1}.
      \]
      Assuming that $\textbf{C}_{n-1} = \textbf{D}_{n-1}$, 
      we have that $\Cn^{(i)} = \Dn^{(i)}$ for all $i=1,\hdots,n$.
   \item 
      From part (b), $\textbf{C}_{n-1} = \textbf{D}_{n-1}$ implies 
      $\Cn^{(i)} = \Dn^{(i)}$ for all $i=1,\hdots,n$,
      which then implies that the 
      coefficients of all $x_i$ are equal.
      This then implies that
      $\Cn = \Dn$ since they are both polynomials.
      Since $\textbf{C}_1 = \textbf{D}_1 = 1$ and 
      $\textbf{C}_2 = \textbf{D}_2 = x_1x_2$, we have that 
      $\Cn = \Dn$ for all $n \geq 1$ from induction.
   \end{enumerate}
   \newpage
   \section{Ordering the letters of MATHEMATICS}
      Let $U$ be the set of orderings of MATHEMATICS
      without restriction.
      Let $M$, $A$, and $T$ be 
      the sets of ways of ordering MATHEMATICS
      with a consecutive repeated `M', `A', and `T' respectively.
      The number of orderings of MATHEMATICS is therefore
      $|U \setminus M \cup A \cup T|$.

      There are $|U| = \frac{10!}{2!2!2!}$ ways to order MATHEMATICS
      with no restrictions. 
      Grouping pairs of letters togther, 
      there are $|M| = |A| = |T| = \frac{9!}{2!2!}$ ways to order MATHEMATICS 
      such that a two characters appear consecutively.
      Similarly, there are $|M \cap A| = |M \cap T| = |A \cap T| = \frac{8!}{2!}$ 
      ways for two characters to appear consecutively twice
      and $|M \cap A \cap T| = 7!$ ways 
      for all three characters to appear consecutively twice.
      Using the inclusion-exclusion principle,
      the number of ways to order MATHEMATICS 
      where each letter does not appear consecutively is
      \begin{align*}
         &|U \setminus M \cup A \cup T| \\
         = &|U| - |M| - |A| - |T|
            + |M \cap A| + |M \cap T| + |A \cap T| 
            - |M \cap A \cap T| \\
         = &\frac{10!}{2!2!2!} - 3\left(\frac{9!}{2!2!}\right)
            + 3\left(\frac{8!}{2!}\right) - 7! \\
         = &236880.
      \end{align*}
   \newpage 
   \section{Circle of Marriage}
   \begin{enumerate}
      \item Let $A_i$ be the set of lines where $i$th couple 
         is standing next to each other.
         By grouping the couples together, the number of ways for $j$ couples 
         to stand next to each other is $2^j(2n-j)!$.
         Using the inclusion-exclusion principle, 
         the number of ways to have everyone stand apart from their spouse is 
         \begin{align*}
            &(2n)! - |A_1 \cup \hdots \cup A_n| \\
            = &(2n)! - \sum_{j=1}^n (-1)^{j-1} \sum_{1\leq i_1<\hdots<i_j\leq n} |A_{i_1}\cap\hdots \cap A_{i_h}| \\
            = &(2n)! - \sum_{j=1}^n (-1)^{j-1} 2^j \binom{n}{j} (2n-j)! \\
            = &(2n)! + \sum_{j=1}^n (-2)^{j}\binom{n}{j} (2n-j)!. \\
         \end{align*}
      \item Let $B_i$ be the set of circles where $i$th couple 
         is standing next to each other.
         Note that the number of circular permutations of $n$ things is $(n-1)!$.
         Using inclusion-exclusion yields 
         \begin{align*}
            &(2n-1)! - |B_1 \cup \hdots \cup B_n| \\
            = &(2n-1)! - \sum_{j=1}^n (-1)^{j-1} \sum_{1\leq i_1<\hdots<i_j\leq n} |B_{i_1}\cap\hdots \cap B_{i_h}| \\
            = &(2n-1)! - \sum_{j=1}^n (-1)^{j-1} 2^j \binom{n}{j} (2n-j-1)! \\
            = &(2n-1)! + \sum_{j=1}^n (-2)^{j}\binom{n}{j} (2n-j-1)!. \\
         \end{align*}
   \end{enumerate}
   \newpage
   \section{Vector Subspace Morbin'}
      The set of $r$-dimensional subspaces $Z$ where $X \subseteq Z \subseteq Y$
      are in bijection with ($r$ - $\dim$ X)-dimensional subspaces of $Y/X$.
      Using the q-binomial theorem with $t=-1$ yields that 
         \begin{align*}
            \sum_{Z \in [X, Y]} \mu(X, Z)
            &= \sum_{Z \in [X, Y]} (-1)^{\dim Z - \dim X} q^{\binom{\dim Z - \dim X}{2}} \\
            &= \sum_{Q \subseteq Y/X} (-1)^{\dim Q} q^{\binom{\dim Q}{2}} \\
            &= \sum_{k=0}^d \qbinom{d}{k} (-1)^{k} q^{\binom{k}{2}} \\
            &= \prod_{k=0}^{d-1}(1-q^k) \\
            &= \delta_{0, d} 
            = \delta_{X, Y}
         \end{align*}
      Since this equality characterizes the Mobius function,
      we have that $\mu(X, Y) = (-1)^dq^{\binom{d}{2}}$.
   \newpage
   \section{Number of Connected Labeled Graphs}
         Let $x$ be a set partitions of $[n]$,
         let $x_i$ be the size of the $i$th block of $x$, and 
         let $|x|$ be the number of blocks in $x$. 
         Let $g(x)$ be the number of labeled graphs such that there no edges between 
         vertices in different blocks.
         Let $f(x)$ be the number of labeled graphs such that there are no edges 
         between blocks and each block is connected.
         Since $g(y) = \sum_{x \leq y}f(x)$ for all set partitions $y$, we have that 
         \begin{align*}
            f(y) 
            &= \sum_{x \leq y}g(x)\mu(x, y) \\
            &= \sum_{x \leq y}2^{\sum_{i=1}^{|x|} \binom{x_i}{2}}\mu(x, y).
         \end{align*}
         If $y$ is the set partition with a single block,
         then $f(y)$ counts the number of connected labeled graphs.

\end{document}