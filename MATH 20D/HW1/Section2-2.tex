\subsection*{Section 2.2}
\textbf{Problem 3}
Is this equation separable? 
\[
    \der{s}{t} = t\ln(s^{2t}) + 8t^2
\]
\solution
It is separable and can be written as 
\[
    \frac{1}{\ln(s^2)+8}\,ds = t^2 \,dt
\]
\textbf{Problem 5}
Is this equation separable?
\[
    (xy^2 + 3y^2)dy - 2xdx = 0
\]
\solution 
It is separable and can be written as 
\[
    y^2 \,dy = \frac{2x}{x+3} \,dx
\]
\textbf{Problem 9}
Solve the equation
\[
    \der{x}{t} = \frac{t}{xe^{t+2x}}
\]
\solution 
Separating the equation yields 
\begin{align*}
    \der{x}{t} = \frac{t}{xe^{t+2x}}
    &\implies (xe^{t+2x})\der{x}{t} = t \\
    &\implies (xe^{2x}) \,dx = te^{-t} \,dt \\
    &\implies \int (xe^{2x}) \,dx = \int te^{-t} \,dt \\
    &\implies \frac{1}{2}xe^{2x} - \int \frac{1}{2}e^{2x} \,dx =
              -te^{-t} - \int -e^{-t} \,dt \\
    &\implies \frac{1}{2}xe^{2x} - \frac{1}{4}e^{2x} = -te^{-t} - e^{-t} + C \\
    &\implies e^{2x}(2x-1) + 4e^{-t}(t+1) = C
\end{align*}
$\frac{1}{xe^{2x}} \neq 0$, so there are no constant solutions. \\
\textbf{Problem 11}
Solve the equation 
\[
    x\der{v}{x} = \frac{1-4v^2}{3v}
\]
\solution 
Separating the equation yields 
\begin{align*}
    x\der{v}{x} = \frac{1-4v^2}{3v}
    &\implies \frac{3v}{1-4v^2} \,dv = \frac{1}{x} \,dx \\
    &\implies \int 3v(1-4v^2)\inv \,dv = \int \frac{1}{x} \,dx \\
    &\implies -\frac{3}{8}\ln(1-4v^2) = \ln x + C \\
    &\implies 1-4v^2 = Cx^{\frac{8}{3}} \\
    &\implies v = \pm \frac{\sqrt{1-Cx^{\frac{8}{3}}}}{2}
\end{align*}
$\frac{1-4v^2}{3v} = 0$ at $v=\frac{1}{2}$, which is a constant solution. \\
\textbf{Problem 12}
Solve the equation 
\[
    \der{y}{x} = \frac{\sec^2 y}{1+x^2}
\]
\solution 
Separating the equation yields 
\begin{align*}
    \der{y}{x} = \frac{\sec^2 y}{1+x^2}
    &\implies \cos^2 y \,dy = \frac{1}{1+x^2} \,dx \\
    &\implies \int \frac{1 + \cos 2y}{2} \,dy = \int \frac{1}{1+x^2} \,dx \\
    &\implies  \frac{1}{2}y + \frac{1}{4}\sin 2y = \tan\inv x  + C\\ 
\end{align*}
$\sec^2 y \neq 0$, so there are no constant solutions. \\
\textbf{Problem 18}
Solve $y' = x^3(1-y)$ with $y(0) = 3$ \\
\solution 
Separating the equation yields 
\begin{align*}
    y' = x^3(1-y)
    &\implies \frac{1}{1-y} \,dy = x^3 \,dx \\
    &\implies -\ln(y) = \frac{1}{4}x^4 + C \\
    &\implies y = e^{-\frac{1}{4}x^4 + C}
\end{align*}
Pluggin in the initial value yields 
\[
    3 = e^{C} \implies C = \ln(3)
\]
Therefore, the solution is 
\[
    y = e^{-\frac{1}{4}x^4 + \ln(3)}
\]
\textbf{Problem 20}
Solve the equation at $y(1) = 1$
\[
    x^2 \der{y}{x} = \frac{4x^2-x-2}{(x+1)(y+1)}
\]
\solution 
Separating the equation yields 
\begin{align*}
    x^2 \der{y}{x} = \frac{4x^2-x-2}{(x+1)(y+1)} 
    &\implies (y+1) \,dy = \frac{4x^2-x-2}{x^2(x+1)} \,dx \\
    &\implies \int y+1 \,dy 
        = \int -\frac{2}{x^2} + \frac{3}{x+1} + \frac{1}{x} \,dx\\
    &\implies \frac{1}{2}y^2 + y 
        = \frac{2}{x} + 3\ln(x+1) + \ln(x) + C \\
\end{align*}
Plugging in the initial value yields
\[
    \frac{3}{2} = 2 + 3\ln(2) + C 
    \implies C = -\frac{1}{2} - 3\ln(2)
\]
Therefore the implicit solution is 
\[
    \frac{1}{2}y^2 + y 
    = \frac{2}{x} + 3\ln(x+1) + \ln(x) -\frac{1}{2} - 3\ln(2)
\]
\textbf{Problem 22}
Solve the equation at $y(0) = 2$
\[
    x^2 dx + 2y dy = 0
\]
\solution 
Separating the equation yields 
\begin{align*}
    x^2 dx + 2y dy = 0
    &\implies \int 2y dy = \int -x^2 dx \\
    &\implies y^2 = -\frac{1}{3} x^3 + C
\end{align*}
Plugging in the initial value gives
\[
    4 = 0 + C \implies C = 4
\]
Therefore the implicit solution is 
\[
    y^2 = -\frac{1}{3} x^3 + 4
\]
\textbf{Problem 26}
Solve the equation at $y(0) = 1$
\[
    \sqrt{y} dx + (1+x) dy = 0
\]
\solution 
Separating the equation yields 
\begin{align*}
    \sqrt{y} dx + (1+x) dy = 0
    &\implies \int y^{-\frac{1}{2}} dy = \int -\frac{1}{1+x} dx \\
    &\implies 2\sqrt{y} = -\ln(1+x) + C \\
    &\implies y  = \frac{(\ln(x+1) + C)^2}{4}
\end{align*}
Plugging in the initial value gives 
\[
    1 = \frac{C^2}{4} \implies C = 2
\]
Therefore the solution is 
\[
    y  = \frac{(-\ln(x+1) + 2)^2}{4}
\]
\textbf{Problem 30}
\begin{enumerate}
    \item Separate $\der{y}{x} = (x-3)(y+1)^{\frac{2}{3}}$
    \item Show that $y=-1$ satisfies the original equation 
    \item Show that there is no choice of $C$ that will yield $y=-1$.
\end{enumerate}
\solution 
\begin{enumerate}
    \item Separating yields 
        \begin{align*}
            \der{y}{x} = (x-3)(y+1)^{\frac{2}{3}}
            &\implies \int (y+1)^{-\frac{2}{3}} \,dy = \int (x-3) \,dx \\
            &\implies 3(y+1)^\frac{1}{3} = \frac{1}{2}x^2-3x + C \,dx \\ 
            &\implies y = -1 + (\frac{x^2}{6}-x+C)^3
        \end{align*}
    \item At $y=-1$, $\der{y}{x} = (x-3)\cdot 0 = 0$, 
        which is true for constants.
    \item For $y=1$, $(\frac{x^2}{6}-x+C)^3 = 0$ has to be true for all $x$,
        but this is not the case.
\end{enumerate}

