\subsection*{Section 4.6}
\textbf{Problem 3}
The root of the auxillary equation is 
\[
    r^2-2r+1 \implies r = 1
\]
From this,
\[
    y_p = v_1(t)e^t + v_2(t)te^t
\]
The denominator of the integrals is
\[
    1(e^te^t(t+1) - e^tte^t)) = e^{2t}
\]
Taking the integrals,
\[
    v_1 
    = \int -(t\inv e\inv)te^t \,dt
    = \int -1 \,dt
    = -t
\]
\[
    v_2
    = \int (t\inv e\inv)e^t \,dt
    = \int t\inv \,dt
    = \ln t
\]
Therefore the general solution is 
\[
    y = C_1e^t + C_2te^t + \ln (t) te^t
\]
\textbf{Problem 6}
The homogeneous equation has general solution 
\[
    C_1\cos 3t + C_2\sin 3t
\]
The denominator of the integrals is 
\[
    1(3\cos^2t + 3\sin^2t) = 3
\]
Taking the integrals,
\[
    v_1
    = \int -\frac{\sec^2 (3t)\sin 3t}{3} \,dt
    = -\frac{1}{9} \sec 3t
\]
\[
    v_2 
    = \int \frac{\sec^2 (3t)\cos 3t}{3} \,dt
    = \frac{1}{9} \ln |\sec 3t + \tan 3t|
\]
Therefore the general solution is 
\[
    y = C_1\cos 3t + C_2\sin 3t - \frac{1}{9} + \frac{1}{9} \sin 3t \ln |\sec 3t + \tan 3t|
\]
\textbf{Problem 9}
The auxillary roots are 
\[
    r^2 - 1 \implies r = -1, 1
\]
Therefore the particular solution has equation 
\[
    y_p = (A_1t + A_0) + B_0
\]
The only constants that solve the equation are 
\[
    y_p = -2t-4
\]
Using variation of parameters,
\[
    y_p = v_1(t)e^{-t} + v_2(t)e^t
\]
The denominator is
\[
    1(1+1) = 2
\]
Taking the integrals,
\[
    v_1
    = \int -(t+2)(e^t) \,dt
    = -e^t(t+1)
\]
\[
    v_2
    = \int (t+2)e^{-t} \,dt 
    = -e^{-t}(t+3)
\]
Therefore, 
\[
    y_p = -t-1-t-3 = -2t-4
\]
Using undetermined coefficients is a lot quicker. \\
\textbf{Problem 16}
We can use the method of undetermined coefficients.
The roots of the auxillary equation are 
\[
    r^2+5r+6 = (r+3)(r+2) \implies r=-3,-2
\]
Therefore the particular solution and its derivatives have form
\[
    y_p = A_2t^2 + A_1t + A_0
\]
\[
    y_p' = 2A_2t + A_1
\]
\[
    y_p'' = 2A_2
\]
Solving for the constants and adding the homogenous solution gives
\[
    y = 3t^2 - 5t + \frac{19}{6} + c_1e^{-2t} + c_2e^{-3t}
\]
\textbf{Problem 18}
We can use the method of variation of parameters.
The auxillary equation has roots
\[
    r^2-6r+ 9 = (r-3)^2 \implies r = 3
\]
The denominator has form
\[
    1(e^{3t}(3te^{3t} + e^{3t}) - (3e^{3t})(te^{3t}))
    = e^{6t}(3t+1)-3te^{6t}
    = e^{6t}
\]
Taking the integrals 
\[
    v_1
    = \int -\frac{t^{-3}e^{3t}te^{3t}}{e^{6t}} \,dt 
    = \int -\frac{1}{t^2} \,dt
    = t\inv
\]
\[
    v_2
    = \int \frac{t^{-3}e^{3t}e^{3t}}{e^{6t}} \,dt 
    = \int t^{-3} \,dt 
    = -\frac{1}{2t^2}
\]
Thus the general solution is 
\[
    y = \frac{e^{3t}}{2t} + c_1e^{3t} + c_2te^{3t}
\]

