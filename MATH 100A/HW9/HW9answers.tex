\documentclass{article}

\usepackage{amsmath}
\usepackage{amssymb}
\usepackage{hyperref}
\usepackage{mathrsfs}
\usepackage{enumerate}
\usepackage{bm}
\usepackage{physics}
\setlength{\parindent}{0pt}
\usepackage[parfill]{parskip}
\usepackage[margin=1in]{geometry}

\DeclareMathOperator{\im}{im}

\begin{document}
\begin{center}
	\huge{\bf Math 100A: Homework 9} \\
	Merrick Qiu
\end{center}

\section*{Problem 1}
We can show that the both representations have the same trace
to show that they are isomorphic.
Each element $g \in D_{2n}$ can be written in the form
$g = x^ay^b$ for $0 \leq a < n$ and $0 \leq y < 2$.

If $b = 0$ then 
\begin{align*}
	\chi_{\rho_1}(x^a) &= 
	\tr
	\left(\begin{bmatrix}
		\cos(\frac{2\pi}{n}) & -\sin(\frac{2\pi}{n})\\
		\sin(\frac{2\pi}{n}) & \cos(\frac{2\pi}{n})
	\end{bmatrix} \right) \\ &= 2\cos(\frac{2\pi}{n}) \\ \\
	\chi_{\rho_1}(x^a) &= 
	\tr
	\left(\begin{bmatrix}
		e^{2\pi i/n} & 0\\
		0 & e^{-2\pi i/n}
	\end{bmatrix} \right) \\ &= 
	\left(\cos(\frac{2\pi}{n}) + i\sin(\frac{2\pi}{n})\right) +
	\left(\cos(\frac{2\pi}{n}) - i\sin(\frac{2\pi}{n})\right) \\ &=
	2\cos(\frac{2\pi}{n})
\end{align*}

If $b=1$ then 
\begin{align*}
	\chi_{\rho_1}(x^ay) &= 
	\tr
	\left(\begin{bmatrix}
		\cos(\frac{2\pi}{n}) & \sin(\frac{2\pi}{n})\\
		\sin(\frac{2\pi}{n}) & -\cos(\frac{2\pi}{n})
	\end{bmatrix} \right) \\ &= 0 \\ \\
	\chi_{\rho_1}(x^a) &= 
	\tr
	\left(\begin{bmatrix}
		0 & e^{2\pi i/n} \\
		e^{-2\pi i/n} & 0
	\end{bmatrix} \right) \\ &= 
	0
\end{align*}

Since the traces of both representations are equal,
the representations are isomorphic.
\newpage 

\section*{Problem 2}
If $a \in \mathbb{Z}/n\mathbb{Z}$ then the $k$th representation 
is 
\[
	\rho_k(a) = e^{ak(2\pi i/n)}
\]
This is a homomorphism for all $k$ since 
\[
	\rho_k(a+b) = e^{(a+b)k(2\pi i/n)} = e^{ak(2\pi i/n)}\cdot e^{bk(2\pi i/n)} = \rho_k(a)\rho_k(b)
\]
\newpage 

\section*{Problem 3}
Let $C_n \subseteq D_{2n}$ be the cyclic normal subgroup of size $n$.
Restricting $\rho$ to $C_n$, we can write $V$ as a direct
sum of irreducible representations of $C_n$ by Maschke's theorem.
In particular, we have a nonzero eigenvector $v \in V$
such that $gv = \lambda(g)v$ for every $g \in C_n$
by Schur's lemma. Let $W = \operatorname{Span}(v, y\cdot v)$.
We will now prove that $W$ is a  $D_{2n}$ invariant subspace of $V$.

Let $w \in W$ with $w = av + by\cdot v$ for $a,b \in \mathbb{C}$.
Let $x^a \in D_{2n}$.
\begin{align*}
	\rho(x^a)(av + b\rho(y)v) &=
	a\rho(x^a)v + b\rho(x^ay)v \\
	&=a\rho(x^a)v + b\rho(y)\rho(x^{-a})v \\
	&= a\lambda(x^a) v + b\lambda(x^{-a})\rho(y) v \\
\end{align*}
Similarly for $x^ay \in D_{2n}$,
\begin{align*}
	\rho(x^ay)(av + b\rho(y)v) &=
	a\rho(x^ay)v + b\rho(x^a)v \\
	&=a\rho(y)\rho(x^{-a})v + b\rho(x^{a})v \\
	&= a\lambda(x^{-a})\rho(y) v + b\lambda(x^a) v \\
\end{align*}

Therefore $W$ is a $D_{2n}$ invariant subspace of $V$.
Since $\rho$ is an irreducible representation,
we have that $W=V$ and that $\dim(V) \leq 2$
since $\dim(W) = 2$.

	




\end{document}