\documentclass{article}

\usepackage{amsmath}
\usepackage{amssymb}
\usepackage{hyperref}
\usepackage{mathrsfs}
\usepackage{enumerate}
\usepackage{bm}
\usepackage{physics}
\setlength{\parindent}{0pt}
\usepackage[parfill]{parskip}
\usepackage[margin=1in]{geometry}

\DeclareMathOperator{\im}{im}

\begin{document}
\begin{center}
	\huge{\bf Math 100A: Homework 8} \\
	Merrick Qiu
\end{center}

\section*{Problem 1}
Since the trace is the sum of the diagonal entires 
and each diagonal entry is the dot product of the 
associated row in $X$ and column in $Y$, then
\begin{align*}
	\tr(XY) &= \sum_{i=1}^n \sum_{j=1}^n X_{i,j}Y_{j,i} \\
	&= \sum_{j=1}^n \sum_{i=1}^n Y_{j,i}X_{i,j} \\
	&= \tr(YX)
\end{align*}
As a corollary, we have that
\[
	\tr(SXS^{-1}) = \tr(SS^{-1}X) = \tr(X)
\]
\newpage 

\section*{Problem 2}
($\implies$) Suppose there existed nonzero $w \in V$
such that $\textbf{C}[G](w) = W \neq V$.
$W \neq \{0\}$ since $W = \{0\}$ would imply $\rho(g)(w) = 0$
for all $g$.  Since $\rho(g)$ is invertible,
that implies $w=0$ which is a contradiction.

$W$ is a nontrivial G-invariant subspace
since it is the span of $g\cdot w$ for all $g\in G$.
Applying a group action $g$ to an element in that span 
will still yield an element in that span.
Therefore if $V$ is irreducible it must be that 
$\textbf{C}[G](v) = V$ for all nonzero $v\in V$.

($\impliedby$) 
If $W \subseteq V$ was a nontrivial G-invariant subspace,
then $\rho(g)(w) \in W$ for all $g$, and so 
$\textbf{C}[G](w) \subseteq W$.
However this contradicts the fact that 
$\textbf{C}[G](v) = V$ for all $v \neq 0$ in $V$.
Therefore $V$ does not contain any G-invariant subspaces 
other than $\{0\}$ and $V$.
\newpage 

\section*{Problem 3}
Let $W= \{(v_1,\ldots,v_n): v_1+\cdots+v_n = 0\}$.
Suppose $w=(w_1,\ldots,w_n) \in W$ is not zero.
It is sufficient to show that $\textbf{C}[S_n](w) = W$ to show that 
$W$ is irreducible by the previous question.
There exists $i<j$ so that $w_i \neq w_j$ since $w \neq 0$.
Let $\tau$ be the permutation of $S_n$ that 
only exchanges $i$ with $j$.


Notice that $\tau\cdot w - w \in \textbf{C}[S_n](w)$
(since it is a linear combination of permutations of $w$)
and that $\tau\cdot w - w = \alpha(e_i - e_j)$ for some $\alpha \neq 0$. 
Let $\sigma_k$ be the permutation that exchanges $j$ and $n$ and then
exchanges $i$ with $k$.
Then $\sigma_k(\tau\cdot w - w) = \alpha(e_{k} - e_{n}) \in \textbf{C}[S_n](w)$ too.
Since any $w \in W$ can be written as a linear combination of the $e_{k} - e_{n}$ terms,
and each  $e_{k} - e_{n}$ is in $\textbf{C}[S_n](w)$,
we have that $\textbf{C}[S_n](w) = W$.



\end{document}