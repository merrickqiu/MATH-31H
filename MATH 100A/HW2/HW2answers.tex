\documentclass{article}

\usepackage{amsmath}
\usepackage{amssymb}
\usepackage{hyperref}
\usepackage{mathrsfs}
\usepackage{enumerate}
\usepackage{bm}
\usepackage{physics}
\setlength{\parindent}{0pt}
\usepackage[parfill]{parskip}
\usepackage[margin=1in]{geometry}

\begin{document}
\begin{center}
	\huge{\bf Math 100A: Homework 2} \\
	Merrick Qiu
\end{center}

\section*{Problem 1}
\begin{align*}
	ab &= a^{15}b \\
	&= a^{12}(a^3b) \\
	&= a^{12}ba^3 \\
	&= ba^{15} \\
	&= ba
\end{align*}

\section*{Problem 2}
Suppose $ab$ has order $n$. 
This means that $(ab)^n = 1$.
\begin{align*}
	1 &= (ab)^n \\
	\iff 1 &= a(ba)^{n-1}b \\
	\iff a^{-1}b^{-1} &= (ba)^{n-1} \\
	\iff ba(a^{-1}b^{-1}) &= ba(ba)^{n-1} \\
	\iff 1 &= (ba)^n
\end{align*}
Thus $(ab)^n = 1$ iff $(ba)^n = 1$.

\section*{Problem 3}
If $G$ has no proper subgroup, it must be cyclic.
If it was not cyclic, then the group generated by an element of $G$
would be a proper subgroup.

Since $G$ is cyclic and it doesn't have any proper subgroups, it must be finite.
If it was infinite, then it would have a proper subgroup.
For example if $g$ generates an infinite cyclic group $G$, then the group generated by $g^2$ 
would be a proper subgroup.

Let $G$ be the finite cyclic group generated by $g$.
$G$ must either have order $1$ or order $p$.
Suppose the order, $n$, of $G$ can be written as the product of two integers greater than $1$,
$n=pq$.
Then the group generated by $g^p$ would be a proper subgroup,
which is a contradiction.

\section*{Problem 4}
Suppose $a^m = 1$ and $b^n = 1$.
Since $G$ is abelian, we have that 
\[
	(ab)^{mn} = a^{mn}b^{mn} = (a^m)^n (b^n)^m = 1
\]
Thus $ab$ has finite order.

Take $SL_2(\mathbb{R})$ to be an example of a non-abelian group.
\begin{align*}
	&a = \begin{bmatrix}
		-1 & 0 \\
		0 & 1
	\end{bmatrix} 
	&ab = 
	\begin{bmatrix}
		1 & -1 \\
		0 & 1 
	\end{bmatrix} \\
	&b = 
	\begin{bmatrix}
		-1 & 1 \\
		0 & 1
	\end{bmatrix} 
	&(ab)^n = 
	\begin{bmatrix}
		1 & -n \\
		0 & 1 
	\end{bmatrix}
\end{align*}
$a^2 = b^2 = I$ but $(ab)^n$ is never $I$,
so this proves $ab$ need not have finite order 
in a non-abelian group.

\section*{Problem 5}
Since $G$ is cyclic, each element $g \in G$ can be written 
as $g = x^n$ for some integer $n$.
Since $\varphi$ is surjective, each element $h \in G'$ 
can be written as $h = \varphi(g)$ for some $g \in G$.
Thus $G'$ is a cyclic group generated by $\varphi(x)$ since
each element $h$ can be written as
\[
	h = \varphi(g) = \varphi(x^n) = \varphi(x)^n.
\]

Suppose $G$ is abelian.
Let $h, h' \in G'$ with $h = \varphi(g)$ and $h' = \varphi(g')$
for some $g, g' \in G$.
Then $G'$ is abelian since
\begin{align*}
	hh' &= \varphi(g)\varphi(g') \\
	&= \varphi(gg') \\
	&= \varphi(g'g) \\
	&= \varphi(g')\varphi(g) \\
	&= h'h.
\end{align*}



\end{document}