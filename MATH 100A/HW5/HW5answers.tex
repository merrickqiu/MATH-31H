\documentclass{article}

\usepackage{amsmath}
\usepackage{amssymb}
\usepackage{hyperref}
\usepackage{mathrsfs}
\usepackage{enumerate}
\usepackage{bm}
\usepackage{physics}
\setlength{\parindent}{0pt}
\usepackage[parfill]{parskip}
\usepackage[margin=1in]{geometry}

\DeclareMathOperator{\im}{im}

\begin{document}
\begin{center}
	\huge{\bf Math 100A: Homework 5} \\
	Merrick Qiu
\end{center}

\section*{Problem 1}
Since $yxy^{-1} = x^{-1}$ and $m$ is odd,
\[
	x = y^m x y^{-m} = x^{-1}
\]
Therefore $x^2 = 1$.
\newpage 

\section*{Problem 2}
Let $n = 2k+1$.
Since $x^2 = 1$ and $x^n = x^{2k+1} = 1$,
this implies that $x^{2k+1}(x^{2})^{-k} = x = 1$.
Substituting $x=1$, we have the relations $y^m = 1$ and $yy^{-1} = 1$,
which is a cyclic group of order $m$ with generator $y$.
\newpage 

\section*{Problem 3}
The last relation implies that $yx = x^{-1}y$, meaning we can
conjugate all the $y$ to the right and write all elements in $G_{n,m}$
in the form $x^ay^b$ for 
$a \in \{0, \ldots, n-1 \}$ and $b \in \{0, \ldots, m-1 \}$.
Thus there are at most $nm$ elements.
\newpage 

\section*{Problem 4}
The cyclic group $N = \langle x \rangle$ is of order $n$
so it is isomorphic to $\mathbb{Z}/n\mathbb{Z}$.
It is a normal subgroup of $G_{n,m}$ because of the relation $yxy^{-1} = x^{-1}$.

The cyclic group $M = \langle y \rangle$ is of order $m$
so it is isomorphic to $\mathbb{Z}/m\mathbb{Z}$.
$N \cap M = \{1\}$ since the set of symbols $x^k$ is distinct from the set 
of symbols $y^k$ apart from identity.

Since every element in $G_{n,m}$ can be written as $x^ay^b$
for integers $a,b$ we have that $G_{n,m} = NM$.
Let the map $\Psi : G'_{n,m} \to G_{n,m}$ be given by $\Psi(x^a,y^b) = x^ay^b$
for integers $a,b$.

By the third relation $(\varphi \circ \pi)(y^m)$ is an automorphism
that is conjugation by $y^m$
\[
	(\varphi \circ \pi)(y^m)(x^n) = 
	\begin{cases}
		x^n & \text{ if m is even} \\
		x^{-n} & \text{ if m is odd} \\
	\end{cases} = 
	y^m x^n y^{-m}.
\]
$\Psi$ is an homomorphism since
\begin{align*}
	\Psi((x^a, y^b)(x^c, y^d)) &= \Psi((x^a\varphi(\pi(y^b))(x^c), y^by^d))\\
	&=x^a\varphi(\pi(y^b))(x^c) y^by^d \\
	&= x^ay^bx^cy^{-b}y^by^d \\
	&= x^ay^bx^cy^d \\
	&=\Psi(x^a, y^b)\Psi(x^c, y^d)
\end{align*}

$\Psi$ is injective since $N \cap M = \{1\}$
and it is surjective since $G_{n,m} = NM$.
Therefore $\Psi$ is an isomorphism between
$\Psi : G'_{n,m}$ and $G_{n,m}$.












\end{document}