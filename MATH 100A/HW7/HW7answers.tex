\documentclass{article}

\usepackage{amsmath}
\usepackage{amssymb}
\usepackage{hyperref}
\usepackage{mathrsfs}
\usepackage{enumerate}
\usepackage{bm}
\usepackage{physics}
\setlength{\parindent}{0pt}
\usepackage[parfill]{parskip}
\usepackage[margin=1in]{geometry}

\DeclareMathOperator{\im}{im}

\begin{document}
\begin{center}
	\huge{\bf Math 100A: Homework 7} \\
	Merrick Qiu
\end{center}

\section*{Problem 1}
Every orientation reversing isometry can 
be written as a glide-reflection.
We can write $m = t_a r_\ell$ for $a$ parallel to line $\ell$.
Note that $r_\ell t_a r_\ell = t_a$ since translations are a normal subgroup.
Therefore 
\[
	m^2 = t_a r_\ell t_a r_\ell = t_{2a}
\]
\newpage

\section*{Problem 2}
\begin{align*}
	x^2yxyx^3yx^4 &= x^2x^{-1}yx^{-3}yx^{-4}y \\
	&= x^2x^{-1}x^{3}yx^{4}y^2 \\ 
	&= x^2x^{-1}x^{3}x^{-4}y^3 \\ 
	&= x^0y^1
\end{align*}
\newpage 

\section*{Problem 3}
First we will prove that $\Gamma \supseteq 3L$.
For each $\Gamma \subseteq L$ with index 3, there is the
associated homomorphism $L \to L/\Gamma$ that sends each element to its coset.
Notice that $\Gamma$ is the kernel of this homomorphism.
Since $\Gamma$ has index 3, $L/\Gamma$ has order 3
and so it is isomorphic to the cyclic group $\mathbb{Z}_3$.
Therefore we can characterize each subgroup of index 3 as being the kernel
of of a homomorphism $f: L \to \mathbb{Z}_3$.
If $a,b \in \mathbb{Z}$ and $3av_1 + 3bv_2 \in 3L$ then
\begin{align*}
	f(3av_1 + 3bv_2) &= f(3av_1) + f(3bv_2) \\
	 &= 3f(av_1) + 3f(bv_2) \\
	 &= 0 \mod 3.
\end{align*}
Therefore $3L$ is in the kernel of $f$ and $3L \subseteq \Gamma$.
This implies that if $av_1 + bv_2 \in \Gamma$, then 
$a'v_1 + b'v_2 \in \Gamma$ as well if $a' = a \mod 3$ and $b' = b \mod 3$.
Therefore we only need to consider whether 
$av_1 + bv_2$ with $a,b \in \mathbb{Z}_3$ are in $\Gamma$.
These are the eight nonzero vectors with coefficients unique modulo $3$.

\begin{table}[h!]
	\centering
	\begin{tabular}{|c|c|c|c|}
		\hline
		$w_i$      & $2w_i$ & Subgroup elements & Cosets \\ \hline
		$1v_1 + 0v_2$ & $2v_1 + 0v_2$ & $av_1 + 3bv_2$ & $(0 + \Gamma)$, $(v_2 + \Gamma)$, $(2v_2 + \Gamma)$  \\ \hline
		$0v_1 + 1v_2$ & $0v_1 + 2v_2$ & $3bv_1 + av_2$ & $(0 + \Gamma)$, $(v_1 + \Gamma)$, $(2v_1 + \Gamma)$  \\ \hline
		$1v_1 + 1v_2$ & $2v_1 + 2v_2$ & $(a+3b)v_1 + (a + 3b)v_2$ & $(0 + \Gamma)$, $(v_1-v_2 + \Gamma)$, $(2v_1-2v_2 + \Gamma)$  \\ \hline
		$1v_1 + 2v_2$ & $2v_1 + 1v_2$ & $(a+3b)v_1 - (a + 3b)v_2$ & $(0 + \Gamma)$, $(v_1+v_2 + \Gamma)$, $(2v_1+2v_2 + \Gamma)$  \\ \hline
	\end{tabular}
\end{table}

In the table above, these eight vectors are listed in the first two columns.
Notice that the first two columns are integer multiples of each other,
so we only need to consider if $\Gamma$ contains the four vectors 
in the first column denoted $w_i$.
$\Gamma$ must contain just one $w_i$, because 
if it contained more than one then $\Gamma = L$.
Therefore there are at most $4$ subgroups of index $3$.
Since each of these subgroups are unique,
there are exactly four subgroups of index $3$.
Each subgroup can be be written as $\Gamma =\langle 3L \cup \{w_i\} \rangle$.
\newpage

\section*{Problem 4}
$(\impliedby)$First we will prove that if $g \in M_2(\mathbf{Z})$,
then $(w_1, w_2)$, where $(w_1,w_2) = (v_1,v_2)g$,
is a lattice basis for $L$.
If $g = 
\begin{bmatrix}
	a & b \\
	c & d
\end{bmatrix}$ with determinant $\pm 1$,
then $w_1 = av_1 + cv_2$ $w_2 = bv_1 + dv_2$ with 
$ad-bc = \pm 1$.

If $v = \mathbf{Z}w_1 + \mathbf{Z}w_2$, it is in $L$.
Writing $v = k_1w_1 + k_2w_2$ for constants $k_1, k_2 \in \mathbf{Z}$
then
\begin{align}
	v &= k_1w_1 + k_2w_2 \\
	  &= k_1(av_1 + cv_2) + k_2(bv_1 + dv_2) \\
	  &= (k_1a + k_2b)v_1 + (k_1c + k_2d)v_2.
\end{align}
If $v \in L$ then it is in $\mathbf{Z}w_1 + \mathbf{Z}w_2$.
We can write $v = k_1v_1 + k_2v_2$
for constants $k_1, k_2 \in \mathbf{Z}$.
Since $(w_1,w_2)g^{-1} = (v_1,v_2)$ and $g^{-1} = \frac{1}{ad-bc}\begin{bmatrix}
	d & -b \\
	-c & a
\end{bmatrix}$, then 
$v_1 = \frac{1}{ad-bc} (dw_1 - cw_2)$,
$v_2 = \frac{1}{ad-bc}(-bw_1 + acw_1)$.
Therefore we have that
\begin{align}
	v &= k_1v_1 + k_2v_2 \\
	  &= \frac{1}{ad-bc}\left( k_1(dw_1 - cw_2) + k_2((-bw_1 + aw_2))\right) \\
	  &= \frac{1}{ad-bc}\left(  (k_1d - k_2b)w_1 + (-k_1c + k_2a)w_2\right). 
\end{align}
Therefore $\mathbf{Z}v_1 + \mathbf{Z}v_2 = \mathbf{Z}w_1 + \mathbf{Z}w_2$ 
and$(w_1, w_2)$ is a lattice basis if $g \in M_2(\mathbf{Z})$.

$(\implies)$Next we will prove that if $(w_1,w_2)$ is a lattice basis for $L$,
we can write it in the form $(w_1,w_2) = (v_1,v_2)g$
for $g \in M_2(\mathbf{Z})$ with determinant $\pm 1$.
If $(w_1,w_2)$ is a lattice basis for $L$, then 
for all $v \in \mathbf{Z}w_1 + \mathbf{Z}w_2$ with $v = k_1w_1 + k_2w_2$, there exists
constants $s_1, s_2 \in \mathbf{Z}$ such that
$v = s_1v_1 + s_2v_2$.

When $v = w_1$, let $a$ and $c$ be the constants such that 
\[
	w_1 = av_1 + cv_2.
\]
When $v = w_2$, let $b$ and $d$ be the constants such that 
\[
	w_2 = bv_1 + dv_2.
\]
Therefore we can write,
$(w_1,w_2) = (v_1,v_2)\begin{bmatrix}
	a & b \\
	c & d
\end{bmatrix} = (v_1,v_2)g$.
From the coefficients in equation (6),
we must have that $\frac{1}{ad-bc}$ is an integer
in order that every $v\in \mathbf{Z}v_1 + \mathbf{Z}v_2$ is also in $\mathbf{Z}w_1 + \mathbf{Z}w_2$.
Since $a,b,c,d$ are all integers, this is only 
possible when $ad-bc = \pm 1$, meaning that the 
determinant of $g$ is $\pm 1$.
\newpage 
\section*{Problem 5}
Suppose $f,g$ are the rotations about distinct points $P,Q$ respectively.
We can write each element of $f$ in the form $t_P R_\theta t_{-P}$ for some $\theta$.
Likewise we can write each element of $g$ as $t_Q R_\theta t_{-Q}$ for some $\theta$.
If we rotate around $P$ by $\pi$ and around $Q$ by $\pi$ then we get a translation.
For all $v \in \mathbf{R}^2$,
\begin{align*}
	(t_P R_{\pi} t_{-P})(t_Q R_\pi t_{-Q})v &= t_P R_{-\pi} t_{-P}t_Q R_\pi (v-Q) \\
	&= t_P R_{\pi} t_{-P}t_Q  (-v+Q) \\
	&= t_P R_{\pi} t_{-P}  (-v+2Q) \\
	&= t_P R_{\pi}  (-v+2Q-P) \\
	&= t_P  (v-2Q+P) \\
	&= v-2Q+2P \\
	&= (t_{2P-2Q})v
\end{align*}
Therefore the group generated by $f$ and $g$ contains 
a non-trivial translation.

\newpage





\end{document}