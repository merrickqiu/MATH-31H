\documentclass{article}

\usepackage{amsmath}
\usepackage{amssymb}
\usepackage{hyperref}
\usepackage{mathrsfs}
\usepackage{enumerate}
\usepackage{bm}
\usepackage{physics}
\setlength{\parindent}{0pt}
\usepackage[parfill]{parskip}
\usepackage[margin=1in]{geometry}

\DeclareMathOperator{\im}{im}

\begin{document}
\begin{center}
	\huge{\bf Math 100A: Homework 4} \\
	Merrick Qiu
\end{center}

\section*{Problem 1}
By the Chinese Remainder Theorem,
the statement is true for $n=2$.

Let $r_1,r_2,\dots, r_k$ be pairwise coprime positive integers.
Assume that the canonical map 
\[
	\mathbb{Z}/(r_1\cdots r_{k-1}\mathbb{Z}) \rightarrow (\mathbb{Z}/r_1\mathbb{Z})\cross\cdots\cross(\mathbb{Z}/r_{k-1}\mathbb{Z})
\]
is an isomorphism for $n=k-1$.
Since $r_k$ is coprime with $r_1$ and $r_2$,
we can write 
\[
	ar_1 + br_k = 1
\]
\[
	cr_2 + dr_k = 1
\]
Multiplying these two equations yields
\begin{align*}
	(ar_1 + br_k)(cr_2 + dr_k) &= acr_1r_2 + bcr_2r_k + adr_1r_k + bd r_k^2 \\
	&= ac(r_1r_2) + (bcr_2 + adr_1 + bd r_k)r_k \\
	&= 1
\end{align*}
Therefore $r_k$ is coprime with $r_1r_2$.
By induction, $r_k$ is coprime with the product $r_1r_2\dots r_{k-1}$.
Applying the chinese remainder theorem on $r_1\cdots r_{k-1}$ and $r_k$ yields 
\begin{align*}
	\mathbb{Z}/((r_1\cdots r_{k-1})r_k\mathbb{Z}) &\rightarrow 
	\mathbb{Z}/(r_1\cdots r_{k-1}\mathbb{Z}) \cross (\mathbb{Z}/r_k\mathbb{Z})  \\ &\rightarrow
	((\mathbb{Z}/r_1\mathbb{Z})\cross\cdots\cross(\mathbb{Z}/r_{k-1}\mathbb{Z})) \cross (\mathbb{Z}/r_k\mathbb{Z})
\end{align*}

Therefore
\[
	\mathbb{Z}/(r_1\cdots r_{k}\mathbb{Z}) \rightarrow (\mathbb{Z}/r_1\mathbb{Z})\cross\cdots\cross(\mathbb{Z}/r_{k}\mathbb{Z})
\]
is an isomorphism, 
which by induction shows that the statement is true for all $n$.

\newpage
\section*{Problem 2}

When an element $x \in (\mathbb{Z}/p\mathbb{Z})^{\cross}$
is equal to its inverse, then $x^2 \equiv 1 \mod p$.
This implies that 
\[
	(x-1)(x+1) \equiv 0 \mod p
\]
so $x$ is equal to its inverse if and only if 
$x \equiv 1 \mod p$ or $x \equiv -1 \equiv p-1 \mod p$.
This implies that $(p-2)! \equiv 1 \mod p$ since each element in the 
product $2\cdot 3 \dotsm p-2$ has a distinct inverse
that is also in the product.
Therefore 
\begin{align*}
	(p-1)! &\equiv (p-2)!\cdot(p-1) \\
	&\equiv 1\cdot(-1) \\
	&\equiv -1 \mod p.
\end{align*}

\newpage
\section*{Problem 3}
$x$ and $y$ are equivalent to 
one of $0,1,2,3 \mod 4$.
Note that 
\begin{align*}
	0^2 &\equiv 0 \mod 4 \\
	1^2 &\equiv 1 \mod 4 \\
	2^2 &\equiv 0 \mod 4 \\
	3^2 &\equiv 1 \mod 4.
\end{align*}
Thus the sum of the squares $x^2+y^2$
can only be equal to $0,1,2 \mod 4$.
Therefore there does not exist integers 
$x^2+y^2 = n$ when $n \equiv 3 \mod 4$.

\newpage 
\section*{Problem 4}
Since $p \equiv 1 \mod 4$, we can write $p = 4n+1$ for some $n$.
Therefore the multiplicative group modulo $p$ has $4n$ elements.


Wilson's theorem says that the square of the  product of the numbers $1$ to $\frac{p-1}{2} = 2n$
is $-1 \mod p$.

\begin{align*}
	(p-1)! &\equiv 1 \cdot 2 \cdot \ldots \cdot \left(\frac{p-1}{2}\right)\cdot\left(\frac{p+1}{2}\right)\cdot \ldots \cdot (p-2)\cdot (p-1)\\
	&\equiv  1 \cdot 2 \cdot \ldots \cdot \left(\frac{p-1}{2}\right)\cdot\left(1 - \frac{p+1}{2}\right)\cdot \ldots \cdot -2 \cdot -1 \\
	&\equiv \Pi_{i=1}^{2n} i\cdot(-i) \\
	&\equiv \Pi_{i=1}^{2n} i^2 \\
	&\equiv -1 \mod p
\end{align*}

Therefore we can choose $x = \equiv \Pi_{i=1}^{2n} i$ to 
satisfy the equation $x^2 + 1 \equiv 0 \mod p$.











\end{document}