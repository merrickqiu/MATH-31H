\documentclass{article}

\usepackage{amsmath}
\usepackage{amssymb}
\usepackage{hyperref}
\usepackage{mathrsfs}
\usepackage{enumerate}
\usepackage{bm}
\usepackage{physics}
\setlength{\parindent}{0pt}
\usepackage[parfill]{parskip}
\usepackage[margin=1in]{geometry}

\DeclareMathOperator{\im}{im}

\begin{document}
\begin{center}
	\huge{\bf Math 100A: Homework 3} \\
	Merrick Qiu
\end{center}

\section*{Problem 1}
When $n = 1$,
any element $g \in G$ distinct from the identity has order $p$.
This is because $|\langle g \rangle|$ must divide $p$ by Lagrange's theorem.

Assume that there
exists an element with order $p$ for all groups with order 
$p^1, p^2, \cdots, p^{k-1}$.
Let $G$ be a group with order $p^k$.
Choose an element $g \in G$ distinct from the identity.
The subgroup $\langle g \rangle$ must have order that divides $p^{k}$
by Lagrange's theorem, meaning that it has order $p^1, p^2, \cdots, p^{k-1}, p^k$.

If the order is $p^k$, then $g$ generates $G$ and we can choose 
$g^{(p^{k-1})}$ as our element of order $p$.

Otherwise, there is an element $h \in \langle g \rangle$
with order $p$ by our inductive hypothesis since $\langle g \rangle$
has order $p^1, p^2, \cdots, p^{k-1}, p^k$.
Since $\langle g \rangle$ is a subgroup of $G$,
the same element $h$ is in $G$ with order $p$ as well.

\section*{Problem 2}
Since $G$ contains an element of order 10, 
10 must divide $|G|$.
Since $G$ contains an element of order 6,
6 must divide $|G|$.
Thus $|G|$ must be a multiple of 30
since 30 is the LCM of 10 and 6.

\section*{Problem 3}
By corollary 11,
$|G| = |\ker(\varphi)|\cdot|\im(\varphi)|$,
$|\ker(\varphi)|$ divides $|G|$, and 
$|\im(\varphi)|$ divides $|G|$ and $|G'|$.
Since $\phi$ is non-trivial we know that 
$|\im(\varphi)| \neq 1$.
Thus $|\im(\varphi)| = 3$ since it must divide both $15$ and $18$.
This then implies that $|\ker(\varphi)| = 6$ since 
$|G| = |\ker(\varphi)|\cdot|\im(\varphi)|$.

\section*{Problem 4}
Since $N$ is a subgroup of $G$ of index 2, $G$ is the disjoint
union of two cosets, $N$ and $G \setminus N$.
Let $g \in G$.
We want to show that $gN = Ng$.

In the case when $g \in N$, then these two sets are equal
since $gN = N = Ng$.

Let $g \notin N$.
This implies that $gN \neq N$ and so $gN = G \setminus N$.
Similarly $Ng \neq N$ so $Ng = G \setminus N$.
Therefore $gN = G \setminus N = Ng$.







\end{document}