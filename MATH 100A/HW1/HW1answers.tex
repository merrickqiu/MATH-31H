\documentclass{article}

\usepackage{amsmath}
\usepackage{amssymb}
\usepackage{hyperref}
\usepackage{mathrsfs}
\usepackage{enumerate}
\usepackage{bm}
\usepackage{physics}
\setlength{\parindent}{0pt}
\usepackage[parfill]{parskip}
\usepackage[margin=1in]{geometry}

\begin{document}
\begin{center}
	\huge{\bf Math 100A: Homework 1} \\
	Merrick Qiu
\end{center}

\section*{Problem 1}
Since $G \subset S$ and the law of composition is associative on $S$,
the same law of composition is associative on $G$.
Since $1 \in S$ and $1$ is its own inverse, $1 \in G$ so $G$ has an identity.
From the definition of $G$, each element in $G$ has an inverse.
If $x \in G$ and $y \in G$ then $xy \in G$ since 
$xy \in S$ and $xy$ has inverse $y^{-1}x^{-1}$, so $G$ is closed.

\section*{Problem 2}
Matrix multiplication for $\mathrm{SL}_2(\bf{Z})$ is associative since 
we can expand the matrix multiplications below to show they are equal.
\[
	\left(
	\begin{bmatrix}
		a & b \\
		c & d
	\end{bmatrix}
	\begin{bmatrix}
		e & f \\
		g & h
	\end{bmatrix}
	\right)
	\begin{bmatrix}
		i & j \\
		k & l
	\end{bmatrix}
	=
	\begin{bmatrix}
		a & b \\
		c & d
	\end{bmatrix}
	\left(
	\begin{bmatrix}
		e & f \\
		g & h
	\end{bmatrix}
	\begin{bmatrix}
		i & j \\
		k & l
	\end{bmatrix}
	\right)
\]
We have the identity element 
\[
	I = 
	\begin{bmatrix}
		1 & 0 \\
		0 & 1
	\end{bmatrix}
\]
Since each matrix has nonzero determinant so each matrix in $\mathrm{SL}_2(\bf{Z})$.
is invertible, and the inverse of a matrix with determinant 1 also has determinant 1.
The product of two matricies with determinant 1 also has determinant 1 so 
$\mathrm{SL}_2(\bf{Z})$ is closed under multiplication.

\section*{Problem 3}
We can define $\rho(x) = e^x$.
Thus $\rho(x+y) = e^{x+y} = e^xe^y = \rho(x)\rho(y)$.
\end{document}