\documentclass{article}

\usepackage{amsmath}
\usepackage{amssymb}
\usepackage{hyperref}
\usepackage{mathrsfs}
\usepackage{enumerate}
\usepackage{bm}
\usepackage{physics}
\setlength{\parindent}{0pt}
\usepackage[parfill]{parskip}
\usepackage[margin=1in]{geometry}

\DeclareMathOperator{\im}{im}

\begin{document}
\begin{center}
	\huge{\bf Math 100A: Homework 6} \\
	Merrick Qiu
\end{center}

\section*{Problem 1}
For any two nonzero elements $(x,y) \in S$ and $(u,v) \in S$,
we can multiply $(x,y)$ by an invertible matrix to get to $(u,v)$.

If $x, y \neq 0$ then
\[
	\begin{bmatrix}
		u \\ v
	\end{bmatrix} = 
	\begin{bmatrix}
		\frac{u}{x} & 0 \\
		0 & \frac{v}{y}
	\end{bmatrix}
	\begin{bmatrix}
		x \\ y
	\end{bmatrix}
\]
If $x = 0$ then
\[
	\begin{bmatrix}
		u \\ v
	\end{bmatrix} = 
	\begin{bmatrix}
		1 & \frac{u}{y} \\
		0 & \frac{v}{y}
	\end{bmatrix}
	\begin{bmatrix}
		0 \\ y
	\end{bmatrix}
\]
If $y = 0$ then 
\[
	\begin{bmatrix}
		u \\ v
	\end{bmatrix} = 
	\begin{bmatrix}
		\frac{u}{x} & 1\\
		\frac{v}{x} & 0
	\end{bmatrix}
	\begin{bmatrix}
		x \\ 0
	\end{bmatrix}
\]
Therefore $\text{GL}_2(\bf{R})$ acts transitively on $S$.
\newpage 
\section*{Problem 2}
A matrix is in the stabilizer if multiplying it 
with $[1,0]$ yields $[1,0]$.
Notice that the second column of a stabilizing matrix 
can be arbitrary
but the first column must be $[1,0]$ for 
the product to equal $[1,0]$.
\[
	\begin{bmatrix}
		1 \\ 0
	\end{bmatrix} =
	\begin{bmatrix}
		1 & a \\
		0 & b
	\end{bmatrix}
	\begin{bmatrix}
		1 \\ 0
	\end{bmatrix}
\]
However for the matrix to be invertible, we need $b \neq 0$
so the stabilizer is
\[
    \text{GL}_2(\textbf{R})_{[1,0]} =
	\left\{
	\begin{bmatrix}
		1 & a \\
		0 & b
	\end{bmatrix} :
	a, b \in \mathbb{R}, b\neq 0
	\right\}.
\]
\newpage
\section*{Problem 3}
\begin{enumerate}[(a)]
	\item Since $g \in \text{SL}_2(\bf{R})$ we have that 
	$ad-bc = 1$.
	If $cz+d = cx+d + i(cy) = 0$, then $cx+d = 0$ and $cy = 0$.
	This implies that $c = 0$ since $y > 0$. This then implies that $d = 0$, however
	this means that $ad-bc = 0$ which is a contradiction,
	so $cz+d \neq 0$.
	\item 
	\begin{align*}
		\Im g \cdot z &= \Im\frac{az+b}{cz+d}\\
		&= \Im \frac{(ax+b) + iay}{(cx+d) + icy} \cdot \frac{(cx+d) - icy}{(cx+d) - icy} \\
		&= \Im \frac{(ax+b)(cx+d) + acy^2 + i(ay(cx+d) - cy(ax+b))}{(cx+d)^2 + (cy)^2} \\
		&= \frac{acxy + ady - acxy - bcy}{|cz+d|^2} \\
		&= \frac{(ad-bc)y }{|cz+d|^2} \\
		&= \frac{y }{|cz+d|^2} \\
	\end{align*}
	\item First note that 
	\[	
		1 \cdot z = \frac{1z + 0}{0z + 1} = z
	\]
	If $G = \begin{bmatrix}
		a & b \\
		c & d
	\end{bmatrix}$ and 
	$H = \begin{bmatrix}
		e & f \\
		g & h
	\end{bmatrix}$
	then

	\begin{align*}
		(GH) \cdot z &=  
		\begin{bmatrix}
			ae+bg & af+bh \\
			ce+dg & cf+dh
		\end{bmatrix} \cdot z \\
		&= \frac{(ae+bg)z + af+bh}{(ce+dg)z + cf+dh} \\
		&= \frac{\frac{aez + af + bgz +bh}{gz + h}}{\frac{cez + cf+dgz + dh}{gz + h}} \\
		&= \frac{a\frac{ez + f}{gz + h} + b}{c\frac{ez + f}{gz + h} + d} \\
		&= G \cdot \frac{ez + f}{gz + h}\\
		&= G \cdot (H \cdot z)
	\end{align*}
\end{enumerate}
\newpage
\section*{Problem 4}
\begin{enumerate}
	\item For complex numbers $x+iy\in \mathcal{H}$ and $u+iv\in \mathcal{H}$, we need to show that 
	there exists upper triangle matrix $g$ such that $g\cdot (x+iy) = u+iv$.
	Let $g = \begin{bmatrix}
		a & b \\
		0 & d 
	\end{bmatrix}$.
	\begin{align*}
		g\cdot (x+iy) &= \frac{a(x+iy) + b}{d} \\
		&= \left(\frac{ax+b}{d}\right)+i\frac{ay}{d} \\
		&= u + iv
	\end{align*}
	Since $v>0$, we can choose $a = 1$, $d= \frac{y}{v}$, and $b = ud-x$ so that 
	$g\cdot (x+iy) = u+iv$.
	Since it is possible to get from any element to any other element in $\mathcal{H}$
	using group actions from $B$, there is only one orbit and $B$ acts transitively on 
	$\mathcal{H}$. Since $B \subseteq \text{SL}_2(\textbf{R})$, 
	$\text{SL}_2(\textbf{R})$ also has just one orbit and so it also acts transitively on $B$.
	\item Let $g \in \text{SL}(2)$ where $g = \begin{bmatrix}
		a & b \\
		c & d 
	\end{bmatrix}$ and
	$ad-bc = 1$ since the determinant is 1.
	\begin{align*}
		g \cdot i &= \frac{ai + b}{ci+d} \\
		&= \frac{b + ai}{d + ci} \cdot \frac{d - ci}{d - ci} \\
		&= \frac{(bd+ac)+i(ad-bc)}{d^2+c^2} \\
		&= \frac{(bd+ac)+i}{d^2+c^2} \\
	\end{align*}
	We need $d^2+c^2 = 1$ and $bd+ac = 0$ for $g$ to be a stabilizer. This means that 
	the inner product of the second row with itself is $1$ and
	the inner product of the first and second row is $0$.
	It also must be that $a^2+b^2 = 1$ since the determinant is $1$
	and the rows are orthogonal.
	Algebraically we can see this by adding the two following equations
	to get the third
	\[
		1 =(ad-bc)^2 = a^2d^2 - 2abcd + b^2c^2
	\]
	\[
		0 = (ac+bd)^2 = a^2c^2 + 2abcd + b^2d^2
	\]
	\[
		1 = a^2d^2 + a^2c^2 + b^2c^2 + b^2d^2 = a^2(c^d+d^2) + b^2(c^d+d^2) = a^2+b^2
	\]
	Thus $gg^t = g^tg = I$,
	so the stabilizer is $\text{SO}_2(\textbf{R})$.
	Note that this means
	\item If $g \in \text{SL}_2(\textbf{R})$ then there must 
	exist $b \in B$ such that 
	$b\cdot i = g\cdot i$ since $\text{SL}_2(\textbf{R})$
	and $B$ both act transitively on $\mathcal{H}$.
	Therefore $i = b^{-1}g \cdot i$ so $g^{-1}b$ is in the stabilizer of $i$,
	which is $\text{SO}_2(\textbf{R})$.
	Therefore $b^{-1}g = h$ for some $h \in \text{SO}_2(\textbf{R})$,
	ie $g = bh$.

\end{enumerate}

\newpage
\end{document}