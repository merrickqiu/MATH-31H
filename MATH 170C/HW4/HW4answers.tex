\documentclass{article}

\usepackage{amsmath}
\usepackage{amssymb}
\usepackage{hyperref}
\usepackage{mathrsfs}
\usepackage{enumerate}
\usepackage[shortlabels]{enumitem}
\usepackage{bm}
\usepackage{matlab-prettifier}
\setlength{\parindent}{0pt}

\begin{document}
\begin{center}
	\huge{\bf Math 170C: Homework 4} \\
	Merrick Qiu
\end{center}
\section*{Problem 1}
Letting $x_0 = t$, $x_1 = x$, $x_2 = x'$, and $x_3 = x''$ yields
\[
	\begin{cases}
		x'_0 = 1 \\
		x'_1 = x_2 \\
		x'_2 = x_3\\
		x'_3 = e^{x_0} -2x_3 + x_2 + 2x_1
	\end{cases}
\]
with initial condition $X = (8, 3,2,1)^T$
\newpage 

\section*{Problem 2}
We can transform the first problem to the second with the change of variables
$t = 3+4s$,  $y(s) = x(3+4s)$, $y'(s) = 4x'(3+4s)$, $y''(s) = 16x''(3+4s)$
\begin{align*}
	x(3) = \alpha \implies y(0) &= \alpha \\
	x(7) = \beta \implies y(1) &= \beta \\
	x'' = t+x^2-3x' \implies y'' &= 16((3+4s)+y^2-3(y'/4)) \\
	&= 48 +64s + 16y^2 - 12 y'.
\end{align*}
Thus, theorem 2 holds for this problem.
\newpage

\section*{Problem 3}
Suppose we find a solution $x_1$ with initial conditions $x_1(a)$ and $x_1'(a)$
such that $c_{11}x_1(a) + c_{12}x'_1(a) = \alpha$.
Then consider $x_2$ such that $x_2(a) = -c_{12}$ and $x'_2(a) = c_{11}$.
Consider the solution $x_1 + \lambda x_2$.
This satisfies the initial condition at $a$ since 
\begin{align*}
	c_{11}(x_1(a) + \lambda x_2(a)) + c_{12}(x'_1(a) + \lambda x'_2(a)) 
	&= (c_{11}x_1(a) + c_{12}x'_1(a)) + (-\lambda c_{11}c_{12} + \lambda c_{12}c_{11}) \\
	&= \alpha + 0 \\
	&= \alpha
\end{align*}
We then want to select $\lambda$ such that 
\[
	c_{21}(x_1(b) + \lambda x_2(b)) + c_{22}(x'_1(b) + \lambda x'_2(b)) = \beta
\]
Solving for $\lambda$ yields 
\[
	\lambda = \frac{\beta - c_{21}x_1(b) - c_{22}x'_1(b)}{x_2(b) + x'_2(b)}.
\]
\end{document}