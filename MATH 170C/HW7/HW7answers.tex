\documentclass{article}

\usepackage{amsmath}
\usepackage{amssymb}
\usepackage{hyperref}
\usepackage{mathrsfs}
\usepackage{enumerate}
\usepackage[shortlabels]{enumitem}
\usepackage{bm}
\usepackage{matlab-prettifier}
\usepackage[margin=1in]{geometry}
\usepackage{physics}
\setlength{\parindent}{0pt}


\begin{document}
\begin{center}
	\huge{\bf Math 170C: Homework 7} \\
	Merrick Qiu
\end{center}

\section*{Problem 1}
The solution is
\[
	u(x,t) = \sum_{n=1}^N c_n e^{-n^2\pi^2t}\sin(n\pi x)
\]
The solution has the correct initial time condition since
\[
	u(x,0) = \sum_{n=1}^N c_n e^{0}\sin(n\pi x) =  \sum_{n=1}^N c_n \sin(n\pi x).
\]
The solution has the correct initial distance conditions since
\[
	u(0,t) = \sum_{n=1}^N c_n e^{-n^2\pi^2t}\sin(0) = 0
\]
\[
	u(1,t) = \sum_{n=1}^N c_n e^{-n^2\pi^2t}\sin(n\pi) = 0
\]
It satisfies the differential equation since 
\begin{align*}
	u_{xx} &= \frac{\partial}{\partial x} n\pi \sum_{n=1}^N c_n e^{-n^2\pi^2t}\cos(n\pi x) \\
	&= -n^2\pi^2\sum_{n=1}^N c_n e^{-n^2\pi^2t}\sin(n\pi x) \\
	&= u_t
\end{align*}
\newpage

\section*{Problem 2}
First we find $g$ such that $\nabla^2 g = f$ in $\Omega$.
Then we solve the dirichlet problem in $\Omega$, using $-g$ for the boundary values.
Call this solution $v$ so that 
\[
	\begin{cases}
		\nabla^2 v = 0 & \text{in } \Omega \\
		v = -g & \text{on } \partial \Omega
	\end{cases}
\]
Thus, $u = v + g$ will equal $0$ on the boundary and it will equal $\nabla^2 u = f$ in $\Omega$,
which solves the problem.
\newpage 

\section*{Problem 3}
By the Cauchy-Riemann equations we have that $u_x = v_y$ and $u_y = -v_x$.
We need to prove that $u$ and $v$ are harmonic, meaning that $\nabla^2 u = 0$ and $\nabla^2 v = 0$.
These directly follow from the Cauchy-Riemann equations since
\begin{align*}
	\nabla^2 u &= u_{xx} + u_{yy} \\
	&= v_{yx} - v_{xy} \\
	&= 0
\end{align*}
\begin{align*}
	\nabla^2 v &= v_{xx} + v_{yy} \\
	&= - u_{yx} + u_{xy} \\
	&= 0
\end{align*}
\newpage




\end{document}