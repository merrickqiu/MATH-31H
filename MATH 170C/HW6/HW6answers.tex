\documentclass{article}

\usepackage{amsmath}
\usepackage{amssymb}
\usepackage{hyperref}
\usepackage{mathrsfs}
\usepackage{enumerate}
\usepackage[shortlabels]{enumitem}
\usepackage{bm}
\usepackage{matlab-prettifier}
\usepackage[margin=1in]{geometry}
\usepackage{physics}
\setlength{\parindent}{0pt}


\begin{document}
\begin{center}
	\huge{\bf Math 170C: Homework 6} \\
	Merrick Qiu
\end{center}

\section*{Problem 1}
We are attempting to solve the system of equations $X' = AX$ where 
\[
	A = 
	\begin{bmatrix}
		3 & -5 \\
		2 & 1
	\end{bmatrix}.
\]
This has solution $X(t) = e^{\lambda t}V$
where $\lambda$ and $V$ are the eigenvalue and eigenvector of $A$.

The characteristic polynomial and its eigenvalues are
\[
	\left|\begin{matrix}
		\lambda - 3 & 5 \\
		-2 & \lambda - 1
	\end{matrix} \right|
	= (\lambda - 3)(\lambda - 1) + 10
	= \lambda^2 -4\lambda + 13
\]
\[
	\lambda = \frac{4 \pm \sqrt{16-52}}{2} = 2 \pm 3i.
\]
The eigenvectors for $\lambda_1 = 2 + 3i$ and $\lambda_2 = 2 - 3i$ are
\[
\begin{bmatrix}
	-1+3i & 5 \\
	-2 & 1+3i
\end{bmatrix}V_1 = 0 \implies 
V_1 = 
\begin{bmatrix}
	1+3i \\ 2
\end{bmatrix},
\]
\[
\begin{bmatrix}
	-1-3i & 5 \\
	-2 & 1-3i
\end{bmatrix}V_2 = 0 \implies 
V_1 = 
\begin{bmatrix}
	1-3i \\ 2
\end{bmatrix}.
\]
Thus for a constants $C_1$ and $C_2$, the solutions are 
\[
	X(t) = C_1e^{(2+3i)t}
	\begin{bmatrix}
		1+3i \\ 2
	\end{bmatrix} +
	C_2e^{(2-3i)t}
	\begin{bmatrix}
		1-3i \\ 2
	\end{bmatrix}
\]
\newpage 

\section*{Problem 2}
If $x_1 = x$ and $x_2 = x'$ then we can rewrite the problem as
\[
	\begin{cases}
		x_1' = x_2 \\
		x_2' = -19x_1 -20x_2
	\end{cases}
\]
with $X(0) = [2,-20]^T$.
Note that 
\[
	A =
	\begin{bmatrix}
		0 & 1 \\
		-19 & -20
	\end{bmatrix} = 
	\begin{bmatrix}
		-1 & -1 \\
		19 & 1
	\end{bmatrix}
	\begin{bmatrix}
		-19 & 0 \\
		0 & -1
	\end{bmatrix}
	\begin{bmatrix}
		\frac{1}{18} & \frac{1}{18} \\
		-\frac{19}{18} & -\frac{1}{18}
	\end{bmatrix}
\]
It has solution 
\begin{align*}
	X(t) &= e^{At}X(0) \\
	&= \begin{bmatrix}
		0 & 1 \\
		-19 & -20
	\end{bmatrix} = 
	\begin{bmatrix}
		-1 & -1 \\
		19 & 1
	\end{bmatrix}
	\begin{bmatrix}
		e^{-19t} & 0 \\
		0 & e^{-t}
	\end{bmatrix}
	\begin{bmatrix}
		\frac{1}{18} & \frac{1}{18} \\
		-\frac{19}{18} & -\frac{1}{18}
	\end{bmatrix}
	\begin{bmatrix}
		2 \\ -20
	\end{bmatrix} \\
	&= 
	\begin{bmatrix}
		e^{-19t} + e^{-t} \\
		-19e^{-19t} - e^{-t}
	\end{bmatrix}.
\end{align*}

Looking at the eigenvalues and the solution,
we can see that the problem is stiff.
$-19$ is much larger in magnitude than $-1$ and so there is a 
``wide disparity in the time scales of the components in the vector solution".
$e^{-19}$ is transient and we need very small time steps to account for its effects.
% For example, if we were to use Euler's method on the system of equations,
% \[
% 	X_{n+1} = X_n + hAX_n = (I + hA)X_n
% \]
% \[
% 	X_n = (I + hA)^n X(0)
% \]
\newpage 

\section*{Problem 3}
The relevant polynomials are
\[
	p(z) = z^2 + \alpha z - (1+\alpha)
\]
\[
	q(z) = - \frac{\alpha}{2}z^2 + \frac{4+3\alpha}{2}z
\]
\[
	\phi(z) = \left(1+\lambda h\frac{\alpha}{2}\right)z^2 + \left(\alpha - \lambda h\frac{4+3\alpha}{2}\right)z - (1+\alpha)
\]
For stability we need that all the roots of $p$ lie in the disk $|z| \leq 1$ and
roots of modulus $1$ are simple.
We need $-2 < \alpha < 0$ since the roots of $p$ are 
\[
	z = \frac{-\alpha \pm \sqrt{\alpha^2 + 4(1+\alpha)}}{2} 
	= \frac{-\alpha \pm (\alpha+2)}{2}
	= 1, -\alpha-1
\]
Our method is already consistent since $p(1) = 0$ and $p'(1) = q(1)$ for all $\alpha$.
\[
	p(1) = 1 + \alpha - (1+\alpha) = 0
\]
\[
	p'(1) = 2 + \alpha 
\]
\[
	q(1) = - \frac{\alpha}{2} + \frac{4+3\alpha}{2} = 2 + \alpha
\]

Convergence directly follows from stability and consistency. 

The method is second order since
\begin{align*}
	d_0 &= a_2 + a_1 + a_0 \\
	&= 1 + \alpha - (1+\alpha)\\
	&= 0
\end{align*}
\begin{align*}
	d_1 &= (2a_2 - b_2) + (a_1 - b_1) - b_0 \\
	&= \left(2 + \frac{\alpha}{2}\right) + \left(\alpha -\frac{(4+3\alpha)}{2}\right) - 0\\
	&= 0
\end{align*}

For A-stablility, we need the roots of $\phi$ ti
lie in the unit disk $|z| < 1$.
For this to happen, we can choose $\alpha = -1$ so that
\[
	\phi(z) = \left(1-\frac{h\lambda}{2}\right)z^2 - \left(1+\frac{h\lambda}{2}\right)z.
\]
Then our roots lie inside of the unit circle when $\Re(h\lambda) < 0$.
\[
	z=0 \quad z = \frac{2+h\lambda}{2-h\lambda}
\]
\newpage


\end{document}