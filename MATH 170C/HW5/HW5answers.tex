\documentclass{article}

\usepackage{amsmath}
\usepackage{amssymb}
\usepackage{hyperref}
\usepackage{mathrsfs}
\usepackage{enumerate}
\usepackage[shortlabels]{enumitem}
\usepackage{bm}
\usepackage{matlab-prettifier}
\usepackage[margin=0.5in]{geometry}
\setlength{\parindent}{0pt}

\usepackage[active,tightpage]{preview}

\renewcommand{\PreviewBorder}{1in}

\newcommand{\Newpage}{\end{preview}\begin{preview}}
\begin{document}
\begin{center}
	\huge{\bf Math 170C: Homework 5} \\
	Merrick Qiu
\end{center}
\begin{preview}

\section*{Problem 1}
\lstinputlisting[style=Matlab-editor]{BVP_shooting.m}

\begin{verbatim}
>> [x,t]=BVP_shooting(@(t, y, yp) exp(t) + y * cos(t) - (t + 1) * yp,0,1,1,3,0.1,0.0001,10)

x =

	1.0000
	1.1865
	1.3752
	1.5668
	1.7617
	1.9602
	2.1623
	2.3678
	2.5764
	2.7874
	3.0000


t =

			0
	0.1000
	0.2000
	0.3000
	0.4000
	0.5000
	0.6000
	0.7000
	0.8000
	0.9000
	1.0000
\end{verbatim}
\begin{verbatim}
>> [x,t]=BVP_shooting(@(t, y, yp) exp(t) + y * cos(t) - (t + 1) * yp,0,1,1,3,0.05,0.0001,10)

x =

	1.0000
	1.0930
	1.1865
	1.2805
	1.3752
	1.4706
	1.5668
	1.6638
	1.7617
	1.8605
	1.9602
	2.0608
	2.1623
	2.2647
	2.3678
	2.4718
	2.5764
	2.6816
	2.7874
	2.8935
	3.0000


t =

			0
	0.0500
	0.1000
	0.1500
	0.2000
	0.2500
	0.3000
	0.3500
	0.4000
	0.4500
	0.5000
	0.5500
	0.6000
	0.6500
	0.7000
	0.7500
	0.8000
	0.8500
	0.9000
	0.9500
	1.0000
\end{verbatim}
\Newpage
\section*{Problem 2}
\lstinputlisting[style=Matlab-editor]{BVP_finitediff.m}
We can see that with finite differences there is a jump between $y_n$ and $y_{n+1}$
that gets smaller as $h \to 0$.
This jump does not exist with the shooting method.
\begin{verbatim}
>> [x,t] = BVP_finitediff(@(t)exp(t),@(t)cos(t),@(t) -(t+1),0,1,1,3,0.1)

x =

    1.0000
    1.1273
    1.2624
    1.4054
    1.5563
    1.7147
    1.8804
    2.0528
    2.2312
    2.4148
    3.0000


t =

         0
    0.1000
    0.2000
    0.3000
    0.4000
    0.5000
    0.6000
    0.7000
    0.8000
    0.9000
    1.0000
\end{verbatim}
\begin{verbatim}
>> [x,t] = BVP_finitediff(@(t)exp(t),@(t)cos(t),@(t) -(t+1),0,1,1,3,0.05)

x =

    1.0000
    1.0769
    1.1551
    1.2345
    1.3154
    1.3975
    1.4812
    1.5662
    1.6527
    1.7406
    1.8299
    1.9206
    2.0127
    2.1060
    2.2006
    2.2963
    2.3931
    2.4909
    2.5896
    2.6890
    3.0000


t =

         0
    0.0500
    0.1000
    0.1500
    0.2000
    0.2500
    0.3000
    0.3500
    0.4000
    0.4500
    0.5000
    0.5500
    0.6000
    0.6500
    0.7000
    0.7500
    0.8000
    0.8500
    0.9000
    0.9500
    1.0000
\end{verbatim}
\end{preview}



\end{document}