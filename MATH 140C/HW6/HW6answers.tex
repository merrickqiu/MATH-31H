\documentclass{article}

\usepackage{amsmath}
\usepackage{amssymb}
\usepackage{hyperref}
\usepackage{mathrsfs}
\usepackage{enumerate}
\usepackage{bm}
\setlength{\parindent}{0pt}
\usepackage[parfill]{parskip}
\usepackage[margin=1in]{geometry}


\begin{document}
\begin{center}
	\huge{\bf Math 140C: Homework 6} \\
	Merrick Qiu
\end{center}

\section*{Rudin 11.1}
Let $E_n$ be the subset of $E$ on which $f(x) > \frac{1}{n}$.
Write $A = \bigcup E_n$.
If $\mu(A) = 0$ then $\mu(E_n) = 0$ since $A \supset E_n$.
If $\mu(E_n) = 0$ then $\mu(A) = 0$ since for the disjoint sets 
$E_n' = E_n \setminus \bigcup_1^{n-1} E_i$, $\mu(E') = 0$ and $\bigcup E' = \bigcup E = A$ so $\mu(A) = 0$ by countable additivity.


We know that $\mu(E_n) = 0$ since $0 \leq \frac{1}{n}\mu(E_n) \leq\int_{E_n} f \,du \leq \int_E f \,du = 0$.
Since $A$ is the set where $f(x) > 0$ and $f(x) \geq 0$, $\mu(E_n) = 0$ implies that 
$\mu(A) = 0$, which then implies that $f(x) = 0$ almost everywhere.
\newpage 

\section*{Rudin 11.2}
We can apply the conclusion of the previous problem twice on 
the set of non-negative values and non-positive values to show that 
$f^+(x) = 0$ almost everywhere and $f^-(x) = 0$.  
Thus $f(x) = 0$ almost everywhere on $E$ since the 
set of values where $f(x) \neq 0$ is the union of the set of values where
$f(x) < 0$ and $f(x) > 0$.
\newpage 

\section*{Rudin 11.5}
For $0 \leq x \leq \frac{1}{2}$, the $\liminf$ can be achieved by the subsequence of even elements, $f_{2k}$. 
For $\frac{1}{2} < x \leq 1$  the $\liminf$ can be achieved by taking the subsequence of odd elements, $f_{2k+1}$.
Thus we have that
\[
	f(x) = \liminf_{n \to \infty} f_n(x) = 0.
\]
However each $f_n$ is a simple function with integral 
$\int_0^1 f_n(x) \,dx = \frac{1}{2}$.
This problem is in agreement with (77) since
\[
	0 = \int_E f \,d\mu \leq \liminf_{n\to \infty} \int_E f_n \,d\mu = \frac{1}{2}. 
\]
\newpage 

\section*{Rudin 11.6}
Since $|f_n(x) - 0| < \frac{1}{n}$, $f_n(x) \to 0$
uniformly as $n \to \infty$.

However all $f_n$ cannot be bounded by a function $g \in \mathscr{L}$,
so Theorem 11.32 fails, as given by the fact that
$\int_{- \infty}^\infty f_n \,dx = \int_{-n}^n \frac{1}{n} \,dx = 2$.
\newpage


\end{document}