\documentclass{article}

\usepackage{amsmath}
\usepackage{amssymb}
\usepackage{hyperref}
\usepackage{mathrsfs}
\usepackage{enumitem}
\usepackage{bm}
\setlength{\parindent}{0pt}

%User defined commands
\def\upint{\mathchoice%
    {\mkern13mu\overline{\vphantom{\intop}\mkern7mu}\mkern-20mu}%
    {\mkern7mu\overline{\vphantom{\intop}\mkern7mu}\mkern-14mu}%
    {\mkern7mu\overline{\vphantom{\intop}\mkern7mu}\mkern-14mu}%
    {\mkern7mu\overline{\vphantom{\intop}\mkern7mu}\mkern-14mu}%
  \int}
\def\lowint{\mkern3mu\underline{\vphantom{\intop}\mkern7mu}\mkern-10mu\int}
\DeclareMathOperator{\Span}{span}

\begin{document}
\begin{center}
	\huge{\bf Math 140C: Homework 1} \\
	Merrick Qiu
\end{center}

\section*{Problem 1}
\begin{enumerate}
  \item ($\implies$) Since $\bm{y} \in \Span(E)$, we can write $\bm{y} = c_1\bm{v_1} + \cdots + c_r\bm{v_r}$
  for some $c_i \in \mathbb{R}$. Thus we have that
  \[
    c_1\bm{v_1} + \cdots + c_r\bm{v_r} - \bm{y} = 0
  \]
  which implies that $E \cup \{\bm{y}\}$ is linearly dependent.
  
  ($\impliedby$) Since $E \cup \{\bm{y}\}$ is linearly dependent,
  we can write 
  $c_1\bm{v_1} + \cdots + c_r\bm{v_r} + c_{r+1}\bm{y} = 0$
  for some $c_i \in \mathbb{R}$.
  Thus 
  \[
    \bm{y} = -\frac{c_1}{c_{r+1}}\bm{v_1} - \cdots - \frac{c_r}{c_{r+1}}\bm{v_r} 
  \]
  which implies that $\bm{y} \in \Span(E)$.

  \item 
  If $\bm{x} \in \Span(E)$ then $\bm{x} = a_1\bm{v_1} + \cdots + a_r\bm{v_r}$ 
  for some $a_i \in \mathbb{R}$.
  It is also true that $\bm{x} = a_1\bm{v_1} + \cdots + a_r\bm{v_r} + 0\bm{y}$ 
  so $\bm{x} \in \Span(E\cup \{\bm{y}\})$. 

  If $\bm{x} \in \Span(E \cup \{\bm{y}\})$ then
  $\bm{x} = a_1\bm{v_1} + \cdots + a_r\bm{v_r} + a_{r+1}\bm{y}$  for some $a_i \in \mathbb{R}$.
  Since $E \cup \{\bm{y}\}$ is linearly dependent, $\bm{y} \in \Span(E)$ so 
  $\bm{y} = c_1\bm{v_1} + \cdots + c_r\bm{v_r}$ for some $c_i \in \mathbb{R}$.
  Thus 
  \[
    \bm{x} = (a_1 + a_{r+1}c_1)\bm{v_1} + \cdots + (a_r + a_{r+1}c_r)\bm{v_r}
  \]
  so $\bm{x} \in \Span(E)$
\end{enumerate}
\newpage 
\section*{Rudin 9.1}
If $\bm{x} \in \Span(S)$ and $\bm{y} \in \Span{S}$ then for some 
set of $\bm{v_1},\cdots,\bm{v_n} \in S$  and constants $a_i, b_i \in \mathbb{R}$, we can write
\[
  \bm{x} = \sum_{i=1}^{n} a_i \bm{v_i} \qquad  \bm{y} = \sum_{i=1}^{n} b_i \bm{v_i}
\]
$\Span(S)$ is a vector space since for all $c \in \mathbb{R}$ and $\bm{x}, \bm{y} \in \Span(S)$,
\[
  c\bm{x} = \sum_{i=1}^{n} ca_i \bm{v_i} \in \Span(S)
\]
\[
  \bm{x} + \bm{y} = \sum_{i=1}^{n} (a_i + b_i) \bm{v_i} \in \Span(S).
\]
\newpage

\section*{Rudin 9.2}
If $A$ and $B$ are linear transformations in $X$ then for all $\bm{x}, \bm{v_1}, \bm{v_2} \in X$
\begin{align*}
  BA(\bm{v_1} + \bm{v_2}) &= B(A(\bm{v_1} + \bm{v_2})) \\
  &= B(A\bm{v_1} + A\bm{v_2}) \\
  &= BA\bm{v_1} + BA\bm{v_2} 
\end{align*}

\[
  BA(c\bm{x}) = B(cA\bm{x}) = cBA\bm{x}
\]

Thus $BA$ is also a linear transformation.

If $A$ is one-to-one from $X$ onto $X$ then for all $\bm{x}, \bm{v_1}, \bm{v_2} \in X$
we can write 
\[
  \bm{x} = A\bm{y} \qquad \bm{v_1} = A\bm{v_1} \qquad \bm{v_2} = A\bm{v_2}
\]
for some vectors $\bm{y}, \bm{v_1}, \bm{v_2} \in X$.
$A^{-1}$ is a linear operator since
\begin{align*}
  A^{-1}(\bm{v_1} + \bm{v_1}) &= A^{-1}(A\bm{v_1} + A\bm{v_2}) \\
  &= A^{-1}A(\bm{v_1} + \bm{v_2}) \\
  &= \bm{v_1} + \bm{v_2} \\
  &= A^{-1}\bm{v_1} + A^{-1}\bm{v_1}
\end{align*}
\[
  A^{-1}(c\bm{x}) =  A^{-1}(cA\bm{y}) = A^{-1}A(c\bm{y}) = c\bm{y} = cA^{-1}\bm{x}.
\]
The inverse of $A^{-1}$ is $A$ since 
\begin{align*}
  A(A^{-1}\bm{x}) =  A(A^{-1}A\bm{y}) = A\bm{y} = \bm{x}
\end{align*}
\newpage

\section*{Rudin 9.3}
Suppose $A$ is not 1-1. 
Then for some $\bm{y} \in Y$, there exists distinct $\bm{v}, \bm{w} \in X$
such that $A\bm{v} = \bm{y}$ and  $A\bm{w} = \bm{y}$.
Subtracting these two equations implies that
\[
  A(\bm{v}-\bm{w}) = 0
\]
which contradicts our assumption that $A\bm{x} = 0$ only when $\bm{x} = 0$.
\newpage

\section*{Rudin 9.4}
Let $A \in L(X, Y)$ be a linear transformation
Let $\bm{x}, \bm{y} \in \mathcal{N}(A)$.
$\mathcal{N}(A)$ is a vector space since 
\[
  A(\bm{x} + \bm{y}) = A\bm{x} + A\bm{y} = \bm{0}
\]
\[
  A(c\bm{x}) = cA\bm{x} = \bm{0}
\]
Let $\bm{x}, \bm{y} \in \mathcal{R}(A)$.
We can write $\bm{x} = A\bm{p}$ and $\bm{y} = A\bm{q}$
for some $\bm{p}, \bm{q} \in X$.
By the linearity of $A$,
\[
  \bm{x} + \bm{y} = A\bm{p} + A\bm{q} = A(\bm{p} + \bm{q}) \in \mathcal{R}(A)
\]
\[
  c\bm{x} = cA\bm{p} = A(c\bm{p}) \in \mathcal{R}(A)
\]
\newpage 

\section*{Rudin 9.5}
Let $\bm{x} = \bm{x}_1\bm{e_1} + \cdots \bm{x}_n\bm{e_n}$ 
for the standard basis vectors $\bm{e}_i$.
If we let $\bm{y} \in \mathbb{R}^n$ with $\bm{y}_i = A\bm{e_i}$ then
\[
  A\bm{x} = A(\bm{x}_1\bm{e_1} + \cdots \bm{x}_n\bm{e_n}) =
   c_1\bm{y}_1 + \cdots + c_n\bm{y}_n =
   \bm{x} \cdot \bm{y}.
\]
It is unique since if there was $\bm{z}$ such that $A\bm{x} = \bm{x} \cdot \bm{z}$,
then 
\[
  \bm{y} - \bm{z}|^2 = 
  \bm{y}\cdot \bm{y} - \bm{y}\cdot \bm{z} - \bm{z}\cdot \bm{y} - \bm{z}\cdot \bm{z} =
  A(\bm{y}) - A(\bm{y}) - A(\bm{z}) + A(\bm{z}).
\]
By the Schwarz inequality,
\begin{align*}
  ||A|| = \sup |A \bm{x}|
        = \sup |\bm{x} \cdot \bm{y}|
        \leq \sup |\bm{x}||\bm{y}|.
\end{align*}
which implies that $||A|| \leq |y|$.
Also note that $A\left(\frac{\bm{y}}{|\bm{y}|}\right) = \frac{\bm{y}}{|\bm{y}|}\cdot \bm{y} = |\bm{y}|$
so  $||A|| \geq |y|$.
\end{document}