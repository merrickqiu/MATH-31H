\documentclass{article}

\usepackage{amsmath}
\usepackage{amssymb}
\usepackage{hyperref}
\usepackage{mathrsfs}
\usepackage{enumerate}
\usepackage{bm}
\setlength{\parindent}{0pt}
\usepackage[parfill]{parskip}
\usepackage[margin=1in]{geometry}


\begin{document}
\begin{center}
	\huge{\bf Math 140C: Homework 7} \\
	Merrick Qiu
\end{center}

\section*{Problem 1}
By Theorem 11.24, we can treat $\phi(A)$ as a measure where
\[
	\phi(A) = \int_A x^\alpha \,dx.
\]
Let $A_n = (\frac{1}{n}, 1)$ and $A = (0,1)$.
Since $x^\alpha$ is Riemann integrable on $(\frac{1}{n},1)$, we can write
\begin{align*}
	\int_0^1 x^\alpha &= \phi(A) \\
	&= \lim_{n \to \infty} \phi(A_n) \\
	&= \lim_{n \to \infty} \int_\frac{1}{n}^1 x^\alpha \\
	&= \lim_{n \to \infty} \left[\frac{x^{\alpha+1}}{\alpha+1}\right]_{1/n}^1 \\
	&= \frac{1}{\alpha + 1}
\end{align*}

Thus, the function is Lebesgue integrable when $\alpha > -1$.
When $\alpha \leq -1$, the sequence of integrals diverges so the Lebesgue integral
diverges as well.
\newpage

\section*{Rudin 11.8}
Theorem 6.20 says that $F'(x) = f(x)$ when $f$ is continuous.
Theorem 11.33 says that $f \in \mathcal{R}$ iff $f$ is continuous almost everywhere.
Therefore $F'(x) = f(x)$ almost everywhere on $[a,b]$.
\newpage

\section*{Rudin 11.9}
We can show that $F$ is continuous at $x$ if 
$F(x_n) \to F(x)$ for any sequence $x_n \to x$.
To do so, we can apply the dominated convergence theorem.
For any sequence $x_n \to x$, we can define $f_n \to f$ by
\[
	f_n(x) = 
	\begin{cases}
		f(x) & a \leq x < x_n \\
		0	& \text{otherwise}
	\end{cases}.
\]
Then choose $g = |f|$ to be the dominating function.
Then we have that
\[
	\lim_{n\to \infty} F(x_n) = \lim_{n\to \infty} \int_a^x f_n = \int_a^x f = F(x)
\]
which implies that $F$ is continuous since $x$ was arbitrary.
\newpage

\section*{Rudin 11.10}
$f \in \mathscr{L}^2(\mu)$ on $X$ implies 
\[
	\int_X |f|^2 \,d\mu < \infty.
\]
We can break $X$ into two sets:
let $X_1$ be the set where $|f(x)| > 1$ and 
$X_2$ be the set where $|f(x)| \leq 1$.
Note that $\mu(X_2) < \infty$ since $\mu(X) < \infty$.
Thus 
\begin{align*}
	\int_X |f| \,d\mu &= \int_{X_1} |f| \,d\mu + \int_{X_2} |f| \,d\mu \\
	&< \int_{X_1} |f|^2 \,d\mu  + \mu(X_2) \\
	&< \infty.
\end{align*}
If we choose $X = \mathcal{R}$, then $f(x) = \frac{1}{1+|x|}$ is in
$\mathscr{L}^2$ since
\begin{align*}
	\int_{\mathbb{R}} \left(\frac{1}{1+|x|}\right)^2 \,d\mu 
	&= 2\int_0^\infty \frac{1}{(1+x)^2} \,d\mu  \\
	&< 2\int_0^\infty \frac{1}{x^2} \,d\mu \\
	&<\infty 
\end{align*}
but $f \notin \mathscr{L}$ since
\begin{align*}
	\int_{\mathbb{R}} \frac{1}{1+|x|} \,d\mu &\geq \int_0^\infty \frac{1}{1+x} \\
	&= \left[\ln(1+x)\right]_0^\infty \\
	&\to \infty
\end{align*}

\newpage

\section*{Rudin 11.11}
Suppose $\{f_n\}$ is a Cauchy sequence.
Thus, we can find a sequence $\{n_k\}$ so that
\[
	|| || < \frac{1}{}
\]
\[
	\int_X |f_m - f_n| \,du < \epsilon
\]
\newpage

\section*{Rudin 11.12}
\newpage


\end{document}